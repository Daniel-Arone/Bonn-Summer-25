\section{Lecture 1 -- 7th April 2025}\label{sec: lectuer 1}
We begin by a consideration of the theory of smoothness, first in the local case. This is done by defining the sheaves of K\"{a}hler differentials on schemes -- in the local picture, the module of differentials on a ring. 
\begin{definition}[Derivation]\label{def: derivation}
    Let $B$ be an $A$-algebra and $M$ a $B$-module. An morphism of $A$-modules $D:B\to M$ is an $A$-derivation if it satisfies the Leibniz rule $\dform(xy)=x\dform(y)+y\dform(x)$ for all $x,y\in B$. Denote the set of $A$-derivations in $M$ by $\mathrm{Der}_{A}(B,M)$. 
\end{definition}
\begin{remark}
    It is necessary that $M$ is a $B$-module, since the Leibniz rule involves elements of $B$. 
\end{remark}
\begin{remark}\label{rmk: map from A is zero}
    Observe that the composition $A\to B\to M$is zero since $a=a\cdot 1_{B}$ and computing we get $\dform(a\cdot 1_{B})=a\dform(1_{B})$ by $A$-linearity, but on the other hand $\dform(a\cdot 1_{B})=a\dform(1_{B})+1_{B}\dform(a)$ by the Leibniz rule, so $a\dform(1_{B})=0$ showing $\dform(1_{B})=0$ and thus $\dform(a)=0$. 
\end{remark}
The K\"{a}hler differentials of a ring map is the universal recipient of an $A$-algebra $B$ in the following sense. 
\begin{definition}[Module of K\"{a}hler Differentials]
    Let $B$ be an $A$-algebra. The module of K\"{a}hler differentials of $B$ over $A$ is a $B$-module $\Omega^{1}_{B/A}$ with an $A$-derivation $\dform:B\to\Omega_{B/A}^{1}$ that is initial amongst $B$-modules recieving an $A$-derivation from $B$. 
\end{definition}
Unwinding the universal property, if $M$ is a $B$-module recieving an $A$-derivation from $B$ by $f:B\to M$, there is a unique factorization over $\Omega^{1}_{B/A}$ as follows.
$$% https://q.uiver.app/#q=WzAsMyxbMCwxLCJCIl0sWzIsMCwiTSJdLFswLDAsIlxcT21lZ2FeezF9X3tCL0F9Il0sWzIsMV0sWzAsMiwiXFxleGlzdHMhIiwwLHsic3R5bGUiOnsiYm9keSI6eyJuYW1lIjoiZGFzaGVkIn19fV0sWzAsMV1d
\begin{tikzcd}
	{\Omega^{1}_{B/A}} && M \\
	B
	\arrow[from=1-1, to=1-3]
	\arrow["{\exists!}", dashed, from=2-1, to=1-1]
	\arrow[from=2-1, to=1-3]
\end{tikzcd}$$
In particular, there is a bijection $\mathrm{Der}_{A}(B,M)\leftrightarrow\Hom_{\Mod_{B}}(\Omega_{B/A},M)$ functorial in $M$. 
\begin{proposition}\label{prop: uniqueness of differentials}
    Let $B$ be an $A$-algebra. The $B$-module $\Omega^{1}_{B/A}$ and the $A$-derivation $\dform:B\to\Omega^{1}_{B/A}$ exist and are unique up to unique isomorphism. 
\end{proposition}
\begin{proof}
    The module $\Omega^{1}_{B/A}$ can be constructed as the free $B$-module on elements $\dform x$ for $x\in B$ modulo the relations generated by the Leibniz rule and $\dform a=0$ for $a\in A$. Uniqueness up to unique isomorphism is clear from the universal property and Yoneda's lemma. 
\end{proof}
In special cases, the module of K\"{a}hler differentials can be described explicitly. 
\begin{example}
    Let $A=k,B=k[x_{1},\dots,x_{n}]$. $\Omega_{B/A}^{1}$ is a free module of rank $n$ with basis $\dform x_{i}$. The map $f\mapsto\sum_{i=1}^{n}\frac{\partial f}{\partial x_{i}}\cdot \dform x_{i}$ is an $A$-derivation and the map $\dform x_{i}\mapsto x_{i}$ defines an isomorphism $\Omega_{B/A}^{1}\to B^{\oplus n}$. 
\end{example}
K\"{a}hler differentials are also fairly easy to understand in the case of ring localizations and ring quotients. These will be important in understanding the sheaves of K\"{a}hler differentials of open and closed immersions in the case of schemes, respectively. 
\begin{proposition}\label{prop: differentials of open and closed immersions}
    Let $A$ be a ring. 
    \begin{enumerate}[label=(\roman*)]
        \item If $B=S^{-1}A$, then $\Omega^{1}_{B/A}=0$. 
        \item If $B=A/I$ for $I\subseteq A$ an ideal, then $\Omega^{1}_{B/A}=0$. 
    \end{enumerate}
\end{proposition}
\begin{proof}[Proof of (i)]
    We already have that $\dform a=0$ for all $a\in A$. We then observe that writing $a=s\cdot\frac{a}{s}$ we have 
    \begin{align*}
        0=\dform(a)=\dform\left(s\cdot\frac{a}{s}\right) &= s\dform\left(\frac{a}{s}\right)+\frac{a}{s}\dform(s) \\
        &= s\dform\left(\frac{a}{s}\right) && s\in A\Rightarrow \dform s=0 
    \end{align*}
    so $s\dform(\frac{a}{s})=0$ and $\dform(\frac{a}{s})=0$ whence the claim. 
\end{proof}
\begin{proof}[Proof of (ii)]
    The map $A\to B$ is surjective, so this is precisely the situation \Cref{rmk: map from A is zero}. 
\end{proof}
We can additionally understand sheaves of K\"{a}hler differentials in towers. Let $A\to B\to C$ be maps of rings. There is a natural $C$-linear map $\Omega^{1}_{B/A}\otimes_{B}C\to\Omega_{C/A}^{1}$ which is a $C$-module homomorphism induced by the diagram 
$$% https://q.uiver.app/#q=WzAsNCxbMCwwLCJCIl0sWzIsMCwiQyJdLFs0LDAsIlxcT21lZ2Ffe0MvQX1eezF9Il0sWzAsMSwiXFxPbWVnYV97Qi9BfV57MX0iXSxbMSwyLCJcXGRmb3JtX3tDL0F9Il0sWzAsMV0sWzAsMywiXFxkZm9ybV97Qi9BfSIsMl0sWzMsMiwiXFxleGlzdHMhIiwyLHsic3R5bGUiOnsiYm9keSI6eyJuYW1lIjoiZGFzaGVkIn19fV1d
\begin{tikzcd}
	B && C && {\Omega_{C/A}^{1}} \\
	{\Omega_{B/A}^{1}}
	\arrow[from=1-1, to=1-3]
	\arrow["{\dform_{B/A}}"', from=1-1, to=2-1]
	\arrow["{\dform_{C/A}}", from=1-3, to=1-5]
	\arrow["{\exists!}"', dashed, from=2-1, to=1-5]
\end{tikzcd}$$
where the top row is both $A$ and $B$-linear inducing a unique $B$-module map $\Omega_{B/A}^{1}\to\Omega_{C/A}^{1}$, considering the latter as a $B$-module. By the extension-restriction adjunction, however, we have 
$$\Hom_{\Mod_{B}}(\Omega_{B/A},\Omega_{C/A}|_{B})\leftrightarrow\Hom_{\Mod_{C}}(\Omega_{B/A}\otimes_{B}C,\Omega_{C/A})$$
hence the data of the dotted map in the diagram above gives rise to a unique map $\Omega_{B/A}\otimes_{B}C\to\Omega_{C/A}$. Arguing similarly, there is a $C$-linear map $\Omega^{1}_{C/A}\to\Omega_{C/B}^{1}$ induced by 
$$% https://q.uiver.app/#q=WzAsMyxbMCwwLCJDIl0sWzIsMCwiXFxPbWVnYV97Qy9CfV57MX0iXSxbMCwxLCJcXE9tZWdhX3tDL0F9XnsxfSJdLFswLDEsIlxcZGZvcm1fe0MvQn0iXSxbMCwyLCJcXGRmb3JtX3tDL0F9IiwyXSxbMiwxLCJcXGV4aXN0cyEiLDIseyJzdHlsZSI6eyJib2R5Ijp7Im5hbWUiOiJkYXNoZWQifX19XV0=
\begin{tikzcd}
	C && {\Omega_{C/B}^{1}} \\
	{\Omega_{C/A}^{1}}
	\arrow["{\dform_{C/B}}", from=1-1, to=1-3]
	\arrow["{\dform_{C/A}}"', from=1-1, to=2-1]
	\arrow["{\exists!}"', dashed, from=2-1, to=1-3]
\end{tikzcd}$$
where the map is induced by the universal property as any $B$-derivation is also an $A$-derivation. 

The maps in the preceding discussion assemble to give the following proposition. 
\begin{proposition}\label{prop: tensor exact sequence}
    Let $A\to B\to C$ be maps of rings. There is an exact sequence 
    $$\Omega_{B/A}^{1}\otimes_{B}C\to \Omega_{C/A}^{1}\to\Omega_{C/B}^{1}\to0.$$
\end{proposition}
\begin{proof}
    The above discussion gives the existence of such maps, so it remains to show exactness at $\Omega^{1}_{C/A}$ and surjectivity of the map $\Omega^{1}_{C/A}\to\Omega^{1}_{C/B}$. 
    
    We begin with the latter, where by the quotient construction of \Cref{prop: uniqueness of differentials} it suffices to observe that $\Omega^{1}_{C/B}$ is a quotient of $\Omega^{1}_{C/A}$. 

    For the former, we note that for a fixed $C$-module $M$ we have an exact sequence 
    $$0\to\mathrm{Der}_{B}(C,M)\to\mathrm{Der}_{A}(C,M)\to\mathrm{Der}_{A}(B,M|_{B})$$
    since an $A$-derivation $(\dform:C\to M)$ is taken to the composite $B\to C\to M$ which is zero when the map is also a $B$-derivation. Rewriting this using the universal property, this is 
    $$0\to\Hom_{\Mod_{C}}(\Omega^{1}_{C/B},M)\to\Hom_{\Mod_{C}}(\Omega_{C/A}^{1},M)\to\Hom_{\Mod_{C}}(\Omega_{B/A}^{1}\otimes_{B}C,M)$$
    which by contravariant exatness of the Hom-functor (see \cite[\href{https://stacks.math.columbia.edu/tag/0582}{Tag 0582}]{stacks-project} for the precise statement), is the claim. 
\end{proof}
As a corollary, we can deduce the following fact about localizations. 
\begin{corollary}\label{corr: localization}
    Let $B$ be an $A$-algebra and $S$ a multiplicative subset of $B$. Then $S^{-1}\Omega_{B/A}^{1}\cong\Omega^{1}_{S^{-1}B/A}$. 
\end{corollary}
\begin{proof}
    Apply \Cref{prop: tensor exact sequence} to $C=S^{-1}B$ and note that $\Omega^{1}_{C/B}=0$ so the map $S^{-1}\Omega^{1}_{B/A}\to\Omega^{1}_{S^{-1}B/A}$ is surjective. To prove injectivity, we produce an inverse map which is an $A$-derivation of $S^{-1}B$ to $S^{-1}\Omega_{B/A}^{1}$ by $\dform(\frac{b}{s})\mapsto \frac{1}{s}\dform(b)-\frac{1}{s^{2}}b\dform(s)$ which by the universal property can be seen to be the inverse. 
\end{proof}
Note that in general $\Omega^{1}_{B/A}\otimes_{B}C\to\Omega^{1}_{C/A}$ is rarely injective. 
\begin{example}
    Let $A=k, B=k[x], C=k[x]/(x)$. So $\Omega_{B/A}^{1}\cong B\dform x$ but $\Omega_{C/A}=\Omega_{k/k}=0$. 
\end{example}
On the other hand, there are situations in which the exact sequence of \Cref{prop: tensor exact sequence} extends to a short exact sequence. 
\begin{example}
    Let $B$ be an $A$-algebra and $C=B[x_{1},\dots,x_{n}]$. We then have a split short exact sequence 
    $$0\to\Omega^{1}_{B/A}\otimes_{B}C\to\Omega_{C/A}^{1}\to\Omega_{C/B}^{1}\to 0$$
    where denoting the map $\Omega_{C/A}^{1}\to\Omega_{C/B}^{1}$ by $\varphi$, we have the splitting $\Omega_{C/A}^{1}\to(\Omega_{B/A}^{1}\otimes_{B}C)\oplus\Omega_{C/B}^{1}$ prescribed by the $C$-derivation $f\mapsto \dform_{B/A}(f)+\varphi(f)$ under the bijection 
    $$\Hom_{\Mod_{C}}(\Omega_{C/A},(\Omega_{B/A}\otimes_{B}C)\oplus\Omega_{C/B})\leftrightarrow\mathrm{Der}_{A}(C,(\Omega_{B/A}\otimes_{B}C)\oplus\Omega_{C/B}).$$
\end{example}
The following proposition describes the behavior of the module of K\"{a}hler differentials with respect to tensor products. 
\begin{proposition}\label{prop: differentials of pushouts}
    Let $B,A'$ be $A$-algebras. Then there is an isomorphism of $B$-modules $\Omega_{B/A}^{1}\otimes_{B}(B\otimes_{A}A')\cong\Omega^{1}_{(B\otimes_{A}A')/A'}$. 
\end{proposition}
\begin{proof}
    We contemplate the diagram 
    $$% https://q.uiver.app/#q=WzAsMyxbMCwwLCJCXFxvdGltZXNfe0F9QSciXSxbMiwwLCJcXE9tZWdhXnsxfV97Qi9BfVxcb3RpbWVzX3tBfUEnIl0sWzAsMSwiXFxPbWVnYV97KEJcXG90aW1lc197QX1BJykvQSd9XnsxfSJdLFswLDJdLFsyLDEsIlxcZXhpc3RzISIsMix7InN0eWxlIjp7ImJvZHkiOnsibmFtZSI6ImRhc2hlZCJ9fX1dLFswLDFdXQ==
    \begin{tikzcd}
        {B\otimes_{A}A'} && {\Omega^{1}_{B/A}\otimes_{A}A'} \\
        {\Omega_{(B\otimes_{A}A')/A'}^{1}}
        \arrow[from=1-1, to=1-3]
        \arrow[from=1-1, to=2-1]
        \arrow["{\exists!}"', dashed, from=2-1, to=1-3]
    \end{tikzcd}$$
    where the solid arrows are $B\otimes_{A}A'$-linear with $\Omega^{1}_{B/A}\otimes_{A}A'\cong(\Omega^{1}_{B/A}\otimes_{B}(B\otimes_{A}A'))$ and the dotted arrow induced by the universal property of $\Omega^{1}_{(B\otimes_{A}A')/A'}$. By applying the tensor-hom adjunction and the universal property of derivations, prescribing an inverse map to the dotted arrow is equivalent to producing an $A$-derivation of $B$ in $\Omega^{1}_{(B\otimes_{A}A')/A'}$ and one observes that the map $b\mapsto \dform_{(B\otimes_{A}A')/A'}(b\otimes 1)$ gives an inverse, whence the claim. 
\end{proof}
We now treat the case of quotients. 
\begin{proposition}\label{prop: ideal exact sequence}
    Let $A\to B\to C$ be maps of rings where $C\cong B/\bfrak$ for some ideal $\bfrak\subseteq B$. There is an exact sequence 
    $$\bfrak/\bfrak^{2}\to\Omega_{B/A}^{1}\otimes_{B}C\to\Omega_{C/A}^{1}\to0.$$
\end{proposition}
\begin{proof}
    We first observe that $\Omega^{1}_{C/B}=0$ by \Cref{prop: differentials of open and closed immersions} and $\Omega^{1}_{B/A}\otimes_{B}C\cong\Omega_{B/A}^{1}/\bfrak\Omega^{1}_{B/A}$. 
    
    We denote $\bfrak/\bfrak^{2}\to\Omega^{1}_{B/A}\otimes_{B}C$ by $\delta$, $b\mapsto \dform b\otimes 1$. We first show $\delta$ is well-defined. For this, we want to show that $\dform(b_{1}b_{2})\otimes 1$ is zero for $b_{1},b_{2}\in\bfrak$. Indeed, using the Leibniz rule, we have 
    $$\dform(b_{1}b_{2})\otimes 1 = \dform(f_{2})\otimes f_{1}+\dform(f_{1})\otimes f_{2}\in\bfrak\Omega_{B/A}$$
    hence zero in the quotient, showing the map is well-defined. 
    
    The diagram is a complex as $db$ maps to zero in $\Omega^{1}_{C/A}$. The kernel of $\Omega^{1}_{B/A}\to\Omega^{1}_{C/A}$ is generated by the $B$-submodule $\bfrak\Omega_{B/A}^{1}$ and the elements $\dform b$ for $b\in\bfrak$, showing exactness of the complex in the middle. 
\end{proof}
This specializes to finite type algebras. 
\begin{corollary}\label{corr: kahler differentials are fg module}
    Let $C$ be a finite type $A$-algebra -- that is, the quotient of $B=A[x_{1},\dots,x_{n}]$. Then $\Omega^{1}_{C/A}$ is a finitely generated $C$-module. 
\end{corollary}
\begin{proof}
    Set $B=A[x_{1},\dots,x_{n}]$ for which $C=B/\bfrak$. Exactness of the sequence in \Cref{prop: ideal exact sequence} gives a surjection $\Omega_{B/A}^{1}\otimes_{B}C\to\Omega_{C/A}^{1}$, and observing that $\Omega^{1}_{B/A}\otimes_{B}C\cong B^{\oplus n}\otimes_{B}C\cong C^{\oplus n}$ gives a surjection $C^{\oplus n}\to\Omega_{C/A}^{1}$, showing that it is finitely generated. 
\end{proof}
Let us consider the case of quotients of multivariate polynomial rings by a single polynomial. 
\begin{example}
    Let $A$ be a ring, $B=A[x_{1},\dots,x_{n}]$, and $C=B/(f)$ for $f\in B$. By \Cref{prop: ideal exact sequence} and \Cref{corr: kahler differentials are fg module}, we have that $\Omega_{C/A}^{1}$ is the cokernel of the map $\delta:(f)/(f)^{2}\to\Omega_{B/A}^{1}\otimes_{B}C\cong C^{\oplus n}$ of \Cref{prop: ideal exact sequence}, so is the quotient $\left(\bigoplus_{i=1}^{n}C\dform x_{i}\right)/\dform f$. 
\end{example}
We can also consider the case of $k$-algebras. 
\begin{corollary}\label{corr: residue fields}
    Let $A$ be a $k$-algebra and $\mfrak$ a maximal ideal in $A$ such that $\kappa(\mfrak)=A/\mfrak\cong k$. Then $\Omega_{A/k}^{1}\otimes_{k}\kappa(\mfrak)\cong\mfrak/\mfrak^{2}$. 
\end{corollary}
\begin{proof}
    This is precisely \Cref{prop: ideal exact sequence} for $k\to k[x_{1},\dots,x_{n}]\to A$, and the map $\mfrak/\mfrak^{2}\to\Omega^{1}_{A/k}\otimes_{k}\kappa(\mfrak)$ is a surjection between vector spaces of the same dimension, hence an isomorphism. 
\end{proof}
Note that this is the dual of the Zariski tangent space $\Hom_{\mathsf{Vec}_{\kappa(\mfrak)}}(\mfrak/\mfrak^{2},\kappa(\mfrak))$, motivating the connection to schemes. 