\section{Lecture 5 -- 24th April 2025}\label{sec: lecture 5}
We begin a discussion of flatness, which intuitively corresponds to varying ``nicely'' in a family over the base. As always, we first define these in the local case. 
\begin{definition}[Flat Module]\label{def: flat module}
    Let $A$ be a ring and $M$ an $A$-module. $M$ is a flat $A$-module if $-\otimes_{A}M$ is an exact functor. 
\end{definition}
Recall that for $A$-modules $M$, $-\otimes_{A}M$ is always a right exact functor so flatness is equivalent to being left exact, taking short exact sequences 
$$0\to N_{1}\to N_{2}\to N_{3}\to 0$$
to short exact sequences
$$0\to N_{1}\otimes_{A}M\to N_{2}\otimes_{A}M_{2}\to N_{3}\otimes_{A}N_{3}\to0.$$
In fact, it suffices to verify that for all ideals $\afrak\subseteq A$ that $\afrak\otimes_{A}M\to A\otimes_{A}M$ is injective. 

We can apply this to the special case where an $A$-algebra $B$ is considered as an $A$-module. 
\begin{definition}[Flat Algebra]\label{def: flat algebra}
    Let $A$ be a ring and $B$ an $A$-algebra. $B$ is $A$-flat if $B$ is flat as an $A$-module. 
\end{definition}
Let us recall some further properties of flatness. 
\begin{proposition}\label{prop: properties of flatness rings and modules}
    Let $A$ be a ring. 
    \begin{enumerate}[label=(\roman*)]
        \item If $A$ is a local ring and $M$ is a finite $A$-module, then $M$ is flat if and only if $M$ is free. 
        \item If $S\subseteq A$ is a multiplicative subset, $A\to S^{-1}A$ is flat. 
        \item If $A\to B$ is a ring map and $M$ is a flat $A$-module then $M\otimes_{A}B$ is a flat $B$-module. 
        \item If $A\to B$ is a flat ring map and $N$ is a flat $B$-module then its restriction of scalars $N|_{A}$ is a flat $A$-module. 
        \item Let $M$ be an $A$-module. $M$ is flat if and only if $M_{\pfrak}$ is a flat $A_{\pfrak}$-module for all prime ideals (maximal ideals). 
        \item Let $A\to B$ be a flat ring map between Noetherian local rings, and $b\in B$ such that $\overline{b}\in B/\mfrak_{A}B$ is a non-zerodivisor then $B/(b)$ is $A$-flat. 
        \item For a coCartesian diagram 
        $$% https://q.uiver.app/#q=WzAsNCxbMCwwLCJBIl0sWzIsMCwiQiJdLFswLDEsIkMiXSxbMiwxLCJCXFxvdGltZXNfe0N9QSJdLFswLDFdLFsxLDNdLFswLDJdLFsyLDNdXQ==
        \begin{tikzcd}
            A && B \\
            C && {B\otimes_{C}A}
            \arrow[from=1-1, to=1-3]
            \arrow[from=1-1, to=2-1]
            \arrow[from=1-3, to=2-3]
            \arrow[from=2-1, to=2-3]
        \end{tikzcd}$$
        and $M$ a $B$-module that is $A$-flat, then $M\otimes_{B}(C\otimes_{A}B)$ is a flat $C$-module. 
    \end{enumerate}
\end{proposition}
\begin{proof}
    See \cite[\href{https://stacks.math.columbia.edu/tag/00H9}{Tag 00H9}]{stacks-project}.
\end{proof}
We prove the following lemma about flatness on localizations at prime ideals. 
\begin{lemma}\label{lem: flatness on stalks}
    Let $\varphi:A\to B$ be a ring map. $\varphi$ is flat if and only if for all primes $\qfrak\subseteq B$, the map $A_{\varphi^{-1}(\qfrak)}\to B_{\qfrak}\cong A_{\varphi^{-1}(\qfrak)}\otimes_{A}B$ is a flat ring map. 
\end{lemma}
\begin{proof}
    $(\Rightarrow)$ Suppose $A\to B$ is flat. Then any $A_{\varphi^{-1}(\qfrak)}\to B_{\qfrak}$ is factored as $A_{\varphi^{-1}(\qfrak)}\to B_{\varphi^{-1}(\qfrak)}\cong A_{\varphi^{-1}(\qfrak)}\otimes_{A}B\to(B_{\varphi^{-1}(\qfrak)})_{\qfrak}\cong B_{\qfrak}$ which is a composition of a flat map with a localization, the latter flat, hence the composition is flat too. 

    $(\Leftarrow)$ Suppose we have an injection $M\to M'$ of $A$-modules. By exactness of localization, we have $M\otimes_{A}A_{\pfrak}\to M'\otimes_{A}A_{\pfrak}$. $B_{\qfrak}$ is $A_{\pfrak}$-flat so the map remains injective on base change to $B_{\qfrak}$ for all $\qfrak\subseteq B$ so taking $M=A,M'=B$ yields the claim in conjunction with \Cref{prop: properties of flatness rings and modules} (v). 
\end{proof}
We now describe the geometric case. 
\begin{definition}[Flat Sheaves]\label{def: flat sheaves}
    Let $f:X\to Y$ be a morphism of schemes and $\Fcal$ a quasicoherent sheaf on $X$. $\Fcal$ is flat over $Y$ if $\Fcal_{x}$ is flat as an $\Ocal_{Y,f(x)}$-module. 
\end{definition}
\begin{definition}[Flat Morphism]\label{def: flat morphism}
    Let $f:X\to Y$ be a morphism of schemes. $f$ is a flat morphism if $\Ocal_{X}$ is flat over $Y$. 
\end{definition}
We consider some examples. 
\begin{example}
    Let $X$ be a scheme. $\id_{X}:X\to X$ is a flat morphism as affine-locally it is obtained by the identity ring map and rings are flat as modules over themselves. 
\end{example}
\begin{example}
    Let $\Fcal$ be a coherent sheaf on $X$. $\Fcal$ is a flat $\Ocal_{X}$-module if and only if $\Fcal$ is locally free. In the reduced case, this can be checked by the function $x\mapsto\dim_{\kappa(x)}\Fcal_{x}\otimes_{\Ocal_{X,x}}\kappa(x)$ being locally constant. 
\end{example}
\begin{example}
    \Cref{prop: properties of flatness rings and modules} (ii) shows that open immersions are flat, though closed immersions are often not flat uness they are isomorphisms. 
\end{example}
\begin{proposition}\label{prop: dimension criterion for flatness}
    Let $f:X\to Y$ be a morphism between locally Noetherian schemes. Then $\dim(\Ocal_{X_{y}},x)\geq\dim(\Ocal_{X,x})-\dim(\Ocal_{Y,y})$ and equality occurs when $f$ is flat.
\end{proposition}
\begin{proof}
    We proceed by induction. Since the property is local on target, we can take $Y$ to be affine, and since flatness is preserved by base change, we can take $Y=\spec(\Ocal_{Y,y})$ and $X=\spec(\Ocal_{X,x})$. 

    We proceed by induction on $\dim(Y)$. If $\dim(Y)=0$, the nilradical is contained in the maximal ideal. Using the inclusion $\mfrak_{x}B\hookrightarrow\Nil(B)$ we have 
    $$\dim(\Ocal_{X,x})=\dim(B/\Nil(B))=\dim(B/\mfrak_{x}B)=\dim(\Ocal_{X_{y},x})$$ 
    by $(B/\Nil(B))/(\Nil(B)/\mfrak_{x}B)\cong B/\Nil(B)$ so we in fact have equality $\dim(\Ocal_{X_{y},x})=\dim(\Ocal_{X,x})$ as $\dim(\Ocal_{Y,y})=0$. In particular, the desired inequality holds. 

    If $\dim(Y)\geq 1$, we can, without loss of generality, take $Y$ to be reduced by base changing to the reduction -- an operation that preserves all dimensions involved. Let $t\in\Ocal_{Y,y}$ be neither a zerodivisor nor invertible with image $\overline{t}$ in $\Ocal_{X,x}$. We then have 
    $$\dim(\Ocal_{Y,y}/(t))=\dim(\Ocal_{Y,y})-1, \dim(\Ocal_{X,x}/(\overline{t}))\geq\dim(\Ocal_{X,x})-1$$
    where if $f$ is flat, then $\Ocal_{X,x}$ is flat over $\Ocal_{Y,y}$ and $t$ is a non-zerodivisor in $\Ocal_{X,x}$ giving $\dim(\Ocal_{X,x}/(\overline{t}))=\dim(\Ocal_{X,x})-1$. 

    Now set $Y'=\spec(\Ocal_{Y,y}/(t))$ and $X'=\spec(\Ocal_{X,x}\otimes_{\Ocal_{Y,y}}(\Ocal_{Y,y}/(t)))\cong\spec(\Ocal_{X,x}/(\overline{t}))$ where by the induction hypothesis we have 
    \begin{equation}\label{eqn: flat induction}
        \dim(\Ocal_{X'_{y},x})\geq\dim(\Ocal_{X',x})-\dim(\Ocal_{Y',y})
    \end{equation}
    which is an equality if $f$ is flat since flatness is preserved under base change. But on the fiber we have $X'_{y}=X_{y}$ since $X_{y}$ is the fiber over $y\in\spec(\Ocal_{Y,y}/(t))\subseteq\spec(\Ocal_{Y,y})$ so we have 
    \begin{align*}
        \dim(\Ocal_{X'_{y},x}) &\geq \dim(\Ocal_{X',x})-\dim(\Ocal_{Y,y}) \\
        &= (\dim(\Ocal_{X,x})-1) - (\dim(\Ocal_{Y,y})-1) \\
        &= \dim(\Ocal_{X,x})-\dim(\Ocal_{Y,y}) 
    \end{align*}
    with equality in the flat case by the (\ref{eqn: flat induction}), as desired. 
\end{proof}
In the case of locally finite type schemes, this can be detected fiberwise. 
\begin{corollary}\label{corr: fiberwise flatness}
    Let $f:X\to Y$ be a morphism of $k$-schemes where $X,Y$ are locally of finite type, $Y$ is irreducible, and $X$ equidimensional. If for all $y\in Y$, $X_{y}$ is equidimensional of $\dim(X)-\dim(Y)$ then $f$ is flat and surjective. 
\end{corollary}
\begin{proof}
    Without loss of generality, let $X$ be irreducible and take $x\in X$ closed. Then $\dim(X_{y})=\dim(\Ocal_{X_{y},x})$. By hypothesis we have 
    $$\dim(\Ocal_{X,x})=\dim(X)-\dim(\overline{\{x\}})$$
    which holds for finite-type $k$-algebras. On the other hand, we have 
    \begin{align*}
        \dim(\Ocal_{Y,y}) &= \dim(Y)-\dim(\overline{\{y\}}) \\
        &= \dim(Y)-\mathrm{trdeg}(\kappa(y)/k) \\
        &= \dim(Y)-\dim\{\overline{x}\} \\
        &= \dim(Y)-\dim(X)+\dim(\Ocal_{X,y})
    \end{align*}
    which gives the equality of \Cref{prop: dimension criterion for flatness} and from which the claim follows. 
\end{proof}
Moreover, flatness behaves especially well over smooth curves. 
\begin{proposition}\label{prop: flatness over curves}
    Let $f:X\to Y$ be a morphism of schemes where $Y$ is the spectrum of a discrete valuation. If $f$ is flat, then $\overline{X_{\eta}}=X$. 
\end{proposition}
In particular, there are no closed irreducible components $Z\subseteq Z$ over the (unique) closed point of $Y$. 
\begin{proof}
    $(\Rightarrow)$ Suppose $f$ is flat and there is $Z\subseteq Z$ irreducible in the fiber over the closed point $\mfrak\in Y$. Then $\dim(Z)\leq \dim(X_{\mfrak})=\dim(X)-1$. A contradiction to $X$ equidimensional. Otherwise, we apply \Cref{corr: fiberwise flatness} to $Z$ obtaining a contradiction once more. 

    $(\Leftarrow)$ This is \cite[Prop III.9.7]{Hartshorne}.
\end{proof}