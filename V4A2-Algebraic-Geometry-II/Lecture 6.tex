\section{Lecture 6 -- 28th April 2025}\label{sec: lecture 6}
As a corollary of the previous discussion about flatness, we show that flatness can be extended over a closed fiber. 
\begin{corollary}\label{corr: properness of the hilbert scheme}
    Let $f:X\to Y\setminus\{y\}$ be flat and projective with $Y$ Dedekind and $y\in Y$ closed. 
    $$% https://q.uiver.app/#q=WzAsNCxbMCwwLCJYIl0sWzIsMCwiXFxQUF57bn1fe1lcXHNldG1pbnVzIFxce3lcXH19Il0sWzIsMSwiXFxQUF57bn1fe1l9Il0sWzAsMSwiXFxvdmVybGluZXtYfSJdLFswLDFdLFsxLDJdLFswLDNdLFszLDJdXQ==
    \begin{tikzcd}
        X && {\PP^{n}_{Y\setminus \{y\}}} \\
        {\overline{X}} && {\PP^{n}_{Y}}
        \arrow[from=1-1, to=1-3]
        \arrow[from=1-1, to=2-1]
        \arrow[from=1-3, to=2-3]
        \arrow[from=2-1, to=2-3]
    \end{tikzcd}$$
    Then the induced map $\overline{X}\to Y$ is flat. 
\end{corollary}
\begin{proof}
    Recall that a Dedekind scheme is a Noetherian regular integral scheme of dimension 1. We have the diagram of the statement of the corollary and by \Cref{prop: flatness over curves} we have $X_{\eta}=(\overline{X})_{\eta}\subseteq\overline{X}$ which is dense and $X_{\eta}$ is dense in $X$ so $X_{\eta}\subseteq \overline{X}$ is dense. This shows that $\overline{X}$ is flat over $Y$. 
\end{proof}
We show that images of flat maps are dense. 
\begin{proposition}\label{prop: flat implies dense image}
    Let $f:X\to Y$ be a flat map of schemes with $Y$ irreducible. If $U\subseteq X$ is a nonempty open, then $f(U)\subseteq Y$ is dense. 
\end{proposition}
\begin{proof}
    Without loss of generality, $Y=\spec(A)$ is affine and irreiducible so $A/\Nil(A)$ is integral and $\Frac(A/\Nil(A))=K$. For $U=\spec(B)$ affine, consider $B/\Nil(A)B=B\otimes_{A}(A/\Nil(A))$. Using the injection $A/\Nil(A)\hookrightarrow K$ and $-\otimes_{A}B$ being exact, we have an injective map $(A/\Nil(A))\otimes_{A}B\hookrightarrow K\otimes_{A}B$ injective, where $K\otimes_{A}B=\Ocal_{X}(U_{\eta})$, here denoting $U_{\eta}=\spec(B\otimes_{A}K)$. 

    It suffices to rpove that $U_{\eta}$ is nonempty, or equivalently that $\Ocal_{X}(U_{\eta})$ is nonzero. If $K\otimes_{A}B$ was zero, then $\Nil(A)B=B$ and 1 is nilpotent, a contradiction. 
\end{proof}
We can show the following, weaker, version of a base change statement. 
\begin{proposition}\label{prop: weak flat base change}
    Let $f:X\to \spec(A)$ be a separated quasicompact morphism and $A\to A'$ a flat ring extension. The Cartesian diagram 
    $$% https://q.uiver.app/#q=WzAsNCxbMCwwLCJYXFx0aW1lc197QX1cXHNwZWMoQScpIl0sWzAsMSwiXFxzcGVjKEEnKSJdLFsyLDEsIlxcc3BlYyhBKSJdLFsyLDAsIlgiXSxbMywyLCJmIl0sWzAsMywiZyJdLFswLDFdLFsxLDJdXQ==
    \begin{tikzcd}
        {X\times_{A}\spec(A')} && X \\
        {\spec(A')} && {\spec(A)}
        \arrow["g", from=1-1, to=1-3]
        \arrow[from=1-1, to=2-1]
        \arrow["f", from=1-3, to=2-3]
        \arrow[from=2-1, to=2-3]
    \end{tikzcd}$$
    induces for all quasicoherent sheaves $\Fcal$ on $X$ an isomorphism 
    $$H^{0}(X,\Fcal)\otimes_{A}A'\longrightarrow H^{0}(X\times_{A}\spec(A'),g^{*}\Fcal).$$
\end{proposition}
\begin{proof}
    Choose an affine open covering $\{U_{i}\}$ of $X$. The sheaf condition gives an exact sequence of $A$-modules
    $$0\to H^{0}(X,\Fcal)\to\prod_{i}H^{0}(U_{i},\Fcal|_{U_{i}})\to\prod_{i,j}H^{0}(U_{ij},\Fcal|_{U_{ij}}).$$
    This remains exact after base changing to $A'$, which precisely the exact sequence for $g^{*}\Fcal$ on $X\times_{A}\spec(A')$. 
\end{proof}
Having set up some basic constructions surrounding flatness, we discuss its relation to the Hilbert polynomial. 
\begin{definition}[Hilbert Polynomial]\label{def: hilbert polynomial}
    Let $X$ be projective $k$-scheme with a choice of embedding $i:X\to\PP^{n}_{k}$ and $\Fcal$ a coherent sheaf on $X$. The Hilbert polynomial of $\Fcal$ is $P(X,\Fcal)(m)=\chi(X,\Fcal\otimes\Ocal_{\PP^{n}_{k}}(m)|_{X})$. 
\end{definition}
The Hilbert polynomial is in fact a numerical polynomial, that is, a map $\ZZ\to\ZZ$. Moreover, we can show that this polynomial characterizes flatness on the fibers. The proof of the statement is the globalization of the followimg lemma. 
\begin{lemma}\label{lem: local constant hilbert polynomial}
    Let $A$ be an integral Noetherian local ring and $\Fcal$ a coherent sheaf on $\PP^{n}_{A}$. The following are equivalent:\todo{Finish proof.}
    \begin{enumerate}[label=(\alph*)]
        \item $\Fcal$ is flat over $\spec(A)$. 
        \item $H^{0}(X,\Fcal(m))$ is a free $A$-module for $m$ sufficiently large. 
        \item The Hilbert polynomial $P(\PP^{n}_{\kappa(\pfrak)},\Fcal|_{\kappa(\pfrak)})(m)$ is independent of $\pfrak$.
    \end{enumerate}
\end{lemma}
%\begin{proof}
    %(a)$\Rightarrow$(b) We use the standard \v{C}ech cover of $\PP^{n}_{A}$ to observe that by flatness of $\Fcal$, each term $C^{i}(\Ucal,\Fcal(m))$ is a flat $A$-module. Taking $m$ large enough such that $\Fcal(m)$ has vanishing higher cohomology, we note that the long exact sequence 
    %$$0\to H^{0}(X,\Fcal(m))\to C^{0}(\Ucal,\Fcal(m))\to C^{1}(X,\Fcal(m))\to\dots\to C^{n}(X,\Fcal(m))\to0$$
    %is an acyclic resolution of $H^{0}(X,\Fcal(m))$. This implies that sequences of the form 
    %\begin{align*}
        %0&\to\ker(C^{i}(X,\Fcal(m))\to C^{i+1}(X,\Fcal(m)))\\
        %&\hspace{0.5cm}\to C^{i}(X,\Fcal(m))\to\coker(C^{i+1}(X,\Fcal(m))\to C^{i+2}(X,\Fcal(m)))\to0
    %\end{align*}
    %are short exact. The middle term $C^{i}(X,\Fcal(m))$ is flat over an integral Noetherian local ring, hence free by a result in commutative algebra (vis. \cite[\href{https://stacks.math.columbia.edu/tag/00NZ}{Tag 00NZ}]{stacks-project}). 
%\end{proof}
\begin{proposition}\label{prop: constant hilbert polynomial}
    Let $f:X\to Y$ be a projective morphism with $Y$ integral Noetherian. Let $\Fcal$ be a coherent sheaf on $Y$. $f$ is flat if and only if the Hilbert polynomial of $\Fcal$ on the fibers is constant. 
\end{proposition}
\begin{proof}
    The question is local on target, and flatness is preserved along base change, so it suffices to check the condition along the base change $\spec(\Ocal_{Y,y})\to Y$, putting us in the situation of \Cref{lem: local constant hilbert polynomial}, in which case the desired statement is immediate. 
\end{proof}
\begin{example}
    Blowups are not flat. Let $f$ be the projection from the blowup of $\A^{2}_{k}$ at the origin to $\A^{2}_{k}$. The fiber over the origin is of a higher dimension. 
\end{example}
This applies in particular to the structure sheaf. 
\begin{corollary}\label{corr: flatness by Hilbert polynomial}
    Let $f:X\to Y$ be a projective morphism with $Y$ integral Noetherian. $f$ is flat if and only if $P(X_{y},\Ocal_{X_{y}})(m)$ is constant.
\end{corollary}
\begin{proof}
    This is an immediate application of \Cref{prop: constant hilbert polynomial} to $\Ocal_{X}=\Fcal$. 
\end{proof}
Moreover, we can show that flatness is open and universally open. 
\begin{proposition}\label{prop: flatness is universally open}
    Let $f:X\to Y$ be a finite type morphism of schemes with $Y$ Noetherian. Then $f$ is open and universally open.\todo{Check.}
\end{proposition}
\begin{proof}
    We use the following input from the theory of spectral spaces: 
    \begin{itemize}
        \item If $f:X\to Y$ is a finite type morphism with $Y$ Noetherian then the image of $f$ is constructible in $Y$\cite[\href{https://stacks.math.columbia.edu/tag/054K}{Tag 054K}]{stacks-project}.
        \item Let $V\subseteq Y$ be a subset. $V$ is constructible and stable under specializations if and only if $V$ is open in $Y$ \cite[\href{https://stacks.math.columbia.edu/tag/0542}{Tag 0542}]{stacks-project}. 
    \end{itemize}
    It suffices to show that $f(X)$ is open. We know already $f(X)$ is constructible so it suffices to show that $f(X)$ is stable under generalizations. Take $x\in X$ and $y=f(x)\in Y$. Assume $y\in\overline{\{y'\}}$. We want to show $y'\in f(X)$. Let $B=\Ocal_{X,x}$ with $x\in U\subseteq X$ and $A=\Ocal_{Y,y}$ which contains the prime ideal $\pfrak_{y'}$. The ring map $A\to B$ is a local homomorphism of local rings so $\mfrak_{x}\cap A=\mfrak_{y}$ and $\pfrak_{y'}B\subseteq\mfrak_{y}B\subseteq\mfrak_{x}\subsetneq B$ showing $B\otimes_{A}A_{\pfrak_{y'}}\neq0$. If it were zero, then for all $b\in B$ there would be $t\in A_{\pfrak_{y'}}$ such that $tb=0$ but the multiplication by $t$-map $A\to A$ is injective and remains injective after base-changing to $B$. This shows that $B\otimes_{A_{\pfrak_{y'}}}\kappa(\pfrak_{y'})$ is nonzero, and hence lies in the image. 
\end{proof}