\documentclass{amsart}
\usepackage[margin=1.5in]{geometry} 
\usepackage{amsmath}
\usepackage{tcolorbox}
\usepackage{amssymb}
\usepackage{amsthm}
\usepackage{lastpage}
\usepackage{fancyhdr}
\usepackage{accents}
\usepackage{hyperref}
\usepackage{xcolor}
\usepackage{color}
% Fields
\newcommand{\A}{\mathbb{A}}
\newcommand{\CC}{\mathbb{C}}
\newcommand{\EE}{\mathbb{E}}
\newcommand{\RR}{\mathbb{R}}
\newcommand{\QQ}{\mathbb{Q}}
\newcommand{\ZZ}{\mathbb{Z}}
\newcommand{\HH}{\mathbb{H}}
\newcommand{\KK}{\mathbb{K}}
\newcommand{\NN}{\mathbb{N}}
\newcommand{\FF}{\mathbb{F}}
\newcommand{\PP}{\mathbb{P}}
\newcommand{\GG}{\mathbb{G}}
\newcommand{\LL}{\mathbb{L}}
\newcommand{\WW}{\mathbb{W}}

% mathcal letters
\newcommand{\Acal}{\mathcal{A}}
\newcommand{\Bcal}{\mathcal{B}}
\newcommand{\Ccal}{\mathcal{C}}
\newcommand{\Dcal}{\mathcal{D}}
\newcommand{\Ecal}{\mathcal{E}}
\newcommand{\Fcal}{\mathcal{F}}
\newcommand{\Gcal}{\mathcal{G}}
\newcommand{\Hcal}{\mathcal{H}}
\newcommand{\Ical}{\mathcal{I}}
\newcommand{\Jcal}{\mathcal{J}}
\newcommand{\Kcal}{\mathcal{K}}
\newcommand{\Lcal}{\mathcal{L}}
\newcommand{\Mcal}{\mathcal{M}}
\newcommand{\Ncal}{\mathcal{N}}
\newcommand{\Ocal}{\mathcal{O}}
\newcommand{\Pcal}{\mathcal{P}}
\newcommand{\Qcal}{\mathcal{Q}}
\newcommand{\Rcal}{\mathcal{R}}
\newcommand{\Scal}{\mathcal{S}}
\newcommand{\Tcal}{\mathcal{T}}
\newcommand{\Ucal}{\mathcal{U}}
\newcommand{\Vcal}{\mathcal{V}}
\newcommand{\Wcal}{\mathcal{W}}
\newcommand{\Xcal}{\mathcal{X}}
\newcommand{\Ycal}{\mathcal{Y}}
\newcommand{\Zcal}{\mathcal{Z}}

% abstract categories
\newcommand{\Asf}{\mathsf{A}}
\newcommand{\Bsf}{\mathsf{B}}
\newcommand{\Csf}{\mathsf{C}}
\newcommand{\Dsf}{\mathsf{D}}
\newcommand{\Esf}{\mathsf{E}}
\newcommand{\Ssf}{\mathsf{S}}

% algebraic geometry
\newcommand{\spec}{\operatorname{Spec}}
\newcommand{\proj}{\operatorname{Proj}}

% categories 
\newcommand{\id}{\mathrm{id}}
\newcommand{\Obj}{\mathrm{Obj}}
\newcommand{\Mor}{\mathrm{Mor}}
\newcommand{\Hom}{\mathrm{Hom}}
\newcommand{\Ext}{\mathrm{Ext}}
\newcommand{\Aut}{\mathrm{Aut}}
\newcommand{\Sets}{\mathsf{Sets}}
\newcommand{\SSets}{\mathsf{SSets}}
\newcommand{\kVect}{\mathsf{Vect}_{k}}
\newcommand{\Vect}{\mathsf{Vect}}
\newcommand{\Alg}{\mathsf{Alg}}
\newcommand{\Ring}{\mathsf{Ring}}
\newcommand{\Mod}{\mathsf{Mod}}
\newcommand{\Grp}{\mathsf{Grp}}
\newcommand{\AbGrp}{\mathsf{AbGrp}}
\newcommand{\PSh}{\mathsf{PSh}}
\newcommand{\Sh}{\mathsf{Sh}}
\newcommand{\PSch}{\mathsf{PSch}}
\newcommand{\Sch}{\mathsf{Sch}}
\newcommand{\Top}{\mathsf{Top}}
\newcommand{\Com}{\mathsf{Com}}
\newcommand{\Coh}{\mathsf{Coh}}
\newcommand{\QCoh}{\mathsf{QCoh}}
\newcommand{\Opens}{\mathsf{Opens}}
\newcommand{\Opp}{\mathsf{Opp}}
\newcommand{\Cat}{\mathsf{Cat}}
\newcommand{\NatTrans}{\mathrm{NatTrans}}
\newcommand{\pr}{\mathrm{pr}}
\newcommand{\Fun}{\mathrm{Fun}}
\newcommand{\colim}{\mathrm{colim}}
\newcommand{\lifts}{\boxslash}
\DeclareMathOperator\squarediv{\lifts}
\newcommand{\Kan}{\mathsf{Kan}}
\newcommand{\Path}{\mathsf{Path}}
\newcommand{\SPSh}{\mathsf{SPSh}}
\newcommand{\SSh}{\mathsf{SSh}}
\newcommand{\Bord}{\mathsf{Bord}}

% simplicial sets
\newcommand{\DDelta}{\Updelta}
\newcommand{\Sing}{\operatorname{Sing}}

% ideal theory
\newcommand{\mfrak}{\mathfrak{m}}
\newcommand{\afrak}{\mathfrak{a}}
\newcommand{\bfrak}{\mathfrak{b}}
\newcommand{\pfrak}{\mathfrak{p}}
\newcommand{\qfrak}{\mathfrak{q}}

% number theory
\newcommand{\Tr}{\mathrm{Tr}}
\newcommand{\Nm}{\mathrm{Nm}}
\newcommand{\Gal}{\mathrm{Gal}}
\newcommand{\Frob}{\mathrm{Frob}}

\newcommand{\SL}{\mathrm{SL}}
\newcommand{\GL}{\mathrm{GL}}
\newcommand{\Li}{\mathrm{Li}}
\newcommand{\sfPic}{\mathsf{Pic}}
\newcommand{\img}{\mathrm{Im}}
\newcommand{\Reg}{\mathrm{Reg}}
\newcommand{\Dscr}{\EuScript{D}}
\newcommand{\Ani}{\mathsf{Ani}}
\newcommand{\Proj}{\mathsf{Proj}}
\newcommand{\Free}{\mathsf{Free}}
\newcommand{\CMon}{\mathsf{CMon}}
\newcommand{\cond}{\mathsf{cond}}
\newcommand{\cont}{\mathsf{cont}}
\newcommand{\Liq}{\mathsf{Liq}}
\newcommand{\Gas}{\mathsf{Gas}}
\newcommand{\ku}{\mathsf{ku}}
\newcommand{\KU}{\mathsf{KU}}
\newcommand{\SSS}{\mathbb{S}}
\newcommand{\univ}{\mathrm{univ}}
\newcommand{\DMod}{\mathsf{DMod}}
\newcommand{\FGauge}{\mathsf{FGauge}}
\newcommand{\prism}{\mathbbl{\Delta}}
\newcommand{\Hab}{\mathsf{Hab}}
\newcommand{\Hdg}{\mathsf{Hdg}}
\newcommand{\HHdg}{\mathcal{H}\mathsf{dg}}
\newcommand{\dash}{\text{-}}
\newcommand{\gh}{\mathrm{gh}}
\newcommand{\CAlg}{\mathsf{CAlg}}
\newcommand{\Fil}{\mathrm{Fil}}
\setlength{\headheight}{40pt}


\newenvironment{solution}
  {\renewcommand\qedsymbol{$\blacksquare$}
  \begin{proof}[Solution]}
  {\end{proof}}
\renewcommand\qedsymbol{$\blacksquare$}

\usepackage{amsmath, amssymb, tikz, amsthm, csquotes, multicol, footnote, tablefootnote, biblatex, wrapfig, float, quiver, mathrsfs, cleveref, enumitem, upgreek, stmaryrd, marginnote, todonotes}
\addbibresource{refs.bib}
\theoremstyle{definition}
\newtheorem{theorem}{Theorem}[section]
\newtheorem{lemma}[theorem]{Lemma}
\newtheorem{corollary}[theorem]{Corollary}
\newtheorem{exercise}[theorem]{Exercise}
\newtheorem{question}[theorem]{Question}
\newtheorem{example}[theorem]{Example}
\newtheorem{proposition}[theorem]{Proposition}
\newtheorem{conjecture}[theorem]{Conjecture}
\newtheorem{remark}[theorem]{Remark}
\newtheorem{definition}[theorem]{Definition}
\numberwithin{equation}{section}
\setuptodonotes{color=blue!20, size=tiny}
\begin{document}
\large
\title[Algebraic Geometry II -- Bonn, Summer 2025]{V4A2 -- Algebraic Geometry II \\ Summer Semester 2025}
\author{Wern Juin Gabriel Ong}
\address{Universit\"{a}t Bonn, Bonn, D-53113}
\email{wgabrielong@uni-bonn.de}
\urladdr{https://wgabrielong.github.io/}
\maketitle
\section*{Preliminaries}
These notes roughly correspond to the course \textbf{V4A2 -- Algebraic Geometry II} taught by Prof. Daniel Huybrechts at the Universit\"{a}t Bonn in the Summer 2025 semester. These notes are \LaTeX-ed after the fact with significant alteration and are subject to misinterpretation and mistranscription. Use with caution. Any errors are undoubtedly my own and any virtues that could be ascribed to these notes ought be attributed to the instructor and not the typist. Knowledge of commutative algebra, topology, and category theory will be assumed. 
\newpage
\tableofcontents
\section{Lecture 1 -- 7th April 2025}\label{sec: lectuer 1}
We begin by a consideration of the theory of smoothness, first in the local case. This is done by defining the sheaves of K\"{a}hler differentials on schemes -- in the local picture, the module of differentials on a ring. 
\begin{definition}[Derivation]\label{def: derivation}
    Let $B$ be an $A$-algebra and $M$ a $B$-module. An morphism of $A$-modules $D:B\to M$ is an $A$-derivation if it satisfies the Leibniz rule $\dform(xy)=x\dform(y)+y\dform(x)$ for all $x,y\in B$. Denote the set of $A$-derivations in $M$ by $\mathrm{Der}_{A}(B,M)$. 
\end{definition}
\begin{remark}
    It is necessary that $M$ is a $B$-module, since the Leibniz rule involves elements of $B$. 
\end{remark}
\begin{remark}\label{rmk: map from A is zero}
    Observe that the composition $A\to B\to M$is zero since $a=a\cdot 1_{B}$ and computing we get $\dform(a\cdot 1_{B})=a\dform(1_{B})$ by $A$-linearity, but on the other hand $\dform(a\cdot 1_{B})=a\dform(1_{B})+1_{B}\dform(a)$ by the Leibniz rule, so $a\dform(1_{B})=0$ showing $\dform(1_{B})=0$ and thus $\dform(a)=0$. 
\end{remark}
The K\"{a}hler differentials of a ring map is the universal recipient of an $A$-algebra $B$ in the following sense. 
\begin{definition}[Module of K\"{a}hler Differentials]
    Let $B$ be an $A$-algebra. The module of K\"{a}hler differentials of $B$ over $A$ is a $B$-module $\Omega^{1}_{B/A}$ with an $A$-derivation $\dform:B\to\Omega_{B/A}^{1}$ that is initial amongst $B$-modules recieving an $A$-derivation from $B$. 
\end{definition}
Unwinding the universal property, if $M$ is a $B$-module recieving an $A$-derivation from $B$ by $f:B\to M$, there is a unique factorization over $\Omega^{1}_{B/A}$ as follows.
$$% https://q.uiver.app/#q=WzAsMyxbMCwxLCJCIl0sWzIsMCwiTSJdLFswLDAsIlxcT21lZ2FeezF9X3tCL0F9Il0sWzIsMV0sWzAsMiwiXFxleGlzdHMhIiwwLHsic3R5bGUiOnsiYm9keSI6eyJuYW1lIjoiZGFzaGVkIn19fV0sWzAsMV1d
\begin{tikzcd}
	{\Omega^{1}_{B/A}} && M \\
	B
	\arrow[from=1-1, to=1-3]
	\arrow["{\exists!}", dashed, from=2-1, to=1-1]
	\arrow[from=2-1, to=1-3]
\end{tikzcd}$$
In particular, there is a bijection $\mathrm{Der}_{A}(B,M)\leftrightarrow\Hom_{\Mod_{B}}(\Omega_{B/A},M)$ functorial in $M$. 
\begin{proposition}\label{prop: uniqueness of differentials}
    Let $B$ be an $A$-algebra. The $B$-module $\Omega^{1}_{B/A}$ and the $A$-derivation $\dform:B\to\Omega^{1}_{B/A}$ exist and are unique up to unique isomorphism. 
\end{proposition}
\begin{proof}
    The module $\Omega^{1}_{B/A}$ can be constructed as the free $B$-module on elements $\dform x$ for $x\in B$ modulo the relations generated by the Leibniz rule and $\dform a=0$ for $a\in A$. Uniqueness up to unique isomorphism is clear from the universal property and Yoneda's lemma. 
\end{proof}
In special cases, the module of K\"{a}hler differentials can be described explicitly. 
\begin{example}
    Let $A=k,B=k[x_{1},\dots,x_{n}]$. $\Omega_{B/A}^{1}$ is a free module of rank $n$ with basis $\dform x_{i}$. The map $f\mapsto\sum_{i=1}^{n}\frac{\partial f}{\partial x_{i}}\cdot \dform x_{i}$ is an $A$-derivation and the map $\dform x_{i}\mapsto x_{i}$ defines an isomorphism $\Omega_{B/A}^{1}\to B^{\oplus n}$. 
\end{example}
K\"{a}hler differentials are also fairly easy to understand in the case of ring localizations and ring quotients. These will be important in understanding the sheaves of K\"{a}hler differentials of open and closed immersions in the case of schemes, respectively. 
\begin{proposition}\label{prop: differentials of open and closed immersions}
    Let $A$ be a ring. 
    \begin{enumerate}[label=(\roman*)]
        \item If $B=S^{-1}A$, then $\Omega^{1}_{B/A}=0$. 
        \item If $B=A/I$ for $I\subseteq A$ an ideal, then $\Omega^{1}_{B/A}=0$. 
    \end{enumerate}
\end{proposition}
\begin{proof}[Proof of (i)]
    We already have that $\dform a=0$ for all $a\in A$. We then observe that writing $a=s\cdot\frac{a}{s}$ we have 
    \begin{align*}
        0=\dform(a)=\dform\left(s\cdot\frac{a}{s}\right) &= s\dform\left(\frac{a}{s}\right)+\frac{a}{s}\dform(s) \\
        &= s\dform\left(\frac{a}{s}\right) && s\in A\Rightarrow \dform s=0 
    \end{align*}
    so $s\dform(\frac{a}{s})=0$ and $\dform(\frac{a}{s})=0$ whence the claim. 
\end{proof}
\begin{proof}[Proof of (ii)]
    The map $A\to B$ is surjective, so this is precisely the situation \Cref{rmk: map from A is zero}. 
\end{proof}
We can additionally understand sheaves of K\"{a}hler differentials in towers. Let $A\to B\to C$ be maps of rings. There is a natural $C$-linear map $\Omega^{1}_{B/A}\otimes_{B}C\to\Omega_{C/A}^{1}$ which is a $C$-module homomorphism induced by the diagram 
$$% https://q.uiver.app/#q=WzAsNCxbMCwwLCJCIl0sWzIsMCwiQyJdLFs0LDAsIlxcT21lZ2Ffe0MvQX1eezF9Il0sWzAsMSwiXFxPbWVnYV97Qi9BfV57MX0iXSxbMSwyLCJcXGRmb3JtX3tDL0F9Il0sWzAsMV0sWzAsMywiXFxkZm9ybV97Qi9BfSIsMl0sWzMsMiwiXFxleGlzdHMhIiwyLHsic3R5bGUiOnsiYm9keSI6eyJuYW1lIjoiZGFzaGVkIn19fV1d
\begin{tikzcd}
	B && C && {\Omega_{C/A}^{1}} \\
	{\Omega_{B/A}^{1}}
	\arrow[from=1-1, to=1-3]
	\arrow["{\dform_{B/A}}"', from=1-1, to=2-1]
	\arrow["{\dform_{C/A}}", from=1-3, to=1-5]
	\arrow["{\exists!}"', dashed, from=2-1, to=1-5]
\end{tikzcd}$$
where the top row is both $A$ and $B$-linear inducing a unique $B$-module map $\Omega_{B/A}^{1}\to\Omega_{C/A}^{1}$, considering the latter as a $B$-module. By the extension-restriction adjunction, however, we have 
$$\Hom_{\Mod_{B}}(\Omega_{B/A},\Omega_{C/A}|_{B})\leftrightarrow\Hom_{\Mod_{C}}(\Omega_{B/A}\otimes_{B}C,\Omega_{C/A})$$
hence the data of the dotted map in the diagram above gives rise to a unique map $\Omega_{B/A}\otimes_{B}C\to\Omega_{C/A}$. Arguing similarly, there is a $C$-linear map $\Omega^{1}_{C/A}\to\Omega_{C/B}^{1}$ induced by 
$$% https://q.uiver.app/#q=WzAsMyxbMCwwLCJDIl0sWzIsMCwiXFxPbWVnYV97Qy9CfV57MX0iXSxbMCwxLCJcXE9tZWdhX3tDL0F9XnsxfSJdLFswLDEsIlxcZGZvcm1fe0MvQn0iXSxbMCwyLCJcXGRmb3JtX3tDL0F9IiwyXSxbMiwxLCJcXGV4aXN0cyEiLDIseyJzdHlsZSI6eyJib2R5Ijp7Im5hbWUiOiJkYXNoZWQifX19XV0=
\begin{tikzcd}
	C && {\Omega_{C/B}^{1}} \\
	{\Omega_{C/A}^{1}}
	\arrow["{\dform_{C/B}}", from=1-1, to=1-3]
	\arrow["{\dform_{C/A}}"', from=1-1, to=2-1]
	\arrow["{\exists!}"', dashed, from=2-1, to=1-3]
\end{tikzcd}$$
where the map is induced by the universal property as any $B$-derivation is also an $A$-derivation. 

The maps in the preceding discussion assemble to give the following proposition. 
\begin{proposition}\label{prop: tensor exact sequence}
    Let $A\to B\to C$ be maps of rings. There is an exact sequence 
    $$\Omega_{B/A}^{1}\otimes_{B}C\to \Omega_{C/A}^{1}\to\Omega_{C/B}^{1}\to0.$$
\end{proposition}
\begin{proof}
    The above discussion gives the existence of such maps, so it remains to show exactness at $\Omega^{1}_{C/A}$ and surjectivity of the map $\Omega^{1}_{C/A}\to\Omega^{1}_{C/B}$. 
    
    We begin with the latter, where by the quotient construction of \Cref{prop: uniqueness of differentials} it suffices to observe that $\Omega^{1}_{C/B}$ is a quotient of $\Omega^{1}_{C/A}$. 

    For the former, we note that for a fixed $C$-module $M$ we have an exact sequence 
    $$0\to\mathrm{Der}_{B}(C,M)\to\mathrm{Der}_{A}(C,M)\to\mathrm{Der}_{A}(B,M|_{B})$$
    since an $A$-derivation $(\dform:C\to M)$ is taken to the composite $B\to C\to M$ which is zero when the map is also a $B$-derivation. Rewriting this using the universal property, this is 
    $$0\to\Hom_{\Mod_{C}}(\Omega^{1}_{C/B},M)\to\Hom_{\Mod_{C}}(\Omega_{C/A}^{1},M)\to\Hom_{\Mod_{C}}(\Omega_{B/A}^{1}\otimes_{B}C,M)$$
    which by contravariant exatness of the Hom-functor (see \cite[\href{https://stacks.math.columbia.edu/tag/0582}{Tag 0582}]{stacks-project} for the precise statement), is the claim. 
\end{proof}
As a corollary, we can deduce the following fact about localizations. 
\begin{corollary}\label{corr: localization}
    Let $B$ be an $A$-algebra and $S$ a multiplicative subset of $B$. Then $S^{-1}\Omega_{B/A}^{1}\cong\Omega^{1}_{S^{-1}B/A}$. 
\end{corollary}
\begin{proof}
    Apply \Cref{prop: tensor exact sequence} to $C=S^{-1}B$ and note that $\Omega^{1}_{C/B}=0$ so the map $S^{-1}\Omega^{1}_{B/A}\to\Omega^{1}_{S^{-1}B/A}$ is surjective. To prove injectivity, we produce an inverse map which is an $A$-derivation of $S^{-1}B$ to $S^{-1}\Omega_{B/A}^{1}$ by $\dform(\frac{b}{s})\mapsto \frac{1}{s}\dform(b)-\frac{1}{s^{2}}b\dform(s)$ which by the universal property can be seen to be the inverse. 
\end{proof}
Note that in general $\Omega^{1}_{B/A}\otimes_{B}C\to\Omega^{1}_{C/A}$ is rarely injective. 
\begin{example}
    Let $A=k, B=k[x], C=k[x]/(x)$. So $\Omega_{B/A}^{1}\cong B\dform x$ but $\Omega_{C/A}=\Omega_{k/k}=0$. 
\end{example}
On the other hand, there are situations in which the exact sequence of \Cref{prop: tensor exact sequence} extends to a short exact sequence. 
\begin{example}
    Let $B$ be an $A$-algebra and $C=B[x_{1},\dots,x_{n}]$. We then have a split short exact sequence 
    $$0\to\Omega^{1}_{B/A}\otimes_{B}C\to\Omega_{C/A}^{1}\to\Omega_{C/B}^{1}\to 0$$
    where denoting the map $\Omega_{C/A}^{1}\to\Omega_{C/B}^{1}$ by $\varphi$, we have the splitting $\Omega_{C/A}^{1}\to(\Omega_{B/A}^{1}\otimes_{B}C)\oplus\Omega_{C/B}^{1}$ prescribed by the $C$-derivation $f\mapsto \dform_{B/A}(f)+\varphi(f)$ under the bijection 
    $$\Hom_{\Mod_{C}}(\Omega_{C/A},(\Omega_{B/A}\otimes_{B}C)\oplus\Omega_{C/B})\leftrightarrow\mathrm{Der}_{A}(C,(\Omega_{B/A}\otimes_{B}C)\oplus\Omega_{C/B}).$$
\end{example}
The following proposition describes the behavior of the module of K\"{a}hler differentials with respect to tensor products. 
\begin{proposition}\label{prop: differentials of pushouts}
    Let $B,A'$ be $A$-algebras. Then there is an isomorphism of $B$-modules $\Omega_{B/A}^{1}\otimes_{B}(B\otimes_{A}A')\cong\Omega^{1}_{(B\otimes_{A}A')/A'}$. 
\end{proposition}
\begin{proof}
    We contemplate the diagram 
    $$% https://q.uiver.app/#q=WzAsMyxbMCwwLCJCXFxvdGltZXNfe0F9QSciXSxbMiwwLCJcXE9tZWdhXnsxfV97Qi9BfVxcb3RpbWVzX3tBfUEnIl0sWzAsMSwiXFxPbWVnYV97KEJcXG90aW1lc197QX1BJykvQSd9XnsxfSJdLFswLDJdLFsyLDEsIlxcZXhpc3RzISIsMix7InN0eWxlIjp7ImJvZHkiOnsibmFtZSI6ImRhc2hlZCJ9fX1dLFswLDFdXQ==
    \begin{tikzcd}
        {B\otimes_{A}A'} && {\Omega^{1}_{B/A}\otimes_{A}A'} \\
        {\Omega_{(B\otimes_{A}A')/A'}^{1}}
        \arrow[from=1-1, to=1-3]
        \arrow[from=1-1, to=2-1]
        \arrow["{\exists!}"', dashed, from=2-1, to=1-3]
    \end{tikzcd}$$
    where the solid arrows are $B\otimes_{A}A'$-linear with $\Omega^{1}_{B/A}\otimes_{A}A'\cong(\Omega^{1}_{B/A}\otimes_{B}(B\otimes_{A}A'))$ and the dotted arrow induced by the universal property of $\Omega^{1}_{(B\otimes_{A}A')/A'}$. By applying the tensor-hom adjunction and the universal property of derivations, prescribing an inverse map to the dotted arrow is equivalent to producing an $A$-derivation of $B$ in $\Omega^{1}_{(B\otimes_{A}A')/A'}$ and one observes that the map $b\mapsto \dform_{(B\otimes_{A}A')/A'}(b\otimes 1)$ gives an inverse, whence the claim. 
\end{proof}
We now treat the case of quotients. 
\begin{proposition}\label{prop: ideal exact sequence}
    Let $A\to B\to C$ be maps of rings where $C\cong B/\bfrak$ for some ideal $\bfrak\subseteq B$. There is an exact sequence 
    $$\bfrak/\bfrak^{2}\to\Omega_{B/A}^{1}\otimes_{B}C\to\Omega_{C/A}^{1}\to0.$$
\end{proposition}
\begin{proof}
    We first observe that $\Omega^{1}_{C/B}=0$ by \Cref{prop: differentials of open and closed immersions} and $\Omega^{1}_{B/A}\otimes_{B}C\cong\Omega_{B/A}^{1}/\bfrak\Omega^{1}_{B/A}$. 
    
    We denote $\bfrak/\bfrak^{2}\to\Omega^{1}_{B/A}\otimes_{B}C$ by $\delta$, $b\mapsto \dform b\otimes 1$. We first show $\delta$ is well-defined. For this, we want to show that $\dform(b_{1}b_{2})\otimes 1$ is zero for $b_{1},b_{2}\in\bfrak$. Indeed, using the Leibniz rule, we have 
    $$\dform(b_{1}b_{2})\otimes 1 = \dform(f_{2})\otimes f_{1}+\dform(f_{1})\otimes f_{2}\in\bfrak\Omega_{B/A}$$
    hence zero in the quotient, showing the map is well-defined. 
    
    The diagram is a complex as $db$ maps to zero in $\Omega^{1}_{C/A}$. The kernel of $\Omega^{1}_{B/A}\to\Omega^{1}_{C/A}$ is generated by the $B$-submodule $\bfrak\Omega_{B/A}^{1}$ and the elements $\dform b$ for $b\in\bfrak$, showing exactness of the complex in the middle. 
\end{proof}
This specializes to finite type algebras. 
\begin{corollary}\label{corr: kahler differentials are fg module}
    Let $C$ be a finite type $A$-algebra -- that is, the quotient of $B=A[x_{1},\dots,x_{n}]$. Then $\Omega^{1}_{C/A}$ is a finitely generated $C$-module. 
\end{corollary}
\begin{proof}
    Set $B=A[x_{1},\dots,x_{n}]$ for which $C=B/\bfrak$. Exactness of the sequence in \Cref{prop: ideal exact sequence} gives a surjection $\Omega_{B/A}^{1}\otimes_{B}C\to\Omega_{C/A}^{1}$, and observing that $\Omega^{1}_{B/A}\otimes_{B}C\cong B^{\oplus n}\otimes_{B}C\cong C^{\oplus n}$ gives a surjection $C^{\oplus n}\to\Omega_{C/A}^{1}$, showing that it is finitely generated. 
\end{proof}
Let us consider the case of quotients of multivariate polynomial rings by a single polynomial. 
\begin{example}
    Let $A$ be a ring, $B=A[x_{1},\dots,x_{n}]$, and $C=B/(f)$ for $f\in B$. By \Cref{prop: ideal exact sequence} and \Cref{corr: kahler differentials are fg module}, we have that $\Omega_{C/A}^{1}$ is the cokernel of the map $\delta:(f)/(f)^{2}\to\Omega_{B/A}^{1}\otimes_{B}C\cong C^{\oplus n}$ of \Cref{prop: ideal exact sequence}, so is the quotient $\left(\bigoplus_{i=1}^{n}C\dform x_{i}/\dform f\right)$. 
\end{example}
We can also consider the case of $k$-algebras. 
\begin{corollary}\label{corr: residue fields}
    Let $A$ be a $k$-algebra and $\mfrak$ a maximal ideal in $A$ such that $\kappa(\mfrak)=A/\mfrak\cong k$. Then $\Omega_{A/k}^{1}\otimes_{k}\kappa(\mfrak)\cong\mfrak/\mfrak^{2}$. 
\end{corollary}
\begin{proof}
    This is precisely \Cref{prop: ideal exact sequence} for $k\to k[x_{1},\dots,x_{n}]\to A$, and the map $\mfrak/\mfrak^{2}\to\Omega^{1}_{A/k}\otimes_{k}\kappa(\mfrak)$ is a surjection between vector spaces of the same dimension, hence an isomorphism. 
\end{proof}
Note that this is the dual of the Zariski tangent space $\Hom_{\mathsf{Vec}_{\kappa(\mfrak)}}(\mfrak/\mfrak^{2},\kappa(\mfrak))$, motivating the connection to schemes. 
\section{Lecture 2 -- 2nd May 2025}\label{sec: lecture 2}
The goal of this course is to develop a theory of Habiro cohomology, a functor that associates to a smooth $\ZZ$-scheme $X$ its Habiro cohomology -- a module over the Habiro ring, or more generally its ``category of constructible sheaves'' which in this case we tentatively denote $\Dscr_{\Hab}(X)$ of ``variations of Habiro structure.''

We begin with an exploration of what these structures are in terms of coordinates, and we will later show that the constructions we discuss are in fact independent of these coordinates. Let us make the notion of coordinates precise. 
\begin{definition}[Framed Algebra]\label{def: framed algebra}
    A framed algebra is a pair $(R,\square)$ where $R$ is a smooth $\ZZ$-algebra and a map $\square:\spec(R)\to\A^{d}_{\ZZ}$ or $\square:\spec(R)\to\GG_{m}^{d}$. 
\end{definition}
\begin{remark}
    It is often simpler to consider the case where the coordinates are invertible, that is, the case of $\GG_{m}^{d}$. 
\end{remark}
As a first pass, let us contemplate these constructions in the case where $X$ is affine and equal to either $\A^{d}_{\ZZ}$ or $\GG_{m}^{d}$ and only later consider the generalization to the case where $X$ is \'{e}tale over one of these spaces. Moreover, under these assumptions, we need not make any completions and one can work over $\ZZ[q^{\pm}]$.

Recall Habiro cohomology subsumes de Rham cohomology in an appropriate sense, and takes the $q$-derivative -- the Gaussian $q$-analogue of the derivative -- as an input. These $q$-derivatives were first investigated by Jackson \cite{Jackson}.
\begin{definition}[$q$-Derivative]\label{def: q-derivative}
    Let $R$ be $\ZZ[q^{\pm}][T_{1},\dots,T_{d}]$ or $\ZZ[q^{\pm}][T_{1}^{\pm},\dots,T_{d}^{\pm}]$. The $q$-derivative $\nabla_{i}^{q}:R\to R$ for $1\leq i\leq d$ is defined by 
    $$\nabla_{i}^{q}(f(T_{1},\dots,T_{d}))=\frac{f(T_{1},\dots,qT_{i},\dots,T_{d})-f(T_{1},\dots,T_{i},\dots,T_{d})}{qT_{i}-T_{i}}.$$
\end{definition}
\begin{remark}
    More explicitly, this operation is given on monomials by 
    $$\nabla_{i}^{q}(T_{1}^{n_{1}}\dots T_{d}^{n_{d}})=[n_{i}]_{q}\cdot T_{1}^{n_{1}}\dots T_{i}^{n_{i}-1}\dots T_{d}^{n_{d}}$$
    where $[n]_{q}=\frac{1-q^{n}}{1-q}$ is the Gaussian $q$-analogue of $n$. 
\end{remark}
\begin{remark}\label{rmk: gamma i maps}
    $\nabla_{i}^{q}$ is closely related $\gamma_{i}:R\to R$ the automorphism by 
    $$T_{j}\mapsto\begin{cases}
        T_{j} & j\neq i \\
        qT_{i} & j=i
    \end{cases}$$
    allowing us to write $\nabla_{i}^{q}(f)=\frac{\gamma_{i}(f)-f}{(q-1)T_{i}}$. 
\end{remark}
The $q$-derivative does not satisfy the Leibniz rule on the nose, but does so up to a twist by the automorphism $\gamma_{i}$ of \Cref{rmk: gamma i maps}. 
\begin{lemma}\label{lem: twisted q-leibniz}
    Let $R$ be $\ZZ[q^{\pm}][T_{1},\dots,T_{d}]$ or $\ZZ[q^{\pm}][T_{1}^{\pm},\dots,T_{d}^{\pm}]$. Then for $f,g\in R$ we have equalities 
    $$\nabla_{i}^{q}(fg)=\gamma_{i}(f)\cdot\nabla_{i}^{q}(g)+g\cdot\nabla_{i}^{q}(f)=f\cdot\nabla_{i}^{q}(g)+\gamma_{i}(g)\cdot\nabla^{i}_{q}(f).$$
\end{lemma}
\begin{proof}
    We first show the second equality. We use \Cref{rmk: gamma i maps} to observe that the latter two terms are given by 
    $$\gamma_{i}(f)\cdot\frac{\gamma_{i}(g)-g}{(q-1)T_{i}}+g\cdot\frac{\gamma_{i}(f)-f}{(q-1)T_{i}}=\frac{\gamma_{i}(f)\gamma_{i}(g)-\gamma_{i}(f)g+\gamma_{i}(f)g-fg}{(q-1)T_{i}}=\frac{\gamma_{i}(f)\gamma_{i}(g)-fg}{(q-1)T_{i}}$$
    and 
    $$f\cdot\frac{\gamma_{i}(g)-g}{(q-1)T_{i}}+\gamma_{i}(g)\frac{\gamma_{i}(f)-f}{(q-1)T_{i}}=\frac{\gamma_{i}(g)f-fg+\gamma_{i}(f)\gamma_{i}(g)-\gamma_{i}(g)f}{(q-1)T_{i}}=\frac{\gamma_{i}(f)\gamma_{i}(g)-fg}{(q-1)T_{i}}$$
    respectively, which are evidently equal. 

    We now show the first equality. Note that $\gamma_{i}$ is an automorphism $R\to R$, and in particular a homomorphism so $\gamma_{i}(fg)=\gamma_{i}(f)\gamma_{i}(g)$ in which case we have 
    $$\frac{\gamma_{i}(fg)-fg}{(q-1)T_{i}}=\frac{\gamma_{i}(f)\gamma_{i}(g)-fg}{(q-1)T_{i}}$$
    whence the claim. 
\end{proof}
We can now define the $q$-de Rham complex following Aomoto \cite{Aomoto}. 
\begin{definition}[$q$-de Rham Complex of $\A^{d}_{\ZZ}$ and $\GG_{m}^{d}$]\label{def: q-dR complex}
    Let $R=\ZZ[q^{\pm}][\underline{T}]$ be $\ZZ[q^{\pm}][T_{1},\dots,T_{d}]$ or $\ZZ[q^{\pm}][T_{1}^{\pm},\dots,T_{d}^{\pm}]$. The $q$-de Rham complex of $\spec(R)$ is the complex
    \begin{equation}\label{eqn: q-dR complex of R}
        \footnotesize 
        % https://q.uiver.app/#q=WzAsOCxbMCwwLCIwIl0sWzEsMCwiXFxaWltxXntcXHBtfV1bXFx1bmRlcmxpbmV7VH1dIl0sWzIsMCwiXFxaWltxXntcXHBtfV1bXFx1bmRlcmxpbmV7VH1dXntcXG9wbHVzIGR9Il0sWzMsMCwiXFxiaWdvcGx1c197aTxqfVxcWlpbcV57XFxwbX1dW1xcdW5kZXJsaW5le1R9XSJdLFs0LDAsIlxcZG90cyJdLFszLDEsIlxcZG90cyJdLFs0LDEsIlxcWlpbcV57XFxwbX1dW1xcdW5kZXJsaW5le1R9XSJdLFs1LDEsIjAiXSxbMCwxXSxbMSwyXSxbMiwzXSxbNSw2XSxbMyw0XSxbNiw3XV0=
        \begin{tikzcd}
            0 & {\ZZ[q^{\pm}][\underline{T}]} & {\ZZ[q^{\pm}][\underline{T}]^{\oplus d}} & {\bigoplus_{i<j}\ZZ[q^{\pm}][\underline{T}]} & \dots \\
            &&& \dots & {\ZZ[q^{\pm}][\underline{T}]} & 0
            \arrow[from=1-1, to=1-2]
            \arrow[from=1-2, to=1-3]
            \arrow[from=1-3, to=1-4]
            \arrow[from=1-4, to=1-5]
            \arrow[from=2-4, to=2-5]
            \arrow[from=2-5, to=2-6]
        \end{tikzcd}
        \normalsize
    \end{equation}
    with differentials given by the differentials for the Koszul complex of commuting operators $\nabla_{1}^{q},\dots,\nabla_{n}^{q}$. 
\end{definition}
\begin{remark}
    Recall that these are precisely the differentials for the classical de Rham complex. See \cite[\href{https://stacks.math.columbia.edu/tag/0FKF}{Tag 0FKF}]{stacks-project} for an explicit description via equations. 
\end{remark}
\begin{remark}
    Since the first differential $\ZZ[q^{\pm}][\underline{T}]\to\ZZ[q^{\pm}][\underline{T}]^{\oplus d}$ by $(\nabla_{1}^{q},\dots,\nabla_{d}^{q})$ does not satisfy the ordinary Leibniz rule, the complex (\ref{eqn: q-dR complex of R}) is not a differential graded algebra. Later, we will see that working in the derived ($\infty$-)category, one can endow this with the structure of a commutative ring.  
\end{remark}
The complex (\ref{eqn: q-dR complex of R}) computes $q$-de Rham cohomology, or Aomoto-Jackson cohomology of $\spec(R)$. But to compute Habiro cohomology, we use a closely related variant based on a modified $q$-derivative. 
\begin{definition}[Modified $q$-Derivative]\label{def: modified q-derivative}
    Let $R$ be $\ZZ[q^{\pm}][T_{1},\dots,T_{d}]$ or $\ZZ[q^{\pm}][T_{1}^{\pm},\dots,T_{d}^{\pm}]$. The modified $q$-derivative is given by 
    $$\widetilde{\nabla}_{i}^{q}(f(T_{1},\dots,T_{d}))=\frac{f(T_{1},\dots,qT_{i},\dots,T_{d})-f(T_{1},\dots,T_{i},\dots,T_{d})}{T_{i}}.$$
\end{definition}
\begin{remark}
    In other words, $\widetilde{\nabla}_{i}^{q}(f)=(q-1)\nabla_{i}^{q}(f)=\frac{\gamma_{i}(f)-f}{T_{i}}$.  
\end{remark}
Recomputing everything using this modified derivative gives the $q$-Hodge complex. 
\begin{definition}[$q$-Hodge Complex of $\A^{d}_{\ZZ}$ and $\GG_{m}^{d}$]\label{def: q-Hodge complex}
    Let $R=\ZZ[q^{\pm}][\underline{T}]$ be $\ZZ[q^{\pm}][T_{1},\dots,T_{d}]$ or $\ZZ[q^{\pm}][T_{1}^{\pm},\dots,T_{d}^{\pm}]$. The $q$-Hodge complex of $\spec(R)$ is the complex
    \begin{equation}\label{eqn: q-Hodge complex of R}
        \footnotesize 
        % https://q.uiver.app/#q=WzAsOCxbMCwwLCIwIl0sWzEsMCwiXFxaWltxXntcXHBtfV1bXFx1bmRlcmxpbmV7VH1dIl0sWzIsMCwiXFxaWltxXntcXHBtfV1bXFx1bmRlcmxpbmV7VH1dXntcXG9wbHVzIGR9Il0sWzMsMCwiXFxiaWdvcGx1c197aTxqfVxcWlpbcV57XFxwbX1dW1xcdW5kZXJsaW5le1R9XSJdLFs0LDAsIlxcZG90cyJdLFszLDEsIlxcZG90cyJdLFs0LDEsIlxcWlpbcV57XFxwbX1dW1xcdW5kZXJsaW5le1R9XSJdLFs1LDEsIjAiXSxbMCwxXSxbMSwyXSxbMiwzXSxbNSw2XSxbMyw0XSxbNiw3XV0=
        \begin{tikzcd}
            0 & {\ZZ[q^{\pm}][\underline{T}]} & {\ZZ[q^{\pm}][\underline{T}]^{\oplus d}} & {\bigoplus_{i<j}\ZZ[q^{\pm}][\underline{T}]} & \dots \\
            &&& \dots & {\ZZ[q^{\pm}][\underline{T}]} & 0
            \arrow[from=1-1, to=1-2]
            \arrow[from=1-2, to=1-3]
            \arrow[from=1-3, to=1-4]
            \arrow[from=1-4, to=1-5]
            \arrow[from=2-4, to=2-5]
            \arrow[from=2-5, to=2-6]
        \end{tikzcd}
        \normalsize
    \end{equation}
    with differentials given by the differentials for the Koszul complex of commuting operators $\widetilde{\nabla}_{1}^{1},\dots,\widetilde{\nabla}_{d}^{q}$. 
\end{definition}
\begin{remark}
    The nomenclature of \Cref{def: q-dR complex,def: q-Hodge complex} are justified by the fact that they recover the ordinary de Rham and Hodge complexes at $q=1$.
\end{remark} 
\begin{remark}
    An automorphism of $\A^{d}_{\ZZ}$ or $\GG_{m}^{d}$ would give rise to an automorphism of the complexes (\ref{eqn: q-dR complex of R}) and (\ref{eqn: q-Hodge complex of R}), at least as an object in the derived category, but it is extremely difficult to understand these automorphisms from this explicit perspective. 
\end{remark}\marginpar{The instructor remarks that he does not believe in non-flat connections. We will henceforth omit the adjective ``flat.'' \\\\ Note that a $q$-connection is additional data on a module.}
The classical correspondence between $D$-modules and modules with flat connection suggest that an appropriate category of modules with connection could play the role of $\Dscr_{\Hab}(X)$ alluded to earlier. To make this precise, we consider modules with $q$-connection. To simplify matters, we make these considerations on the Abelian and not $\infty$-categorical level. 
\begin{definition}[$q$-Connections on Modules]\label{def: q-connections on modules}
    Let $R=\ZZ[q^{\pm}][\underline{T}]$ be $\ZZ[q^{\pm}][T_{1},\dots,T_{d}]$ or $\ZZ[q^{\pm}][T_{1}^{\pm},\dots,T_{d}^{\pm}]$. A module with (flat) $q$-connection is a $\ZZ[q^{\pm}][\underline{T}]$-module with commuting $\ZZ[q^{\pm}]$-linear operations $\nabla_{i,M}^{q}:M\to M$ which satisfy the $q$-Leibniz rule 
    $$\nabla_{i,M}^{q}(fm)=\gamma_{i}(f)\cdot\nabla_{i,M}^{q}(m)+\nabla_{i}^{q}(f)\cdot m$$
    for $f\in \ZZ[q^{\pm}][\underline{T}]$ and $m\in M$.
\end{definition}
\begin{remark}
    To unwind any possible confusion between the similar-looking $\nabla_{i}^{q}:\ZZ[q^{\pm}][\underline{T}]\to\ZZ[q^{\pm}][\underline{T}],\nabla_{i,M}^{q}:M\to M$, we have 
    $$\underbrace{\underbrace{\gamma_{i}(f)}_{\in \ZZ[q^{\pm}][\underline{T}]}\cdot\underbrace{\nabla_{i,M}^{q}(m)}_{\in M}}_{\in M}+\underbrace{\underbrace{\nabla_{i}^{q}(f)}_{\in \ZZ[q^{\pm}][\underline{T}]}\cdot\underbrace{m}_{\in M}}_{\in M}$$
    so everything type-checks. 
\end{remark}
\begin{example}\label{ex: A1 with Weyl algebra}
    If $X=\A^{1}_{\ZZ}$ then recall that modules with connection are equivalent to modules over the Weyl algebra $\ZZ[q^{\pm}]\{T,\partial_{q}\}/(qT\partial_{q}-\partial_{q}T+1)$ since we have the operators $T\partial_{q},\partial_{q}T$ take $T^{n}$ to $q[n]_{q}T^{n},[n+1]_{q}T^{n}$, respectively, but $q[n]_{q}-[n+1]_{q}=q\cdot\frac{1-q^{n}}{1-q}-\frac{1-q^{n+1}}{1-q}=-1$. Passing to the associated-graded of the degree filtration, one gets commuting variables with the correct $q$-twists. 
\end{example}
Similarly, we can construct modules with a modified $q$-connection. 
\begin{definition}[Modified $q$-Connections on Modules]\label{def: modified q-connections on modules}
    Let $R=\ZZ[q^{\pm}][\underline{T}]$ be $\ZZ[q^{\pm}][T_{1},\dots,T_{d}]$ or $\ZZ[q^{\pm}][T_{1}^{\pm},\dots,T_{d}^{\pm}]$. A module with modified $q$-connection is a $\ZZ[q^{\pm}][\underline{T}]$-module with commuting $\ZZ[q^{\pm}]$-linear operations $\widetilde{\nabla}_{i,M}^{q}:M\to M$ which satisfy the $q$-Leibniz rule 
    $$\widetilde{\nabla}_{i,M}^{q}(fm)=\gamma_{i}(f)\cdot\widetilde{\nabla}_{i,M}^{q}(m)+\widetilde{\nabla}_{i}^{q}(f)\cdot m$$
    for $f\in \ZZ[q^{\pm}][\underline{T}]$ and $m\in M$.
\end{definition}
\begin{remark}\label{rmk: invertible case}
    Let $T_{i}$ be invertible. Unwinding the definition of the modified $q$-derivative, we have
    $$\widetilde{\nabla}_{i,M}^{q}(fm)=\gamma_{i}(f)\cdot\widetilde{\nabla}_{i,M}^{q}(m)+(q-1)\nabla_{i}^{q}(f)\cdot m$$
    where in particular we observe that the second summand has denominator $T_{i}$. Define a new operator 
    $$\widetilde{\widetilde{\nabla}}_{i,M}^{q}=T_{i}\cdot\widetilde{\nabla}_{i,M}^{q}$$
    which satisfies 
    \begin{align*}
        \widetilde{\widetilde{\nabla}}_{i,M}^{q}(fm) &= \gamma_{i}(f)\cdot\widetilde{\widetilde{\nabla}}_{i,M}^{q}(m)+(\gamma_{i}(f)-f)m \\
        &= \gamma_{i}(f)\left(\widetilde{\widetilde{\nabla}}_{i,M}^{q}(m)+m\right)-fm.
    \end{align*}
    In particular, 
    $$\left(\widetilde{\widetilde{\nabla}}_{i,M}^{q}+\id_{M}\right)(fm) = \gamma_{i}(f)\left(\widetilde{\widetilde{\nabla}}_{i,M}^{q}+\id_{M}\right)(m)$$
    so denoting $\gamma_{i,M}=\left(\widetilde{\widetilde{\nabla}}_{i,M}^{q}+\id_{M}\right)$, we have $\gamma_{i,M}(fm)=\gamma_{i}(f)\gamma_{i,M}(m)$ simplyfing the relation. 
\end{remark}
The preceding discussion of \Cref{rmk: invertible case} implies the following. 
\begin{corollary}\label{corr: equivalence of q-connection modules and semilinear}
    Let $R=\ZZ[q^{\pm}][T_{1}^{\pm},\dots,T_{d}^{\pm}]$. There is an equivalence of categories between $R$-modules with modified $q$-connection and $R$-modules with commuting $\gamma_{i}:R\to R$-semilinear endomorphisms $\gamma_{i,M}:M\to M$. 
\end{corollary}
Note that for $R=\ZZ[q^{\pm}][\underline{T}]$, $(-)\otimes_{R}(-)$ does not define a symmetric monoidal structure on the category of modules with $q$-connection: for $(M,\nabla_{i,M}^{q}),(N,\nabla_{i,N}^{q})$ two modules with $q$-connecition, $$(M\otimes_{R} N,\nabla_{i,M}^{q}\otimes_{R}\id_{N}+\id_{M}\otimes_{R}\nabla_{i,N}^{q})$$ is not a module with $q$-connection. One needs instead to take the twist $$(M\otimes_{R} N,\nabla_{i,M}^{q}\otimes_{R}\id_{N}+\gamma_{i,M}\otimes_{R}\nabla_{i,N}^{q}),$$
defining $\gamma_{i,M}:M\to M$ in an analogous way to \Cref{rmk: invertible case}. While \emph{a priori} appearing assymetric in $M,N$, there is in fact a canonical isomorphism between them. 
\begin{proposition}
    Let $R=\ZZ[q^{\pm}][T_{1}^{\pm},\dots,T_{d}^{\pm}]$. The category of $R$-modules with $q$-connection is symmetric monoidal. 
\end{proposition}
\begin{proof}[Proof Outline]
    Using the equivalence of \Cref{corr: equivalence of q-connection modules and semilinear}, the latter category is symmetric monoidal, hence the former can be promoted to a symmetric monoidal category. 
\end{proof}
\begin{proposition}
    Let $R=\ZZ[q^{\pm}][T_{1}^{\pm},\dots,T_{d}^{\pm}]$. There is a fully faithful embedding from $(q-1)$-torsion free $R$-modules with $q$-connection and $R$-modules with modified $q$-connection by $(M,\nabla_{i,M}^{q})\mapsto(M,\widetilde{\nabla}_{i,M}^{q})$ with essential image those that are $(q-1)$-torsion free and such that $\widetilde{\nabla}_{i,M}^{q}\equiv0\pmod{(q-1)}$. 
\end{proposition}
The discussion thus far has been done entirely in terms of coordinates. This prompts:
\begin{question}
    To what extent are the cohomologies and categories discussed thus far independent of coordinates? 
\end{question}
Let us consider the following example. 
\begin{example}\marginpar{The instructor remarks that in the theory of analytic geometry the quotient would be the Tate elliptic curve for $d=1$. See \cite{AnalyticStacks}.}
    Let $X=\GG_{m}^{d}$. The modules with modified $q$-connection are quasicoherent sheaves on $(\GG_{m}/q^{\ZZ})^{d}$ -- the $\gamma_{i}$'s act by multiplication by $q$ on the coordinates so the data of the endomorphisms $\gamma_{i,M}$ on the modules prescribe descent data to the quotient stack (ie. as an fpqc quotient).
\end{example}
Let us relate the discussion of complexes \Cref{def: q-dR complex,def: q-Hodge complex}, their cohomologies, and these categories of modules with (modified) $q$-connections. 
\begin{proposition}
    \begin{enumerate}[label=(\roman*)]
        \item The $q$ de Rham complex computes $R\Hom(\mathbbm{1},\mathbbm{1})$ in the derived category of modules with $q$-connection on $\spec(R)$. 
        \item The $q$-Hodge complex computes $R\Hom(\mathbbm{1},\mathbbm{1})$ in the derived category of modules with modified $q$-connection on $\spec(R)$. 
    \end{enumerate}
\end{proposition}
\begin{proof}[Proof Outline of (i)]
    Using the equvialence between modules with $q$-connection and modules over the Weyl algebra, we compute a resolution of the symmetric monoidal unit $\ZZ[q^{\pm}][\underline{T}]$ in the category of modules over the Weyl algebra -- which precisely recovers the de Rham complex, whence the claim. 
\end{proof}
\begin{example}
    Consider the case $\ZZ[q^{\pm}][T]$. We compute $R\Hom(\ZZ[q^{\pm}][T],\ZZ[q^{\pm}][T])$ as $R\Hom(-,\ZZ[q^{\pm}][T])$ of a free resolution of $\ZZ[q^{\pm}][T]$ in the category of modules over the Weyl algebra $\ZZ[q^{\pm}]\{T,\partial_{q}\}/(qT\partial_{q}-\partial_{q}T+1)$ (vis. \Cref{ex: A1 with Weyl algebra}). This produces 
    $$0\to\ZZ[q^{\pm}][T]\xrightarrow{\nabla_{1}^{q}}\ZZ[q^{\pm}][T]\to0$$
    which is the $q$-de Rham complex (after passing back to modules with $q$-connection along the equivalence). 
\end{example}
Moreover, in the setting of higher algebra, these promote canonically to commutative algebra objects. 
\begin{corollary}
    The $q$-de Rham complex and $q$-Hodge complex have canonical structures as $\EE_{\infty}$-rings. 
\end{corollary}
\section{Lecture 3 -- 14th April 2025}\label{sec: lecture 3}
Recall that the cotangent sheaf on $\A^{n}_{A},\PP^{n}_{A}$ over $\spec(A)$ are locally free sheaves by \Cref{ex: affine space scheme differentials,ex: projective space scheme differentials}. One is then led to consider for what other $S$-schemes $X$ is $\Omega^{1}_{X/S}$ locally free. This is roughly captured by smoothness. Moreover, as suggested by \Cref{prop: isomorphism to Zariski tangent space}, the notion of smoothness is connected to the Zariski tangent space, which in the case of algebraic geometry -- unlike differential geometry -- need not coincide with the geometric tangent space, especially in characteristic $p$ situations. 

We recall the definition of the Zariski tangent space. 
\begin{definition}[Zariski Tangent Space of a Ring]\label{def: Zariski tangent space}
    Let $A$ be a local ring with maximal ideal $\mfrak$ and residue field $\kappa=A/\mfrak$. The Zariski tangent space of $A$ is the $\kappa$-vector space $(\frac{\mfrak}{\mfrak^{2}})^{\vee}=\Hom_{\Vect_{\kappa}}(\frac{\mfrak}{\mfrak^{2}},\kappa)$. 
\end{definition}
\begin{definition}[Zariski Tangent Space of a Scheme]\label{def: Zariski tangent space scheme}
    Let $X$ be a scheme and $x\in X$ a point. The Zariski tangent space $T_{X,x}$ is the Zariski tangent space of the local ring $\Ocal_{X,x}$. 
\end{definition}
\begin{example}\label{ex: spec A example}
    Let $A$ be a ring and $\pfrak\subseteq\spec(A)$. The Zariski tangent space $T_{\spec(A),\pfrak}$ is given by $(\frac{\pfrak A_{\pfrak}}{(\pfrak A_{\pfrak})^{2}})^{\vee}$. This is a vector space over the field $\kappa(\pfrak)=A_{\pfrak}/\pfrak A_{\pfrak}$. 
\end{example}
\begin{example}\label{ex: map to hom}
    Let $k$ be a field and $x\in \A^{n}_{k}(k)$ a closed $k$-rational point hence of the form $(x_{1}-a_{1},\dots,x_{n}-a_{n})$. Define a map $D_{x}:k[x_{1},\dots,x_{n}]\to\Hom_{\Vect_{k}}(k^{n},k)$ by $$f\mapsto\left[(\alpha_{i})_{i=1}^{n}\mapsto\sum_{i=1}^{n}\alpha_{i}\frac{\partial f}{\partial x_{i}}(x)\right].$$
    The map is $k$-linear and satisfies the Leibniz rule, hence defines a $k$-linear derivation which is a $k$-vector space. This defines an isomorphism between $(\frac{\mfrak}{\mfrak^{2}})^{2}$ and $\Hom_{\Vect_{k}}(k^{n},k)$ by considering the Taylor expansion of a polynomial 
    $$f=f(x)+\sum_{i=1}^{n}\frac{\partial f}{\partial x_{i}}(x)(x_{i}-a_{i})+\underbrace{O(x^{2})}_{\in \mfrak^{2}}$$
    hence the map is zero on $f\in\mfrak^{2}$ showing that $(\frac{\mfrak}{\mfrak^{2}})^{\vee}\to\Hom_{\Vect_{k}}(k^{n},k)$ by $(x_{i}-a_{i})\mapsto e_{i}^{\vee}$ is an injection between vector spaces of the same dimension and hence an isomorphism. 
\end{example}
\begin{example}
    In general, one can still define a map on non-rational points with target $\Hom_{\Vect_{\kappa(\pfrak)}}(\kappa(\pfrak)^{n},\kappa(\pfrak))$ which may fail to be injective. Let $k$ be a field of characteristic $p$ and consider $(x^{p}-a)\subseteq k[x]$ which is maximal when $a^{1/p}\notin k$. We have $\frac{\mfrak}{\mfrak^{2}}=\frac{(x^{p}-a)}{(x^{p}-a)^{2}}\cong k$ which defines a map $\Hom_{\Vect_{k}}(k,k)$ by \Cref{ex: map to hom} which is the zero map as $px^{p-1}=0$. 
\end{example}
In what follows, we will use the following result for closed subschemes of affine spaces. 
\begin{proposition}\label{prop: annihilator is differentials}
    Let $\afrak\subseteq k[x_{1},\dots,x_{n}]$ be an ideal defining $X=V(\afrak)\subseteq \A^{n}_{k}$. Let $x\in X(k)\subseteq\A^{n}_{k}(k)$. Then $T_{X,x}$ is the annihilator of the image of $\afrak$ under $D_{x}$
\end{proposition}
\begin{proof}
    We have a short exact sequence 
    $$0\to\afrak\to\widetilde{\mfrak}\to\mfrak\to 0$$
    inducing 
    \begin{equation}\label{eqn: maxl ideal intersection SES}
        0\to\frac{\afrak}{\afrak\cap\widetilde{\mfrak}^{2}}\to\frac{\widetilde{\mfrak}}{\widetilde{\mfrak}^{2}}\to\frac{\mfrak}{\mfrak^{2}}\to0.
    \end{equation}
    Applying the right-exact functor $\Hom_{\Vect_{k}}(-,k)$ we get 
    $$\left(\frac{\afrak}{\afrak\cap\widetilde{\mfrak}^{2}}\right)^{\vee}\to T_{\A^{n}_{k},x}^{\vee}\to T_{X,x}^{\vee}\to0$$
    where the map $\left(\frac{\afrak}{\afrak\cap\widetilde{\mfrak}^{2}}\right)^{\vee}\to \Hom_{\Vect_{k}}(k^{n},k)$ by taking $\afrak$-derivations as in \Cref{ex: map to hom}. 
\end{proof}
An analogous proof can be used to show that the Zariski tangent space is the cokernel of the Jacobian matrix. 
\begin{corollary}\label{corr: dimension is corank jacobian}
    Let $\afrak\subseteq k[x_{1},\dots,x_{n}]$ be an ideal defining $X=V(\afrak)\subseteq \A^{n}_{k}$. Let $x\in X(k)\subseteq\A^{n}_{k}(k)$. Then $T_{X,x}^{\vee}\cong\coker(J_{x})$ where $J_{x}$ is the Jacobian at $x$. 
\end{corollary}
\begin{proof}
    We use the short exact sequence (\ref{eqn: maxl ideal intersection SES}) and observe that the map $\frac{\afrak}{\afrak\cap \widetilde{\mfrak}^{2}}\to\frac{\widetilde{\mfrak}}{\widetilde{\mfrak}^{2}}$ is given by multiplication by the Jacobian, giving the claim. 
\end{proof}
Having related this to the Zariski tangent space, we want to relate the Jacobian matrix to the sheaf/module of K\"{a}hler differentials, an analogy suggested by \Cref{prop: isomorphism to Zariski tangent space}. 
\begin{proposition}\label{prop: corank of Jacobian is base change of differentials}
    Let $\afrak\subseteq k[x_{1},\dots,x_{n}]$ be an ideal defining $X=V(\afrak)\subseteq \A^{n}_{k}$. Let $x\in X(k)\subseteq\A^{n}_{k}(k)$. Then the corank of the Jacobian $J_{x}$ is equal to $\dim_{\kappa(x)}\Omega_{X/k}^{1}\otimes\kappa(x)$. 
\end{proposition} 
\begin{proof}
    Applying $-\otimes\kappa(x)$ to the short exact sequence of \Cref{prop: ideal exact sequence}, we have 
    $$\frac{\afrak}{\afrak^{2}}\otimes\kappa(x)\to\Omega^{1}_{\A^{n}_{k}/k}\otimes\kappa(x)\to\Omega_{X/k}^{1}\otimes\kappa(x)\to0$$
    which factors over the image of the Jacobian $J_{x}$. As such, we get that $\dim_{\kappa(x)}\Omega_{X/k}^{1}=n-\dim(\img(J_{x}))=n-\mathrm{rank}(J_{x})$ which is precisely the corank. 
\end{proof}
Moreover, for a general point $x$, the property of the Zariski tangent space being isomorphic to the scalar extension of the sheaf of K\"{a}hler differentials. 
\begin{proposition}
    Let $\afrak\subseteq k[x_{1},\dots,x_{n}]$ be an ideal defining $X=V(\afrak)\subseteq \A^{n}_{k}$. Let $x\in X$. $\kappa(x)$ is a separable extension of $k$ if and only if $T_{X,x}^{\vee}\cong\Omega_{X/k}^{1}\otimes\kappa(x)$. 
\end{proposition}
\begin{proof}
    $(\Rightarrow)$ If $\kappa(x)$ is separable over $k$, then $\Omega_{\kappa(x)/k}^{1}=0$ by \Cref{lem: differentials of separable extension are zero} so $\frac{\mfrak}{\mfrak^{2}}\to\Omega_{X,k}^{1}\otimes\kappa(x)$ is surjective, but this shows that we have a surjection of $\kappa(x)$-vector spaces of the same dimension, hence an isomorphism. 

    $(\Leftarrow)$ If the Zariski cotangent space is isomorphic to teh sheaf of differentials, then the cokernel $\Omega^{1}_{\kappa(x)/k}$ of the exact sequence \Cref{prop: ideal exact sequence} is zero, showing that $\kappa(x)/k$ is separable. 
\end{proof}
Note that the equality $\dim(T_{X,x})=\dim_{\kappa(x)}\Omega^{1}_{X/k}\otimes\kappa(x)$ does not imply the natural map is an isomorphism when $\kappa(x)$ is not separable over $k$. 
\begin{example}
    Let $k$ be a field of characteristic $p$ and consider $(x^{p}-a)\subseteq k[x]$ for $a^{1/p}\notin k$. Denoting $X=V(x^{p}-a)\subseteq\A^{1}_{k}$, we have $\dim(T_{X,x})=1=\dim_{\kappa(x)}\Omega^{1}_{X/k}\otimes\kappa(x)$ but $\Omega^{1}_{\kappa(x)/k}$ is nonzero as $\kappa(x)$ is not a separable extension of $k$. 
\end{example}
We arrive at the notion of smoothness for schemes. 
\begin{definition}[Smooth Scheme]\label{def: smooth scheme}
    Let $X$ be a scheme of finite type over a field $k$. $X$ is smooth of pure dimension $d$ if 
    \begin{enumerate}[label=(\roman*)]
        \item each of the finitely many irreducible components of $X$ are of dimension $d$, and 
        \item every point $x\in X$ is contained in an affine open neighborhood where the Jacobian matrix is of corank $d$. 
    \end{enumerate}
\end{definition}
\begin{remark}\label{rmk: smoothness is relative}
    Smoothness is a relative notion, determined by the structure map to $\spec(k)$. 
\end{remark}
\begin{remark}
    By \Cref{ex: spec A example}, this construction is independent of the choice of chart. 
\end{remark}
\begin{remark}\label{rmk: check smoothness on closed points}
    It suffices to verify this condition on closed points, as if the Jacobian is rank-deficient at some non-closed point, then it is rank-deficient at any specialization. 
\end{remark}
Intuitively, we can view $X$ locally as the fiber of a map $\A^{n}_{k}\to\A^{r}_{k}$ defined by the $r$ polynomials $f_{1},\dots,f_{r}\in k[x_{1},\dots,x_{n}]$, and where $x$ being in the fiber over zero implies that the tangent space $T_{X,x}$ is the kernel of the map $(f_{1},\dots,f_{r})$ hence equal to the dimension of the fiber. 
We introduce the notion of being geometrically smooth. 
\begin{definition}[Geometrically Smooth Scheme]\label{def: geometrically smooth scheme}
    Let $X$ be a scheme of finite type over a field $k$. $X$ is geometrically smooth if the base change $X_{\overline{k}}$ to the algebraic closure is smooth over $\overline{k}$. 
\end{definition}
This is in fact equivalent to the condition of being smooth. 
\begin{lemma}\label{lem: smooth iff geometrically smooth}
    Let $X$ be a scheme of finite type over a field $k$. $X$ is a smooth $k$-scheme if and only if it is geometrically smooth. 
\end{lemma}
\begin{proof}
    We use the Cartesian square 
    $$% https://q.uiver.app/#q=WzAsNCxbMCwwLCJYX3tcXG92ZXJsaW5le2t9fSJdLFswLDEsIlxcc3BlYyhcXG92ZXJsaW5le2t9KSJdLFsyLDAsIlgiXSxbMiwxLCJcXHNwZWMoaykiXSxbMCwyXSxbMiwzXSxbMCwxXSxbMSwzXV0=
    \begin{tikzcd}
        {X_{\overline{k}}} && X \\
        {\spec(\overline{k})} && {\spec(k)}
        \arrow[from=1-1, to=1-3]
        \arrow[from=1-1, to=2-1]
        \arrow[from=1-3, to=2-3]
        \arrow[from=2-1, to=2-3]
    \end{tikzcd}$$
    where using \Cref{prop: tensor exact sequence}, we have $\Omega_{X/k}^{1}\otimes\kappa(x)\cong\Omega^{1}_{X_{\overline{k}}/\overline{k}}\otimes\kappa(y)$ where $y$ is the closed point corresponding to $x$ in $X_{\overline{k}}$. This isomorphism of sheaves characterizes the Jacobian being full rank at $x,y$, hence the smoothness conditions are equivalent. 
\end{proof}
While smoothness depends on the structure of $X$ as a $k$-scheme, it is closely related to the absolute notion of regularity. 

We recall the relevant definitons from commutative algebra. 
\begin{definition}[Regular Local Ring]\label{def: regular local ring}
    Let $(A,\mfrak)$ be a Noetherian local ring with residue field $\kappa=A/\mfrak$. $A$ is a regular local ring if $\dim(A)=\dim_{\kappa}(\frac{\mfrak}{\mfrak^{2}})$. 
\end{definition}
\begin{definition}[Regular Ring]\label{def: regular ring}
    Let $A$ be a Noetherian ring. $A$ is a regular ring if for all primes $\pfrak\subseteq A$, the localization $A_{\pfrak}$ is a regular local ring. 
\end{definition}
\begin{remark}
    Checking regularity of an arbitrary Noetherian ring can be done on maximal ideals by reasoning analogous to that of \Cref{rmk: check smoothness on closed points}. 
\end{remark}
This allows us to define regularity of schemes. 
\begin{definition}[Regular Scheme]\label{def: regular scheme}
    Let $X$ be a locally Noetherian scheme. $X$ is regular if for all closed points $x\in X$, the local ring $\Ocal_{X,x}$ is regular. 
\end{definition}
\begin{remark}
    By \Cref{def: regular ring}, this is equivalent to each point admitting an affine neighborhood given by the Zariski spectrum of a regular ring. 
\end{remark}
\begin{remark}
    In contrast to \Cref{rmk: smoothness is relative}, regularity is absolute and does not depend on any structure map of $X$. 
\end{remark}
The notions of regularity and smoothness are connected by the following proposition. 
\begin{proposition}\label{prop: smooth iff regular alg closed}
    Let $X$ be a scheme of finite type over an algebraically closed field $k$. $X$ is $k$-smooth if and only if $X$ is regular. 
\end{proposition}
\begin{proof}
    Both conditions can be checked affine-locally, so without loss of generality, we can take $X=V(\afrak)\subseteq\A^{n}_{k}$. By \Cref{prop: annihilator is differentials} the dimension of the Zariski tangent space of any $x\in X$ is dimension of the image of the map $D_{x}$ defined in \Cref{ex: map to hom}, which is equal to the rank of the Jacobian $J_{x}$. This is of rank equal to the Zariski tangent space (ie. $X$ is regular) if and only if the corank of the Jacobian is $\dim(X)$ (ie. $X$ is smooth). 
\end{proof}
Over general fields, smoothness implies regularity, but not the converse. 
\begin{corollary}\label{corr: smooth implies regular}
    Let $X$ be a scheme of finite type over a field $k$. If $X$ is $k$-smooth, then $X$ is regular. 
\end{corollary}
\begin{proof}
    By locality, we reduce once more to $X=V(\afrak)\subseteq\A^{n}_{k}$. By smoothness, $D_{x}(\afrak)=\mathrm{rank}(J_{x})$ and by the short exact sequence (\ref{eqn: maxl ideal intersection SES}) we have 
    $$\dim_{\kappa(\mfrak)}\left(\frac{\mfrak}{\mfrak^{2}}\right)=\dim_{\kappa(\mfrak)}\left(\frac{\widetilde{\mfrak}}{\widetilde{\mfrak}^{2}}\right)-\dim_{\mfrak}\left(\frac{\afrak}{\afrak\cap\widetilde{\mfrak}^{2}}\right)$$
    showing that the dimension of the Zariski tangent space at $x$ is equal to the dimension of $X$, hence $X$ is regular. 
\end{proof}
We now see an example of a regular non-smooth scheme. 
\begin{example}
    Let $k$ be a field of characteristic $p$ and consider $X=V(x^{p}-a)\subseteq \A^{1}_{k}$ and where $a^{1/p}\notin k$. $\spec(\frac{k[x]}{(x^{p}-a)})$ is the Zariski spectrum of a field, hence regular, but $X$ is not geometrically smooth and hence not smooth. 
\end{example}
\section{Lecture 4 -- 23rd May 2025}\label{sec: lecture 4}
Using the gluing procedure of (\ref{eqn: gluing map Frobenius}) allows us to correct for the overspecification of prescribing a local algebra $R^{(m)}$ for each positive integer $m$ in characteristic $p$ -- that is, gluing $R^{(m)},R^{(m')}$ where $m_{0}$ is coprime to $p$ and $m=m_{0}p^{a},m'=m_{0}p^{b}$ using the Frobenius. 

More generally, we can define the Habiro ring of a smooth framed $\ZZ$-algebra $(R,\square)$ by passing to the limit of the rings $R_{n}$ where there are surjective transition maps $R_{pm}\to R_{m}$ given by the (necessarily unique) lift of the isomorphism (\ref{eqn: gluing map Frobenius}) along the (necessarily unique) deformation of \'{e}tale algebras $R^{(m)}$ to $R_{m}$.  
\begin{definition}[Habiro Ring of Framed Algebra]\label{def: Habiro ring of framed algebra}
    Let $(R,\square)$ be a smooth framed $\ZZ$-algebra. The Habiro ring $\Hcal_{(R,\square)}$ is given by the limit 
    $$\Hcal_{(R,\square)}=\lim_{n\in\NN}R_{n}$$
    where $R_{n}$ is the completed root of unity algebra of \Cref{def: completed root of unity algebra}. 
\end{definition}
\begin{proposition}\label{prop: explicit elements of HR}
    Let $(R,\square)$ be a smooth framed $\ZZ$-algebra. The Habiro ring $\Hcal_{(R,\square)}$ of $(R,\square)$ is given by 
    {\footnotesize
    \begin{equation}\label{eqn: Habiro ring of framed algebra}
        \Hcal_{(R,\square)}=\left\{(f_{m})_{m\geq 1}\in\prod_{m\geq 1}R^{(m)}[[q-\zeta_{m}]]:\substack{\forall m\in\NN,\text{ }\forall p\text{ prime} \\\varphi_{p}(f_{pm})=f_{m}\in (R^{(m)})_{p}^{\wedge}[[q-\zeta_{m}]]\cong (R^{(pm)})_{p}^{\wedge}[[q-\zeta_{pm}]]}\right\}
    \end{equation}
    \normalsize}where $\varphi_{p}$ lifts the Frobenius on $R^{(m)}/(p)$ by raising each variable to the $p$-th power and fixes $q$ and $\zeta_{m}$. 
\end{proposition}
\begin{remark}
    There is an obvious map from the Habiro ring of the torus \Cref{def: Habiro ring of base} $\Hcal_{\ZZ[\underline{T}^{\pm}]}\to\Hcal_{(R,\square)}$ endowing the Habiro ring of $(R,\square)$ with the structure of a $\Hcal_{\ZZ[\underline{T}^{\pm}]}$-algebra.  
\end{remark}
Let us consider some explicit elements of the Habiro ring. 
\begin{example}\label{ex: element of Habiro ring}\marginpar{The lecture contained a fairly substantive sketch of the proof \Cref{ex: element of Habiro ring}, which the author has defered to \Cref{appdx: explicit elements} for continuity of exposition.}
    Let $R=\ZZ[T_{1},\dots,T_{d},\frac{1}{1-T_{1}-\dots-T_{d}}]$ with framing $\square:\ZZ[T_{1},\dots,T_{d}]\to R$. The element 
    $$\sum_{k_{1},\dots,k_{d}\geq0}\left[\substack{k_{1}+\dots+k_{d} \\ k_{1}\text{ }\dots\text{ }k_{d}}\right]_{q}T_{1}^{k_{1}}\dots T_{d}^{k_{d}}\in\ZZ[q][[\underline{T}]]$$
    is an element of the Habiro ring $\Hcal_{(R,\square)}$ where 
    $$\left[\substack{k_{1}+\dots+k_{d} \\ k_{1}\text{ }\dots\text{ }k_{d}}\right]_{q}=\frac{(q;q)_{k_{1}+\dots+k_{d}}}{(q;q)_{k_{1}}\dots(q;q)_{k_{d}}}$$
    is the $q$-deformation of the multinomial $\binom{k_{1}+\dots+k_{d}}{k_{1}\dots k_{d}}$. More generally, explicit elements of the Habiro ring can be constructed by considering $q$-deformations of rational functions (vis. \Cref{ex: legendre family} and surrounding discussion). 
\end{example}
Returning to a discussion of Habiro cohomology of a smooth $\ZZ$-algebra with framing $\square:\spec(R)\to(\GG_{m})^{d}$, we recall that there are lifts of the automorphism $\gamma_{i}$ to $\Hcal_{(R,\square)}$: more explicitly, for a section $(f_{m})_{m\geq0}$, the action $\gamma_{i}$ acts by $(f_{m})_{m\geq1}\mapsto (\gamma_{i}^{(m)}(f_{m}))_{m\geq1}$ where $\gamma_{i}^{(m)}$ is the automorphism given in \Cref{def: root of unity algebra}. This produces a $\ZZ^{d}$-action on $\Hcal_{(R,\square)}$, and we can define Habiro-Hodge cohomology to be the group cohomology of the action of $\ZZ^{d}$ on $\Hcal_{(R,\square)}$. 
\begin{definition}[$q$-Habiro-Hodge Cohomology]\label{def: q-Habiro-Hodge cohomology}
    Let $(R,\square)$ be a smooth framed $\ZZ$-algebra. The $q$-Habiro-Hodge cohomology is the cohomology of the $q$-Habiro-Hodge complex $q\dash\HHdg_{(R,\square)}$ given by 
    \begin{equation}\label{eqn: q-Habiro-Hodge complex}
        \Hcal_{(R,\square)}\xrightarrow{(\gamma_{i}-1)_{i=1}^{d}}\bigoplus_{i=1}^{d}\Hcal_{(R,\square)}\xrightarrow{(\gamma_{i}-1)_{i=1}^{d}}\bigoplus_{i<j}\Hcal_{(R,\square)}\longrightarrow\dots.
    \end{equation}
\end{definition}
For this to be functorial, we would expect this to be coordinate independent, at least at the level of derived categories. As a first step, we study the cohomology of the complex modulo $(1-q^{m})$ -- that is, at specalizations to roots of unity. 

If $m=1$, then $\Hcal_{(R,\square)}/(1-q)\cong R$ and all differentials are zero, so 
\begin{equation}\label{eqn: de Rham complex at trivial root of unity}
    H^{i}\left(q\dash\HHdg_{(R,\square)}/(1-q)\right)\cong R^{\oplus\binom{d}{i}}\cong \Omega^{i}_{R/\ZZ}
\end{equation}
and is therefore independent of coordinates since the middle term is so. 
\begin{remark}\label{rmk: Bockstein operator}
    While \emph{a priori} we only have a isomorphism to a free module of a certain rank, there is additonal structure that allows us to identify this with the module of K\"{a}hler differentials: the Bockstein map associated to the triangle 
    {\footnotesize 
    $$q\dash\HHdg_{(R,\square)}/(1-q)\xrightarrow{\times(1-q)}q\dash\HHdg_{(R,\square)}/(1-q)^{2}\longrightarrow q\dash\HHdg_{(R,\square)}/(1-q)\longrightarrow \left(q\dash\HHdg_{(R,\square)}/(1-q)\right)[1]$$
    \normalsize}inducing 
    $$H^{i}\left(q\dash\HHdg_{(R,\square)}/(1-q)\right)\longrightarrow H^{i+1}\left(q\dash\HHdg_{(R,\square)}/(1-q)\right)$$
    which gives a derivation 
    $$H^{0}\left(q\dash\HHdg_{(R,\square)}/(1-q)\right)\longrightarrow H^{1}\left(q\dash\HHdg_{(R,\square)}/(1-q)\right)$$
    and hence an isomorphism $H^{1}\left(q\dash\HHdg_{(R,\square)}/(1-q)\right)\to\Omega^{1}_{R/\ZZ}$. In addition, the ring structure on cohomology induces the structure of a commutative differential graded algebra on $H^{\bullet}\left(q\dash\HHdg_{(R,\square)}/(1-q)\right)$ and this structure is in fact independent of coordinates on the nose and not just up to quasi-isomorphism.
\end{remark}

For general $m$, $H^{\bullet}\left(q\dash\HHdg_{(R,\square)}/(1-q^{m})\right)$ has the strucuture of a commutative differential graded algebra that is coordinate independent. 
\begin{theorem}[Wagner; {\cite[Prop. 5.7]{WagnerMSThesis}}]\label{thm: surjection from q witt vectors}
    Let $R$ be a smooth framed $\ZZ$-algebra. There is a canonical surjection 
    $$W_{m}(R)[q]/(1-q^{m})^{\bullet}\longrightarrow H^{0}\left(q\dash\HHdg_{(R,\square)}/(1-q^{m})\right)$$
    inducing 
    $$\Omega_{W_{m}(R)[q]/(1-q^{m})}\longrightarrow H^{\bullet}\left(q\dash\HHdg_{(R,\square)}/(1-q)\right)$$ 
    which is coordinate independent, degreewise surjective, and with kernel independent of coordinates. 
\end{theorem}
\begin{proof}[Proof Outline]
    For every commutative differential graded algebra $B$ receiving a map from a commutative ring $A$ in 0th cohomology, there is an induced map from the initial commutative differential graded algebra generated by $A$ to $B$ -- the latter being the de Rham complex. 
\end{proof}
This produces a description of $H^{i}\left(q\dash\HHdg_{(R,\square)}/(1-q)\right)$ that is visibly independent of coordinates, being the quotient of coordinate-independent objects. 

In fact we can do better. For any $R$, there is a notion of $q$-Witt vectors $q\dash W_{m}(R)$ and $q$-de Rham-Witt complexes $q\dash W_{m}\Omega_{R}$ which is a commutative differential graded algebra with first term $q\dash W_{m}(R)$ isomorphic to $H^{\bullet}\left(q\dash\HHdg_{(R,\square)}/(1-q^{m})\right)$. 
\begin{theorem}[Wagner; {\cite[Thm. 5.7]{WagnerMSThesis}}]\label{thm: }
    Let $R$ be a smooth framed $\ZZ$-algebra. There is an isomorphism 
    $$q\dash W_{m}\Omega_{R}^{\bullet}\longrightarrow H^{\bullet}\left(q\dash\HHdg_{(R,\square)}/(1-q)\right).$$
\end{theorem}
\begin{remark}
    This is related to the classical construction of the de Rham-Witt complex, though the sense in which the preceding constructions are $q$-deformations are quite subtle. 
\end{remark}
\begin{remark}
    One can often reduce to the case of computing on the torus, since many of the constructions ``commute with \'{e}tale maps'' in the sense that they are preserved under \'{e}tale base change. 
\end{remark}
Based on this, one might hope that these complexes are independent of coordinates. 
\begin{example}\label{ex: q-Habiro-Hodge cohomology of torus}
    Let $R=\ZZ[T^{\pm}]$. The $q$-Habiro-Hodge complex is given by 
    $$\ZZ[q][T^{\pm}]/(1-q^{m})\xrightarrow{\gamma-1}\ZZ[q][T^{\pm}]/(1-q^{m})$$
    by $T^{k}\mapsto(q^{k}-1)T^{k}$. We can compute the kernel of this map -- the 0th cohomology -- by noting that the map preserves the degree of $T$, we can compute the kernel in each degree to see that it is given by 
    $$\bigoplus_{k\in\ZZ}\left(\frac{\frac{q^{m}-1}{q^{\gcd(k,m)}-1}\ZZ[q]}{(q^{m}-1)\ZZ[q]}\right)T^{k}\cong\bigoplus_{k\in\ZZ}\left(\ZZ[q]/(1-q^{\gcd(k,m)})\ZZ[q]\right)T^{k}.$$
    We similarly compute first cohomology to see it is also given by 
    $$\bigoplus_{k\in\ZZ}\left(\ZZ[q]/(1-q^{\gcd(k,m)})\ZZ[q]\right)T^{k}.$$
    Indeed, when $m=p$ is prime, the 0th cohomology is a subring of $\ZZ[q][T^{\pm}]/(1-q^{p})$ (hence a subring of $\ZZ[T^{\pm}]\times\ZZ[\zeta_{p}][T^{\pm p}]\subseteq\ZZ[T^{\pm}]\times\ZZ[\zeta_{p}][T^{\pm}]$) and is generated by $T^{p}$ and $[p]_{q}T^{i}$ for $1\leq i\leq p-1$. 
\end{example}
The computations of \Cref{ex: q-Habiro-Hodge cohomology of torus} is suggestive of a connection to Witt vectors since the cohomology lies in the product of rings $\ZZ[T^{\pm}]\times\ZZ[\zeta_{p}][T^{\pm p}]$. Recall that for a $p$-torsion free ring $R$, the $p$-th Witt vectors $W_{p}(R)$ consists of elements $(x_{0},x_{1},\dots)$ has ghost maps $\gh_{1},\gh_{p}:W_{p}(R)\to R$ by $(x_{0},x_{1},\dots)\mapsto x_{0}$ and $(x_{0},x_{1},\dots)\mapsto x_{0}^{p}+px_{1}$, respectively. The image of $(\gh_{1},\gh_{p}):W_{p}(R)\to R\times R$ consists precisely of those pairs $(x,y)\in R\times R$ where $y\equiv x^{p}\pmod{p}$. 
\begin{proposition}[Wagner]\label{prop: q-Witt vectors}
    Let $R=\ZZ[T^{\pm}]$ with the identity framing and $q\dash\HHdg_{(R,\square)}$ its $q$-Habiro-Hodge complex. There is a canonical embedding 
    $$W_{p}(R)\hookrightarrow H^{0}\left(q\dash\HHdg_{(R,\square)}/(1-q^{p})\right)$$
    rendering the diagram 
    {\footnotesize
    $$% https://q.uiver.app/#q=WzAsNyxbMSwxLCJIXnswfVxcbGVmdChxXFxkYXNoXFxISGRnX3soUixcXHNxdWFyZSl9LygxLXFee3B9KVxccmlnaHQpIl0sWzEsMiwiV197cH0oUikiXSxbMywxLCJcXFpaW1Ree1xccG19XVxcdGltZXNcXFpaW1xcemV0YV97cH1dW1Ree1xccG0gcH1dIl0sWzMsMiwiUlxcdGltZXMgUiJdLFswLDMsIih4X3swfSx4X3sxfSkiXSxbNCwzLCIoeF97MH0seF97MH1ee3B9K3B4X3sxfSkiXSxbMCwwLCJcXHZhcnBoaV97cH0oeF97MH0pK1twXV97cX14X3sxfSJdLFsxLDAsIiIsMCx7InN0eWxlIjp7InRhaWwiOnsibmFtZSI6Imhvb2siLCJzaWRlIjoidG9wIn19fV0sWzEsMywiKFxcZ2hfezF9LFxcZ2hfe3B9KSIsMl0sWzMsMl0sWzAsMiwiIiwwLHsic3R5bGUiOnsidGFpbCI6eyJuYW1lIjoiaG9vayIsInNpZGUiOiJ0b3AifX19XSxbNCw1LCIiLDIseyJzdHlsZSI6eyJ0YWlsIjp7Im5hbWUiOiJtYXBzIHRvIn19fV0sWzQsNiwiIiwwLHsic3R5bGUiOnsidGFpbCI6eyJuYW1lIjoibWFwcyB0byJ9fX1dXQ==
    \begin{tikzcd}
        {\varphi_{p}(x_{0})+[p]_{q}x_{1}} \\
        & {H^{0}\left(q\dash\HHdg_{(R,\square)}/(1-q^{p})\right)} && {\ZZ[T^{\pm}]\times\ZZ[\zeta_{p}][T^{\pm p}]} \\
        & {W_{p}(R)} && {R\times R} \\
        {(x_{0},x_{1})} &&&& {(x_{0},x_{0}^{p}+px_{1})}
        \arrow[hook, from=2-2, to=2-4]
        \arrow[hook, from=3-2, to=2-2]
        \arrow["{(\gh_{1},\gh_{p})}"', from=3-2, to=3-4]
        \arrow[from=3-4, to=2-4]
        \arrow[maps to, from=4-1, to=1-1]
        \arrow[maps to, from=4-1, to=4-5]
    \end{tikzcd}$$
    \normalsize}commutative. 
\end{proposition}
\begin{remark}
    On the $q$-Habiro-Hodge cohomologies, we can relate the different specializations by Frobenii and Verschiebungen 
    $$% https://q.uiver.app/#q=WzAsMixbMCwwLCJIXntpfVxcbGVmdChxXFxkYXNoXFxISGRnX3soUixcXHNxdWFyZSl9LygxLXFee21rfSlcXHJpZ2h0KSJdLFsyLDAsIkhee2l9XFxsZWZ0KHFcXGRhc2hcXEhIZGdfeyhSLFxcc3F1YXJlKX0vKDEtcV57bX0pXFxyaWdodCkuIl0sWzAsMSwiRl97a30iLDAseyJvZmZzZXQiOi0xfV0sWzEsMCwiVl97a309XFx0aW1lc1xcZnJhY3sxLXFee21rfX17MS1xXnttfX0iLDAseyJvZmZzZXQiOi0xfV1d
    \begin{tikzcd}
        {H^{i}\left(q\dash\HHdg_{(R,\square)}/(1-q^{mk})\right)} && {H^{i}\left(q\dash\HHdg_{(R,\square)}/(1-q^{m})\right).}
        \arrow["{F_{k}}", shift left, from=1-1, to=1-3]
        \arrow["{V_{k}=\times\frac{1-q^{mk}}{1-q^{m}}}", shift left, from=1-3, to=1-1]
    \end{tikzcd}$$
\end{remark}
More generally, we have the following. 
\begin{proposition}
    Let $R$ be a flat $\ZZ$-algebra. There is a commutative diagram 
    $$% https://q.uiver.app/#q=WzAsNSxbMiwwLCJXX3ttfShSKSJdLFs0LDAsIlxccHJvZF97ZHxtfVIiXSxbNCwxLCJcXHByb2Rfe2R8bX1SW1xcemV0YV97ZH1dIl0sWzIsMSwicVxcZGFzaCBXX3ttfShSKSJdLFswLDEsIldfe219KFIpW3FdLygxLXFee219KSJdLFs0LDMsIiIsMCx7InN0eWxlIjp7ImhlYWQiOnsibmFtZSI6ImVwaSJ9fX1dLFswLDRdLFswLDMsIiIsMix7InN0eWxlIjp7InRhaWwiOnsibmFtZSI6Imhvb2siLCJzaWRlIjoidG9wIn19fV0sWzAsMSwiKFxcZ2hfe2R9KV97ZHxtfSIsMCx7InN0eWxlIjp7InRhaWwiOnsibmFtZSI6Imhvb2siLCJzaWRlIjoidG9wIn19fV0sWzMsMiwiKHFcXGRhc2hcXGdoX3tkfSlfe2R8bX0iLDIseyJzdHlsZSI6eyJ0YWlsIjp7Im5hbWUiOiJob29rIiwic2lkZSI6InRvcCJ9fX1dLFsxLDIsIihSKV97ZH1cXHRvKFJbXFx6ZXRhX3ttL2R9XSlfe20vZH0iLDAseyJzdHlsZSI6eyJ0YWlsIjp7Im5hbWUiOiJob29rIiwic2lkZSI6InRvcCJ9fX1dXQ==
    \begin{tikzcd}
        && {W_{m}(R)} && {\prod_{d|m}R} \\
        {W_{m}(R)[q]/(1-q^{m})} && {q\dash W_{m}(R)} && {\prod_{d|m}R[\zeta_{d}]}
        \arrow["{(\gh_{d})_{d|m}}", hook, from=1-3, to=1-5]
        \arrow[from=1-3, to=2-1]
        \arrow[hook, from=1-3, to=2-3]
        \arrow["{(R)_{d}\to(R[\zeta_{m/d}])_{m/d}}", hook, from=1-5, to=2-5]
        \arrow[two heads, from=2-1, to=2-3]
        \arrow["{(q\dash\gh_{d})_{d|m}}"', hook, from=2-3, to=2-5]
    \end{tikzcd}$$
    where the Frobenii and Verschiebungen are defined on $q\dash W_{m}(R)$. 
\end{proposition}
\begin{remark}
    On the right, the map takes the $d$th factor of the product $\prod_{d|m}R$ to the $\frac{m}{d}$th factor of the product $\prod_{d|m}R[\zeta_{d}]$. 
\end{remark}
\begin{remark}
    There are no restriction maps on the $q$-Witt vectors $q\dash W_{m}(R)$. 
\end{remark}
This shows that on the level of cohomology, the $q$-Habiro-Hodge complex is coordinate independent after specialization. However, due to a theorem of Wagner, this is the best we can do: there is no way to make the $q$-Habiro-Hodge complex itself coordinate independent in the derived category in such a way that remains coordinate independent on specialization. 

\newpage
\printbibliography
\end{document}