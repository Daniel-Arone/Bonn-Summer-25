\documentclass{amsart}
\usepackage[margin=1.5in]{geometry} 
\usepackage{amsmath}
\usepackage{tcolorbox}
\usepackage{amssymb}
\usepackage{amsthm}
\usepackage{lastpage}
\usepackage{fancyhdr}
\usepackage{accents}
\usepackage{hyperref}
\usepackage{xcolor}
\usepackage{color}
% Fields
\newcommand{\A}{\mathbb{A}}
\newcommand{\CC}{\mathbb{C}}
\newcommand{\EE}{\mathbb{E}}
\newcommand{\RR}{\mathbb{R}}
\newcommand{\QQ}{\mathbb{Q}}
\newcommand{\ZZ}{\mathbb{Z}}
\newcommand{\HH}{\mathbb{H}}
\newcommand{\KK}{\mathbb{K}}
\newcommand{\NN}{\mathbb{N}}
\newcommand{\FF}{\mathbb{F}}
\newcommand{\PP}{\mathbb{P}}
\newcommand{\GG}{\mathbb{G}}
\newcommand{\LL}{\mathbb{L}}
\newcommand{\WW}{\mathbb{W}}

% mathcal letters
\newcommand{\Acal}{\mathcal{A}}
\newcommand{\Bcal}{\mathcal{B}}
\newcommand{\Ccal}{\mathcal{C}}
\newcommand{\Dcal}{\mathcal{D}}
\newcommand{\Ecal}{\mathcal{E}}
\newcommand{\Fcal}{\mathcal{F}}
\newcommand{\Gcal}{\mathcal{G}}
\newcommand{\Hcal}{\mathcal{H}}
\newcommand{\Ical}{\mathcal{I}}
\newcommand{\Jcal}{\mathcal{J}}
\newcommand{\Kcal}{\mathcal{K}}
\newcommand{\Lcal}{\mathcal{L}}
\newcommand{\Mcal}{\mathcal{M}}
\newcommand{\Ncal}{\mathcal{N}}
\newcommand{\Ocal}{\mathcal{O}}
\newcommand{\Pcal}{\mathcal{P}}
\newcommand{\Qcal}{\mathcal{Q}}
\newcommand{\Rcal}{\mathcal{R}}
\newcommand{\Scal}{\mathcal{S}}
\newcommand{\Tcal}{\mathcal{T}}
\newcommand{\Ucal}{\mathcal{U}}
\newcommand{\Vcal}{\mathcal{V}}
\newcommand{\Wcal}{\mathcal{W}}
\newcommand{\Xcal}{\mathcal{X}}
\newcommand{\Ycal}{\mathcal{Y}}
\newcommand{\Zcal}{\mathcal{Z}}

% abstract categories
\newcommand{\Asf}{\mathsf{A}}
\newcommand{\Bsf}{\mathsf{B}}
\newcommand{\Csf}{\mathsf{C}}
\newcommand{\Dsf}{\mathsf{D}}
\newcommand{\Esf}{\mathsf{E}}
\newcommand{\Ssf}{\mathsf{S}}

% algebraic geometry
\newcommand{\spec}{\operatorname{Spec}}
\newcommand{\proj}{\operatorname{Proj}}

% categories 
\newcommand{\id}{\mathrm{id}}
\newcommand{\Obj}{\mathrm{Obj}}
\newcommand{\Mor}{\mathrm{Mor}}
\newcommand{\Hom}{\mathrm{Hom}}
\newcommand{\Ext}{\mathrm{Ext}}
\newcommand{\Aut}{\mathrm{Aut}}
\newcommand{\Sets}{\mathsf{Sets}}
\newcommand{\SSets}{\mathsf{SSets}}
\newcommand{\kVect}{\mathsf{Vect}_{k}}
\newcommand{\Vect}{\mathsf{Vect}}
\newcommand{\Alg}{\mathsf{Alg}}
\newcommand{\Ring}{\mathsf{Ring}}
\newcommand{\Mod}{\mathsf{Mod}}
\newcommand{\Grp}{\mathsf{Grp}}
\newcommand{\AbGrp}{\mathsf{AbGrp}}
\newcommand{\PSh}{\mathsf{PSh}}
\newcommand{\Sh}{\mathsf{Sh}}
\newcommand{\PSch}{\mathsf{PSch}}
\newcommand{\Sch}{\mathsf{Sch}}
\newcommand{\Top}{\mathsf{Top}}
\newcommand{\Com}{\mathsf{Com}}
\newcommand{\Coh}{\mathsf{Coh}}
\newcommand{\QCoh}{\mathsf{QCoh}}
\newcommand{\Opens}{\mathsf{Opens}}
\newcommand{\Opp}{\mathsf{Opp}}
\newcommand{\Cat}{\mathsf{Cat}}
\newcommand{\NatTrans}{\mathrm{NatTrans}}
\newcommand{\pr}{\mathrm{pr}}
\newcommand{\Fun}{\mathrm{Fun}}
\newcommand{\colim}{\mathrm{colim}}
\newcommand{\lifts}{\boxslash}
\DeclareMathOperator\squarediv{\lifts}
\newcommand{\Kan}{\mathsf{Kan}}
\newcommand{\Path}{\mathsf{Path}}
\newcommand{\SPSh}{\mathsf{SPSh}}
\newcommand{\SSh}{\mathsf{SSh}}
\newcommand{\Bord}{\mathsf{Bord}}

% simplicial sets
\newcommand{\DDelta}{\Updelta}
\newcommand{\Sing}{\operatorname{Sing}}

% ideal theory
\newcommand{\mfrak}{\mathfrak{m}}
\newcommand{\afrak}{\mathfrak{a}}
\newcommand{\bfrak}{\mathfrak{b}}
\newcommand{\pfrak}{\mathfrak{p}}
\newcommand{\qfrak}{\mathfrak{q}}

% number theory
\newcommand{\Tr}{\mathrm{Tr}}
\newcommand{\Nm}{\mathrm{Nm}}
\newcommand{\Gal}{\mathrm{Gal}}
\newcommand{\Frob}{\mathrm{Frob}}

\newcommand{\SL}{\mathrm{SL}}
\newcommand{\GL}{\mathrm{GL}}
\newcommand{\Li}{\mathrm{Li}}
\newcommand{\sfPic}{\mathsf{Pic}}
\newcommand{\img}{\mathrm{Im}}
\newcommand{\Reg}{\mathrm{Reg}}
\newcommand{\Dscr}{\EuScript{D}}
\newcommand{\Ani}{\mathsf{Ani}}
\newcommand{\Proj}{\mathsf{Proj}}
\newcommand{\Free}{\mathsf{Free}}
\newcommand{\CMon}{\mathsf{CMon}}
\newcommand{\cond}{\mathsf{cond}}
\newcommand{\cont}{\mathsf{cont}}
\newcommand{\Liq}{\mathsf{Liq}}
\newcommand{\Gas}{\mathsf{Gas}}
\newcommand{\ku}{\mathsf{ku}}
\newcommand{\KU}{\mathsf{KU}}
\newcommand{\SSS}{\mathbb{S}}
\newcommand{\univ}{\mathrm{univ}}
\newcommand{\DMod}{\mathsf{DMod}}
\newcommand{\FGauge}{\mathsf{FGauge}}
\newcommand{\prism}{\mathbbl{\Delta}}
\newcommand{\Hab}{\mathsf{Hab}}
\newcommand{\Hdg}{\mathsf{Hdg}}
\newcommand{\HHdg}{\mathcal{H}\mathsf{dg}}
\newcommand{\dash}{\text{-}}
\newcommand{\gh}{\mathrm{gh}}
\newcommand{\CAlg}{\mathsf{CAlg}}
\newcommand{\Fil}{\mathrm{Fil}}
\setlength{\headheight}{40pt}


\newenvironment{solution}
  {\renewcommand\qedsymbol{$\blacksquare$}
  \begin{proof}[Solution]}
  {\end{proof}}
\renewcommand\qedsymbol{$\blacksquare$}

\usepackage{amsmath, amssymb, tikz, amsthm, csquotes, multicol, footnote, tablefootnote, biblatex, wrapfig, float, quiver, mathrsfs, cleveref, enumitem, upgreek, stmaryrd, marginnote, todonotes}
\addbibresource{refs.bib}
\theoremstyle{definition}
\newtheorem{theorem}{Theorem}[section]
\newtheorem{lemma}[theorem]{Lemma}
\newtheorem{corollary}[theorem]{Corollary}
\newtheorem{exercise}[theorem]{Exercise}
\newtheorem{question}[theorem]{Question}
\newtheorem{example}[theorem]{Example}
\newtheorem{proposition}[theorem]{Proposition}
\newtheorem{conjecture}[theorem]{Conjecture}
\newtheorem{remark}[theorem]{Remark}
\newtheorem{definition}[theorem]{Definition}
\numberwithin{equation}{section}
\setuptodonotes{color=blue!20, size=tiny}
\begin{document}
\large
\title[Algebraic Geometry II -- Bonn, Summer 2025]{V4A2 -- Algebraic Geometry II \\ Summer Semester 2025}
\author{Wern Juin Gabriel Ong}
\address{Universit\"{a}t Bonn, Bonn, D-53113}
\email{wgabrielong@uni-bonn.de}
\urladdr{https://wgabrielong.github.io/}
\maketitle
\section*{Preliminaries}
These notes roughly correspond to the course \textbf{V4A2 -- Algebraic Geometry II} taught by Prof. Daniel Huybrechts at the Universit\"{a}t Bonn in the Summer 2025 semester. These notes are \LaTeX-ed after the fact with significant alteration and are subject to misinterpretation and mistranscription. Use with caution. Any errors are undoubtedly my own and any virtues that could be ascribed to these notes ought be attributed to the instructor and not the typist. Knowledge of commutative algebra, topology, and category theory will be assumed. 
\newpage
\tableofcontents
\section{Lecture 1 -- 7th April 2025}\label{sec: lectuer 1}
We begin by a consideration of the theory of smoothness, first in the local case. This is done by defining the sheaves of K\"{a}hler differentials on schemes -- in the local picture, the module of differentials on a ring. 
\begin{definition}[Derivation]\label{def: derivation}
    Let $B$ be an $A$-algebra and $M$ a $B$-module. An morphism of $A$-modules $D:B\to M$ is an $A$-derivation if it satisfies the Leibniz rule $\dform(xy)=x\dform(y)+y\dform(x)$ for all $x,y\in B$. Denote the set of $A$-derivations in $M$ by $\mathrm{Der}_{A}(B,M)$. 
\end{definition}
\begin{remark}
    It is necessary that $M$ is a $B$-module, since the Leibniz rule involves elements of $B$. 
\end{remark}
\begin{remark}\label{rmk: map from A is zero}
    Observe that the composition $A\to B\to M$is zero since $a=a\cdot 1_{B}$ and computing we get $\dform(a\cdot 1_{B})=a\dform(1_{B})$ by $A$-linearity, but on the other hand $\dform(a\cdot 1_{B})=a\dform(1_{B})+1_{B}\dform(a)$ by the Leibniz rule, so $a\dform(1_{B})=0$ showing $\dform(1_{B})=0$ and thus $\dform(a)=0$. 
\end{remark}
The K\"{a}hler differentials of a ring map is the universal recipient of an $A$-algebra $B$ in the following sense. 
\begin{definition}[Module of K\"{a}hler Differentials]
    Let $B$ be an $A$-algebra. The module of K\"{a}hler differentials of $B$ over $A$ is a $B$-module $\Omega^{1}_{B/A}$ with an $A$-derivation $\dform:B\to\Omega_{B/A}^{1}$ that is initial amongst $B$-modules recieving an $A$-derivation from $B$. 
\end{definition}
Unwinding the universal property, if $M$ is a $B$-module recieving an $A$-derivation from $B$ by $f:B\to M$, there is a unique factorization over $\Omega^{1}_{B/A}$ as follows.
$$% https://q.uiver.app/#q=WzAsMyxbMCwxLCJCIl0sWzIsMCwiTSJdLFswLDAsIlxcT21lZ2FeezF9X3tCL0F9Il0sWzIsMV0sWzAsMiwiXFxleGlzdHMhIiwwLHsic3R5bGUiOnsiYm9keSI6eyJuYW1lIjoiZGFzaGVkIn19fV0sWzAsMV1d
\begin{tikzcd}
	{\Omega^{1}_{B/A}} && M \\
	B
	\arrow[from=1-1, to=1-3]
	\arrow["{\exists!}", dashed, from=2-1, to=1-1]
	\arrow[from=2-1, to=1-3]
\end{tikzcd}$$
In particular, there is a bijection $\mathrm{Der}_{A}(B,M)\leftrightarrow\Hom_{\Mod_{B}}(\Omega_{B/A},M)$ functorial in $M$. 
\begin{proposition}\label{prop: uniqueness of differentials}
    Let $B$ be an $A$-algebra. The $B$-module $\Omega^{1}_{B/A}$ and the $A$-derivation $\dform:B\to\Omega^{1}_{B/A}$ exist and are unique up to unique isomorphism. 
\end{proposition}
\begin{proof}
    The module $\Omega^{1}_{B/A}$ can be constructed as the free $B$-module on elements $\dform x$ for $x\in B$ modulo the relations generated by the Leibniz rule and $\dform a=0$ for $a\in A$. Uniqueness up to unique isomorphism is clear from the universal property and Yoneda's lemma. 
\end{proof}
In special cases, the module of K\"{a}hler differentials can be described explicitly. 
\begin{example}
    Let $A=k,B=k[x_{1},\dots,x_{n}]$. $\Omega_{B/A}^{1}$ is a free module of rank $n$ with basis $\dform x_{i}$. The map $f\mapsto\sum_{i=1}^{n}\frac{\partial f}{\partial x_{i}}\cdot \dform x_{i}$ is an $A$-derivation and the map $\dform x_{i}\mapsto x_{i}$ defines an isomorphism $\Omega_{B/A}^{1}\to B^{\oplus n}$. 
\end{example}
K\"{a}hler differentials are also fairly easy to understand in the case of ring localizations and ring quotients. These will be important in understanding the sheaves of K\"{a}hler differentials of open and closed immersions in the case of schemes, respectively. 
\begin{proposition}\label{prop: differentials of open and closed immersions}
    Let $A$ be a ring. 
    \begin{enumerate}[label=(\roman*)]
        \item If $B=S^{-1}A$, then $\Omega^{1}_{B/A}=0$. 
        \item If $B=A/I$ for $I\subseteq A$ an ideal, then $\Omega^{1}_{B/A}=0$. 
    \end{enumerate}
\end{proposition}
\begin{proof}[Proof of (i)]
    We already have that $\dform a=0$ for all $a\in A$. We then observe that writing $a=s\cdot\frac{a}{s}$ we have 
    \begin{align*}
        0=\dform(a)=\dform\left(s\cdot\frac{a}{s}\right) &= s\dform\left(\frac{a}{s}\right)+\frac{a}{s}\dform(s) \\
        &= s\dform\left(\frac{a}{s}\right) && s\in A\Rightarrow \dform s=0 
    \end{align*}
    so $s\dform(\frac{a}{s})=0$ and $\dform(\frac{a}{s})=0$ whence the claim. 
\end{proof}
\begin{proof}[Proof of (ii)]
    The map $A\to B$ is surjective, so this is precisely the situation \Cref{rmk: map from A is zero}. 
\end{proof}
We can additionally understand sheaves of K\"{a}hler differentials in towers. Let $A\to B\to C$ be maps of rings. There is a natural $C$-linear map $\Omega^{1}_{B/A}\otimes_{B}C\to\Omega_{C/A}^{1}$ which is a $C$-module homomorphism induced by the diagram 
$$% https://q.uiver.app/#q=WzAsNCxbMCwwLCJCIl0sWzIsMCwiQyJdLFs0LDAsIlxcT21lZ2Ffe0MvQX1eezF9Il0sWzAsMSwiXFxPbWVnYV97Qi9BfV57MX0iXSxbMSwyLCJcXGRmb3JtX3tDL0F9Il0sWzAsMV0sWzAsMywiXFxkZm9ybV97Qi9BfSIsMl0sWzMsMiwiXFxleGlzdHMhIiwyLHsic3R5bGUiOnsiYm9keSI6eyJuYW1lIjoiZGFzaGVkIn19fV1d
\begin{tikzcd}
	B && C && {\Omega_{C/A}^{1}} \\
	{\Omega_{B/A}^{1}}
	\arrow[from=1-1, to=1-3]
	\arrow["{\dform_{B/A}}"', from=1-1, to=2-1]
	\arrow["{\dform_{C/A}}", from=1-3, to=1-5]
	\arrow["{\exists!}"', dashed, from=2-1, to=1-5]
\end{tikzcd}$$
where the top row is both $A$ and $B$-linear inducing a unique $B$-module map $\Omega_{B/A}^{1}\to\Omega_{C/A}^{1}$, considering the latter as a $B$-module. By the extension-restriction adjunction, however, we have 
$$\Hom_{\Mod_{B}}(\Omega_{B/A},\Omega_{C/A}|_{B})\leftrightarrow\Hom_{\Mod_{C}}(\Omega_{B/A}\otimes_{B}C,\Omega_{C/A})$$
hence the data of the dotted map in the diagram above gives rise to a unique map $\Omega_{B/A}\otimes_{B}C\to\Omega_{C/A}$. Arguing similarly, there is a $C$-linear map $\Omega^{1}_{C/A}\to\Omega_{C/B}^{1}$ induced by 
$$% https://q.uiver.app/#q=WzAsMyxbMCwwLCJDIl0sWzIsMCwiXFxPbWVnYV97Qy9CfV57MX0iXSxbMCwxLCJcXE9tZWdhX3tDL0F9XnsxfSJdLFswLDEsIlxcZGZvcm1fe0MvQn0iXSxbMCwyLCJcXGRmb3JtX3tDL0F9IiwyXSxbMiwxLCJcXGV4aXN0cyEiLDIseyJzdHlsZSI6eyJib2R5Ijp7Im5hbWUiOiJkYXNoZWQifX19XV0=
\begin{tikzcd}
	C && {\Omega_{C/B}^{1}} \\
	{\Omega_{C/A}^{1}}
	\arrow["{\dform_{C/B}}", from=1-1, to=1-3]
	\arrow["{\dform_{C/A}}"', from=1-1, to=2-1]
	\arrow["{\exists!}"', dashed, from=2-1, to=1-3]
\end{tikzcd}$$
where the map is induced by the universal property as any $B$-derivation is also an $A$-derivation. 

The maps in the preceding discussion assemble to give the following proposition. 
\begin{proposition}\label{prop: tensor exact sequence}
    Let $A\to B\to C$ be maps of rings. There is an exact sequence 
    $$\Omega_{B/A}^{1}\otimes_{B}C\to \Omega_{C/A}^{1}\to\Omega_{C/B}^{1}\to0.$$
\end{proposition}
\begin{proof}
    The above discussion gives the existence of such maps, so it remains to show exactness at $\Omega^{1}_{C/A}$ and surjectivity of the map $\Omega^{1}_{C/A}\to\Omega^{1}_{C/B}$. 
    
    We begin with the latter, where by the quotient construction of \Cref{prop: uniqueness of differentials} it suffices to observe that $\Omega^{1}_{C/B}$ is a quotient of $\Omega^{1}_{C/A}$. 

    For the former, we note that for a fixed $C$-module $M$ we have an exact sequence 
    $$0\to\mathrm{Der}_{B}(C,M)\to\mathrm{Der}_{A}(C,M)\to\mathrm{Der}_{A}(B,M|_{B})$$
    since an $A$-derivation $(\dform:C\to M)$ is taken to the composite $B\to C\to M$ which is zero when the map is also a $B$-derivation. Rewriting this using the universal property, this is 
    $$0\to\Hom_{\Mod_{C}}(\Omega^{1}_{C/B},M)\to\Hom_{\Mod_{C}}(\Omega_{C/A}^{1},M)\to\Hom_{\Mod_{C}}(\Omega_{B/A}^{1}\otimes_{B}C,M)$$
    which by contravariant exatness of the Hom-functor (see \cite[\href{https://stacks.math.columbia.edu/tag/0582}{Tag 0582}]{stacks-project} for the precise statement), is the claim. 
\end{proof}
As a corollary, we can deduce the following fact about localizations. 
\begin{corollary}\label{corr: localization}
    Let $B$ be an $A$-algebra and $S$ a multiplicative subset of $B$. Then $S^{-1}\Omega_{B/A}^{1}\cong\Omega^{1}_{S^{-1}B/A}$. 
\end{corollary}
\begin{proof}
    Apply \Cref{prop: tensor exact sequence} to $C=S^{-1}B$ and note that $\Omega^{1}_{C/B}=0$ so the map $S^{-1}\Omega^{1}_{B/A}\to\Omega^{1}_{S^{-1}B/A}$ is surjective. To prove injectivity, we produce an inverse map which is an $A$-derivation of $S^{-1}B$ to $S^{-1}\Omega_{B/A}^{1}$ by $\dform(\frac{b}{s})\mapsto \frac{1}{s}\dform(b)-\frac{1}{s^{2}}b\dform(s)$ which by the universal property can be seen to be the inverse. 
\end{proof}
Note that in general $\Omega^{1}_{B/A}\otimes_{B}C\to\Omega^{1}_{C/A}$ is rarely injective. 
\begin{example}
    Let $A=k, B=k[x], C=k[x]/(x)$. So $\Omega_{B/A}^{1}\cong B\dform x$ but $\Omega_{C/A}=\Omega_{k/k}=0$. 
\end{example}
On the other hand, there are situations in which the exact sequence of \Cref{prop: tensor exact sequence} extends to a short exact sequence. 
\begin{example}
    Let $B$ be an $A$-algebra and $C=B[x_{1},\dots,x_{n}]$. We then have a split short exact sequence 
    $$0\to\Omega^{1}_{B/A}\otimes_{B}C\to\Omega_{C/A}^{1}\to\Omega_{C/B}^{1}\to 0$$
    where denoting the map $\Omega_{C/A}^{1}\to\Omega_{C/B}^{1}$ by $\varphi$, we have the splitting $\Omega_{C/A}^{1}\to(\Omega_{B/A}^{1}\otimes_{B}C)\oplus\Omega_{C/B}^{1}$ prescribed by the $C$-derivation $f\mapsto \dform_{B/A}(f)+\varphi(f)$ under the bijection 
    $$\Hom_{\Mod_{C}}(\Omega_{C/A},(\Omega_{B/A}\otimes_{B}C)\oplus\Omega_{C/B})\leftrightarrow\mathrm{Der}_{A}(C,(\Omega_{B/A}\otimes_{B}C)\oplus\Omega_{C/B}).$$
\end{example}
The following proposition describes the behavior of the module of K\"{a}hler differentials with respect to tensor products. 
\begin{proposition}\label{prop: differentials of pushouts}
    Let $B,A'$ be $A$-algebras. Then there is an isomorphism of $B$-modules $\Omega_{B/A}^{1}\otimes_{B}(B\otimes_{A}A')\cong\Omega^{1}_{(B\otimes_{A}A')/A'}$. 
\end{proposition}
\begin{proof}
    We contemplate the diagram 
    $$% https://q.uiver.app/#q=WzAsMyxbMCwwLCJCXFxvdGltZXNfe0F9QSciXSxbMiwwLCJcXE9tZWdhXnsxfV97Qi9BfVxcb3RpbWVzX3tBfUEnIl0sWzAsMSwiXFxPbWVnYV97KEJcXG90aW1lc197QX1BJykvQSd9XnsxfSJdLFswLDJdLFsyLDEsIlxcZXhpc3RzISIsMix7InN0eWxlIjp7ImJvZHkiOnsibmFtZSI6ImRhc2hlZCJ9fX1dLFswLDFdXQ==
    \begin{tikzcd}
        {B\otimes_{A}A'} && {\Omega^{1}_{B/A}\otimes_{A}A'} \\
        {\Omega_{(B\otimes_{A}A')/A'}^{1}}
        \arrow[from=1-1, to=1-3]
        \arrow[from=1-1, to=2-1]
        \arrow["{\exists!}"', dashed, from=2-1, to=1-3]
    \end{tikzcd}$$
    where the solid arrows are $B\otimes_{A}A'$-linear with $\Omega^{1}_{B/A}\otimes_{A}A'\cong(\Omega^{1}_{B/A}\otimes_{B}(B\otimes_{A}A'))$ and the dotted arrow induced by the universal property of $\Omega^{1}_{(B\otimes_{A}A')/A'}$. By applying the tensor-hom adjunction and the universal property of derivations, prescribing an inverse map to the dotted arrow is equivalent to producing an $A$-derivation of $B$ in $\Omega^{1}_{(B\otimes_{A}A')/A'}$ and one observes that the map $b\mapsto \dform_{(B\otimes_{A}A')/A'}(b\otimes 1)$ gives an inverse, whence the claim. 
\end{proof}
We now treat the case of quotients. 
\begin{proposition}\label{prop: ideal exact sequence}
    Let $A\to B\to C$ be maps of rings where $C\cong B/\bfrak$ for some ideal $\bfrak\subseteq B$. There is an exact sequence 
    $$\bfrak/\bfrak^{2}\to\Omega_{B/A}^{1}\otimes_{B}C\to\Omega_{C/A}^{1}\to0.$$
\end{proposition}
\begin{proof}
    We first observe that $\Omega^{1}_{C/B}=0$ by \Cref{prop: differentials of open and closed immersions} and $\Omega^{1}_{B/A}\otimes_{B}C\cong\Omega_{B/A}^{1}/\bfrak\Omega^{1}_{B/A}$. 
    
    We denote $\bfrak/\bfrak^{2}\to\Omega^{1}_{B/A}\otimes_{B}C$ by $\delta$, $b\mapsto \dform b\otimes 1$. We first show $\delta$ is well-defined. For this, we want to show that $\dform(b_{1}b_{2})\otimes 1$ is zero for $b_{1},b_{2}\in\bfrak$. Indeed, using the Leibniz rule, we have 
    $$\dform(b_{1}b_{2})\otimes 1 = \dform(f_{2})\otimes f_{1}+\dform(f_{1})\otimes f_{2}\in\bfrak\Omega_{B/A}$$
    hence zero in the quotient, showing the map is well-defined. 
    
    The diagram is a complex as $db$ maps to zero in $\Omega^{1}_{C/A}$. The kernel of $\Omega^{1}_{B/A}\to\Omega^{1}_{C/A}$ is generated by the $B$-submodule $\bfrak\Omega_{B/A}^{1}$ and the elements $\dform b$ for $b\in\bfrak$, showing exactness of the complex in the middle. 
\end{proof}
This specializes to finite type algebras. 
\begin{corollary}\label{corr: kahler differentials are fg module}
    Let $C$ be a finite type $A$-algebra -- that is, the quotient of $B=A[x_{1},\dots,x_{n}]$. Then $\Omega^{1}_{C/A}$ is a finitely generated $C$-module. 
\end{corollary}
\begin{proof}
    Set $B=A[x_{1},\dots,x_{n}]$ for which $C=B/\bfrak$. Exactness of the sequence in \Cref{prop: ideal exact sequence} gives a surjection $\Omega_{B/A}^{1}\otimes_{B}C\to\Omega_{C/A}^{1}$, and observing that $\Omega^{1}_{B/A}\otimes_{B}C\cong B^{\oplus n}\otimes_{B}C\cong C^{\oplus n}$ gives a surjection $C^{\oplus n}\to\Omega_{C/A}^{1}$, showing that it is finitely generated. 
\end{proof}
Let us consider the case of quotients of multivariate polynomial rings by a single polynomial. 
\begin{example}
    Let $A$ be a ring, $B=A[x_{1},\dots,x_{n}]$, and $C=B/(f)$ for $f\in B$. By \Cref{prop: ideal exact sequence} and \Cref{corr: kahler differentials are fg module}, we have that $\Omega_{C/A}^{1}$ is the cokernel of the map $\delta:(f)/(f)^{2}\to\Omega_{B/A}^{1}\otimes_{B}C\cong C^{\oplus n}$ of \Cref{prop: ideal exact sequence}, so is the quotient $\left(\bigoplus_{i=1}^{n}C\dform x_{i}/\dform f\right)$. 
\end{example}
We can also consider the case of $k$-algebras. 
\begin{corollary}\label{corr: residue fields}
    Let $A$ be a $k$-algebra and $\mfrak$ a maximal ideal in $A$ such that $\kappa(\mfrak)=A/\mfrak\cong k$. Then $\Omega_{A/k}^{1}\otimes_{k}\kappa(\mfrak)\cong\mfrak/\mfrak^{2}$. 
\end{corollary}
\begin{proof}
    This is precisely \Cref{prop: ideal exact sequence} for $k\to k[x_{1},\dots,x_{n}]\to A$, and the map $\mfrak/\mfrak^{2}\to\Omega^{1}_{A/k}\otimes_{k}\kappa(\mfrak)$ is a surjection between vector spaces of the same dimension, hence an isomorphism. 
\end{proof}
Note that this is the dual of the Zariski tangent space $\Hom_{\mathsf{Vec}_{\kappa(\mfrak)}}(\mfrak/\mfrak^{2},\kappa(\mfrak))$, motivating the connection to schemes. 
\section{Lecture 2 -- 2nd May 2025}\label{sec: lecture 2}
The goal of this course is to develop a theory of Habiro cohomology, a functor that associates to a smooth $\ZZ$-scheme $X$ its Habiro cohomology -- a module over the Habiro ring, or more generally its ``category of constructible sheaves'' which in this case we tentatively denote $\Dscr_{\Hab}(X)$ of ``variations of Habiro structure.''

We begin with an exploration of what these structures are in terms of coordinates, and we will later show that the constructions we discuss are in fact independent of these coordinates. Let us make the notion of coordinates precise. 
\begin{definition}[Framed Algebra]\label{def: framed algebra}
    A framed algebra is a pair $(R,\square)$ where $R$ is a smooth $\ZZ$-algebra and a map $\square:\spec(R)\to\A^{d}_{\ZZ}$ or $\square:\spec(R)\to\GG_{m}^{d}$. 
\end{definition}
\begin{remark}
    It is often simpler to consider the case where the coordinates are invertible, that is, the case of $\GG_{m}^{d}$. 
\end{remark}
As a first pass, let us contemplate these constructions in the case where $X$ is affine and equal to either $\A^{d}_{\ZZ}$ or $\GG_{m}^{d}$ and only later consider the generalization to the case where $X$ is \'{e}tale over one of these spaces. Moreover, under these assumptions, we need not make any completions and one can work over $\ZZ[q^{\pm}]$.

Recall Habiro cohomology subsumes de Rham cohomology in an appropriate sense, and takes the $q$-derivative -- the Gaussian $q$-analogue of the derivative -- as an input. These $q$-derivatives were first investigated by Jackson \cite{Jackson}.
\begin{definition}[$q$-Derivative]\label{def: q-derivative}
    Let $R$ be $\ZZ[q^{\pm}][T_{1},\dots,T_{d}]$ or $\ZZ[q^{\pm}][T_{1}^{\pm},\dots,T_{d}^{\pm}]$. The $q$-derivative $\nabla_{i}^{q}:R\to R$ for $1\leq i\leq d$ is defined by 
    $$\nabla_{i}^{q}(f(T_{1},\dots,T_{d}))=\frac{f(T_{1},\dots,qT_{i},\dots,T_{d})-f(T_{1},\dots,T_{i},\dots,T_{d})}{qT_{i}-T_{i}}.$$
\end{definition}
\begin{remark}
    More explicitly, this operation is given on monomials by 
    $$\nabla_{i}^{q}(T_{1}^{n_{1}}\dots T_{d}^{n_{d}})=[n_{i}]_{q}\cdot T_{1}^{n_{1}}\dots T_{i}^{n_{i}-1}\dots T_{d}^{n_{d}}$$
    where $[n]_{q}=\frac{1-q^{n}}{1-q}$ is the Gaussian $q$-analogue of $n$. 
\end{remark}
\begin{remark}\label{rmk: gamma i maps}
    $\nabla_{i}^{q}$ is closely related $\gamma_{i}:R\to R$ the automorphism by 
    $$T_{j}\mapsto\begin{cases}
        T_{j} & j\neq i \\
        qT_{i} & j=i
    \end{cases}$$
    allowing us to write $\nabla_{i}^{q}(f)=\frac{\gamma_{i}(f)-f}{(q-1)T_{i}}$. 
\end{remark}
The $q$-derivative does not satisfy the Leibniz rule on the nose, but does so up to a twist by the automorphism $\gamma_{i}$ of \Cref{rmk: gamma i maps}. 
\begin{lemma}\label{lem: twisted q-leibniz}
    Let $R$ be $\ZZ[q^{\pm}][T_{1},\dots,T_{d}]$ or $\ZZ[q^{\pm}][T_{1}^{\pm},\dots,T_{d}^{\pm}]$. Then for $f,g\in R$ we have equalities 
    $$\nabla_{i}^{q}(fg)=\gamma_{i}(f)\cdot\nabla_{i}^{q}(g)+g\cdot\nabla_{i}^{q}(f)=f\cdot\nabla_{i}^{q}(g)+\gamma_{i}(g)\cdot\nabla^{i}_{q}(f).$$
\end{lemma}
\begin{proof}
    We first show the second equality. We use \Cref{rmk: gamma i maps} to observe that the latter two terms are given by 
    $$\gamma_{i}(f)\cdot\frac{\gamma_{i}(g)-g}{(q-1)T_{i}}+g\cdot\frac{\gamma_{i}(f)-f}{(q-1)T_{i}}=\frac{\gamma_{i}(f)\gamma_{i}(g)-\gamma_{i}(f)g+\gamma_{i}(f)g-fg}{(q-1)T_{i}}=\frac{\gamma_{i}(f)\gamma_{i}(g)-fg}{(q-1)T_{i}}$$
    and 
    $$f\cdot\frac{\gamma_{i}(g)-g}{(q-1)T_{i}}+\gamma_{i}(g)\frac{\gamma_{i}(f)-f}{(q-1)T_{i}}=\frac{\gamma_{i}(g)f-fg+\gamma_{i}(f)\gamma_{i}(g)-\gamma_{i}(g)f}{(q-1)T_{i}}=\frac{\gamma_{i}(f)\gamma_{i}(g)-fg}{(q-1)T_{i}}$$
    respectively, which are evidently equal. 

    We now show the first equality. Note that $\gamma_{i}$ is an automorphism $R\to R$, and in particular a homomorphism so $\gamma_{i}(fg)=\gamma_{i}(f)\gamma_{i}(g)$ in which case we have 
    $$\frac{\gamma_{i}(fg)-fg}{(q-1)T_{i}}=\frac{\gamma_{i}(f)\gamma_{i}(g)-fg}{(q-1)T_{i}}$$
    whence the claim. 
\end{proof}
We can now define the $q$-de Rham complex following Aomoto \cite{Aomoto}. 
\begin{definition}[$q$-de Rham Complex of $\A^{d}_{\ZZ}$ and $\GG_{m}^{d}$]\label{def: q-dR complex}
    Let $R=\ZZ[q^{\pm}][\underline{T}]$ be $\ZZ[q^{\pm}][T_{1},\dots,T_{d}]$ or $\ZZ[q^{\pm}][T_{1}^{\pm},\dots,T_{d}^{\pm}]$. The $q$-de Rham complex of $\spec(R)$ is the complex
    \begin{equation}\label{eqn: q-dR complex of R}
        \footnotesize 
        % https://q.uiver.app/#q=WzAsOCxbMCwwLCIwIl0sWzEsMCwiXFxaWltxXntcXHBtfV1bXFx1bmRlcmxpbmV7VH1dIl0sWzIsMCwiXFxaWltxXntcXHBtfV1bXFx1bmRlcmxpbmV7VH1dXntcXG9wbHVzIGR9Il0sWzMsMCwiXFxiaWdvcGx1c197aTxqfVxcWlpbcV57XFxwbX1dW1xcdW5kZXJsaW5le1R9XSJdLFs0LDAsIlxcZG90cyJdLFszLDEsIlxcZG90cyJdLFs0LDEsIlxcWlpbcV57XFxwbX1dW1xcdW5kZXJsaW5le1R9XSJdLFs1LDEsIjAiXSxbMCwxXSxbMSwyXSxbMiwzXSxbNSw2XSxbMyw0XSxbNiw3XV0=
        \begin{tikzcd}
            0 & {\ZZ[q^{\pm}][\underline{T}]} & {\ZZ[q^{\pm}][\underline{T}]^{\oplus d}} & {\bigoplus_{i<j}\ZZ[q^{\pm}][\underline{T}]} & \dots \\
            &&& \dots & {\ZZ[q^{\pm}][\underline{T}]} & 0
            \arrow[from=1-1, to=1-2]
            \arrow[from=1-2, to=1-3]
            \arrow[from=1-3, to=1-4]
            \arrow[from=1-4, to=1-5]
            \arrow[from=2-4, to=2-5]
            \arrow[from=2-5, to=2-6]
        \end{tikzcd}
        \normalsize
    \end{equation}
    with differentials given by the differentials for the Koszul complex of commuting operators $\nabla_{1}^{q},\dots,\nabla_{n}^{q}$. 
\end{definition}
\begin{remark}
    Recall that these are precisely the differentials for the classical de Rham complex. See \cite[\href{https://stacks.math.columbia.edu/tag/0FKF}{Tag 0FKF}]{stacks-project} for an explicit description via equations. 
\end{remark}
\begin{remark}
    Since the first differential $\ZZ[q^{\pm}][\underline{T}]\to\ZZ[q^{\pm}][\underline{T}]^{\oplus d}$ by $(\nabla_{1}^{q},\dots,\nabla_{d}^{q})$ does not satisfy the ordinary Leibniz rule, the complex (\ref{eqn: q-dR complex of R}) is not a differential graded algebra. Later, we will see that working in the derived ($\infty$-)category, one can endow this with the structure of a commutative ring.  
\end{remark}
The complex (\ref{eqn: q-dR complex of R}) computes $q$-de Rham cohomology, or Aomoto-Jackson cohomology of $\spec(R)$. But to compute Habiro cohomology, we use a closely related variant based on a modified $q$-derivative. 
\begin{definition}[Modified $q$-Derivative]\label{def: modified q-derivative}
    Let $R$ be $\ZZ[q^{\pm}][T_{1},\dots,T_{d}]$ or $\ZZ[q^{\pm}][T_{1}^{\pm},\dots,T_{d}^{\pm}]$. The modified $q$-derivative is given by 
    $$\widetilde{\nabla}_{i}^{q}(f(T_{1},\dots,T_{d}))=\frac{f(T_{1},\dots,qT_{i},\dots,T_{d})-f(T_{1},\dots,T_{i},\dots,T_{d})}{T_{i}}.$$
\end{definition}
\begin{remark}
    In other words, $\widetilde{\nabla}_{i}^{q}(f)=(q-1)\nabla_{i}^{q}(f)=\frac{\gamma_{i}(f)-f}{T_{i}}$.  
\end{remark}
Recomputing everything using this modified derivative gives the $q$-Hodge complex. 
\begin{definition}[$q$-Hodge Complex of $\A^{d}_{\ZZ}$ and $\GG_{m}^{d}$]\label{def: q-Hodge complex}
    Let $R=\ZZ[q^{\pm}][\underline{T}]$ be $\ZZ[q^{\pm}][T_{1},\dots,T_{d}]$ or $\ZZ[q^{\pm}][T_{1}^{\pm},\dots,T_{d}^{\pm}]$. The $q$-Hodge complex of $\spec(R)$ is the complex
    \begin{equation}\label{eqn: q-Hodge complex of R}
        \footnotesize 
        % https://q.uiver.app/#q=WzAsOCxbMCwwLCIwIl0sWzEsMCwiXFxaWltxXntcXHBtfV1bXFx1bmRlcmxpbmV7VH1dIl0sWzIsMCwiXFxaWltxXntcXHBtfV1bXFx1bmRlcmxpbmV7VH1dXntcXG9wbHVzIGR9Il0sWzMsMCwiXFxiaWdvcGx1c197aTxqfVxcWlpbcV57XFxwbX1dW1xcdW5kZXJsaW5le1R9XSJdLFs0LDAsIlxcZG90cyJdLFszLDEsIlxcZG90cyJdLFs0LDEsIlxcWlpbcV57XFxwbX1dW1xcdW5kZXJsaW5le1R9XSJdLFs1LDEsIjAiXSxbMCwxXSxbMSwyXSxbMiwzXSxbNSw2XSxbMyw0XSxbNiw3XV0=
        \begin{tikzcd}
            0 & {\ZZ[q^{\pm}][\underline{T}]} & {\ZZ[q^{\pm}][\underline{T}]^{\oplus d}} & {\bigoplus_{i<j}\ZZ[q^{\pm}][\underline{T}]} & \dots \\
            &&& \dots & {\ZZ[q^{\pm}][\underline{T}]} & 0
            \arrow[from=1-1, to=1-2]
            \arrow[from=1-2, to=1-3]
            \arrow[from=1-3, to=1-4]
            \arrow[from=1-4, to=1-5]
            \arrow[from=2-4, to=2-5]
            \arrow[from=2-5, to=2-6]
        \end{tikzcd}
        \normalsize
    \end{equation}
    with differentials given by the differentials for the Koszul complex of commuting operators $\widetilde{\nabla}_{1}^{1},\dots,\widetilde{\nabla}_{d}^{q}$. 
\end{definition}
\begin{remark}
    The nomenclature of \Cref{def: q-dR complex,def: q-Hodge complex} are justified by the fact that they recover the ordinary de Rham and Hodge complexes at $q=1$.
\end{remark} 
\begin{remark}
    An automorphism of $\A^{d}_{\ZZ}$ or $\GG_{m}^{d}$ would give rise to an automorphism of the complexes (\ref{eqn: q-dR complex of R}) and (\ref{eqn: q-Hodge complex of R}), at least as an object in the derived category, but it is extremely difficult to understand these automorphisms from this explicit perspective. 
\end{remark}\marginpar{The instructor remarks that he does not believe in non-flat connections. We will henceforth omit the adjective ``flat.'' \\\\ Note that a $q$-connection is additional data on a module.}
The classical correspondence between $D$-modules and modules with flat connection suggest that an appropriate category of modules with connection could play the role of $\Dscr_{\Hab}(X)$ alluded to earlier. To make this precise, we consider modules with $q$-connection. To simplify matters, we make these considerations on the Abelian and not $\infty$-categorical level. 
\begin{definition}[$q$-Connections on Modules]\label{def: q-connections on modules}
    Let $R=\ZZ[q^{\pm}][\underline{T}]$ be $\ZZ[q^{\pm}][T_{1},\dots,T_{d}]$ or $\ZZ[q^{\pm}][T_{1}^{\pm},\dots,T_{d}^{\pm}]$. A module with (flat) $q$-connection is a $\ZZ[q^{\pm}][\underline{T}]$-module with commuting $\ZZ[q^{\pm}]$-linear operations $\nabla_{i,M}^{q}:M\to M$ which satisfy the $q$-Leibniz rule 
    $$\nabla_{i,M}^{q}(fm)=\gamma_{i}(f)\cdot\nabla_{i,M}^{q}(m)+\nabla_{i}^{q}(f)\cdot m$$
    for $f\in \ZZ[q^{\pm}][\underline{T}]$ and $m\in M$.
\end{definition}
\begin{remark}
    To unwind any possible confusion between the similar-looking $\nabla_{i}^{q}:\ZZ[q^{\pm}][\underline{T}]\to\ZZ[q^{\pm}][\underline{T}],\nabla_{i,M}^{q}:M\to M$, we have 
    $$\underbrace{\underbrace{\gamma_{i}(f)}_{\in \ZZ[q^{\pm}][\underline{T}]}\cdot\underbrace{\nabla_{i,M}^{q}(m)}_{\in M}}_{\in M}+\underbrace{\underbrace{\nabla_{i}^{q}(f)}_{\in \ZZ[q^{\pm}][\underline{T}]}\cdot\underbrace{m}_{\in M}}_{\in M}$$
    so everything type-checks. 
\end{remark}
\begin{example}\label{ex: A1 with Weyl algebra}
    If $X=\A^{1}_{\ZZ}$ then recall that modules with connection are equivalent to modules over the Weyl algebra $\ZZ[q^{\pm}]\{T,\partial_{q}\}/(qT\partial_{q}-\partial_{q}T+1)$ since we have the operators $T\partial_{q},\partial_{q}T$ take $T^{n}$ to $q[n]_{q}T^{n},[n+1]_{q}T^{n}$, respectively, but $q[n]_{q}-[n+1]_{q}=q\cdot\frac{1-q^{n}}{1-q}-\frac{1-q^{n+1}}{1-q}=-1$. Passing to the associated-graded of the degree filtration, one gets commuting variables with the correct $q$-twists. 
\end{example}
Similarly, we can construct modules with a modified $q$-connection. 
\begin{definition}[Modified $q$-Connections on Modules]\label{def: modified q-connections on modules}
    Let $R=\ZZ[q^{\pm}][\underline{T}]$ be $\ZZ[q^{\pm}][T_{1},\dots,T_{d}]$ or $\ZZ[q^{\pm}][T_{1}^{\pm},\dots,T_{d}^{\pm}]$. A module with modified $q$-connection is a $\ZZ[q^{\pm}][\underline{T}]$-module with commuting $\ZZ[q^{\pm}]$-linear operations $\widetilde{\nabla}_{i,M}^{q}:M\to M$ which satisfy the $q$-Leibniz rule 
    $$\widetilde{\nabla}_{i,M}^{q}(fm)=\gamma_{i}(f)\cdot\widetilde{\nabla}_{i,M}^{q}(m)+\widetilde{\nabla}_{i}^{q}(f)\cdot m$$
    for $f\in \ZZ[q^{\pm}][\underline{T}]$ and $m\in M$.
\end{definition}
\begin{remark}\label{rmk: invertible case}
    Let $T_{i}$ be invertible. Unwinding the definition of the modified $q$-derivative, we have
    $$\widetilde{\nabla}_{i,M}^{q}(fm)=\gamma_{i}(f)\cdot\widetilde{\nabla}_{i,M}^{q}(m)+(q-1)\nabla_{i}^{q}(f)\cdot m$$
    where in particular we observe that the second summand has denominator $T_{i}$. Define a new operator 
    $$\widetilde{\widetilde{\nabla}}_{i,M}^{q}=T_{i}\cdot\widetilde{\nabla}_{i,M}^{q}$$
    which satisfies 
    \begin{align*}
        \widetilde{\widetilde{\nabla}}_{i,M}^{q}(fm) &= \gamma_{i}(f)\cdot\widetilde{\widetilde{\nabla}}_{i,M}^{q}(m)+(\gamma_{i}(f)-f)m \\
        &= \gamma_{i}(f)\left(\widetilde{\widetilde{\nabla}}_{i,M}^{q}(m)+m\right)-fm.
    \end{align*}
    In particular, 
    $$\left(\widetilde{\widetilde{\nabla}}_{i,M}^{q}+\id_{M}\right)(fm) = \gamma_{i}(f)\left(\widetilde{\widetilde{\nabla}}_{i,M}^{q}+\id_{M}\right)(m)$$
    so denoting $\gamma_{i,M}=\left(\widetilde{\widetilde{\nabla}}_{i,M}^{q}+\id_{M}\right)$, we have $\gamma_{i,M}(fm)=\gamma_{i}(f)\gamma_{i,M}(m)$ simplyfing the relation. 
\end{remark}
The preceding discussion of \Cref{rmk: invertible case} implies the following. 
\begin{corollary}\label{corr: equivalence of q-connection modules and semilinear}
    Let $R=\ZZ[q^{\pm}][T_{1}^{\pm},\dots,T_{d}^{\pm}]$. There is an equivalence of categories between $R$-modules with modified $q$-connection and $R$-modules with commuting $\gamma_{i}:R\to R$-semilinear endomorphisms $\gamma_{i,M}:M\to M$. 
\end{corollary}
Note that for $R=\ZZ[q^{\pm}][\underline{T}]$, $(-)\otimes_{R}(-)$ does not define a symmetric monoidal structure on the category of modules with $q$-connection: for $(M,\nabla_{i,M}^{q}),(N,\nabla_{i,N}^{q})$ two modules with $q$-connecition, $$(M\otimes_{R} N,\nabla_{i,M}^{q}\otimes_{R}\id_{N}+\id_{M}\otimes_{R}\nabla_{i,N}^{q})$$ is not a module with $q$-connection. One needs instead to take the twist $$(M\otimes_{R} N,\nabla_{i,M}^{q}\otimes_{R}\id_{N}+\gamma_{i,M}\otimes_{R}\nabla_{i,N}^{q}),$$
defining $\gamma_{i,M}:M\to M$ in an analogous way to \Cref{rmk: invertible case}. While \emph{a priori} appearing assymetric in $M,N$, there is in fact a canonical isomorphism between them. 
\begin{proposition}
    Let $R=\ZZ[q^{\pm}][T_{1}^{\pm},\dots,T_{d}^{\pm}]$. The category of $R$-modules with $q$-connection is symmetric monoidal. 
\end{proposition}
\begin{proof}[Proof Outline]
    Using the equivalence of \Cref{corr: equivalence of q-connection modules and semilinear}, the latter category is symmetric monoidal, hence the former can be promoted to a symmetric monoidal category. 
\end{proof}
\begin{proposition}
    Let $R=\ZZ[q^{\pm}][T_{1}^{\pm},\dots,T_{d}^{\pm}]$. There is a fully faithful embedding from $(q-1)$-torsion free $R$-modules with $q$-connection and $R$-modules with modified $q$-connection by $(M,\nabla_{i,M}^{q})\mapsto(M,\widetilde{\nabla}_{i,M}^{q})$ with essential image those that are $(q-1)$-torsion free and such that $\widetilde{\nabla}_{i,M}^{q}\equiv0\pmod{(q-1)}$. 
\end{proposition}
The discussion thus far has been done entirely in terms of coordinates. This prompts:
\begin{question}
    To what extent are the cohomologies and categories discussed thus far independent of coordinates? 
\end{question}
Let us consider the following example. 
\begin{example}\marginpar{The instructor remarks that in the theory of analytic geometry the quotient would be the Tate elliptic curve for $d=1$. See \cite{AnalyticStacks}.}
    Let $X=\GG_{m}^{d}$. The modules with modified $q$-connection are quasicoherent sheaves on $(\GG_{m}/q^{\ZZ})^{d}$ -- the $\gamma_{i}$'s act by multiplication by $q$ on the coordinates so the data of the endomorphisms $\gamma_{i,M}$ on the modules prescribe descent data to the quotient stack (ie. as an fpqc quotient).
\end{example}
Let us relate the discussion of complexes \Cref{def: q-dR complex,def: q-Hodge complex}, their cohomologies, and these categories of modules with (modified) $q$-connections. 
\begin{proposition}
    \begin{enumerate}[label=(\roman*)]
        \item The $q$ de Rham complex computes $R\Hom(\mathbbm{1},\mathbbm{1})$ in the derived category of modules with $q$-connection on $\spec(R)$. 
        \item The $q$-Hodge complex computes $R\Hom(\mathbbm{1},\mathbbm{1})$ in the derived category of modules with modified $q$-connection on $\spec(R)$. 
    \end{enumerate}
\end{proposition}
\begin{proof}[Proof Outline of (i)]
    Using the equvialence between modules with $q$-connection and modules over the Weyl algebra, we compute a resolution of the symmetric monoidal unit $\ZZ[q^{\pm}][\underline{T}]$ in the category of modules over the Weyl algebra -- which precisely recovers the de Rham complex, whence the claim. 
\end{proof}
\begin{example}
    Consider the case $\ZZ[q^{\pm}][T]$. We compute $R\Hom(\ZZ[q^{\pm}][T],\ZZ[q^{\pm}][T])$ as $R\Hom(-,\ZZ[q^{\pm}][T])$ of a free resolution of $\ZZ[q^{\pm}][T]$ in the category of modules over the Weyl algebra $\ZZ[q^{\pm}]\{T,\partial_{q}\}/(qT\partial_{q}-\partial_{q}T+1)$ (vis. \Cref{ex: A1 with Weyl algebra}). This produces 
    $$0\to\ZZ[q^{\pm}][T]\xrightarrow{\nabla_{1}^{q}}\ZZ[q^{\pm}][T]\to0$$
    which is the $q$-de Rham complex (after passing back to modules with $q$-connection along the equivalence). 
\end{example}
Moreover, in the setting of higher algebra, these promote canonically to commutative algebra objects. 
\begin{corollary}
    The $q$-de Rham complex and $q$-Hodge complex have canonical structures as $\EE_{\infty}$-rings. 
\end{corollary}
\section{Lecture 3 -- 14th April 2025}\label{sec: lecture 3}
Recall that the cotangent sheaf on $\A^{n}_{A},\PP^{n}_{A}$ over $\spec(A)$ are locally free sheaves by \Cref{ex: affine space scheme differentials,ex: projective space scheme differentials}. One is then led to consider for what other $S$-schemes $X$ is $\Omega^{1}_{X/S}$ locally free. This is roughly captured by smoothness. Moreover, as suggested by \Cref{prop: isomorphism to Zariski tangent space}, the notion of smoothness is connected to the Zariski tangent space, which in the case of algebraic geometry -- unlike differential geometry -- need not coincide with the geometric tangent space, especially in characteristic $p$ situations. 

We recall the definition of the Zariski tangent space. 
\begin{definition}[Zariski Tangent Space of a Ring]\label{def: Zariski tangent space}
    Let $A$ be a local ring with maximal ideal $\mfrak$ and residue field $\kappa=A/\mfrak$. The Zariski tangent space of $A$ is the $\kappa$-vector space $(\frac{\mfrak}{\mfrak^{2}})^{\vee}=\Hom_{\Vect_{\kappa}}(\frac{\mfrak}{\mfrak^{2}},\kappa)$. 
\end{definition}
\begin{definition}[Zariski Tangent Space of a Scheme]\label{def: Zariski tangent space scheme}
    Let $X$ be a scheme and $x\in X$ a point. The Zariski tangent space $T_{X,x}$ is the Zariski tangent space of the local ring $\Ocal_{X,x}$. 
\end{definition}
\begin{example}\label{ex: spec A example}
    Let $A$ be a ring and $\pfrak\subseteq\spec(A)$. The Zariski tangent space $T_{\spec(A),\pfrak}$ is given by $(\frac{\pfrak A_{\pfrak}}{(\pfrak A_{\pfrak})^{2}})^{\vee}$. This is a vector space over the field $\kappa(\pfrak)=A_{\pfrak}/\pfrak A_{\pfrak}$. 
\end{example}
\begin{example}\label{ex: map to hom}
    Let $k$ be a field and $x\in \A^{n}_{k}(k)$ a closed $k$-rational point hence of the form $(x_{1}-a_{1},\dots,x_{n}-a_{n})$. Define a map $D_{x}:k[x_{1},\dots,x_{n}]\to\Hom_{\Vect_{k}}(k^{n},k)$ by $$f\mapsto\left[(\alpha_{i})_{i=1}^{n}\mapsto\sum_{i=1}^{n}\alpha_{i}\frac{\partial f}{\partial x_{i}}(x)\right].$$
    The map is $k$-linear and satisfies the Leibniz rule, hence defines a $k$-linear derivation which is a $k$-vector space. This defines an isomorphism between $(\frac{\mfrak}{\mfrak^{2}})^{2}$ and $\Hom_{\Vect_{k}}(k^{n},k)$ by considering the Taylor expansion of a polynomial 
    $$f=f(x)+\sum_{i=1}^{n}\frac{\partial f}{\partial x_{i}}(x)(x_{i}-a_{i})+\underbrace{O(x^{2})}_{\in \mfrak^{2}}$$
    hence the map is zero on $f\in\mfrak^{2}$ showing that $(\frac{\mfrak}{\mfrak^{2}})^{\vee}\to\Hom_{\Vect_{k}}(k^{n},k)$ by $(x_{i}-a_{i})\mapsto e_{i}^{\vee}$ is an injection between vector spaces of the same dimension and hence an isomorphism. 
\end{example}
\begin{example}
    In general, one can still define a map on non-rational points with target $\Hom_{\Vect_{\kappa(\pfrak)}}(\kappa(\pfrak)^{n},\kappa(\pfrak))$ which may fail to be injective. Let $k$ be a field of characteristic $p$ and consider $(x^{p}-a)\subseteq k[x]$ which is maximal when $a^{1/p}\notin k$. We have $\frac{\mfrak}{\mfrak^{2}}=\frac{(x^{p}-a)}{(x^{p}-a)^{2}}\cong k$ which defines a map $\Hom_{\Vect_{k}}(k,k)$ by \Cref{ex: map to hom} which is the zero map as $px^{p-1}=0$. 
\end{example}
In what follows, we will use the following result for closed subschemes of affine spaces. 
\begin{proposition}\label{prop: annihilator is differentials}
    Let $\afrak\subseteq k[x_{1},\dots,x_{n}]$ be an ideal defining $X=V(\afrak)\subseteq \A^{n}_{k}$. Let $x\in X(k)\subseteq\A^{n}_{k}(k)$. Then $T_{X,x}$ is the annihilator of the image of $\afrak$ under $D_{x}$
\end{proposition}
\begin{proof}
    We have a short exact sequence 
    $$0\to\afrak\to\widetilde{\mfrak}\to\mfrak\to 0$$
    inducing 
    \begin{equation}\label{eqn: maxl ideal intersection SES}
        0\to\frac{\afrak}{\afrak\cap\widetilde{\mfrak}^{2}}\to\frac{\widetilde{\mfrak}}{\widetilde{\mfrak}^{2}}\to\frac{\mfrak}{\mfrak^{2}}\to0.
    \end{equation}
    Applying the right-exact functor $\Hom_{\Vect_{k}}(-,k)$ we get 
    $$\left(\frac{\afrak}{\afrak\cap\widetilde{\mfrak}^{2}}\right)^{\vee}\to T_{\A^{n}_{k},x}^{\vee}\to T_{X,x}^{\vee}\to0$$
    where the map $\left(\frac{\afrak}{\afrak\cap\widetilde{\mfrak}^{2}}\right)^{\vee}\to \Hom_{\Vect_{k}}(k^{n},k)$ by taking $\afrak$-derivations as in \Cref{ex: map to hom}. 
\end{proof}
An analogous proof can be used to show that the Zariski tangent space is the cokernel of the Jacobian matrix. 
\begin{corollary}\label{corr: dimension is corank jacobian}
    Let $\afrak\subseteq k[x_{1},\dots,x_{n}]$ be an ideal defining $X=V(\afrak)\subseteq \A^{n}_{k}$. Let $x\in X(k)\subseteq\A^{n}_{k}(k)$. Then $T_{X,x}^{\vee}\cong\coker(J_{x})$ where $J_{x}$ is the Jacobian at $x$. 
\end{corollary}
\begin{proof}
    We use the short exact sequence (\ref{eqn: maxl ideal intersection SES}) and observe that the map $\frac{\afrak}{\afrak\cap \widetilde{\mfrak}^{2}}\to\frac{\widetilde{\mfrak}}{\widetilde{\mfrak}^{2}}$ is given by multiplication by the Jacobian, giving the claim. 
\end{proof}
Having related this to the Zariski tangent space, we want to relate the Jacobian matrix to the sheaf/module of K\"{a}hler differentials, an analogy suggested by \Cref{prop: isomorphism to Zariski tangent space}. 
\begin{proposition}\label{prop: corank of Jacobian is base change of differentials}
    Let $\afrak\subseteq k[x_{1},\dots,x_{n}]$ be an ideal defining $X=V(\afrak)\subseteq \A^{n}_{k}$. Let $x\in X(k)\subseteq\A^{n}_{k}(k)$. Then the corank of the Jacobian $J_{x}$ is equal to $\dim_{\kappa(x)}\Omega_{X/k}^{1}\otimes\kappa(x)$. 
\end{proposition} 
\begin{proof}
    Applying $-\otimes\kappa(x)$ to the short exact sequence of \Cref{prop: ideal exact sequence}, we have 
    $$\frac{\afrak}{\afrak^{2}}\otimes\kappa(x)\to\Omega^{1}_{\A^{n}_{k}/k}\otimes\kappa(x)\to\Omega_{X/k}^{1}\otimes\kappa(x)\to0$$
    which factors over the image of the Jacobian $J_{x}$. As such, we get that $\dim_{\kappa(x)}\Omega_{X/k}^{1}=n-\dim(\img(J_{x}))=n-\mathrm{rank}(J_{x})$ which is precisely the corank. 
\end{proof}
Moreover, for a general point $x$, the property of the Zariski tangent space being isomorphic to the scalar extension of the sheaf of K\"{a}hler differentials. 
\begin{proposition}
    Let $\afrak\subseteq k[x_{1},\dots,x_{n}]$ be an ideal defining $X=V(\afrak)\subseteq \A^{n}_{k}$. Let $x\in X$. $\kappa(x)$ is a separable extension of $k$ if and only if $T_{X,x}^{\vee}\cong\Omega_{X/k}^{1}\otimes\kappa(x)$. 
\end{proposition}
\begin{proof}
    $(\Rightarrow)$ If $\kappa(x)$ is separable over $k$, then $\Omega_{\kappa(x)/k}^{1}=0$ by \Cref{lem: differentials of separable extension are zero} so $\frac{\mfrak}{\mfrak^{2}}\to\Omega_{X,k}^{1}\otimes\kappa(x)$ is surjective, but this shows that we have a surjection of $\kappa(x)$-vector spaces of the same dimension, hence an isomorphism. 

    $(\Leftarrow)$ If the Zariski cotangent space is isomorphic to teh sheaf of differentials, then the cokernel $\Omega^{1}_{\kappa(x)/k}$ of the exact sequence \Cref{prop: ideal exact sequence} is zero, showing that $\kappa(x)/k$ is separable. 
\end{proof}
Note that the equality $\dim(T_{X,x})=\dim_{\kappa(x)}\Omega^{1}_{X/k}\otimes\kappa(x)$ does not imply the natural map is an isomorphism when $\kappa(x)$ is not separable over $k$. 
\begin{example}
    Let $k$ be a field of characteristic $p$ and consider $(x^{p}-a)\subseteq k[x]$ for $a^{1/p}\notin k$. Denoting $X=V(x^{p}-a)\subseteq\A^{1}_{k}$, we have $\dim(T_{X,x})=1=\dim_{\kappa(x)}\Omega^{1}_{X/k}\otimes\kappa(x)$ but $\Omega^{1}_{\kappa(x)/k}$ is nonzero as $\kappa(x)$ is not a separable extension of $k$. 
\end{example}
We arrive at the notion of smoothness for schemes. 
\begin{definition}[Smooth Scheme]\label{def: smooth scheme}
    Let $X$ be a scheme of finite type over a field $k$. $X$ is smooth of pure dimension $d$ if 
    \begin{enumerate}[label=(\roman*)]
        \item each of the finitely many irreducible components of $X$ are of dimension $d$, and 
        \item every point $x\in X$ is contained in an affine open neighborhood where the Jacobian matrix is of corank $d$. 
    \end{enumerate}
\end{definition}
\begin{remark}\label{rmk: smoothness is relative}
    Smoothness is a relative notion, determined by the structure map to $\spec(k)$. 
\end{remark}
\begin{remark}
    By \Cref{ex: spec A example}, this construction is independent of the choice of chart. 
\end{remark}
\begin{remark}\label{rmk: check smoothness on closed points}
    It suffices to verify this condition on closed points, as if the Jacobian is rank-deficient at some non-closed point, then it is rank-deficient at any specialization. 
\end{remark}
Intuitively, we can view $X$ locally as the fiber of a map $\A^{n}_{k}\to\A^{r}_{k}$ defined by the $r$ polynomials $f_{1},\dots,f_{r}\in k[x_{1},\dots,x_{n}]$, and where $x$ being in the fiber over zero implies that the tangent space $T_{X,x}$ is the kernel of the map $(f_{1},\dots,f_{r})$ hence equal to the dimension of the fiber. 
We introduce the notion of being geometrically smooth. 
\begin{definition}[Geometrically Smooth Scheme]\label{def: geometrically smooth scheme}
    Let $X$ be a scheme of finite type over a field $k$. $X$ is geometrically smooth if the base change $X_{\overline{k}}$ to the algebraic closure is smooth over $\overline{k}$. 
\end{definition}
This is in fact equivalent to the condition of being smooth. 
\begin{lemma}\label{lem: smooth iff geometrically smooth}
    Let $X$ be a scheme of finite type over a field $k$. $X$ is a smooth $k$-scheme if and only if it is geometrically smooth. 
\end{lemma}
\begin{proof}
    We use the Cartesian square 
    $$% https://q.uiver.app/#q=WzAsNCxbMCwwLCJYX3tcXG92ZXJsaW5le2t9fSJdLFswLDEsIlxcc3BlYyhcXG92ZXJsaW5le2t9KSJdLFsyLDAsIlgiXSxbMiwxLCJcXHNwZWMoaykiXSxbMCwyXSxbMiwzXSxbMCwxXSxbMSwzXV0=
    \begin{tikzcd}
        {X_{\overline{k}}} && X \\
        {\spec(\overline{k})} && {\spec(k)}
        \arrow[from=1-1, to=1-3]
        \arrow[from=1-1, to=2-1]
        \arrow[from=1-3, to=2-3]
        \arrow[from=2-1, to=2-3]
    \end{tikzcd}$$
    where using \Cref{prop: tensor exact sequence}, we have $\Omega_{X/k}^{1}\otimes\kappa(x)\cong\Omega^{1}_{X_{\overline{k}}/\overline{k}}\otimes\kappa(y)$ where $y$ is the closed point corresponding to $x$ in $X_{\overline{k}}$. This isomorphism of sheaves characterizes the Jacobian being full rank at $x,y$, hence the smoothness conditions are equivalent. 
\end{proof}
While smoothness depends on the structure of $X$ as a $k$-scheme, it is closely related to the absolute notion of regularity. 

We recall the relevant definitons from commutative algebra. 
\begin{definition}[Regular Local Ring]\label{def: regular local ring}
    Let $(A,\mfrak)$ be a Noetherian local ring with residue field $\kappa=A/\mfrak$. $A$ is a regular local ring if $\dim(A)=\dim_{\kappa}(\frac{\mfrak}{\mfrak^{2}})$. 
\end{definition}
\begin{definition}[Regular Ring]\label{def: regular ring}
    Let $A$ be a Noetherian ring. $A$ is a regular ring if for all primes $\pfrak\subseteq A$, the localization $A_{\pfrak}$ is a regular local ring. 
\end{definition}
\begin{remark}
    Checking regularity of an arbitrary Noetherian ring can be done on maximal ideals by reasoning analogous to that of \Cref{rmk: check smoothness on closed points}. 
\end{remark}
This allows us to define regularity of schemes. 
\begin{definition}[Regular Scheme]\label{def: regular scheme}
    Let $X$ be a locally Noetherian scheme. $X$ is regular if for all closed points $x\in X$, the local ring $\Ocal_{X,x}$ is regular. 
\end{definition}
\begin{remark}
    By \Cref{def: regular ring}, this is equivalent to each point admitting an affine neighborhood given by the Zariski spectrum of a regular ring. 
\end{remark}
\begin{remark}
    In contrast to \Cref{rmk: smoothness is relative}, regularity is absolute and does not depend on any structure map of $X$. 
\end{remark}
The notions of regularity and smoothness are connected by the following proposition. 
\begin{proposition}\label{prop: smooth iff regular alg closed}
    Let $X$ be a scheme of finite type over an algebraically closed field $k$. $X$ is $k$-smooth if and only if $X$ is regular. 
\end{proposition}
\begin{proof}
    Both conditions can be checked affine-locally, so without loss of generality, we can take $X=V(\afrak)\subseteq\A^{n}_{k}$. By \Cref{prop: annihilator is differentials} the dimension of the Zariski tangent space of any $x\in X$ is dimension of the image of the map $D_{x}$ defined in \Cref{ex: map to hom}, which is equal to the rank of the Jacobian $J_{x}$. This is of rank equal to the Zariski tangent space (ie. $X$ is regular) if and only if the corank of the Jacobian is $\dim(X)$ (ie. $X$ is smooth). 
\end{proof}
Over general fields, smoothness implies regularity, but not the converse. 
\begin{corollary}\label{corr: smooth implies regular}
    Let $X$ be a scheme of finite type over a field $k$. If $X$ is $k$-smooth, then $X$ is regular. 
\end{corollary}
\begin{proof}
    By locality, we reduce once more to $X=V(\afrak)\subseteq\A^{n}_{k}$. By smoothness, $D_{x}(\afrak)=\mathrm{rank}(J_{x})$ and by the short exact sequence (\ref{eqn: maxl ideal intersection SES}) we have 
    $$\dim_{\kappa(\mfrak)}\left(\frac{\mfrak}{\mfrak^{2}}\right)=\dim_{\kappa(\mfrak)}\left(\frac{\widetilde{\mfrak}}{\widetilde{\mfrak}^{2}}\right)-\dim_{\mfrak}\left(\frac{\afrak}{\afrak\cap\widetilde{\mfrak}^{2}}\right)$$
    showing that the dimension of the Zariski tangent space at $x$ is equal to the dimension of $X$, hence $X$ is regular. 
\end{proof}
We now see an example of a regular non-smooth scheme. 
\begin{example}
    Let $k$ be a field of characteristic $p$ and consider $X=V(x^{p}-a)\subseteq \A^{1}_{k}$ and where $a^{1/p}\notin k$. $\spec(\frac{k[x]}{(x^{p}-a)})$ is the Zariski spectrum of a field, hence regular, but $X$ is not geometrically smooth and hence not smooth. 
\end{example}
\section{Lecture 4 -- 23rd May 2025}\label{sec: lecture 4}
Using the gluing procedure of (\ref{eqn: gluing map Frobenius}) allows us to correct for the overspecification of prescribing a local algebra $R^{(m)}$ for each positive integer $m$ in characteristic $p$ -- that is, gluing $R^{(m)},R^{(m')}$ where $m_{0}$ is coprime to $p$ and $m=m_{0}p^{a},m'=m_{0}p^{b}$ using the Frobenius. 

More generally, we can define the Habiro ring of a smooth framed $\ZZ$-algebra $(R,\square)$ by passing to the limit of the rings $R_{n}$ where there are surjective transition maps $R_{pm}\to R_{m}$ given by the (necessarily unique) lift of the isomorphism (\ref{eqn: gluing map Frobenius}) along the (necessarily unique) deformation of \'{e}tale algebras $R^{(m)}$ to $R_{m}$.  
\begin{definition}[Habiro Ring of Framed Algebra]\label{def: Habiro ring of framed algebra}
    Let $(R,\square)$ be a smooth framed $\ZZ$-algebra. The Habiro ring $\Hcal_{(R,\square)}$ is given by the limit 
    $$\Hcal_{(R,\square)}=\lim_{n\in\NN}R_{n}$$
    where $R_{n}$ is the completed root of unity algebra of \Cref{def: completed root of unity algebra}. 
\end{definition}
\begin{proposition}\label{prop: explicit elements of HR}
    Let $(R,\square)$ be a smooth framed $\ZZ$-algebra. The Habiro ring $\Hcal_{(R,\square)}$ of $(R,\square)$ is given by 
    {\footnotesize
    \begin{equation}\label{eqn: Habiro ring of framed algebra}
        \Hcal_{(R,\square)}=\left\{(f_{m})_{m\geq 1}\in\prod_{m\geq 1}R^{(m)}[[q-\zeta_{m}]]:\substack{\forall m\in\NN,\text{ }\forall p\text{ prime} \\\varphi_{p}(f_{pm})=f_{m}\in (R^{(m)})_{p}^{\wedge}[[q-\zeta_{m}]]\cong (R^{(pm)})_{p}^{\wedge}[[q-\zeta_{pm}]]}\right\}
    \end{equation}
    \normalsize}where $\varphi_{p}$ lifts the Frobenius on $R^{(m)}/(p)$ by raising each variable to the $p$-th power and fixes $q$ and $\zeta_{m}$. 
\end{proposition}
\begin{remark}
    There is an obvious map from the Habiro ring of the torus \Cref{def: Habiro ring of base} $\Hcal_{\ZZ[\underline{T}^{\pm}]}\to\Hcal_{(R,\square)}$ endowing the Habiro ring of $(R,\square)$ with the structure of a $\Hcal_{\ZZ[\underline{T}^{\pm}]}$-algebra.  
\end{remark}
Let us consider some explicit elements of the Habiro ring. 
\begin{example}\label{ex: element of Habiro ring}\marginpar{The lecture contained a fairly substantive sketch of the proof \Cref{ex: element of Habiro ring}, which the author has defered to \Cref{appdx: explicit elements} for continuity of exposition.}
    Let $R=\ZZ[T_{1},\dots,T_{d},\frac{1}{1-T_{1}-\dots-T_{d}}]$ with framing $\square:\ZZ[T_{1},\dots,T_{d}]\to R$. The element 
    $$\sum_{k_{1},\dots,k_{d}\geq0}\left[\substack{k_{1}+\dots+k_{d} \\ k_{1}\text{ }\dots\text{ }k_{d}}\right]_{q}T_{1}^{k_{1}}\dots T_{d}^{k_{d}}\in\ZZ[q][[\underline{T}]]$$
    is an element of the Habiro ring $\Hcal_{(R,\square)}$ where 
    $$\left[\substack{k_{1}+\dots+k_{d} \\ k_{1}\text{ }\dots\text{ }k_{d}}\right]_{q}=\frac{(q;q)_{k_{1}+\dots+k_{d}}}{(q;q)_{k_{1}}\dots(q;q)_{k_{d}}}$$
    is the $q$-deformation of the multinomial $\binom{k_{1}+\dots+k_{d}}{k_{1}\dots k_{d}}$. More generally, explicit elements of the Habiro ring can be constructed by considering $q$-deformations of rational functions (vis. \Cref{ex: legendre family} and surrounding discussion). 
\end{example}
Returning to a discussion of Habiro cohomology of a smooth $\ZZ$-algebra with framing $\square:\spec(R)\to(\GG_{m})^{d}$, we recall that there are lifts of the automorphism $\gamma_{i}$ to $\Hcal_{(R,\square)}$: more explicitly, for a section $(f_{m})_{m\geq0}$, the action $\gamma_{i}$ acts by $(f_{m})_{m\geq1}\mapsto (\gamma_{i}^{(m)}(f_{m}))_{m\geq1}$ where $\gamma_{i}^{(m)}$ is the automorphism given in \Cref{def: root of unity algebra}. This produces a $\ZZ^{d}$-action on $\Hcal_{(R,\square)}$, and we can define Habiro-Hodge cohomology to be the group cohomology of the action of $\ZZ^{d}$ on $\Hcal_{(R,\square)}$. 
\begin{definition}[$q$-Habiro-Hodge Cohomology]\label{def: q-Habiro-Hodge cohomology}
    Let $(R,\square)$ be a smooth framed $\ZZ$-algebra. The $q$-Habiro-Hodge cohomology is the cohomology of the $q$-Habiro-Hodge complex $q\dash\HHdg_{(R,\square)}$ given by 
    \begin{equation}\label{eqn: q-Habiro-Hodge complex}
        \Hcal_{(R,\square)}\xrightarrow{(\gamma_{i}-1)_{i=1}^{d}}\bigoplus_{i=1}^{d}\Hcal_{(R,\square)}\xrightarrow{(\gamma_{i}-1)_{i=1}^{d}}\bigoplus_{i<j}\Hcal_{(R,\square)}\longrightarrow\dots.
    \end{equation}
\end{definition}
For this to be functorial, we would expect this to be coordinate independent, at least at the level of derived categories. As a first step, we study the cohomology of the complex modulo $(1-q^{m})$ -- that is, at specalizations to roots of unity. 

If $m=1$, then $\Hcal_{(R,\square)}/(1-q)\cong R$ and all differentials are zero, so 
\begin{equation}\label{eqn: de Rham complex at trivial root of unity}
    H^{i}\left(q\dash\HHdg_{(R,\square)}/(1-q)\right)\cong R^{\oplus\binom{d}{i}}\cong \Omega^{i}_{R/\ZZ}
\end{equation}
and is therefore independent of coordinates since the middle term is so. 
\begin{remark}\label{rmk: Bockstein operator}
    While \emph{a priori} we only have a isomorphism to a free module of a certain rank, there is additonal structure that allows us to identify this with the module of K\"{a}hler differentials: the Bockstein map associated to the triangle 
    {\footnotesize 
    $$q\dash\HHdg_{(R,\square)}/(1-q)\xrightarrow{\times(1-q)}q\dash\HHdg_{(R,\square)}/(1-q)^{2}\longrightarrow q\dash\HHdg_{(R,\square)}/(1-q)\longrightarrow \left(q\dash\HHdg_{(R,\square)}/(1-q)\right)[1]$$
    \normalsize}inducing 
    $$H^{i}\left(q\dash\HHdg_{(R,\square)}/(1-q)\right)\longrightarrow H^{i+1}\left(q\dash\HHdg_{(R,\square)}/(1-q)\right)$$
    which gives a derivation 
    $$H^{0}\left(q\dash\HHdg_{(R,\square)}/(1-q)\right)\longrightarrow H^{1}\left(q\dash\HHdg_{(R,\square)}/(1-q)\right)$$
    and hence an isomorphism $H^{1}\left(q\dash\HHdg_{(R,\square)}/(1-q)\right)\to\Omega^{1}_{R/\ZZ}$. In addition, the ring structure on cohomology induces the structure of a commutative differential graded algebra on $H^{\bullet}\left(q\dash\HHdg_{(R,\square)}/(1-q)\right)$ and this structure is in fact independent of coordinates on the nose and not just up to quasi-isomorphism.
\end{remark}

For general $m$, $H^{\bullet}\left(q\dash\HHdg_{(R,\square)}/(1-q^{m})\right)$ has the strucuture of a commutative differential graded algebra that is coordinate independent. 
\begin{theorem}[Wagner; {\cite[Prop. 5.7]{WagnerMSThesis}}]\label{thm: surjection from q witt vectors}
    Let $R$ be a smooth framed $\ZZ$-algebra. There is a canonical surjection 
    $$W_{m}(R)[q]/(1-q^{m})^{\bullet}\longrightarrow H^{0}\left(q\dash\HHdg_{(R,\square)}/(1-q^{m})\right)$$
    inducing 
    $$\Omega_{W_{m}(R)[q]/(1-q^{m})}\longrightarrow H^{\bullet}\left(q\dash\HHdg_{(R,\square)}/(1-q)\right)$$ 
    which is coordinate independent, degreewise surjective, and with kernel independent of coordinates. 
\end{theorem}
\begin{proof}[Proof Outline]
    For every commutative differential graded algebra $B$ receiving a map from a commutative ring $A$ in 0th cohomology, there is an induced map from the initial commutative differential graded algebra generated by $A$ to $B$ -- the latter being the de Rham complex. 
\end{proof}
This produces a description of $H^{i}\left(q\dash\HHdg_{(R,\square)}/(1-q)\right)$ that is visibly independent of coordinates, being the quotient of coordinate-independent objects. 

In fact we can do better. For any $R$, there is a notion of $q$-Witt vectors $q\dash W_{m}(R)$ and $q$-de Rham-Witt complexes $q\dash W_{m}\Omega_{R}$ which is a commutative differential graded algebra with first term $q\dash W_{m}(R)$ isomorphic to $H^{\bullet}\left(q\dash\HHdg_{(R,\square)}/(1-q^{m})\right)$. 
\begin{theorem}[Wagner; {\cite[Thm. 5.7]{WagnerMSThesis}}]\label{thm: }
    Let $R$ be a smooth framed $\ZZ$-algebra. There is an isomorphism 
    $$q\dash W_{m}\Omega_{R}^{\bullet}\longrightarrow H^{\bullet}\left(q\dash\HHdg_{(R,\square)}/(1-q)\right).$$
\end{theorem}
\begin{remark}
    This is related to the classical construction of the de Rham-Witt complex, though the sense in which the preceding constructions are $q$-deformations are quite subtle. 
\end{remark}
\begin{remark}
    One can often reduce to the case of computing on the torus, since many of the constructions ``commute with \'{e}tale maps'' in the sense that they are preserved under \'{e}tale base change. 
\end{remark}
Based on this, one might hope that these complexes are independent of coordinates. 
\begin{example}\label{ex: q-Habiro-Hodge cohomology of torus}
    Let $R=\ZZ[T^{\pm}]$. The $q$-Habiro-Hodge complex is given by 
    $$\ZZ[q][T^{\pm}]/(1-q^{m})\xrightarrow{\gamma-1}\ZZ[q][T^{\pm}]/(1-q^{m})$$
    by $T^{k}\mapsto(q^{k}-1)T^{k}$. We can compute the kernel of this map -- the 0th cohomology -- by noting that the map preserves the degree of $T$, we can compute the kernel in each degree to see that it is given by 
    $$\bigoplus_{k\in\ZZ}\left(\frac{\frac{q^{m}-1}{q^{\gcd(k,m)}-1}\ZZ[q]}{(q^{m}-1)\ZZ[q]}\right)T^{k}\cong\bigoplus_{k\in\ZZ}\left(\ZZ[q]/(1-q^{\gcd(k,m)})\ZZ[q]\right)T^{k}.$$
    We similarly compute first cohomology to see it is also given by 
    $$\bigoplus_{k\in\ZZ}\left(\ZZ[q]/(1-q^{\gcd(k,m)})\ZZ[q]\right)T^{k}.$$
    Indeed, when $m=p$ is prime, the 0th cohomology is a subring of $\ZZ[q][T^{\pm}]/(1-q^{p})$ (hence a subring of $\ZZ[T^{\pm}]\times\ZZ[\zeta_{p}][T^{\pm p}]\subseteq\ZZ[T^{\pm}]\times\ZZ[\zeta_{p}][T^{\pm}]$) and is generated by $T^{p}$ and $[p]_{q}T^{i}$ for $1\leq i\leq p-1$. 
\end{example}
The computations of \Cref{ex: q-Habiro-Hodge cohomology of torus} is suggestive of a connection to Witt vectors since the cohomology lies in the product of rings $\ZZ[T^{\pm}]\times\ZZ[\zeta_{p}][T^{\pm p}]$. Recall that for a $p$-torsion free ring $R$, the $p$-th Witt vectors $W_{p}(R)$ consists of elements $(x_{0},x_{1},\dots)$ has ghost maps $\gh_{1},\gh_{p}:W_{p}(R)\to R$ by $(x_{0},x_{1},\dots)\mapsto x_{0}$ and $(x_{0},x_{1},\dots)\mapsto x_{0}^{p}+px_{1}$, respectively. The image of $(\gh_{1},\gh_{p}):W_{p}(R)\to R\times R$ consists precisely of those pairs $(x,y)\in R\times R$ where $y\equiv x^{p}\pmod{p}$. 
\begin{proposition}[Wagner]\label{prop: q-Witt vectors}
    Let $R=\ZZ[T^{\pm}]$ with the identity framing and $q\dash\HHdg_{(R,\square)}$ its $q$-Habiro-Hodge complex. There is a canonical embedding 
    $$W_{p}(R)\hookrightarrow H^{0}\left(q\dash\HHdg_{(R,\square)}/(1-q^{p})\right)$$
    rendering the diagram 
    {\footnotesize
    $$% https://q.uiver.app/#q=WzAsNyxbMSwxLCJIXnswfVxcbGVmdChxXFxkYXNoXFxISGRnX3soUixcXHNxdWFyZSl9LygxLXFee3B9KVxccmlnaHQpIl0sWzEsMiwiV197cH0oUikiXSxbMywxLCJcXFpaW1Ree1xccG19XVxcdGltZXNcXFpaW1xcemV0YV97cH1dW1Ree1xccG0gcH1dIl0sWzMsMiwiUlxcdGltZXMgUiJdLFswLDMsIih4X3swfSx4X3sxfSkiXSxbNCwzLCIoeF97MH0seF97MH1ee3B9K3B4X3sxfSkiXSxbMCwwLCJcXHZhcnBoaV97cH0oeF97MH0pK1twXV97cX14X3sxfSJdLFsxLDAsIiIsMCx7InN0eWxlIjp7InRhaWwiOnsibmFtZSI6Imhvb2siLCJzaWRlIjoidG9wIn19fV0sWzEsMywiKFxcZ2hfezF9LFxcZ2hfe3B9KSIsMl0sWzMsMl0sWzAsMiwiIiwwLHsic3R5bGUiOnsidGFpbCI6eyJuYW1lIjoiaG9vayIsInNpZGUiOiJ0b3AifX19XSxbNCw1LCIiLDIseyJzdHlsZSI6eyJ0YWlsIjp7Im5hbWUiOiJtYXBzIHRvIn19fV0sWzQsNiwiIiwwLHsic3R5bGUiOnsidGFpbCI6eyJuYW1lIjoibWFwcyB0byJ9fX1dXQ==
    \begin{tikzcd}
        {\varphi_{p}(x_{0})+[p]_{q}x_{1}} \\
        & {H^{0}\left(q\dash\HHdg_{(R,\square)}/(1-q^{p})\right)} && {\ZZ[T^{\pm}]\times\ZZ[\zeta_{p}][T^{\pm p}]} \\
        & {W_{p}(R)} && {R\times R} \\
        {(x_{0},x_{1})} &&&& {(x_{0},x_{0}^{p}+px_{1})}
        \arrow[hook, from=2-2, to=2-4]
        \arrow[hook, from=3-2, to=2-2]
        \arrow["{(\gh_{1},\gh_{p})}"', from=3-2, to=3-4]
        \arrow[from=3-4, to=2-4]
        \arrow[maps to, from=4-1, to=1-1]
        \arrow[maps to, from=4-1, to=4-5]
    \end{tikzcd}$$
    \normalsize}commutative. 
\end{proposition}
\begin{remark}
    On the $q$-Habiro-Hodge cohomologies, we can relate the different specializations by Frobenii and Verschiebungen 
    $$% https://q.uiver.app/#q=WzAsMixbMCwwLCJIXntpfVxcbGVmdChxXFxkYXNoXFxISGRnX3soUixcXHNxdWFyZSl9LygxLXFee21rfSlcXHJpZ2h0KSJdLFsyLDAsIkhee2l9XFxsZWZ0KHFcXGRhc2hcXEhIZGdfeyhSLFxcc3F1YXJlKX0vKDEtcV57bX0pXFxyaWdodCkuIl0sWzAsMSwiRl97a30iLDAseyJvZmZzZXQiOi0xfV0sWzEsMCwiVl97a309XFx0aW1lc1xcZnJhY3sxLXFee21rfX17MS1xXnttfX0iLDAseyJvZmZzZXQiOi0xfV1d
    \begin{tikzcd}
        {H^{i}\left(q\dash\HHdg_{(R,\square)}/(1-q^{mk})\right)} && {H^{i}\left(q\dash\HHdg_{(R,\square)}/(1-q^{m})\right).}
        \arrow["{F_{k}}", shift left, from=1-1, to=1-3]
        \arrow["{V_{k}=\times\frac{1-q^{mk}}{1-q^{m}}}", shift left, from=1-3, to=1-1]
    \end{tikzcd}$$
\end{remark}
More generally, we have the following. 
\begin{proposition}
    Let $R$ be a flat $\ZZ$-algebra. There is a commutative diagram 
    $$% https://q.uiver.app/#q=WzAsNSxbMiwwLCJXX3ttfShSKSJdLFs0LDAsIlxccHJvZF97ZHxtfVIiXSxbNCwxLCJcXHByb2Rfe2R8bX1SW1xcemV0YV97ZH1dIl0sWzIsMSwicVxcZGFzaCBXX3ttfShSKSJdLFswLDEsIldfe219KFIpW3FdLygxLXFee219KSJdLFs0LDMsIiIsMCx7InN0eWxlIjp7ImhlYWQiOnsibmFtZSI6ImVwaSJ9fX1dLFswLDRdLFswLDMsIiIsMix7InN0eWxlIjp7InRhaWwiOnsibmFtZSI6Imhvb2siLCJzaWRlIjoidG9wIn19fV0sWzAsMSwiKFxcZ2hfe2R9KV97ZHxtfSIsMCx7InN0eWxlIjp7InRhaWwiOnsibmFtZSI6Imhvb2siLCJzaWRlIjoidG9wIn19fV0sWzMsMiwiKHFcXGRhc2hcXGdoX3tkfSlfe2R8bX0iLDIseyJzdHlsZSI6eyJ0YWlsIjp7Im5hbWUiOiJob29rIiwic2lkZSI6InRvcCJ9fX1dLFsxLDIsIihSKV97ZH1cXHRvKFJbXFx6ZXRhX3ttL2R9XSlfe20vZH0iLDAseyJzdHlsZSI6eyJ0YWlsIjp7Im5hbWUiOiJob29rIiwic2lkZSI6InRvcCJ9fX1dXQ==
    \begin{tikzcd}
        && {W_{m}(R)} && {\prod_{d|m}R} \\
        {W_{m}(R)[q]/(1-q^{m})} && {q\dash W_{m}(R)} && {\prod_{d|m}R[\zeta_{d}]}
        \arrow["{(\gh_{d})_{d|m}}", hook, from=1-3, to=1-5]
        \arrow[from=1-3, to=2-1]
        \arrow[hook, from=1-3, to=2-3]
        \arrow["{(R)_{d}\to(R[\zeta_{m/d}])_{m/d}}", hook, from=1-5, to=2-5]
        \arrow[two heads, from=2-1, to=2-3]
        \arrow["{(q\dash\gh_{d})_{d|m}}"', hook, from=2-3, to=2-5]
    \end{tikzcd}$$
    where the Frobenii and Verschiebungen are defined on $q\dash W_{m}(R)$. 
\end{proposition}
\begin{remark}
    On the right, the map takes the $d$th factor of the product $\prod_{d|m}R$ to the $\frac{m}{d}$th factor of the product $\prod_{d|m}R[\zeta_{d}]$. 
\end{remark}
\begin{remark}
    There are no restriction maps on the $q$-Witt vectors $q\dash W_{m}(R)$. 
\end{remark}
This shows that on the level of cohomology, the $q$-Habiro-Hodge complex is coordinate independent after specialization. However, due to a theorem of Wagner, this is the best we can do: there is no way to make the $q$-Habiro-Hodge complex itself coordinate independent in the derived category in such a way that remains coordinate independent on specialization. 

\section{Lecture 5 -- 24th April 2025}\label{sec: lecture 5}
We begin a discussion of flatness, which intuitively corresponds to varying ``nicely'' in a family over the base. As always, we first define these in the local case. 
\begin{definition}[Flat Module]\label{def: flat module}
    Let $A$ be a ring and $M$ an $A$-module. $M$ is a flat $A$-module if $-\otimes_{A}M$ is an exact functor. 
\end{definition}
Recall that for $A$-modules $M$, $-\otimes_{A}M$ is always a right exact functor so flatness is equivalent to being left exact, taking short exact sequences 
$$0\to N_{1}\to N_{2}\to N_{3}\to 0$$
to short exact sequences
$$0\to N_{1}\otimes_{A}M\to N_{2}\otimes_{A}M_{2}\to N_{3}\otimes_{A}N_{3}\to0.$$
In fact, it suffices to verify that for all ideals $\afrak\subseteq A$ that $\afrak\otimes_{A}M\to A\otimes_{A}M$ is injective. 

We can apply this to the special case where an $A$-algebra $B$ is considered as an $A$-module. 
\begin{definition}[Flat Algebra]\label{def: flat algebra}
    Let $A$ be a ring and $B$ an $A$-algebra. $B$ is $A$-flat if $B$ is flat as an $A$-module. 
\end{definition}
Let us recall some further properties of flatness. 
\begin{proposition}\label{prop: properties of flatness rings and modules}
    Let $A$ be a ring. 
    \begin{enumerate}[label=(\roman*)]
        \item If $A$ is a local ring and $M$ is a finite $A$-module, then $M$ is flat if and only if $M$ is free. 
        \item If $S\subseteq A$ is a multiplicative subset, $A\to S^{-1}A$ is flat. 
        \item If $A\to B$ is a ring map and $M$ is a flat $A$-module then $M\otimes_{A}B$ is a flat $B$-module. 
        \item If $A\to B$ is a flat ring map and $N$ is a flat $B$-module then its restriction of scalars $N|_{A}$ is a flat $A$-module. 
        \item Let $M$ be an $A$-module. $M$ is flat if and only if $M_{\pfrak}$ is a flat $A_{\pfrak}$-module for all prime ideals (maximal ideals). 
        \item Let $A\to B$ be a flat ring map between Noetherian local rings, and $b\in B$ such that $\overline{b}\in B/\mfrak_{A}B$ is a non-zerodivisor then $B/(b)$ is $A$-flat. 
        \item For a coCartesian diagram 
        $$% https://q.uiver.app/#q=WzAsNCxbMCwwLCJBIl0sWzIsMCwiQiJdLFswLDEsIkMiXSxbMiwxLCJCXFxvdGltZXNfe0N9QSJdLFswLDFdLFsxLDNdLFswLDJdLFsyLDNdXQ==
        \begin{tikzcd}
            A && B \\
            C && {B\otimes_{C}A}
            \arrow[from=1-1, to=1-3]
            \arrow[from=1-1, to=2-1]
            \arrow[from=1-3, to=2-3]
            \arrow[from=2-1, to=2-3]
        \end{tikzcd}$$
        and $M$ a $B$-module that is $A$-flat, then $M\otimes_{B}(C\otimes_{A}B)$ is a flat $C$-module. 
    \end{enumerate}
\end{proposition}
\begin{proof}
    See \cite[\href{https://stacks.math.columbia.edu/tag/00H9}{Tag 00H9}]{stacks-project}.
\end{proof}
We prove the following lemma about flatness on localizations at prime ideals. 
\begin{lemma}\label{lem: flatness on stalks}
    Let $\varphi:A\to B$ be a ring map. $\varphi$ is flat if and only if for all primes $\qfrak\subseteq B$, the map $A_{\varphi^{-1}(\qfrak)}\to B_{\qfrak}\cong A_{\varphi^{-1}(\qfrak)}\otimes_{A}B$ is a flat ring map. 
\end{lemma}
\begin{proof}
    $(\Rightarrow)$ Suppose $A\to B$ is flat. Then any $A_{\varphi^{-1}(\qfrak)}\to B_{\qfrak}$ is factored as $A_{\varphi^{-1}(\qfrak)}\to B_{\varphi^{-1}(\qfrak)}\cong A_{\varphi^{-1}(\qfrak)}\otimes_{A}B\to(B_{\varphi^{-1}(\qfrak)})_{\qfrak}\cong B_{\qfrak}$ which is a composition of a flat map with a localization, the latter flat, hence the composition is flat too. 

    $(\Leftarrow)$ Suppose we have an injection $M\to M'$ of $A$-modules. By exactness of localization, we have $M\otimes_{A}A_{\pfrak}\to M'\otimes_{A}A_{\pfrak}$. $B_{\qfrak}$ is $A_{\pfrak}$-flat so the map remains injective on base change to $B_{\qfrak}$ for all $\qfrak\subseteq B$ so taking $M=A,M'=B$ yields the claim in conjunction with \Cref{prop: properties of flatness rings and modules} (v). 
\end{proof}
We now describe the geometric case. 
\begin{definition}[Flat Sheaves]\label{def: flat sheaves}
    Let $f:X\to Y$ be a morphism of schemes and $\Fcal$ a quasicoherent sheaf on $X$. $\Fcal$ is flat over $Y$ if $\Fcal_{x}$ is flat as an $\Ocal_{Y,f(x)}$-module. 
\end{definition}
\begin{definition}[Flat Morphism]\label{def: flat morphism}
    Let $f:X\to Y$ be a morphism of schemes. $f$ is a flat morphism if $\Ocal_{X}$ is flat over $Y$. 
\end{definition}
We consider some examples. 
\begin{example}
    Let $X$ be a scheme. $\id_{X}:X\to X$ is a flat morphism as affine-locally it is obtained by the identity ring map and rings are flat as modules over themselves. 
\end{example}
\begin{example}
    Let $\Fcal$ be a coherent sheaf on $X$. $\Fcal$ is a flat $\Ocal_{X}$-module if and only if $\Fcal$ is locally free. In the reduced case, this can be checked by the function $x\mapsto\dim_{\kappa(x)}\Fcal_{x}\otimes_{\Ocal_{X,x}}\kappa(x)$ being locally constant. 
\end{example}
\begin{example}
    \Cref{prop: properties of flatness rings and modules} (ii) shows that open immersions are flat, though closed immersions are often not flat uness they are isomorphisms. 
\end{example}
\begin{proposition}\label{prop: dimension criterion for flatness}
    Let $f:X\to Y$ be a morphism between locally Noetherian schemes. Then $\dim(\Ocal_{X_{y}},x)\geq\dim(\Ocal_{X,x})-\dim(\Ocal_{Y,y})$ and equality occurs when $f$ is flat.\todo{Check}
\end{proposition}
\begin{proof}
    By locality on source and target, we can reduce this to the case of $X=\spec(\Ocal_{X,x}),Y=\spec(\Ocal_{Y,y})$. Denote $B=\Ocal_{X,x},A=\Ocal_{Y,y}$. We proceed by induction on the dimension of $Y$. 

    If $\dim(Y)=0$, then the nilradical is in the maximal ideal. In particular the qutoient by the nilradical is a field. Quotienting $B$ by its nilradical, we have $(B/\mfrak_{y}B)/(\Nil(B)/\mfrak_{y}B)\cong B/\Nil(B)$ by $\mfrak_{y}B\hookrightarrow\Nil(B)$. This gives 
    $$\dim(\Ocal_{X,x})=\dim(B/\Nil(B))=\dim(B/\mfrak_{y}B)=\dim(\Ocal_{X_{y},x})$$
    so 
    $$\dim(\Ocal_{X_{y},x})\geq\dim(\Ocal_{X,x})-\underbrace{\dim(\Ocal_{Y,y})}_{=0}$$
    giving the desired (in)equality. 

    Suppose $Y$ is of dimension at least 1. By base changing to the reduction, we can take $Y$ to be reduced by \Cref{prop: properties of flatness rings and modules} (iii). 

    In this case there exists $\pfrak\subsetneq\mfrak_{y}$ and an element $t\in\mfrak_{y}\setminus\pfrak$. In particular, $t$ is a non-zerodivisor. Using that $A$ is Noetherian, Krull's theorem implies every minimal prime over $(t)$ is of dimension 1 so $\dim(\Ocal_{Y,y}/(t))=\dim(\Ocal_{Y,y})-1$. Under $\varphi:A\to B$, $\varphi(t)$ may or may not be a zerodivisor giving $\dim(\Ocal_{Y,y})\geq\dim(\Ocal_{Y,y}/(t))-1$ but if $\varphi$ is flat then $\varphi(t)$ is a nonzerodivisor and gives equality. 
\end{proof}
In the case of locally finite type schemes, this can be detected fiberwise. 
\begin{corollary}\label{corr: fiberwise flatness}
    Let $f:X\to Y$ be a morphism of $k$-schemes where $X,Y$ are locally of finite type, $Y$ is irreducible, and $X$ equidimensional. If for all $y\in Y$, $X_{y}$ is equidimensional of $\dim(X)-\dim(Y)$ then $f$ is flat and surjective. 
\end{corollary}
\begin{proof}
    Without loss of generality, let $X$ be irreducible and take $x\in X$ closed. Then $\dim(X_{y})=\dim(\Ocal_{X_{y},x})$. By hypothesis we have 
    $$\dim(\Ocal_{X,x})=\dim(X)-\dim(\overline{\{x\}})$$
    which holds for finite-type $k$-algebras. On the other hand, we have 
    \begin{align*}
        \dim(\Ocal_{Y,y}) &= \dim(Y)-\dim(\overline{\{y\}}) \\
        &= \dim(Y)-\mathrm{trdeg}(\kappa(y)/k) \\
        &= \dim(Y)-\dim\{\overline{x}\} \\
        &= \dim(Y)-\dim(X)+\dim(\Ocal_{X,y})
    \end{align*}
    which gives the equality of \Cref{prop: dimension criterion for flatness} and from which the claim follows. 
\end{proof}
Moreover, flatness behaves especially well over smooth curves. 
\begin{proposition}\label{prop: flatness over curves}
    Let $f:X\to Y$ be a morphism of schemes where $Y$ is the spectrum of a discrete valuation. If $f$ is flat, then $\overline{X_{\eta}}=X$. 
\end{proposition}
In particular, there are no closed irreducible components $Z\subseteq Z$ over the (unique) closed point of $Y$. 
\begin{proof}
    Suppose $f$ is flat and there is $Z\subseteq Z$ irreducible in the fiber over the closed point $\mfrak\in Y$. Then $\dim(Z)\leq \dim(X_{\mfrak})=\dim(X)-1$. A contradiction to $X$ equidimensional. Otherwise, we apply \Cref{corr: fiberwise flatness} to $Z$ obtaining a contradiction once more. 
\end{proof}
\section{Lecture 6 -- 28th April 2025}\label{sec: lecture 6}
As a corollary of the previous discussion about flatness, we show that flatness can be extended over a closed fiber. 
\begin{corollary}\label{corr: properness of the hilbert scheme}
    Let $f:X\to Y\setminus\{y\}$ be flat and projective with $Y$ Dedekind and $y\in Y$ closed. 
    $$% https://q.uiver.app/#q=WzAsNCxbMCwwLCJYIl0sWzIsMCwiXFxQUF57bn1fe1lcXHNldG1pbnVzIFxce3lcXH19Il0sWzIsMSwiXFxQUF57bn1fe1l9Il0sWzAsMSwiXFxvdmVybGluZXtYfSJdLFswLDFdLFsxLDJdLFswLDNdLFszLDJdXQ==
    \begin{tikzcd}
        X && {\PP^{n}_{Y\setminus \{y\}}} \\
        {\overline{X}} && {\PP^{n}_{Y}}
        \arrow[from=1-1, to=1-3]
        \arrow[from=1-1, to=2-1]
        \arrow[from=1-3, to=2-3]
        \arrow[from=2-1, to=2-3]
    \end{tikzcd}$$
    Then the induced map $\overline{X}\to Y$ is flat. 
\end{corollary}
\begin{proof}
    Recall that a Dedekind scheme is a Noetherian regular integral scheme of dimension 1. We have the diagram of the statement of the corollary and by \Cref{prop: flatness over curves} we have $X_{\eta}=(\overline{X})_{\eta}\subseteq\overline{X}$ which is dense and $X_{\eta}$ is dense in $X$ so $X_{\eta}\subseteq \overline{X}$ is dense. This shows that $\overline{X}$ is flat over $Y$. 
\end{proof}
We show that images of flat maps are dense. 
\begin{proposition}\label{prop: flat implies dense image}
    Let $f:X\to Y$ be a flat map of schemes with $Y$ irreducible. If $U\subseteq X$ is a nonempty open, then $f(U)\subseteq Y$ is dense. 
\end{proposition}
\begin{proof}
    Without loss of generality, $Y=\spec(A)$ is affine and irreiducible so $A/\Nil(A)$ is integral and $\Frac(A/\Nil(A))=K$. For $U=\spec(B)$ affine, consider $B/\Nil(A)B=B\otimes_{A}(A/\Nil(A))$. Using the injection $A/\Nil(A)\hookrightarrow K$ and $-\otimes_{A}B$ being exact, we have an injective map $(A/\Nil(A))\otimes_{A}B\hookrightarrow K\otimes_{A}B$ injective, where $K\otimes_{A}B=\Ocal_{X}(U_{\eta})$, here denoting $U_{\eta}=\spec(B\otimes_{A}K)$. 

    It suffices to rpove that $U_{\eta}$ is nonempty, or equivalently that $\Ocal_{X}(U_{\eta})$ is nonzero. If $K\otimes_{A}B$ was zero, then $\Nil(A)B=B$ and 1 is nilpotent, a contradiction. 
\end{proof}
We can show the following, weaker, version of a base change statement. 
\begin{proposition}\label{prop: weak flat base change}
    Let $f:X\to \spec(A)$ be a separated quasicompact morphism and $A\to A'$ a flat ring extension. The Cartesian diagram 
    $$% https://q.uiver.app/#q=WzAsNCxbMCwwLCJYXFx0aW1lc197QX1cXHNwZWMoQScpIl0sWzAsMSwiXFxzcGVjKEEnKSJdLFsyLDEsIlxcc3BlYyhBKSJdLFsyLDAsIlgiXSxbMywyLCJmIl0sWzAsMywiZyJdLFswLDFdLFsxLDJdXQ==
    \begin{tikzcd}
        {X\times_{A}\spec(A')} && X \\
        {\spec(A')} && {\spec(A)}
        \arrow["g", from=1-1, to=1-3]
        \arrow[from=1-1, to=2-1]
        \arrow["f", from=1-3, to=2-3]
        \arrow[from=2-1, to=2-3]
    \end{tikzcd}$$
    induces for all quasicoherent sheaves $\Fcal$ on $X$ an isomorphism 
    $$H^{0}(X,\Fcal)\otimes_{A}A'\longrightarrow H^{0}(X\times_{A}\spec(A'),g^{*}\Fcal).$$
\end{proposition}
\begin{proof}
    Choose an affine open covering $\{U_{i}\}$ of $X$. The sheaf condition gives an exact sequence of $A$-modules
    $$0\to H^{0}(X,\Fcal)\to\prod_{i}H^{0}(U_{i},\Fcal|_{U_{i}})\to\prod_{i,j}H^{0}(U_{ij},\Fcal|_{U_{ij}}).$$
    This remains exact after base changing to $A'$, which precisely the exact sequence for $g^{*}\Fcal$ on $X\times_{A}\spec(A')$. 
\end{proof} 
We omit the proof of the strong variant, which requires spectral sequences. 
\begin{proposition}\label{prop: flat base change}
    Let $f:X\to Y$ be a separated quasicompact morphism and $g:Y'\to Y$ a flat morphism. The Cartesian diagram 
    $$% https://q.uiver.app/#q=WzAsNCxbMCwwLCJYXFx0aW1lc197WX1ZJyJdLFswLDEsIlknIl0sWzIsMSwiWSJdLFsyLDAsIlgiXSxbMywyLCJmIl0sWzAsMywiZyciXSxbMCwxLCJmJyIsMl0sWzEsMiwiZyIsMl1d
    \begin{tikzcd}
        {X\times_{Y}Y'} && X \\
        {Y'} && Y
        \arrow["{g'}", from=1-1, to=1-3]
        \arrow["{f'}"', from=1-1, to=2-1]
        \arrow["f", from=1-3, to=2-3]
        \arrow["g"', from=2-1, to=2-3]
    \end{tikzcd}$$
    induces for all quasicoherent sheaves $\Fcal$ on $X$ an isomorphism 
    $$g^{*}R^{i}f_{*}\Fcal\cong R^{i}f'_{*}g'^{*}\Fcal$$
    for all $i\geq 0$. If further $Y=\spec(A),Y'=\spec(A')$ then $H^{i}(X,\Fcal)\otimes_{A}A'\cong H^{i}(X',g'^{*}\Fcal)$. 
\end{proposition}
\begin{proof}
    See \cite[\href{https://stacks.math.columbia.edu/tag/02KH}{Tag 02KH}]{stacks-project}.
\end{proof}
Having set up some basic constructions surrounding flatness, we discuss its relation to the Hilbert polynomial. 
\begin{definition}[Hilbert Polynomial]\label{def: hilbert polynomial}
    Let $X$ be projective $k$-scheme with a choice of embedding $i:X\to\PP^{n}_{k}$ and $\Fcal$ a coherent sheaf on $X$. The Hilbert polynomial of $\Fcal$ is $P(X,\Fcal)(m)=\chi(X,\Fcal\otimes\Ocal_{\PP^{n}_{k}}(m)|_{X})$. 
\end{definition}
The Hilbert polynomial is in fact a numerical polynomial, that is, a map $\ZZ\to\ZZ$. Moreover, we can show that this polynomial characterizes flatness on the fibers. The proof of the statement is the globalization of the followimg lemma. 
\begin{lemma}\label{lem: local constant hilbert polynomial}
    Let $A$ be an integral Noetherian local ring and $\Fcal$ a coherent sheaf on $\PP^{n}_{A}$. The following are equivalent:\todo{Finish proof.}
    \begin{enumerate}[label=(\alph*)]
        \item $\Fcal$ is flat over $\spec(A)$. 
        \item $H^{0}(X,\Fcal(m))$ is a free $A$-module for $m$ sufficiently large. 
        \item The Hilbert polynomial $P(\PP^{n}_{\kappa(\pfrak)},\Fcal|_{\kappa(\pfrak)})(m)$ is independent of $\pfrak$.
    \end{enumerate}
\end{lemma}
%\begin{proof}
    %(a)$\Rightarrow$(b) We use the standard \v{C}ech cover of $\PP^{n}_{A}$ to observe that by flatness of $\Fcal$, each term $C^{i}(\Ucal,\Fcal(m))$ is a flat $A$-module. Taking $m$ large enough such that $\Fcal(m)$ has vanishing higher cohomology, we note that the long exact sequence 
    %$$0\to H^{0}(X,\Fcal(m))\to C^{0}(\Ucal,\Fcal(m))\to C^{1}(X,\Fcal(m))\to\dots\to C^{n}(X,\Fcal(m))\to0$$
    %is an acyclic resolution of $H^{0}(X,\Fcal(m))$. This implies that sequences of the form 
    %\begin{align*}
        %0&\to\ker(C^{i}(X,\Fcal(m))\to C^{i+1}(X,\Fcal(m)))\\
        %&\hspace{0.5cm}\to C^{i}(X,\Fcal(m))\to\coker(C^{i+1}(X,\Fcal(m))\to C^{i+2}(X,\Fcal(m)))\to0
    %\end{align*}
    %are short exact. The middle term $C^{i}(X,\Fcal(m))$ is flat over an integral Noetherian local ring, hence free by a result in commutative algebra (vis. \cite[\href{https://stacks.math.columbia.edu/tag/00NZ}{Tag 00NZ}]{stacks-project}). 
%\end{proof}
\begin{proposition}\label{prop: constant hilbert polynomial}
    Let $f:X\to Y$ be a projective morphism with $Y$ integral Noetherian. Let $\Fcal$ be a coherent sheaf on $Y$. $f$ is flat if and only if the Hilbert polynomial of $\Fcal$ on the fibers is constant. 
\end{proposition}
\begin{proof}
    The question is local on target, and flatness is preserved along base change, so it suffices to check the condition along the base change $\spec(\Ocal_{Y,y})\to Y$, putting us in the situation of \Cref{lem: local constant hilbert polynomial}, in which case the desired statement is immediate. 
\end{proof}
\begin{example}
    Blowups are not flat. Let $f$ be the projection from the blowup of $\A^{2}_{k}$ at the origin to $\A^{2}_{k}$. The fiber over the origin is of a higher dimension. 
\end{example}
This applies in particular to the structure sheaf. 
\begin{corollary}\label{corr: flatness by Hilbert polynomial}
    Let $f:X\to Y$ be a projective morphism with $Y$ integral Noetherian. $f$ is flat if and only if $P(X_{y},\Ocal_{X_{y}})(m)$ is constant.
\end{corollary}
\begin{proof}
    This is an immediate application of \Cref{prop: constant hilbert polynomial} to $\Ocal_{X}=\Fcal$. 
\end{proof}
Moreover, we can show that flatness is open and universally open. 
\begin{proposition}\label{prop: flatness is universally open}
    Let $f:X\to Y$ be a finite type morphism of schemes with $Y$ Noetherian. Then $f$ is open and universally open.\todo{Check.}
\end{proposition}
\begin{proof}
    We use the following input from the theory of spectral spaces: 
    \begin{itemize}
        \item If $f:X\to Y$ is a finite type morphism with $Y$ Noetherian then the image of $f$ is constructible in $Y$\cite[\href{https://stacks.math.columbia.edu/tag/054K}{Tag 054K}]{stacks-project}.
        \item Let $V\subseteq Y$ be a subset. $V$ is constructible and stable under specializations if and only if $V$ is open in $Y$ \cite[\href{https://stacks.math.columbia.edu/tag/0542}{Tag 0542}]{stacks-project}. 
    \end{itemize}
    It suffices to show that $f(X)$ is open. We know already $f(X)$ is constructible so it suffices to show that $f(X)$ is stable under generalizations. Take $x\in X$ and $y=f(x)\in Y$. Assume $y\in\overline{\{y'\}}$. We want to show $y'\in f(X)$. Let $B=\Ocal_{X,x}$ with $x\in U\subseteq X$ and $A=\Ocal_{Y,y}$ which contains the prime ideal $\pfrak_{y'}$. The ring map $A\to B$ is a local homomorphism of local rings so $\mfrak_{x}\cap A=\mfrak_{y}$ and $\pfrak_{y'}B\subseteq\mfrak_{y}B\subseteq\mfrak_{x}\subsetneq B$ showing $B\otimes_{A}A_{\pfrak_{y'}}\neq0$. If it were zero, then for all $b\in B$ there would be $t\in A_{\pfrak_{y'}}$ such that $tb=0$ but the multiplication by $t$-map $A\to A$ is injective and remains injective after base-changing to $B$. This shows that $B\otimes_{A_{\pfrak_{y'}}}\kappa(\pfrak_{y'})$ is nonzero, and hence lies in the image. 
\end{proof}
\section{Lecture 7 -- 5th May 2025}\label{sec: lecture 5}
We state without proof flatness in another setting. 
\begin{proposition}[Miracle Flatness]\label{prop: miracle flatness}
    Let $X,Y$ be smooth integral $k$-schemes and $f:X\to Y$ a morphism such that $\dim(X_{y})$ is constant. Then $f$ is flat. 
\end{proposition}
\begin{proof}
   This is local, so we can reduce to the case of $f$ induced by a ring map $B\to A$ which is \cite[\href{https://stacks.math.columbia.edu/tag/00R4}{Tag 00R4}]{stacks-project}.  
\end{proof}
Having discussed flatness, we can turn to a discussion of smoothness of morphisms, generalizing \Cref{def: smooth scheme}. 
\begin{definition}[Smooth Morphism at a Point]\label{def: smooth morphism}
    Let $f:X\to Y$ be a morphism locally of finite type. $f$ is smooth at $x\in X$ if $f$ is flat at $X$ and each fiber $X_{f(x)}$ is smooth over $\kappa(f(x))$. 
\end{definition}
\begin{remark}
    That is, $\Ocal_{X,x}$ is a flat $\Ocal_{Y,f(x)}$-module and $X_{f(x)}=X\times_{Y}\spec(\kappa(f(x)))$ is a $\kappa(f(x))$-scheme by the Cartesian diagram 
    $$% https://q.uiver.app/#q=WzAsNCxbMCwwLCJYX3tmKHgpfSJdLFswLDEsIlxcc3BlYyhcXGthcHBhKGYoeCkpKSJdLFsyLDAsIlgiXSxbMiwxLCJZLiJdLFswLDJdLFsyLDNdLFsxLDNdLFswLDFdXQ==
    \begin{tikzcd}
        {X_{f(x)}} && X \\
        {\spec(\kappa(f(x)))} && {Y.}
        \arrow[from=1-1, to=1-3]
        \arrow[from=1-1, to=2-1]
        \arrow[from=1-3, to=2-3]
        \arrow[from=2-1, to=2-3]
    \end{tikzcd}$$
\end{remark}
This definition globalizes. 
\begin{definition}[Smooth Morphism]\label{def: smooth morphism}
    Let $f:X\to Y$ be a morphism locally of finite type. $f$ is smooth if $f$ is smooth at all $x\in X$. 
\end{definition}
\begin{example}
    Let $W$ be a smooth $k$-scheme and $Y$ any $k$-scheme. Let $X=W\times_{k}Y$ with $f:X\to Y$ the natural projection. $f$ is smooth -- by miracle flatness the dimensions of the fibers are constant and are $W$ which is $\kappa(f(x))$-smooth since smoothness is preserved under base change. 
\end{example}
\begin{example}\label{ex: affine and projective space and bundles are smooth}
    Let $Y$ be any scheme. The projections $\A^{n}_{Y}\to Y, \PP^{n}_{Y}\to Y$ are smooth. More generally for $\Ecal$ locally free of finite rank on $Y$, the (total spaces of the) associated vector bundle $\VV(\Ecal)\to Y$ and projective bundle $\PP(\Ecal)\to Y$ are smooth.  
\end{example}
Recall that for $f:X\to Y$ a morphism of schemes over $k$ the \Cref{prop: tensor exact sequence} gives an exact sequence of sheaves 
$$f^{*}\Omega_{Y/k}^{1}\to\Omega_{X/k}^{1}\to\Omega_{X/Y}^{1}\to0.$$
As it turns out, smoothness of morphisms can be determined by $\Omega^{1}_{X/Y}$, in analogy to how smoothness of schemes is determined by $\Omega^{1}_{X/k}$. For this, we will require the following lemma. 
\begin{lemma}\label{lem: is closed of flat}
    Let $f:X\to Y$ be a locally finite type morphism, $x\in X$ with $f(x)=y$ and $d=\dim(X_{y})$. There exists a closed immersion $X\to W$ over $Y$ such that $W$ is smooth at $x$ over $y$ of dimension $d$. 
\end{lemma}
\begin{proof}
   Since $f$ is locally of finite type, we can factor $f$ as 
   $$% https://q.uiver.app/#q=WzAsMyxbMCwwLCJYIl0sWzIsMCwiXFxBXntufV97WX0iXSxbMSwxLCJZIl0sWzEsMiwicCJdLFswLDIsImYiLDJdLFswLDEsImkiXV0=
   \begin{tikzcd}
       X && {\A^{n}_{Y}} \\
       & Y
       \arrow["i", from=1-1, to=1-3]
       \arrow["f"', from=1-1, to=2-2]
       \arrow["p", from=1-3, to=2-2]
   \end{tikzcd}$$
   where $i$ is a closed immersion and $p$ is the projection. As in \Cref{ex: affine and projective space and bundles are smooth}, $p$ is smooth so $\Omega^{1}_{\A^{n}_{Y}/Y}$ is locally free of rank $n$, but not of rank $d$. We find a closed subscheme of $\A^{n}_{Y}$ such that the sheaf of relative differentials is locally free of rank $d$. 

   Let $\Ical_{X}$ be the ideal sheaf of $X$ in $\A^{n}_{Y}$ so by \Cref{prop: ideal exact sequence}, we have an exact sequence 
   \begin{equation}\label{eqn: ideal exact sequence technical lemma}
    \Ical_{X}/\Ical_{X}^{2}\to\Omega^{1}_{\A^{n}_{Y}/Y}|_{i(X)}\to\Omega_{X/Y}\to0.
   \end{equation}
   Now letting $\Jcal$ be the ideal sheaf of $X_{y}\in\A^{n}_{\kappa(y)}$ we have an exact sequence 
   \begin{equation}\label{eqn: exact sequence technical lemma}
    \Jcal\to\Omega^{1}_{\A^{n}_{\kappa(y)}/\kappa(y)}\to\Omega_{X_{y}/\kappa(y)}^{1}\to0
   \end{equation}
   where $\Omega^{1}_{\A^{n}_{\kappa(y)}/\kappa(y)}$ is locally free of rank $n$ and $\Omega_{X_{y}/\kappa(y)}^{1}$ is locally free of rank $d$. Then 
   $$\ker\left(\Omega^{1}_{\A^{n}_{\kappa(y)}/\kappa(y)}\to\Omega_{X_{y}/\kappa(y)}^{1}\right)$$ 
   is locally free of rank $n-d$. Base changing (\ref{eqn: ideal exact sequence technical lemma}) and (\ref{eqn: exact sequence technical lemma}) by $\kappa(x)$, we get a diagram 
   $$% https://q.uiver.app/#q=WzAsOCxbMCwwLCJcXEljYWxfe1h9L1xcSWNhbF97WH1eezJ9XFxvdGltZXNcXGthcHBhKHgpIl0sWzIsMCwiXFxPbWVnYV57MX1fe1xcQV57bn1fe1l9L1l9fF97aShYKX1cXG90aW1lc1xca2FwcGEoeCkiXSxbNCwwLCJcXE9tZWdhXnsxfV97WC9ZfVxcb3RpbWVzXFxrYXBwYSh4KSJdLFswLDEsIlxcSmNhbFxcb3RpbWVzXFxrYXBwYSh4KSJdLFsyLDEsIlxcT21lZ2FeezF9X3tcXEFee259X3tcXGthcHBhKHkpfS9cXGthcHBhKHkpfVxcb3RpbWVzXFxrYXBwYSh4KSJdLFs0LDEsIlxcT21lZ2FeezF9X3tYX3t5fS9cXGthcHBhKHkpfVxcb3RpbWVzXFxrYXBwYSh4KSJdLFs1LDAsIjAiXSxbNSwxLCIwIl0sWzAsMV0sWzEsMl0sWzQsNV0sWzMsNF0sWzAsMywiIiwxLHsic3R5bGUiOnsiaGVhZCI6eyJuYW1lIjoiZXBpIn19fV0sWzEsNCwiXFx3ciIsMl0sWzIsNSwiXFx3ciIsMl0sWzIsNl0sWzUsN11d
   \begin{tikzcd}
       {\Ical_{X}/\Ical_{X}^{2}\otimes\kappa(x)} && {\Omega^{1}_{\A^{n}_{Y}/Y}|_{i(X)}\otimes\kappa(x)} && {\Omega^{1}_{X/Y}\otimes\kappa(x)} & 0 \\
       {\Jcal\otimes\kappa(x)} && {\Omega^{1}_{\A^{n}_{\kappa(y)}/\kappa(y)}\otimes\kappa(x)} && {\Omega^{1}_{X_{y}/\kappa(y)}\otimes\kappa(x)} & 0
       \arrow[from=1-1, to=1-3]
       \arrow[two heads, from=1-1, to=2-1]
       \arrow[from=1-3, to=1-5]
       \arrow["\wr"', from=1-3, to=2-3]
       \arrow[from=1-5, to=1-6]
       \arrow["\wr"', from=1-5, to=2-5]
       \arrow[from=2-1, to=2-3]
       \arrow[from=2-3, to=2-5]
       \arrow[from=2-5, to=2-6]
   \end{tikzcd}$$
   where the injectivity of the right two vertical arrows implies surjectivity of the leftmost vertical arrow. 

   If $n>d$, there is $g\in\Ical_{X}$ and $g_{2},\dots,g_{n}\in\Ocal_{\A^{n}_{\kappa(y)}}$ such that $\dform g,\dform g_{2},\dots,\dform g_{n}$ freely generate $\Omega^{1}_{\A^{n}_{\kappa(y)}/\kappa(y)}\otimes\kappa(x)$. By Nakayama's lemma, these lift to basis of $\Omega^{1}_{\A^{n}_{\kappa(y)}/\kappa(y)}\otimes\kappa(x')$ for $x'$ in a neighborhood of $x$ in $X$. Observe $X\subseteq V(g)=W\subseteq \A^{n}_{Y}$ then $\dim(W)=\dim(\A^{n}_{Y})-1$ and with $\Omega_{W'_{y}/\kappa(y)}^{1}=\Omega_{\A^{n}_{\kappa(y)}/\kappa(y)}/\dform g$ which is locally free of rank $n-1$. Moreover, since $g$ is a nonzerodivisor in $\Ocal_{X,x}/\mfrak_{y}\Ocal_{X,x}$, $W_{y}\to y$ is flat by \Cref{prop: properties of flatness rings and modules} (vi). 
   
   Iterating this process, we arrive at the desired $W$. 
\end{proof}
\begin{example}
    Let $X=\{y\}\hookrightarrow Y$ a closed point. $\Omega_{X/Y}^{1}=0$ and the fiber dimension is 0, but taking $Z=Y$, $\id_{Y}:Y\to Y$ is smooth. 
\end{example}
We now state and prove the desired result. 
\begin{proposition}\label{prop: smoothness via relative cotangent}
    Let $f:X\to Y$ be a morphism locally of finite type with all fibers of pure dimension $d$. Then $f$ is smooth if and only if $f$ is flat and $\Omega_{X/Y}^{1}$ is locally free of rank $d$. 
\end{proposition}
\begin{proof}
    Recall by \Cref{prop: differentials of pushouts} that for a Cartesian square 
    $$% https://q.uiver.app/#q=WzAsNCxbMCwwLCJYJyJdLFswLDEsIlknIl0sWzIsMCwiWCJdLFsyLDEsIlkiXSxbMCwyLCJnIl0sWzIsM10sWzEsM10sWzAsMV1d
    \begin{tikzcd}
        {X'} && X \\
        {Y'} && Y
        \arrow["g", from=1-1, to=1-3]
        \arrow[from=1-1, to=2-1]
        \arrow[from=1-3, to=2-3]
        \arrow[from=2-1, to=2-3]
    \end{tikzcd}$$
    we have $g^{*}\Omega^{1}_{X/Y}\cong\Omega_{X'/Y'}$ so in particular when $Y'=\spec(\kappa(y))$ then $X'=X_{y}$ and we have $\Omega^{1}_{X_{y}/\kappa(y)}\cong\Omega^{1}_{X/Y}|_{X_{y}}$. We now begin the proof in earnest. 

    $(\Rightarrow)$ Now assume $f$ is smooth. By definition, $f$ is flat, and $\Omega_{X_{y}/\kappa(y)}^{1}$ is locally free of rank $\dim(X_{y})$. In particular, the rank of $\Omega^{1}_{X_{y}/\kappa(y)}\otimes\kappa(x)$ is $d$ for all $x\in X_{y}$. By base change, $\Omega^{1}_{X_{y}/\kappa(y)}\otimes\kappa(x)$ and $\Omega^{1}_{X/Y}\otimes\kappa(x)$ are of the same rank. If $X$ is reduced, this already implies that $\Omega_{X/Y}^{1}$ is locally free of the correct rank. 

    If $X$ is not reduced, we use the factorization produced by \Cref{lem: is closed of flat} to get a factorization 
    $$% https://q.uiver.app/#q=WzAsMyxbMCwwLCJYIl0sWzIsMCwiVyJdLFsxLDEsIlkiXSxbMCwyLCJmIiwyXSxbMCwxLCJpIl0sWzEsMiwicCJdXQ==
    \begin{tikzcd}
        X && W \\
        & Y
        \arrow["i", from=1-1, to=1-3]
        \arrow["f"', from=1-1, to=2-2]
        \arrow["p", from=1-3, to=2-2]
    \end{tikzcd}$$
    where $i$ is a closed immersion and $p$ is smooth with fiber dimension $d$. We have $X_{y}\subseteq W_{y}$ of the same dimension, but $W_{y}$ is smooth hence reduced, so $X_{y}=W_{y}$ yielding the claim. 

    $(\Leftarrow)$ Suppose $f$ is flat and $\Omega_{X/Y}^{1}$ is locally free of rank $d$. The restriction to the fiber $\Omega_{X_{y}/\kappa(y)}^{1}$ is locally free of rank $\dim(X_{y})$ over $\kappa(y)$ hence smooth over $\kappa(y)$ showing the morphism is smooth. 
\end{proof}
As in the case of smoothness \Cref{prop: smooth on open}, the smooth locus of a morphism can be shown to be open on the source. 
\begin{proposition}\label{prop: smooth locus is open on source}
    Let $f:X\to Y$ be a dominant morphism of finite type integral schemes over $k$ with $K(X)/K(Y)$ separable. There exists $U\subseteq X$ dense open such that $f|_{U}:U\to Y$ is smooth. 
\end{proposition}
\begin{proof}
    Since $K(X)/K(Y)$ is separable, $\Omega^{1}_{K(X)/K(Y)}$ is a $K(X)$-vector space of rank the transendence degree of $K(X)/K(Y)$, which is $d=\dim(X)-\dim(Y)$. By \Cref{lem: free modules over local rings}, we have that there is an open set $U$ such that $\Omega^{1}_{X/Y}|_{U}\cong\Omega^{1}_{U/Y}$ is locally free of rank $d$. 

    By \Cref{prop: flatness is universally open}, we can by restricting further take $U$ to be the flat locus of the morphism, which may be empty. On $U$, the fibers over $y\in f(U)$ are of dimension $d$ by \Cref{corr: fiberwise flatness} (applied to $U$ and not $Y$). So by \Cref{prop: smoothness via relative cotangent}, $f|_{U}$ is smooth. 
\end{proof}
\begin{remark}
    The special case of \Cref{prop: smooth locus is open on source} where $Y=\spec(k)$ is precisely \Cref{prop: smooth on open}. 
\end{remark}
Let us consider some examples. 
\begin{example}\label{ex: cotangent ses}
    A smooth morphism between smooth schemes extends the exact sequence of \Cref{prop: tensor exact sequence} to a short exact sequence. Let $f:X\to Y$ be a smooth morphism between smooth $k$-schemes. There is a short exact sequence 
    \begin{equation}\label{eqn: cotangent ses}
        0\to f^{*}\Omega^{1}_{Y/k}\to\Omega_{X/k}^{1}\to\Omega_{X/Y}\to0.
    \end{equation}
    To see this, we observe that since $\Omega_{X/Y}^{1}$ is locally free of rank $\dim(X)-\dim(Y)$ and $\Omega_{X/k}^{1}$ is locally free of rank $\dim(X)$ implying $f^{*}\Omega_{Y/k}^{1}$ locally free of rank the dimension of $Y$. This implies that $f^{*}\Omega_{Y/k}^{1}\to\ker\left(\Omega^{1}_{X/k}\to\Omega^{1}_{X/Y}\right)$ is surjective, but any surjection of locally free modules of the same rank is an isomorphism, whence the claim. 
\end{example}
\begin{example}\label{ex: normal bundle es}
    Let $f:X\to Y$ be a smooth morphism between smooth $k$-schemes. Dualizing (\ref{eqn: cotangent ses}) of \Cref{ex: cotangent ses}, we get 
    $$0\to\Tcal_{X/Y}\to\Tcal_{X/k}\to f^{*}\Tcal_{Y/k}\to0$$
    which on restriction to any closed fiber $X_{y}$ we have 
    $$0\to\Tcal_{X/Y}|_{X_{y}}\cong\Tcal_{X_{y}}\to\Tcal_{X}|_{X_{y}}\to f^{*}\Tcal_{Y}|_{y}\cong\Ncal_{X_{y}/Y}\to0$$
    recovering the normal bundle sequence. 
\end{example}
Observe that for $f:X\to Y$ a smooth morphism between smooth $k$-schemes with $k$-algebraically closed we have for all $x\in X$ an induced map on the Zariski tangent spaces $T_{X,x}\to T_{Y,f(x)}\otimes\kappa(x)$. In particular, if $x\in X(k)$ we have $\kappa(x)\cong k$ and thus $\kappa(y)\cong k$ as $\kappa(y)$ lies in the intermediate extension $\kappa(x)/\kappa(y)/k$. By the injectivity of the idnuced map $\mfrak_{y}/\mfrak_{y}^{2}\to\mfrak_{x}/\mfrak_{x}^{2}$, we have surjectivity of the map on Zariski tangent spaces. This in fact determines smoothness of a morphism under appropriate hypotheses. 
\begin{proposition}\label{prop: surjectivity of tangent spaces and smoothness}
    Let $f:X\to Y$ be a morphism between smooth $k$-schemes with $k$ algebraically closed. If the induced map $T_{X,x}\to T_{Y,f(x)}$ is surjective for all $x\in X(k)$ then $f$ is smooth. 
\end{proposition}
\begin{proof}
    We show first that $f$ is flat. Since flatness is an open condition, it suffices to show flatness on each closed point, since each point lies in an open neighborhood of some closed point. 

    By injectivity of $\mfrak_{y}/\mfrak_{y}^{2}\hookrightarrow\mfrak_{x}/\mfrak_{x}^{2}$ we pick generators $(\overline{a}_{1},\dots,\overline{a}_{n})$ where $n=\dim(Y)$ with images $\overline{b}_{1},\dots\overline{b}_{n}$ for $b_{i}\in\mfrak_{x}$. Injectivity implies that the $\overline{b}_{i}$ remain linearly independent in the quotient so $b_{1},\dots,b_{n}$ is a regular sequence in $\Ocal_{X,x}$ -- each $b_{i}$ is a nonzerodivisor in $\Ocal_{X,x}/(b_{1},\dots,b_{i-1})$. By induction, $\Ocal_{X,x}/(b_{1},\dots,b_{n})$ is flat over $\kappa(y)=\Ocal_{Y,f(x)}/(a_{1},\dots,a_{n})$. This gives flatness. 
    
    Use \Cref{prop: tensor exact sequence}, we have 
    $$f^{*}\Omega^{1}_{Y/k}\to\Omega_{X/k}^{1}\to\Omega_{X/Y}^{1}\to0$$
    where the former two terms are locally free and surjectivity of the Zariski tangent space implies that the map $f^{*}\Omega^{1}_{Y/k}\to\Omega_{X/k}^{1}$ is fiberwise injective. Hence $\Omega^{1}_{X/Y}$ is locally free of $\dim(X)-\dim(Y)$ showing smoothness too. 
\end{proof}
We hint towards an algebro-geometric version of Sard's theorem via the following lemma, which will be used in the proof of the result. 
\begin{lemma}\label{lem: rank locus is bounded}
    Let $f:X\to Y$ be smooth finite type $k$-schemes with $k$ of characterositic 0. Define 
    $$X_{r}=\overline{\left\{x\in X_{\mathrm{cl}}:\mathrm{rank}(T_{X,x}\to T_{Y,f(x)}\otimes\kappa(x))\leq r\right\}}.$$
    Then $\dim(\overline{f(X_{r})})\leq r$. 
\end{lemma}
\begin{proof}
    Pick an irreducible component $X'\subseteq\overline{X_{r}}$ such that $f':X'\to Y'$ is dominant. Given $X',Y'$ the induced reduced subscheme structure so $X',Y'$ are integral. Using that $K(X')/K(Y')$ is separble since $K(Y')$ is of characteristic 0, there exists by \Cref{prop: smooth locus is open on source} $U'$ of $X'$ such that $f'|_{U'}:U'\to Y'$ is smooth. Then take $x\in U'\cap X_{r}$ and contemplate the diagram 
    $$% https://q.uiver.app/#q=WzAsNCxbMCwwLCJUX3tVJyx4fSJdLFsyLDAsIlRfe1gseH0iXSxbMCwxLCJUX3tZJyxmKHgpfSJdLFsyLDEsIlRfe1ksZih4KX0iXSxbMCwxLCIiLDAseyJzdHlsZSI6eyJ0YWlsIjp7Im5hbWUiOiJob29rIiwic2lkZSI6InRvcCJ9fX1dLFsyLDMsIiIsMCx7InN0eWxlIjp7InRhaWwiOnsibmFtZSI6Imhvb2siLCJzaWRlIjoidG9wIn19fV0sWzEsM10sWzAsMiwiIiwxLHsic3R5bGUiOnsiaGVhZCI6eyJuYW1lIjoiZXBpIn19fV1d
    \begin{tikzcd}
        {T_{U',x}} && {T_{X,x}} \\
        {T_{Y',f(x)}} && {T_{Y,f(x)}}
        \arrow[hook, from=1-1, to=1-3]
        \arrow[two heads, from=1-1, to=2-1]
        \arrow[from=1-3, to=2-3]
        \arrow[hook, from=2-1, to=2-3]
    \end{tikzcd}$$
    where the injective horizontal maps and the right vertical map being of rank $r$ imply surjectivity of the left vertical map and smoothness of $U'\to Y'$. Indeed, it suffices to take $x$ to be $k$-rational, since the preceding discussion holds on base change.Thus $T_{Y',f(x)}$ is of rank at most $r$, showing $\dim(\overline{f(X_{r})})\leq r$, as desired. 
\end{proof}
\section{Lecture 8 -- 8th May 2025}\label{sec: lecture 8}
We can now state and prove the algebraic variant of Sard's theorem. 
\begin{theorem}[Algebraic Sard]\label{thm: algebraic sard}
    Let $X,Y$ be smooth integral $k$-schemes with $k$ of charactersitic 0 and $f:X\to Y$ a morphism of $k$-schemes. There exists a dense set $V\subseteq Y$ such that $f^{-1}(V)\to V$ is smooth. 
\end{theorem}
\begin{remark}
    Note that $V$ is nonempty but $f^{-1}(V)=\emptyset$ can occur, for example where $f$ is a closed embedding. 
\end{remark}
\begin{proof}
    Denote the set 
    $$X_{r}=\overline{\left\{x\in X_{\mathrm{cl}}:\mathrm{rank}(T_{X,x}\to T_{Y,f(x)}\otimes\kappa(x))\leq r\right\}}$$
    where we have $\dim(\overline{f(X_{r})})\leq r$ by \Cref{lem: rank locus is bounded}. Apply this to $r=\dim(Y)-1$ so $\overline{f(X_{\dim(Y)-1})}$ is a proper closed subset. Let $V$ be its open complement in $Y$ which is nonempty since $\overline{f(X_{\dim(Y)-1})}$ is a proper closed subset. For all points $x\in X$, with image contained in $V$, we have that $T_{X,x}\to T_{Y,f(x)}\otimes\kappa(x)$ cannot have rank smaller than the dimension of $Y$, so the map is surjective as a map of $\kappa(x)$-vector spaces, hence smooth by \Cref{prop: surjectivity of tangent spaces and smoothness}. 
\end{proof}
There is an easy counterexample in positive characteristic. 
\begin{example}
    Let $k$ be of characteristic $p$. The relative Frobenius $\A^{2}_{k}\to\A^{2}_{k}$ with all fibers non-reduced, so the morphism is nowhere smooth. 
\end{example}
Algebraic Sard's theorem allows us to show that smoothness is generic on hyperplane sections of a scheme, in the sense that the set of smooth hyperplane sections of a quasiprojective scheme is dense in the set of hyperplanes. Let us be more precise: let $X\subseteq\PP^{n}_{k}$ be a smooth quasiprojective variety. For a line bundle $\Lcal\in\Pic(X)$ and $V\subseteq H^{0}(X,\Lcal)$ a basepoint free linear subspace, we can define a morphism $\varphi_{V}:X\to\PP(V^{\vee})$ with the property that $f^{*}\Ocal_{\PP(V^{\vee})}(1)\cong\Lcal$. There exists a nonempty open $U\subseteq V$ such that for all $s\in U$, $V_{X}(s)=\varphi^{-1}(V_{+}(s))\subseteq X$ is smooth. 

For this, we recall the following facts about the universal hypersurface. 
\begin{lemma}\label{lem: universal hypersurface}
    Let $\Ucal_{n,1}\subseteq\PP^{n}_{k}\times_{k}\PP^{N}_{k}$ for $N=n$ be the universal hyperplane defined by $\sum_{i=0}^{n}a_{i}x_{i}$. Then $\pr_{\PP^{N}_{k}}$ is a projective bundle of dimension $n-1$. 
\end{lemma}
\begin{proof}
    It suffices to observe that the fiber over any point $(a_{0},\dots,a_{n})$ is the hyperplane $V_{+}(\sum_{i=0}^{n}a_{i}x_{i})\subseteq\PP^{n}_{k}$. 
\end{proof}
With this in hand, we state and prove the desired result. 
\begin{theorem}[Bertini]\label{thm: Bertini}
    Let $X\subseteq\PP^{n}_{k}$ be a smooth quasiprojective variety with $k$ algebraically closed of characteristic 0. For a line bundle $\Lcal\in\Pic(X)$ and $V\subseteq H^{0}(X,\Lcal)$ a basepoint free linear subspace. There is a nonempty open subset $U\subseteq V$ such that for all $s\in U$, $V_{X}(s)\subseteq X$ is smooth. 
\end{theorem}
\begin{proof}
    Construct the universal hyperplane 
    $$% https://q.uiver.app/#q=WzAsNCxbMiwwLCJcXFVjYWxfe24sMX0iXSxbNCwwLCJcXFBQXntOfV97a309fFxcT2NhbF97XFxQUChWXntcXHZlZX0pfSgxKXwiXSxbMiwxLCJcXFBQKFZee1xcdmVlfSkiXSxbMCwwLCJcXFBQKFZee1xcdmVlfSlcXHRpbWVzX3trfVxcUFBee059X3trfSJdLFswLDIsIlxccHJfe1xcUFAoVl57XFx2ZWV9KX0iLDJdLFswLDEsIlxccHJfe1xcUFBee059X3trfX0iXSxbMCwzLCIiLDIseyJzdHlsZSI6eyJ0YWlsIjp7Im5hbWUiOiJob29rIiwic2lkZSI6ImJvdHRvbSJ9fX1dXQ==
    \begin{tikzcd}
        {\PP(V^{\vee})\times_{k}\PP^{N}_{k}} && {\Ucal_{n,1}} && {\PP^{N}_{k}=|\Ocal_{\PP(V^{\vee})}(1)|} \\
        && {\PP(V^{\vee})}
        \arrow[hook', from=1-3, to=1-1]
        \arrow["{\pr_{\PP^{N}_{k}}}", from=1-3, to=1-5]
        \arrow["{\pr_{\PP(V^{\vee})}}"', from=1-3, to=2-3]
    \end{tikzcd}$$
    We have that $\pr_{\PP^{N}_{k}}:\Ucal_{n,1}\to\PP^{N}_{k}$ is a projective bundle isomorphic to $\PP(\Ecal)$ where 
    $$\Ecal=\ker\left(H^{0}(\PP(V^{\vee}),\Ocal_{\PP(V^{\vee})}(1))\otimes\Ocal_{\PP(V^{\vee})}(1)\to\Ocal_{\PP(V^{\vee})}(1)\right)$$
    since we have a diagram 
    $$% https://q.uiver.app/#q=WzAsNixbMCwxLCJYXFx0aW1lc197a31cXFBQXntufV97a30iXSxbMCwyLCJcXFhjYWw9WFxcdGltZXNfe1xcUFAoVl57XFx2ZWV9KX1cXFVjYWxfe24sMX0iXSxbMiwxLCJcXFBQKFZee1xcdmVlfSlcXHRpbWVzX3trfVxcUFBee059X3trfSJdLFs0LDEsIlxcVWNhbF97biwxfSJdLFs1LDAsIlxcUFAoVl57XFx2ZWV9KSJdLFs1LDIsIlxcUFBee059X3trfSJdLFsxLDAsIiIsMix7InN0eWxlIjp7InRhaWwiOnsibmFtZSI6Imhvb2siLCJzaWRlIjoidG9wIn19fV0sWzMsMiwiIiwyLHsic3R5bGUiOnsidGFpbCI6eyJuYW1lIjoiaG9vayIsInNpZGUiOiJib3R0b20ifX19XSxbMyw1LCJcXHByX3tcXFBQXntOfV97a319IiwyXSxbMCwyLCJcXHZhcnBoaVxcdGltZXNcXGlkX3tcXFBQXntOfV97a319Il0sWzMsNCwiXFxwcl97XFxQUChWXntcXHZlZX0pfSJdXQ==
    \begin{tikzcd}
        &&&&& {\PP(V^{\vee})} \\
        {X\times_{k}\PP^{n}_{k}} && {\PP(V^{\vee})\times_{k}\PP^{N}_{k}} && {\Ucal_{n,1}} \\
        {\Xcal=X\times_{\PP(V^{\vee})}\Ucal_{n,1}} &&&&& {\PP^{N}_{k}}
        \arrow["{\varphi\times\id_{\PP^{N}_{k}}}", from=2-1, to=2-3]
        \arrow["{\pr_{\PP(V^{\vee})}}", from=2-5, to=1-6]
        \arrow[hook', from=2-5, to=2-3]
        \arrow["{\pr_{\PP^{N}_{k}}}"', from=2-5, to=3-6]
        \arrow[hook, from=3-1, to=2-1]
    \end{tikzcd}$$
    where in particular $\pr_{\PP^{N}_{k}}$ is a $\PP^{\dim(V)-2}_{k}$-bundle. Denote $\pi$ the composite $\Xcal\to\PP^{N}_{k}$ obtained by the diagram above. Apply \Cref{thm: algebraic sard} to $\pi$ so there exists $W\subseteq\PP^{N}_{k}$ open such that $\pi^{-1}(W)\to W$ is smooth. For $g:V\setminus\{0\}\to (V\setminus\{0\})/k^{\times}$ be the quotient map and define $U=g^{-1}(W\cap(\PP^{n}_{k})_{\mathrm{cl}})$. Then for all $s\in U$, $V_{+}(s)$ is smooth. 
\end{proof}
\begin{remark}
    If $k$ is not algebraically closed, we need that $W\cap (V\setminus\{0\})/k^{\times}=W(k)$ is nonempty.  
\end{remark}
\begin{remark}
    The set $U$ is open only if $X$ is projective. 
\end{remark}
\begin{remark}
    By passing to the Veronese embedding, the proof of Bertini's theorem holds for hypersurface sections. 
\end{remark}
We can now see some examples. 
\begin{example}[Katz]
    Let $k=\FF_{q}$. $V_{+}(\sum_{i=0}^{n}x_{i}y_{i}^{q}-x^{q}_{i}y_{i})\subseteq\PP^{2n+1}_{k}$ is smooth but no hyperplane section of $X$ is smooth. 
\end{example}
\begin{example}[Poonen]
    Fix $\FF_{q}$. For $n\geq 2,d\geq1$ there exists $X\subseteq\PP^{n}_{\FF_{q}}$ of degree $d$ such that all hypersurface sections of degree at most $d$ are singular. 
\end{example}
\begin{example}[Poonen]
    Recall the definition of the zeta function of a scheme $\zeta_{X}(s)=\exp(\sum_{i=0}^{\infty}\frac{X(\FF_{q^{r}})}{r}q^{-rs})$. Fix $X\subseteq\PP^{n}_{\FF_{q}}$ smooth. 
    $$\frac{\{f\in\FF_{q}[x_{0},\dots,x_{n}]_{d}:V_{+}(f)\cap X\text{ smooth}\}}{\{f\in\FF_{q}[x_{0},\dots,x_{n}]_{d}\}}\sim_{d\to\infty}\zeta_{X}(\dim(X)).$$
\end{example}
\begin{example}
    Let $\Ucal_{n,d}$ be the universal hypersurface of degree $d$ in $\PP^{n}_{k}$. There is a proper closed subset of $|\Ocal_{\PP^{n}_{k}}(d)|$ parametrizing the singular hypersurfaces of degree $(d-1)^{n+1}(n+3)$. 
\end{example}
We begin a discussion of unramified morphisms. 
\begin{definition}[Unramified Morphism at a Point]\label{def: unramified morphism at a point}
    Let $f:X\to Y$ be a locally finite type morphism betewen locally Noetherian schemes. $f$ is unramified at $x\in X$ with image $y=f(x)$ if $\mfrak_{y}\Ocal_{X,x}=\mfrak_{x}$ and $\kappa(x)/\kappa(y)$ is separable. 
\end{definition}
\begin{definition}[Unramified Morphism]\label{def: unramified morphism at a point}
    Let $f:X\to Y$ be a locally finite type morphism betewen locally Noetherian schemes. $f$ is unramified if it is unramified at all $x\in X$. 
\end{definition}
\begin{remark}\label{rmk: immersion}
    Unramified morphisms are the algebro-geometric analogue of immersions in differential topology. 
\end{remark}
Let us now consider some examples. 
\begin{example}
    Let $K/k$ be a separable algebraic extension. Then $\spec(K)\to\spec(k)$ is unramified. 
\end{example}
\begin{example}
    Let $L/K/\QQ$ be number fields inducing on rings of integers $\ZZ\to\Ocal_{K}\xrightarrow{\varphi}\Ocal_{L}$. This induces a map $\spec(\Ocal_{L})\to\spec(\Ocal_{K})$. The map is unramified at the generic point and at all closed points $\qfrak\subseteq\Ocal_{L},\pfrak=\varphi^{-1}(\qfrak)$ that $(\qfrak\cap\Ocal_{K,\pfrak})\Ocal_{L,\qfrak}=\qfrak\subseteq\Ocal_{L,\qfrak}$. 
\end{example}
\begin{example}
    $\A^{1}_{k}\to\A^{1}_{k}$ by $x\mapsto x^{2}$ is ramified at $(x)$ as $k[x]_{(x)}$ is not generated by $x^{2}$, but unramified at all nonzero points away from characteristic 2. Conversely, the morphism is ramified everywhere in characteristic 2. 
\end{example}
\begin{example}
    Closed and open immersions are always unramified. 
\end{example}
\begin{example}
    $\spec(k[x]/(f))\to\spec(k)$ is unramifed over the factors $f_{i}$ of $f$ that $k[x]/(f_{i})$ are separable. 
\end{example}
As it turns out, unramifiedness can be tested on the relative K\"{a}hler differentials, or equivalently on the diagonal being an open embedding. To show this, we will first require the following lemma. 
\begin{lemma}\label{lem: fiberwise unramifiedness}
    Let $f:X\to Y$ be a locally finite type morphism betewen locally Noetherian schemes. $f$ is unramified if and only if for all points $y\in Y$, the fiber $X_{y}$ is reduced, locally finite, and for all $x\in X_{y}$ the field extension $\kappa(x)/\kappa(y)$ is separable. 
\end{lemma}
\begin{proof}
    $(\Rightarrow)$ Assume that $f$ is unramified. We have that $\Ocal_{X_{y},x}\cong\Ocal_{X,x}/\mfrak_{y}\Ocal_{X,x}$ where by unramifiedness, $\mfrak_{y}\Ocal_{X,x}\cong\mfrak_{x}$ so $\Ocal_{X_{y},x}\cong\kappa(x)$ is reduced so the fiber is reduced and locally finite. 

    $(\Leftarrow)$ Suppose for all $y\in Y$, the fiber $X_{y}$ is reduced, locally finite, and the extension of residue fields is separable. We have an injection $\kappa(y)\hookrightarrow\Ocal_{X_{y},x}$ so $\Ocal_{X_{y},x}$ is zero-dimensional, reduced, and locally finite so $\Ocal_{X_{y},x}=\kappa(x)$ showing $\mfrak_{y}\Ocal_{X,x}=\mfrak_{x}$, as desired. 
\end{proof}
\section{Lecture 9 -- 12th May 2025}\label{sec: lecture 9}
We characterize unramifiedness in terms of triviality of the sheaf of relative K\"{a}hler differentials. 
\begin{proposition}\label{prop: unramified iff trivial relative differentials}
    Let $f:X\to Y$ be a locally finite type morphism betewen locally Noetherian schemes. $f$ is unramified if and only if $\Omega^{1}_{X/Y}=0$. 
\end{proposition}
\begin{proof}
    $(\Rightarrow)$ For $y\in Y$, we have $\Omega^{1}_{X/Y}|_{X_{y}}=\Omega^{1}_{X_{y}/\kappa(y)}$ so $\Omega^{1}_{X/Y}\otimes\kappa(x)\cong\Omega^{1}_{X_{y}/\kappa(y)}\otimes\kappa(x)$. \Cref{lem: fiberwise unramifiedness}, we have locally $X_{y}=\spec(\kappa(x))$ and $\kappa(x)/\kappa(y)$ separable, so $\Omega^{1}_{\kappa(x)/\kappa(y)}=0$. So since for each $y\in Y$ we have $\Omega^{1}_{X_{y}/\kappa(y)}=0$, $\Omega^{1}_{X/Y}=0$ too. 

    $(\Leftarrow)$ Assume $\Omega^{1}_{X/Y}=0$ and $X_{y}=\spec(A)$ locally finite type over $\kappa(y)$. We have $\Omega^{1}_{X_{y}/\kappa(y)}=\Omega^{1}_{X/Y}|_{X_{y}}=0$. Using the exact sequence 
    $$\mfrak_{x}/\mfrak_{x}^{2}\to\Omega^{1}_{A/\kappa(y)}\to\Omega^{1}_{\kappa(x)/\kappa(y)}\to0$$
    we have $\Omega^{1}_{A/\kappa(y)}=0$ by assumption and $\Omega^{1}_{\kappa(x)/\kappa(y)}$ by separability. Thus $\mfrak_{x}/\mfrak_{x}^{2}=0$ and using Nakayama's lemma, $\mfrak_{x}=0$ as well. Thus $\mfrak_{y}\Ocal_{X,x}=\mfrak_{x}$ showing $\mfrak_{y}=0$. We can then conclude $f$ is unramified by \Cref{lem: fiberwise unramifiedness}. 
\end{proof}
\begin{example}
    Let $X$ be the nodal affine plane curve. The map from the normalization is unramified. 
\end{example}
We show the property alluded to in \Cref{rmk: immersion}. 
\begin{lemma}\label{lem: unramified implies immersion}
    Let $f:X\to Y$ be a locally finite type morphism betewen locally Noetherian schemes. If $f$ is unramified, then $T_{X,x}\to T_{Y,y}\otimes\kappa(x)$ is injective. 
\end{lemma}
\begin{proof}
    Dualizing, we show that $\frac{\mfrak_{y}}{\mfrak_{y}^{2}}\otimes_{\Ocal_{Y,y}}\kappa(x)\to\frac{\mfrak_{x}}{\mfrak_{x}^{2}}$ is surjective. We can write $\mfrak_{x}=\mfrak_{y}\otimes\Ocal_{X,x}$ so there is an obvious surjection $\frac{(\mfrak_{y}\otimes\Ocal_{X,x})}{(\mfrak_{y}\otimes\Ocal_{X,x})^{2}}\to\frac{\mfrak_{x}}{\mfrak_{x}^{2}}$, whence the claim. 
\end{proof}
\begin{lemma}\label{lem: unramified implies surjective differentials}
    Let $f:X\to Y$ be a locally finite type morphism betewen locally Noetherian schemes. $f$ is unramified if and only if $f^{*}\Omega_{Y/k}^{1}\to\Omega^{1}_{X/k}$ is surjective. 
\end{lemma}
\begin{proof}
    The cokernel of $f^{*}\Omega_{Y/k}^{1}\to\Omega^{1}_{X/k}$ has cokernel $\Omega^{1}_{X/Y}$ so the map is surjective if and only if $\Omega^{1}_{X/Y}=0$, that is, if $f$ is unramified. 
\end{proof}
We can now define \'{e}taleness. 
\begin{definition}[\'{E}tale Morphism at a Point]\label{def: etale at point}
    Let $f:X\to Y$ be a locally finite type morphism betewen locally Noetherian schemes and $x\in X$. $f$ is \'{e}tale at $x$ if $f$ is flat and unramified at $x$. 
\end{definition}
\begin{definition}[\'{E}tale Morphism]\label{def: etale morphism}
    Let $f:X\to Y$ be a locally finite type morphism betewen locally Noetherian schemes. $f$ is \'{e}tale if $f$ is flat and unramified. 
\end{definition}
The normalization of the nodal affine plane curve is not \'{e}tale. 
\begin{example}
    Let $X$ be the nodal affine plane curve. The map from the normalization is unramified, but not \'{e}tale, as it is not flat. 
\end{example}
We can characterize \'{e}taleness as follows. 
\begin{proposition}\label{prop: tfae etale}
    Let $f:X\to Y$ be a locally finite type morphism betewen locally Noetherian schemes. The map from the normalization is unramified. The following are equivalent:
    \begin{enumerate}[label=(\alph*)]
        \item $f$ is \'{e}tale. 
        \item $f$ is flat and unramified. 
        \item $f$ is flat and $\Omega^{1}_{X/Y}=0$. 
        \item $f$ is smooth of relative dimension 0. 
    \end{enumerate}
\end{proposition}
\begin{proof}
    (i)$\Leftrightarrow$(ii) This is the definition. 

    (ii)$\Leftrightarrow$(iii) This is \Cref{prop: unramified iff trivial relative differentials}. 

    (iii)$\Rightarrow$(iv) This is \Cref{prop: smoothness via relative cotangent}.

    (iv)$\Rightarrow$(iii) Suppose $f$ is smooth of relative dimension 0. Then $f$ is flat by definition. Smoothness implies further that $\Omega^{1}_{X/Y}$ is locally free of rank $\dim(X)-\dim(Y)=0$, hence trivial. 
\end{proof}
\'{E}taleness gives surjectivity on tangent spaces, so we have that \'{e}tale morphisms induce isomorphisms on Zariski tangent spaces since these are in particular unramified (cf. \Cref{lem: unramified implies immersion}). 
\begin{lemma}\label{lem: etale is local diffeo}
    Let $f:X\to Y$ be an \'{e}tale morphism between locally Noetherian schemes. Then $T_{X,x}\to T_{Y,y}\otimes\kappa(x)$ is an isomorphism. 
\end{lemma}
\begin{proof}
    \'{E}tale morphisms are in particular unramifed so $T_{X,x}\to T_{Y,y}\otimes\kappa(x)$ is injective by \Cref{lem: unramified implies immersion}. Flatness implies $\mfrak_{y}\otimes\Ocal_{X,x}=\mfrak_{y}\Ocal_{X,x}=\mfrak_{x}$ so the induced map on Zariski tangent spaces is surjective too. 
\end{proof}
Generalizing \Cref{lem: unramified implies surjective differentials}, we can show that for $f$ \'{e}tale, $f^{*}\Omega_{Y/k}^{1}\to\Omega_{X/k}^{1}$ is an isomorphism. For this we will require the following lemma. 
\begin{lemma}\label{lem: unramified implies open relative diagonal}
    Let $f:X\to Y$ be an unramified morphism between locally finite type $k$-schemes. Then $\Delta_{X/Y}$ is an open immersion.\todo{To do.} 
\end{lemma}
We now begin the proof in earnest. 
\begin{proposition}\label{prop: etale implies isomorphic differentials}
    Let $f:X\to Y$ be an \'{e}tale morphism between locally finite type $k$-schemes. Then $f^{*}\Omega_{Y/k}^{1}\to\Omega_{X/k}^{1}$ is an isomorphism. 
\end{proposition}
\begin{proof}
    We use the exact sequence 
    $$f^{*}\Omega^{1}_{Y/k}\to\Omega^{1}_{X/k}\to\Omega^{1}_{X/Y}\to0$$
    where $\Omega^{1}_{X/Y}=0$ since \'{e}tale morphisms are in particular unramified so $f^{*}\Omega_{X/k}^{1}\to\Omega^{1}_{Y/k}$ is surjective as in \Cref{lem: unramified implies surjective differentials}. It remains to show that the induced map on K\"{a}hler differentials is injective. If $X,Y$ are $k$-smooth, then we are done, as $f^{*}\Omega_{X/k}^{1}\to\Omega^{1}_{Y/k}$ is a surjection of locally free sheaves of the same rank, hence an isomorphism. In the general case, we can reduce to where $X,Y$ are affine and $X$ is the spectrum of a finitely generated $\Gamma(Y,\Ocal_{Y})$-algebra. Using the diagrams 
    $$% https://q.uiver.app/#q=WzAsOCxbMCwwLCJYXFx0aW1lc197a31YIl0sWzIsMCwiWCJdLFsyLDEsIlkiXSxbMCwxLCJYXFx0aW1lc197a31ZIl0sWzUsMCwiWVxcdGltZXNfe2t9WCJdLFs1LDEsIllcXHRpbWVzX3trfVkiXSxbNywwLCJYIl0sWzcsMSwiWSJdLFsxLDIsImYiXSxbMywyXSxbMCwxXSxbMCwzXSxbNCw2XSxbNiw3LCJmIl0sWzQsNV0sWzUsN11d
    \begin{tikzcd}
        {X\times_{k}X} && X &&& {Y\times_{k}X} && X \\
        {X\times_{k}Y} && Y &&& {Y\times_{k}Y} && Y
        \arrow[from=1-1, to=1-3]
        \arrow[from=1-1, to=2-1]
        \arrow["f", from=1-3, to=2-3]
        \arrow[from=1-6, to=1-8]
        \arrow[from=1-6, to=2-6]
        \arrow["f", from=1-8, to=2-8]
        \arrow[from=2-1, to=2-3]
        \arrow[from=2-6, to=2-8]
    \end{tikzcd}$$
    we use the preservation of flatness under base change to observe that $X\times_{k}X\to X\times_{k}Y,Y\times_{k}X\to Y\times_{k}Y$ are flat, and under the isomorphism $X\times_{k}Y\cong Y\times_{k}X$ we have that the composite $X\times_{k}X\to Y\times_{k}Y$ is flat. Now for 
    $$% https://q.uiver.app/#q=WzAsNSxbMiwwLCJYXFx0aW1lc197WX1YIl0sWzIsMSwiWSJdLFs0LDAsIlhcXHRpbWVzX3trfVgiXSxbNCwxLCJZXFx0aW1lc197a31ZIl0sWzAsMCwiWCJdLFs0LDAsIlxcRGVsdGFfe1gvWX0iLDAseyJzdHlsZSI6eyJ0YWlsIjp7Im5hbWUiOiJob29rIiwic2lkZSI6InRvcCJ9fX1dLFs0LDEsImYiLDJdLFswLDEsInAiLDJdLFsxLDMsIlxcRGVsdGFfe1l9IiwyXSxbMiwzXSxbMCwyLCJpIl1d
    \begin{tikzcd}
        X && {X\times_{Y}X} && {X\times_{k}X} \\
        && Y && {Y\times_{k}Y}
        \arrow["{\Delta_{X/Y}}", hook, from=1-1, to=1-3]
        \arrow["f"', from=1-1, to=2-3]
        \arrow["i", from=1-3, to=1-5]
        \arrow["p"', from=1-3, to=2-3]
        \arrow[from=1-5, to=2-5]
        \arrow["{\Delta_{Y}}"', from=2-3, to=2-5]
    \end{tikzcd}$$
    with square Cartesian we have that $p$ is flat and $i$ is closed as $\Delta_{Y}$ is. Denote $\Ical,\Jcal$ the ideal sheaves of $i,\Delta_{Y}$, respectively. We have $p^{*}(\Jcal/\Jcal^{2})\cong\Ical/\Ical^{2}$ by flatness of $p$ and $p^{*}(\Jcal/\Jcal^{2})\cong p^{*}\Omega^{1}_{Y/k}$ by definition. So we compute 
    \begin{align*}
        f^{*}\Omega_{Y/k}^{1} &\cong\Delta_{X/Y}^{*}(p^{*}\Omega^{1}_{Y/k}) \\
        &\cong \Delta^{*}_{X/Y}(\Ical/\Ical^{2}).
    \end{align*}
    By $i\circ\Delta_{X/Y}=\Delta_{X}:X\to X\times_{k}X$ and $\Delta_{X/Y}$ is an open immersion by \Cref{lem: unramified implies open relative diagonal}, we have that $\Delta^{*}_{X/Y}(\Ical/\Ical^{2})=\Kcal/\Kcal^{2}$ where $\Kcal$ is the ideal sheaf of $X\to X\times_{k}X$, that is, $\Omega^{1}_{X/k}$, as desired. 
\end{proof}
We now consider some geometric consequences of \'{e}taleness and unramfiedness. 

Let $X,Y$ be smooth integral $k$-schemes and $f:X\to Y$ a dominant morphism such that $K(X)/K(Y)$ is finite and $\alpha:f^{*}\Omega^{1}_{Y/k}\to\Omega^{1}_{X/k}$ the induced map on the K\"{a}hler differentials. Passing to the fiber at the generic point yields a morphism $\Omega^{1}_{K(Y)/k}\otimes K(X)\to\Omega^{1}_{K(X)/k}$ which is an isomorphism by separability, implying that $f^{*}\Omega^{1}_{Y/k}\otimes K(X)\to\Omega^{1}_{X/k}\otimes K(X)$ is an isomorphism as well. So the support of $\ker(\alpha)$ is a proper closed subset of $X$ which is empty by $f^{*}\Omega^{1}_{Y/k}$ locally free. $\alpha$ induces a canonical map $f^{*}\omega_{Y/k}\to\omega_{X/k}$ which can be viewed as a global section $s\in H^{0}(X,f^{*}\omega_{Y/k}^{\vee}\otimes\omega_{X/k})$, the vanishing locus of which can be studied. 
\begin{definition}[Ramification Divisor]\label{def: ramification divisor}
    Let $f:X\to Y$ be a dominant morphism for $X,Y$ smooth integral $k$-schemes. The ramification divisor $R_{f}$ is the vanishing $V_{X}(s)$ for $s\in H^{0}(X,f^{*}\omega_{Y/k}^{\vee}\otimes\omega_{X/k})$ uniquely determined by the map $f^{*}\omega_{Y/k}\to\omega_{X/k}$. 
\end{definition}
We record some elementary properties of the ramification divisor. 
\begin{lemma}\label{lem: properties of ramification divisor}
    Let $f:X\to Y$ be a dominant morphism for $X,Y$ smooth integral $k$-schemes with ramification divisor $R_{f}$. 
    \begin{enumerate}[label=(\roman*)]
        \item $f|_{X\setminus R_{f}}:X\setminus R_{f}\to Y$ is unramified. 
        \item $\omega_{X/k}\cong f^{*}\omega_{Y/k}\otimes\Ocal_{X}(R_{f})$. 
    \end{enumerate}
\end{lemma}
\begin{proof}[Proof of (i)]
    On the complement of $R_{f}$ the induced morphism on K\"{a}hler differentials is an isomorphism which can be checked stalkwise, so the sheaf of relative K\"{a}hler differentials vanishes, whence the claim. 
\end{proof}
\begin{proof}[Proof of (ii)]
    We observe that $\Ocal_{X}(R_{f})$ is the determinant of the normal bundle of $R_{f}$ in $X$, whence the claim follows by the adjunction formula \Cref{prop: adjunction formula}. 
\end{proof}
Let us consider some examples. 
\begin{example}
    $\A^{1}_{k}\to\A^{1}_{k}$ by $x\mapsto x^{2}$ has ramification divisor $(x)$. More generally for the map $x\mapsto x^{n}$ and with the characteristicof $k$ not dividing $n$, the ramification divisor is $(x)$ to order $n-1$. 
\end{example}
\begin{example}\label{ex: etale map has empty ramification divisor}
    If $f$ is \'{e}tale, then $R_{f}=\emptyset$. 
\end{example}
This is especially interesting in the case of curves, giving the Riemann-Hurwitz formula. 
\begin{proposition}[Riemann-Hurwitz Formula]\label{prop: Riemann-Hurwitz}
    Let $f:X\to Y$ be a dominant morphism for $X,Y$ smooth integral curves over $k$ and with $K(X)/K(Y)$ separable. Then 
    $$2g_{X}-2=\deg(f)(2g_{Y}-2)+\deg(R_{f}).$$
\end{proposition}
\begin{proof}
    Note that $\omega_{X/k},\omega_{Y/k}$ are of degrees $2g_{X}-2,2g_{Y}-2$, respectively. Then using \Cref{lem: properties of ramification divisor} (ii), we compute 
    \begin{align*}
        \deg(\omega_{X/k}) &= \deg\left(f^{*}\omega_{Y/k}\otimes\Ocal_{X}(R_{f})\right) \\
        2g_{X}-2 &= \deg(f^{*}\omega_{Y/k})+\deg(\Ocal_{X}(R_{f})) \\
        &= \deg(f)(2g_{Y}-2)+\deg(R_{f})
    \end{align*}
    as desired. 
\end{proof}
\begin{example}
    If $f$ is \'{e}tale then $2g_{X}-2=\deg(f)(2g_{Y}-2)$ (cf. \Cref{ex: etale map has empty ramification divisor}). 
\end{example}
\begin{example}
    A degree 2 map $\PP^{1}_{k}\to\PP^{1}_{k}$ by squaring has a ramification divisor of degree 2 given by $\{[0:1],[1:0]\}$. 
\end{example}
\begin{example}
    If $f:X\to\PP^{1}_{k}$ is a degree 2 map of curves and $\deg(R_{f})=4$ then $g_{X}=1$. If $X$ has a rational point, then $X$ is an elliptic curve. Without loss of generality, we can take the image of $R_{f}$ to be $\{[0:1],[1:0],[1:1],[\lambda:1]\}$. $\lambda$ determines $X$ uniquely. 
\end{example}
Another interesting consequence is that of Lur\"{o}th's problem. 
\begin{corollary}[Lur\"{o}th's Problem]\label{corr: Luroth problem}
    Let $k$ be algebraically closed and $X$ a smooth projective curve over $k$. For a tower $k\subsetneq K\subseteq L$ where $L$ is purely transcendental of transcendence degree 1, then $K$ is purely transcendental of transcendence degree 1 with $L=K$. 
\end{corollary}
\begin{proof}
    Recall that $k(t)$ is the function field of $\PP^{1}_{k}$ and for $L$ as above, there exists a unique smooth projective curve $Y$ over $k$ such that $K=K(Y)$. $K\subseteq k(t)$ can be viewed as induced by a morphism $\PP^{1}_{k}\to X$ which is dominant. By the Riemann-Hurwitz formula, we have 
    $$-2=\deg(f)(2g_{X}-2)+\deg(R_{f})$$
    but the quantity on the right is $\geq0$ if $g_{X}>0$ so $g_{X}=0$, ie. $X\cong\PP^{1}_{k}$ and the function fields are isomorphic. 
\end{proof}
\section{Lecture 10 -- 15th May 2025}\label{sec: lecture 10}
Having stated and proved the Riemann-Hurwitz formula \Cref{prop: Riemann-Hurwitz}, we continue with a discussion of unramifiedness and \'{e}taleness in the case of curves and their consequences. 

Let $f:X\to Y$ be a morphism of smooth projective integral curves over a field $k$. Let $x\in X$ and $y=f(x)\in Y$ be closed points. Recall that the stalks $\Ocal_{X,x},\Ocal_{Y,y}$ are discrete valuation rings and that the map of local rings $\Ocal_{Y,y}\to\Ocal_{X,x}$ extends to a map to $\ZZ$ by taking the value of the image of the uniformizer $\pi_{y}$, for any local ring homomorphisms of discrete valuation rings takes the uniformizer of the source to a power of the uniformizer of the target. This preempts the notion of ramification for morphisms of curves.
\begin{definition}[Ramification Points of Morphisms of Curves]\label{def: ramification points}
    Let $f:X\to Y$ be a morphism of smooth projective integral curves over a field $k$. Let $x\in X$ and $y=f(x)\in Y$ be closed points. $f$ is ramified at $x$ if $\nu_{x}(\pi_{y})=e_{x}>1$ for $\nu_{x}$ the valuation on $\Ocal_{X,x}$ and $\pi_{y}$ the uniformizer of $\Ocal_{Y,y}$. 
\end{definition}
\begin{definition}[Tamely Ramified Morphism of Curves at a Point]\label{def: tamely ramified morphism at a point}
    Let $f:X\to Y$ be a morphism of smooth projective integral curves over a field $k$ ramifed at $x\in X$. $f$ is tamely ramified at $x$ if $\mathrm{char}(k)\nmid e_{x}=\nu_{x}(\pi_{y})$ and $\kappa(x)/\kappa(y)$ is separable. 
\end{definition}
\begin{definition}[Tamely Ramified Morphism of Curves]\label{def: tamely ramified morphism}
    Let $f:X\to Y$ be a morphism of smooth projective integral curves over a field $k$. $f$ is a tamely ramified morphism of curves if it is tamely ramified at each ramification point $x\in X$. 
\end{definition}
This coincides with the language of unramifiedness \Cref{def: unramified morphism at a point,def: unramified morphism}. 
\begin{lemma}\label{lem: unramifiedness notions agree}
    Let $f:X\to Y$ be a morphism of smooth projective integral curves over a field $k$. The following are equivalent:
    \begin{enumerate}[label=(\alph*)]
        \item $f$ is unramified in the sense of \Cref{def: unramified morphism}: for all $x\in X$ with image $y\in Y$ $\mfrak_{y}\Ocal_{X,x}=\mfrak_{x}$ and $\kappa(x)/\kappa(y)$ is separable. 
        \item $f$ is not ramified in the sense of \Cref{def: ramification points}: for all $x\in X$ with image $y\in Y$, $\nu_{x}(\pi_{y})=1$ and $\kappa(x)/\kappa(y)$ is separable. 
    \end{enumerate}
\end{lemma}
\begin{proof}
    (a)$\Rightarrow$(b) If $\mfrak_{y}\Ocal_{X,x}=\mfrak_{y}$ for all $x\in X$ with image $y\in Y$ and $\kappa(x)/\kappa(y)$ is separable then the valuation $\nu_{x}(\pi_{y})=1$ as $\pi_{y}$ generates $\mfrak_{y}\subseteq\Ocal_{Y,y}$ and hence $\mfrak_{x}\subseteq\Ocal_{X,x}$. 

    (b)$\Rightarrow$(a) Suppose $\nu_{x}(\pi_{y})=1$ then $\pi_{y}$ generates $\mfrak_{y}\subseteq\Ocal_{Y,y}$ and hence $\mfrak_{x}\subseteq\Ocal_{X,x}$. 
\end{proof}
Ramification allows us to compute the size of finite automorphism groups of curves of genus at least 2. 
\begin{proposition}\label{prop: bound on automorphism groups of curves}
    Let $X$ be a smooth integral projective curve over an algebraically closed field $k$ of characteristic 0 of genus at least 2. Then if $\Aut_{k}(X)$ is finite, $|\Aut_{k}(X)|\leq 84(g-1)$. 
\end{proposition}
\begin{proof}
    Recall the antiequivalence of categories between smooth integral projective curves over algebraically closed fields $k$ and extensions of $k$ of transcendence degree 1. Let $G\leq\Aut_{k}(X)$ be a subgroup. Define $K(Y)=K(X)^{G}$ which corresponds to a map $F:X\to Y$ of curves. By construction $F$, is $G$-equivariant so for all $y\in Y$ with fiber $\{x_{1},\dots,x_{r}\}\subseteq X$ $G$ acts transitively on the fiber. In particular, the degree of $F$ is the order of $G$ as a group. Applying the Riemann-Hurwitz formula \Cref{prop: Riemann-Hurwitz}, we have 
    \begin{equation}\label{eqn: RH for automorphisms}
        2g_{X}-2=|G|(2g_{Y}-2+\sum_{y\in Y}\frac{e_{y}-1}{e_{y}}).
    \end{equation}
    We consider several cases: 
    \begin{itemize}
        \item If $g_{Y}\geq2$ then $2g_{Y}-2\geq2$ so (\ref{eqn: RH for automorphisms}) is at least 2, showing $|G|\leq g_{X}-1$. 
        \item If $g_{Y}=1$ then $2g_{Y}-2=0$ and $2g_{X}-2=\sum_{y\in Y}\frac{e_{y}-1}{e_{y}}$ which is at least $\frac{1}{2}$ -- if $f$ is unramified then $2g_{X}-2=g_{Y}=0$ a contradiction as the genus of $X$ is at least 2 -- so $\frac{1}{2}|G|\leq 2g_{X}-2$ and thus $|G|\leq 4g_{X}-1$.
        \item If $g_{Y}=0$ then $2g_{X}=\sum_{y\in Y}\frac{e_{y}-1}{e_{y}}$. Let $n=|\{y\in Y:e_{y}>1\}|$. So 
        $$2g_{X}\geq\begin{cases}
            \geq\frac{1}{2} & n\geq 5 \\ \frac{1}{10} & n=4 \\ \frac{1}{42} & n=3.
        \end{cases}$$
    \end{itemize}
    This yields the claim. 
\end{proof}
\begin{remark}
    A deformation theory argument is used to show that the automorphism groups of curves are in fact finite, which we do not produce here. 
\end{remark}
We can see some (counter)examples of this phenomenon. 
\begin{example}
    Let $X=V_{+}(x^{3}y+y^{3}z+z^{3}x)\subseteq\PP^{2}_{k}$ be the Klein quartic curve. $X$ is of genus $\frac{(4-1)(4-2)}{2}=3$ and $|\Aut_{k}(X)|=168$. This is a Hurwitz curve of genus 3. There exist genera $g\in\NN$ for which there is no Hurwitz curve of genus $g$ -- in particular, for $2\leq g\leq 11$ only $g=3,g=7$ admit Hurwitz curves. In these cases $g_{Y}=0$ in the proof above and the bound is attained. 
\end{example}
\begin{example}
    In positive characteristic, one can produce larger automorphism groups (which remain finite). 
\end{example}
We consider some local properties of \'{e}tale morphisms. 
\begin{proposition}\label{prop: smooth is etale and project}
    Let $f:X\to Y$ be smooth of relative dimension $d$. Then for all $x\in X$ there is an affine neighborhood $U\subseteq X$ of $x$ with image contained in an affine open $V\subseteq Y$ such that there exists a commutative diagram 
    $$% https://q.uiver.app/#q=WzAsNSxbMCwwLCJYIl0sWzAsMSwiWSJdLFsyLDAsIlUiXSxbMiwxLCJWIl0sWzQsMCwiXFxBXntkfV97Vn0iXSxbMCwxLCJmIiwyXSxbMiwwLCIiLDIseyJzdHlsZSI6eyJ0YWlsIjp7Im5hbWUiOiJob29rIiwic2lkZSI6ImJvdHRvbSJ9fX1dLFszLDEsIiIsMCx7InN0eWxlIjp7InRhaWwiOnsibmFtZSI6Imhvb2siLCJzaWRlIjoiYm90dG9tIn19fV0sWzIsM10sWzIsNCwiaSJdLFs0LDNdXQ==
    \begin{tikzcd}
        X && U && {\A^{d}_{V}} \\
        Y && V
        \arrow["f"', from=1-1, to=2-1]
        \arrow[hook', from=1-3, to=1-1]
        \arrow["i", from=1-3, to=1-5]
        \arrow[from=1-3, to=2-3]
        \arrow[from=1-5, to=2-3]
        \arrow[hook', from=2-3, to=2-1]
    \end{tikzcd}$$
    where $i$ is \'{e}tale. 
\end{proposition}
\begin{proof}
    See \cite[\href{https://stacks.math.columbia.edu/tag/039P}{Tag 039P}]{stacks-project}.
\end{proof}
\begin{example}
    Let $X$ be a smooth $k$-scheme of dimension $d$. Then there locally exists a map to $\A^{d}_{k}$, but this map need not be an open immersion. Moreover, there are rarely open immersions $\A^{d}_{k}\to X$. 
\end{example}
We introduce the notion of standard \'{e}taleness. 
\begin{definition}[Standard \'{E}tale]\label{def: standard etale}
    Let $f:X\to Y$ be a locally finite type morphism between Noetherian affine schemes. $f$ is standard \'{e}tale if it is of the form $\spec(A[t]_{f}/(g))\to\spec(A)$ where $g$ is monic with dierivative invertible in the localization $A[t]_{f}/(g)$. 
\end{definition}
Any \'{e}tale morphism can be factored over a standard \'{e}tale one. 
\begin{proposition}\label{prop: etale factors as standard etale}
    Let $f:X\to Y$ be an \'{e}tale morphism. Then for all $x\in X$ there exists an affine neighborhood $U\subseteq X$ of $x$ with image contained in an affine open $V\subseteq Y$ such that $f|_{U}:U\to V$ is standard \'{e}tale as a map of affine schemes. 
\end{proposition}
\begin{proof}
    See \cite[\href{https://stacks.math.columbia.edu/tag/02GT}{Tag 02GT}]{stacks-project}.
\end{proof}
We consider formal variants of \'{e}tale, smoothness, and unramifiedness. These are characterized very similarly to the valuative criterion. 
\begin{definition}[Formally Smooth]\label{def: formally smooth}
    Let $f:X\to Y$ be a locally finite type morphism between Noetherian affine schemes. $f$ is formally smooth if for all solid diagrams 
    \begin{equation}\label{eqn: formally smooth unramified etale lifting diagram}
        % https://q.uiver.app/#q=WzAsNCxbMiwwLCJYIl0sWzIsMSwiWSJdLFswLDEsIlxcc3BlYyhBKSJdLFswLDAsIlxcc3BlYyhBL0kpIl0sWzMsMF0sWzAsMV0sWzMsMl0sWzIsMV0sWzIsMCwiIiwxLHsic3R5bGUiOnsiYm9keSI6eyJuYW1lIjoiZGFzaGVkIn19fV1d
    \begin{tikzcd}
        {\spec(A/I)} && X \\
        {\spec(A)} && Y
        \arrow[from=1-1, to=1-3]
        \arrow[from=1-1, to=2-1]
        \arrow[from=1-3, to=2-3]
        \arrow[dashed, from=2-1, to=1-3]
        \arrow[from=2-1, to=2-3]
    \end{tikzcd}
    \end{equation}
    where $I$ is a nilpotent ideal, there is at least one morphism $\spec(A)\to X$ rendering the entire diagram commutative. 
\end{definition}
\begin{definition}[Formally Unramified]\label{def: formally unramified}
    Let $f:X\to Y$ be a locally finite type morphism between Noetherian affine schemes. $f$ is formally unramified if for all solid diagrams (\ref{eqn: formally smooth unramified etale lifting diagram}) where $I$ is a nilpotent ideal, there is at most one morphism $\spec(A)\to X$ rendering the entire diagram commutative. 
\end{definition}
\begin{definition}[Formally \'{E}tale]\label{def: formally etale}
    Let $f:X\to Y$ be a locally finite type morphism between Noetherian affine schemes. $f$ is formally \'{e}tale if for all solid diagrams (\ref{eqn: formally smooth unramified etale lifting diagram}) where $I$ is a nilpotent ideal, there is a unique morphism $\spec(A)\to X$ rendering the entire diagram commutative. 
\end{definition}
These agree with \Cref{def: smooth morphism,def: unramified morphism,def: etale morphism} we have already seen, as we made these constructions in the case of $f$ locally finite type betewen locally Noetherian schemes. 
\begin{proposition}\label{prop: formal is ordinary}
    Let $f:X\to Y$ be a locally finite type morphism between locally Noetherian schemes. Then:
    \begin{enumerate}[label=(\roman*)]
        \item $f$ is smooth in the sense of \Cref{def: smooth morphism} if and only if it is formally smooth in the sense of \Cref{def: formally smooth}. 
        \item $f$ is unramified in the sense of \Cref{def: unramified morphism} if and only if it is formally unramified in the sense of \Cref{def: formally unramified}. 
        \item $f$ is \'{e}tale in the sense of \Cref{def: etale morphism} if and only if it is formally \'{e}tale in the sense of \Cref{def: formally etale}. 
    \end{enumerate}
\end{proposition}
\begin{proof}[Proof of (i)]
    See \cite[\href{https://stacks.math.columbia.edu/tag/02H6}{Tag 02H6}]{stacks-project}.
\end{proof}
\begin{proof}[Proof of (ii)]
    See \cite[\href{https://stacks.math.columbia.edu/tag/02HE}{Tag 02HE}]{stacks-project}.
\end{proof}
\begin{proof}[Proof of (iii)]
    See \cite[\href{https://stacks.math.columbia.edu/tag/02HM}{Tag 02HM}]{stacks-project}.
\end{proof}
\section{Lecture 11 -- 19th May 2025 -- Interlude: The \'{E}tale Toplogy}\label{sec: lecture 11}
We briefly discuss the \'{e}tale topology on schemes. A comprehensive treatment would require an entire course. 

Recall that the Zariski topology is intrinsically defined on a scheme, but it does come with some drawbacks. Two notable ones are that constant sheaves are acyclic on irreducible schemes, and that local triviality in the Zariski topology is at times too strong a condition in practice. We will see examples in what follows. 

We begin by definign the small \'{e}tale site. 
\begin{definition}[Small \'{E}tale Site]\label{def: small etale site}
    Let $X$ be a locally Noetherian scheme. Define $X_{\et}$ to be the full subcategory of $\Sch_{/X}$ spanned by objects $U\to X$ where $U\to X$ is \'{e}tale. 
\end{definition}
\begin{remark}
    A morphism in the small \'{e}tale site $U\to U'$ is given by a diagram 
    $$% https://q.uiver.app/#q=WzAsMyxbMCwwLCJVIl0sWzIsMCwiVSciXSxbMSwxLCJYIl0sWzAsMl0sWzEsMl0sWzAsMV1d
    \begin{tikzcd}
        U && {U'} \\
        & X
        \arrow[from=1-1, to=1-3]
        \arrow[from=1-1, to=2-2]
        \arrow[from=1-3, to=2-2]
    \end{tikzcd}$$
    where $U\to X,U'\to X$ is \'{e}tale, so $U\to U'$ is \'{e}tale by cancellation for \'{e}tale morphisms \cite[\href{https://stacks.math.columbia.edu/tag/02GW}{Tag 02GW}]{stacks-project}. Additionally, being \'{e}tale is preserved under composition and base change, and isomorphisms are \'{e}tale -- these are the necessary conditions to define a Grothendieck pretopology. 
\end{remark}
We want the \'{e}tale site to behave like a topological space. In particular, we want a notion of coverings. 
\begin{definition}[\'{E}tale Covering]\label{def: etale covering}
    Let $X$ be a locally Noetherian scheme and $X_{\et}$ its small \'{e}tale site. If $U\in X_{\et}$ then a family of morphisms $\{U_{i}\to U\}_{i\in I}$ in $X_{\et}$ is a covering if $\sqcup_{i\in I}U_{i}\to U$ is surjective. 
\end{definition}
Let us consider a simple example. 
\begin{example}
    $\A^{1}_{k}\setminus\{0\}\to\A^{1}_{k}\setminus\{0\}$ by $z\mapsto z^{2}$ is \'{e}tale. The two-element family $$\left\{\A^{1}_{k}\setminus\{0\}\xrightarrow{z\mapsto z^{2}}\A^{1}_{k},\A^{1}_{k}\setminus\{1\}\xrightarrow{\id_{\A^{1}_{k}\setminus\{1\}}}\A^{1}_{k}\right\}$$ 
    is an \'{e}tale covering since the maps are jointly surjective. 
\end{example}
We can compare this to the Zariski site associated to the Zariski topology. 
\begin{definition}[Zariski Site]\label{def: Zariski site}
    Let $X$ be a locally Noetherian scheme. Define $X_{\Zar}$ to be the full subcategory of $\Sch_{/X}$ spanned by the objects $U\to X$ where $U\to X$ is an open immersion. 
\end{definition}
\begin{remark}
    As before, open immersions satisfy cancellation, so any morphism $U\to U'$ in $X_{\Zar}$ is automatically an open immersion. 
\end{remark}
Note that open immersions are in particular \'{e}tale so the Zariski site includes into the \'{e}tale site -- in other words, the \'{e}tale site contains more morphisms than the Zariski site. The functor $F:X_{\et}\to X_{\Zar}$ is continuous since the preimage $F^{-1}([U\to X])$ is a map in the \'{e}tale topology for an open immersion $[U\to X]$ in $X_{\Zar}$. 

We can define sheaves and presheaves over the \'{e}tale site. 
\begin{definition}[\'{E}tale Presheaves]\label{def: etale presheaves}
    Let $X$ be a locally Noetherian scheme and $X_{\et}$ its \'{e}tale site. An \'{e}tale presheaf on $X_{\et}$ is a functor $\Fcal:X_{\et}^{\Opp}\to\AbGrp$. 
\end{definition}
\begin{definition}[\'{E}tale Sheaves]\label{def: etale sheaves}
    Let $X$ be a locally Noetherian scheme and $X_{\et}$ its \'{e}tale site. An \'{e}tale sheaf on $X_{\et}$ is an \'{e}tale presheaf $\Fcal$ such that for all \'{e}tale coverings $\{U_{i}\to U\}_{i\in I}$ in $X_{\et}$ the sequence 
    $$% https://q.uiver.app/#q=WzAsNCxbMCwwLCIwIl0sWzEsMCwiXFxGY2FsKFUpIl0sWzIsMCwiXFxwcm9kX3tpXFxpbiBJfVxcRmNhbChVX3tpfSkiXSxbMywwLCJcXHByb2Rfe2ksalxcaW4gSX1cXEZjYWwoVV97aX1cXHRpbWVzX3tVfVVfe2p9KSJdLFswLDFdLFsxLDJdLFsyLDMsIiIsMix7Im9mZnNldCI6LTF9XSxbMiwzLCIiLDIseyJvZmZzZXQiOjF9XV0=
    \begin{tikzcd}
        0 & {\Fcal(U)} & {\prod_{i\in I}\Fcal(U_{i})} & {\prod_{i,j\in I}\Fcal(U_{i}\times_{U}U_{j})}
        \arrow[from=1-1, to=1-2]
        \arrow[from=1-2, to=1-3]
        \arrow[shift left, from=1-3, to=1-4]
        \arrow[shift right, from=1-3, to=1-4]
    \end{tikzcd}$$
    is an equalizer. 
\end{definition}
We can define a functor $F_{*}:\Sh(X_{\et})\to\Sh(X_{\Zar})$ by $\Fcal\mapsto[[U\to X]\mapsto\Fcal(U)]$ by restriction. More generally for a morphism $f:X\to Y$ of locally Noetherian schemes there we can define $f_{*}:\Sh(X_{\et})\to \Sh(Y_{\et}),f^{-1}:\Sh(Y_{\et})\to\Sh(X_{\et})$. 
\begin{definition}[Global Sections of \'{E}tale Sheaves]\label{def: global sections of etale sheaves}
    Let $X$ be a locally Noetherian scheme and $\Fcal$ a sheaf of Abelian groups on $X_{\et}$. The global sections of $\Fcal$ is defined to be $\Gamma(X_{\et},\Fcal)=\Hom_{\PSh(X_{\et})}(*,\Fcal)$ where $*$ is the final object in $\PSh(X_{\et})$. 
\end{definition}
Moreover, the category of Abelian sheaves on a site has enough injectives \cite[\href{https://stacks.math.columbia.edu/tag/01DL}{Tag 01DL}]{stacks-project}, so this allows us to define all higher cohomology groups. 
\begin{definition}[Cohomology of \'{E}tale Sheaves]\label{def: global sections of etale sheaves}
    Let $X$ be a locally Noetherian scheme and $\Fcal$ a sheaf of Abelian groups on $X_{\et}$. The cohomology of $\Fcal$ is defined to be $H^{i}(X_{\et},\Fcal)$ is the cohomology of an injective resolution of $\Fcal$.  
\end{definition}
\begin{remark}
    The notation $H^{i}_{\et}(X,\Fcal)$ is also used in the literature. 
\end{remark}
We can make the analogous constructions for the structure sheaf, stalks, and derived pushforwards. 
\begin{remark}\label{rmk: commutes for QCoh}
    The diagram 
    $$% https://q.uiver.app/#q=WzAsNCxbMCwwLCJcXFNoKFhfe1xcWmFyfSkiXSxbMiwwLCJcXFNoKFhfe1xcZXR9KSJdLFswLDEsIlxcQWJHcnAiXSxbMiwxLCJcXEFiR3JwIl0sWzAsMiwiSF57aX0oWCwtKSIsMl0sWzEsMywiSF57aX0oWF97XFxldH0sLSkiXSxbMCwxXSxbMiwzLCI/IiwyLHsic3R5bGUiOnsiYm9keSI6eyJuYW1lIjoiZG90dGVkIn19fV1d
    \begin{tikzcd}
        {\Sh(X_{\Zar})} && {\Sh(X_{\et})} \\
        \AbGrp && \AbGrp
        \arrow[from=1-1, to=1-3]
        \arrow["{H^{i}(X,-)}"', from=1-1, to=2-1]
        \arrow["{H^{i}(X_{\et},-)}", from=1-3, to=2-3]
        \arrow["{?}"', dotted, from=2-1, to=2-3]
    \end{tikzcd}$$
    need not commute, but does so for quasicoherent sheaves. 
\end{remark}

Let us consider more examples of \'{e}tale sheaves. 
\begin{example}
    Consider the sheaves $\GG_{m},\GG_{a},\mu_{n}$ as sheaves on the \'{e}tale site taking $[U\to X]$ to $\Gamma(U,\Ocal_{U}^{\times}), \Gamma(U,\Ocal_{U}), \{s\in\Ocal_{U}(U):s^{n}=1_{U}\}$. 
\end{example}

As a consequence of \Cref{rmk: commutes for QCoh}, we have the following comparison between the ordinary and the \'{e}tale Picard groups. 
\begin{proposition}\label{prop: picard group comparison}
    Let $X$ be a locally Noetherian scheme. There is an isomorphism of Abelian groups $H^{1}(X,\Ocal_{X}^{\times})\cong H^{1}(X_{\et},\GG_{m})$. 
\end{proposition}
Though already on the level of second cohomology, there are schemes $X$ for which $H^{2}(X,\Ocal_{X}^{\times})\not\cong H^{2}(X_{\et},\GG_{m})$. 

The \'{e}tale sheaves $\GG_{m},\mu_{n}$ fit together in the Kummer sequence. 
\begin{proposition}[Kummer Sequence]\label{prop: Kummer sequence}
    Let $X$ be a locally Noetherian $k$ scheme and $n$ an integer not divisible by the characteristic of $k$. Then there is a short exact sequence of \'{e}tale sheaves in $X_{\et}$
    $$0\to\mu_{n}\to\GG_{m}\xrightarrow{(-)^{n}}\GG_{m}\to0.$$
\end{proposition}
Taking \'{e}tale cohomology of the Kummer sequence of \Cref{prop: Kummer sequence} yields a long exact sequence 
$$% https://q.uiver.app/#q=WzAsOCxbMCwwLCIwIl0sWzEsMCwiSF57MH0oWF97XFxldH0sXFxtdV97bn0pIl0sWzEsMSwiSF57MX0oWF97XFxldH0sXFxtdV97bn0pIl0sWzIsMSwiSF57MX0oWF97XFxldH0sXFxHR197bX0pIl0sWzMsMSwiSF57MX0oWF97XFxldH0sXFxHR197bX0pIl0sWzQsMSwiXFxkb3RzLiJdLFsyLDAsIkheezB9KFhfe1xcZXR9LFxcR0dfe219KSJdLFszLDAsIkheezB9KFhfe1xcZXR9LFxcR0dfe219KSJdLFsyLDNdLFszLDRdLFs0LDVdLFswLDFdLFsxLDZdLFs2LDddLFs3LDJdXQ==
\begin{tikzcd}
	0 & {H^{0}(X_{\et},\mu_{n})} & {H^{0}(X_{\et},\GG_{m})} & {H^{0}(X_{\et},\GG_{m})} \\
	& {H^{1}(X_{\et},\mu_{n})} & {H^{1}(X_{\et},\GG_{m})} & {H^{1}(X_{\et},\GG_{m})} & {\dots.}
	\arrow[from=1-1, to=1-2]
	\arrow[from=1-2, to=1-3]
	\arrow[from=1-3, to=1-4]
	\arrow[from=1-4, to=2-2]
	\arrow[from=2-2, to=2-3]
	\arrow[from=2-3, to=2-4]
	\arrow[from=2-4, to=2-5]
\end{tikzcd}$$
We provide a more explicit description the behavior of the first cohomology groups: 
\begin{itemize}
    \item $H^{1}(X_{\et},\mu_{n})=\{(\Lcal,\varphi):\varphi:\Lcal^{\otimes n}\xrightarrow{\sim}\Ocal_{X}\}$. 
    \item $H^{1}(X_{\et},\GG_{m})\to H^{1}(X_{\et},\GG_{m})$ by $\Lcal\mapsto\Lcal^{\otimes n}$. 
\end{itemize}
We can consider related phenomena. 
\begin{example}
    $H^{1}(X,\mathrm{PGL}_{n})\not\cong H^{1}(X_{\et},\mathrm{PGL}_{n})$. If $P\to X$ is is a morphism for which there exists an \'{e}tale cover $\{U_{i}\to X\}_{i\in I}$ such that $P\times_{X}U_{i}\cong\PP^{n-1}_{U_{i}}$, we cannot concldue that $P$ is also Zariski-locally a projective space. It is possible that there exists no Zarksi open cover such that the fibers are locally projective spaces. 
\end{example}
\begin{example}
    Let $\GG_{m}=P\to X=\GG_{m}$. The squaring map is \'{e}tale so we get a diagram 
    $$% https://q.uiver.app/#q=WzAsNSxbMSwyLCJcXEdHX3ttfSJdLFszLDIsIlxcR0dfe219PVgiXSxbMywxLCJcXEdHX3ttfT1QIl0sWzEsMSwiXFxHR197bX1cXHRpbWVzX3tcXEdHX3ttfX1cXEdHX3ttfSJdLFswLDAsIlxcR0dfe219XFxjb3Byb2RcXEdHX3ttfSJdLFsyLDFdLFswLDEsIigtKV57Mn0iLDJdLFszLDBdLFszLDJdLFs0LDIsIihcXGlkLCgtKV57LTF9KSIsMCx7ImN1cnZlIjotMn1dLFs0LDAsIihcXGlkLFxcaWQpIiwyLHsiY3VydmUiOjJ9XSxbNCwzLCJcXHNpbSJdXQ==
    \begin{tikzcd}
        {\GG_{m}\coprod\GG_{m}} \\
        & {\GG_{m}\times_{\GG_{m}}\GG_{m}} && {\GG_{m}=P} \\
        & {\GG_{m}} && {\GG_{m}=X}
        \arrow["\sim", from=1-1, to=2-2]
        \arrow["{(\id,(-)^{-1})}", curve={height=-12pt}, from=1-1, to=2-4]
        \arrow["{(\id,\id)}"', curve={height=12pt}, from=1-1, to=3-2]
        \arrow[from=2-2, to=2-4]
        \arrow[from=2-2, to=3-2]
        \arrow[from=2-4, to=3-4]
        \arrow["{(-)^{2}}"', from=3-2, to=3-4]
    \end{tikzcd}$$
    which is once again \'{e}tale locally trivial, but not Zariski locally trivial as there does not exist a nonempty open subset $U\subseteq \GG_{m}$ for which $P_{U}=U\coprod U$. 
\end{example}
\begin{example}
    Let $X$ be an elliptic curve over $\CC$ and $x_{0}\in X[2]$ a 2-torsion point. The map $x\mapsto x+x_{0}$ defines a $\ZZ/2\ZZ$ action on $X$. Define $Y$ the quotient of $X$ by this automorphism so $X\to Y$ is \'{e}tale of degree 2. But this map is not Zariski locally trivial. Every nonempty open subset of $U$ has preimage which is not a disjoint union as $X$ is irreducible. 
\end{example}
The technology of the \'{e}tale site also allows us to define an analogue of singular cohomology on schemes. 
\begin{theorem}[Artin -- Comparison Isomorphism]\label{thm: Artin comparison isomorphism}
    Let $X$ be a finite type smooth $\CC$-scheme. Let $X^{\mathsf{an}}$ the set $X(\CC)$ with the complex-analytic topology and $\Lambda$ a finite Abelian group. There is an isomorphism $H^{i}(X_{\et},\underline{\Lambda})\cong H^{i}(X^{\mathsf{an}},\Lambda)$ where on the left we are computing constant sheaf cohomology in the \'{e}tale site and on the right singular cohomology of the complex manifold with $\Lambda$-coefficients. 
\end{theorem}
The reason this holds true is that any \'{e}tale morphism $U\to X$ induces open maps $B_{i}\to X^{\mathsf{an}}$ in the analytic topology where $\bigcup B_{i}$ form an analytic cover of $U^{\mathsf{an}}$ as a complex manifold. 
\begin{remark}
    The comparison map $H^{1}(X_{\et},\GG_{m})\to H^{1}(X^{\mathsf{an}},\Ocal_{X}^{\times})$ beteween \'{e}tale line bundles and complex analytic ones is not in general an isomorphism, but is in the case where $X$ is projective by Serre's GAGA principle. 
\end{remark}
We conclude with a discussion of the \'{e}tale fundamental group. 
\begin{definition}[Finite \'{E}tale Category]\label{def: finite etale category}
    Let $X$ be a locally Noetherian scheme. Denote $\mathsf{FET}(X)$ to be the full category of $\Sch_{/X}$ spanned by finite \'{e}tale maps $Y\to X$. 
\end{definition}
This allows us to define the \'{e}tale fundamental group as follows. 
\begin{definition}[\'{E}tale Fundamental Group]\label{def: etale fundamental group}
    Let $X$ be a locally Noetherian scheme with $x\in X$ and $F:\mathsf{FET}(X)\to\Sets$ by $[Y\to X]\mapsto\Mor_{\Sch_{/X}}(\overline{x},Y)$ where $\overline{x}$ is a finite separable extension of $\kappa(x)$. The \'{e}tale fundamental group $\pi_{1}^{\et}(X,x)$ is defined to be the automorphism group of the functor $F$. 
\end{definition}
This does not agree in general with the topological fundamental group. 
\begin{example}
    Let $X$ be a smooth projective surface over $\CC$ and $\sigma\in\Aut(\CC)$. Consider the Cartesian square 
    $$% https://q.uiver.app/#q=WzAsNCxbMCwwLCJYXntcXHNpZ21hfSJdLFswLDEsIlxcc3BlYyhcXENDKSJdLFsyLDEsIlxcc3BlYyhcXENDKSJdLFsyLDAsIlgiXSxbMSwyLCJcXHNpZ21hIiwyXSxbMywyXSxbMCwzXSxbMCwxXV0=
    \begin{tikzcd}
        {X^{\sigma}} && X \\
        {\spec(\CC)} && {\spec(\CC)}
        \arrow[from=1-1, to=1-3]
        \arrow[from=1-1, to=2-1]
        \arrow[from=1-3, to=2-3]
        \arrow["\sigma"', from=2-1, to=2-3]
    \end{tikzcd}$$
    The \'{e}tale fundamental groups of $X,X^{\sigma}$ agree, but there is no induced map on the analytic manifolds that makes the map on fundamental groups an isomorphism. 
\end{example}

\section{Lecture 12 -- 22nd May 2025}\label{sec: lecture 12}
We begin a discussion of blowups. Recall that Cartier divisors are sections of $\Gamma(X,\Kcal_{X}^{\times}/\Ocal_{X}^{\times})$. Roughly speaking, this provides the data of collections of rational functions on an affine open cover with regular quotients. In this way, Cartier divisors play a fundamental role in the study of schemes. However, divisors may fail to be Cartier for two reasons: singularties and being of the wrong codimension. 

Blowups give a ``universal'' construction to modify a scheme with its closed subscheme to an effective Cartier divisor. 
\begin{definition}[Blowup]\label{def: blowup}
    Let $X$ be locally Noetherian and $Z\subseteq X$ a closed subscheme. The blowup of $X$ along $Z$ is a Cartesian diagram 
    $$% https://q.uiver.app/#q=WzAsNCxbMCwwLCJFX3tafVgiXSxbMiwwLCJcXEJsX3tafVgiXSxbMiwxLCJYIl0sWzAsMSwiWiJdLFszLDJdLFsxLDJdLFswLDNdLFswLDFdXQ==
    \begin{tikzcd}
        {E_{Z}X} && {\Bl_{Z}X} \\
        Z && X
        \arrow[from=1-1, to=1-3]
        \arrow[from=1-1, to=2-1]
        \arrow[from=1-3, to=2-3]
        \arrow[from=2-1, to=2-3]
    \end{tikzcd}$$
    where the exceptional divisor $E_{Z}X$ is an efective Cartier divisor in $\Bl_{Z}X$ and final amongst cartier divisor-scheme pairs $(D,W)$ fitting into Cartesian diagrams 
    $$% https://q.uiver.app/#q=WzAsNCxbMCwwLCJEIl0sWzIsMCwiVyJdLFsyLDEsIlguIl0sWzAsMSwiWiJdLFszLDJdLFsxLDJdLFswLDNdLFswLDFdXQ==
    \begin{tikzcd}
        D && W \\
        Z && {X.}
        \arrow[from=1-1, to=1-3]
        \arrow[from=1-1, to=2-1]
        \arrow[from=1-3, to=2-3]
        \arrow[from=2-1, to=2-3]
    \end{tikzcd}$$
\end{definition}
\begin{remark}
    That is, for any Cartesian square 
    $$% https://q.uiver.app/#q=WzAsNCxbMCwwLCJEIl0sWzIsMCwiVyJdLFsyLDEsIlguIl0sWzAsMSwiWiJdLFszLDJdLFsxLDJdLFswLDNdLFswLDFdXQ==
    \begin{tikzcd}
        D && W \\
        Z && {X}
        \arrow[from=1-1, to=1-3]
        \arrow[from=1-1, to=2-1]
        \arrow[from=1-3, to=2-3]
        \arrow[from=2-1, to=2-3]
    \end{tikzcd}$$
    where $D$ is an effective Cartier divisor in $W$, there exists a factorization of this diagram 
    $$% https://q.uiver.app/#q=WzAsNixbMCwxLCJFX3tafVgiXSxbMiwxLCJcXEJsX3tafVgiXSxbMiwyLCJYIl0sWzAsMiwiWiJdLFswLDAsIkQiXSxbMiwwLCJXIl0sWzMsMl0sWzEsMl0sWzAsM10sWzAsMV0sWzQsNV0sWzUsMV0sWzQsMF1d
    \begin{tikzcd}
        D && W \\
        {E_{Z}X} && {\Bl_{Z}X} \\
        Z && X
        \arrow[from=1-1, to=1-3]
        \arrow[from=1-1, to=2-1]
        \arrow[from=1-3, to=2-3]
        \arrow[from=2-1, to=2-3]
        \arrow[from=2-1, to=3-1]
        \arrow[from=2-3, to=3-3]
        \arrow[from=3-1, to=3-3]
    \end{tikzcd}$$
    where both squares are Cartesian. 
\end{remark}
\begin{remark}
    Having defined blowups in \Cref{def: blowup} by its universal property, it is unique up to unique isomorphism if it exists. 
\end{remark}
\begin{remark}
    If $Z\subseteq X$ is Cartier, then the blowup is just $X$, since the pair $Z\to X$ trivially satisfies the desired universal property. 
\end{remark}
We can show that these exist, first affine-locally, then globally by gluing. 
\begin{lemma}\label{lem: blowup exists affine locally}
    Let $A$ be a Noetherian ring and $I\subseteq Z$ an ideal. The blowup of $\spec(A)$ along $V(I)$ exists. 
\end{lemma}
\begin{proof}
    Since $A$ is Noetherian, $I$ is finitely generated, say by $a_{0},\dots,a_{n}$. Consider $\proj(\bigoplus_{d\geq0}I^{d})$. We show this satisfies the universal property. 

    Note $\beta^{-1}(I)=I\cdot\bigoplus_{d\geq0}I^{d}=\Ocal_{\proj(\bigoplus_{d\geq0}I^{d})}(1)$ which is a Cartier divisor. We can define a map of graded rings $\varphi:A[x_{0},\dots,x_{n}]\to\bigoplus_{d\geq0}I^{d}$ by $x_{i}\mapsto a_{i}$ which is by inspection a surjective morphism of graded rings. This induces contravariantly on $\proj$ 
    $$% https://q.uiver.app/#q=WzAsMyxbMCwwLCJcXHByb2ooXFxiaWdvcGx1c197ZFxcZ2VxMH1JXntkfSkiXSxbMSwxLCJcXHNwZWMoQSkiXSxbMiwwLCJcXFBQXntufV97QX0iXSxbMCwyXSxbMiwxXSxbMCwxXV0=
    \begin{tikzcd}
        {\proj(\bigoplus_{d\geq0}I^{d})} && {\PP^{n}_{A}} \\
        & {\spec(A)}
        \arrow[from=1-1, to=1-3]
        \arrow[from=1-1, to=2-2]
        \arrow[from=1-3, to=2-2]
    \end{tikzcd}$$
    where $\PP^{n}_{A}\to\spec(A)$ and $\proj(\bigoplus_{d\geq0}I^{d})\to\PP^{n}_{A}$ are both closed, so $\proj(\bigoplus_{d\geq0}I^{d})\to\spec(A)$ is closed and the kernel of $\varphi$ is the ideal generated by homogeneous polynomials in $n+1$ variables which vanish at $(a_{0},\dots,a_{n})$. 
    
    Let 
    $$% https://q.uiver.app/#q=WzAsNCxbMCwwLCJEIl0sWzIsMCwiVyJdLFswLDEsIlYoSSkiXSxbMiwxLCJcXHNwZWMoQSkiXSxbMiwzXSxbMSwzLCJmIl0sWzAsMV0sWzAsMl1d
    \begin{tikzcd}
        D && W \\
        {V(I)} && {\spec(A)}
        \arrow[from=1-1, to=1-3]
        \arrow[from=1-1, to=2-1]
        \arrow["f", from=1-3, to=2-3]
        \arrow[from=2-1, to=2-3]
    \end{tikzcd}$$
    be a Cartesian square where $D$ is an effective Cartier divisor in $W$ -- that is, where $f^{-1}I\cdot\Ocal_{W}=\Ical_{D}$. Since $I$ is finitely generated by the $a_{i}$'s, their images $s_{0},\dots,s_{n}$ in $\Ocal_{W}$ generate $\Ical_{D}$. This induces a unique morphism $g:W\to\PP^{n}_{A}$ over $\spec(A)$ such that $g^{*}\Ocal_{\PP^{n}_{A}}(1)\cong\Ical_{D}$ with $s_{i}=g^{-1}(x_{i})$. Moreover, this morphism factors over the closed subscheme $\proj(\bigoplus_{d\geq0}I^{d})\subseteq\PP^{n}_{A}$ -- any element of the kernel $\ker(\varphi)$ of the morphism of graded rings is a homogeneous polynomial of degree $m$ that vanishes on $(a_{0},\dots,a_{n})$ and hence on $(s_{0},\dots,s_{n})$ in $\Gamma(W,\Ical_{D}^{m})$. This shows that $\proj(\bigoplus_{d\geq0}I^{d})$ satisfies the desired universal property. 
\end{proof}
\begin{remark}
    Let $f:X\to Y$ be any morphism and $Z\subseteq Y$ closed with sheaf of ideals $\Ical_{Z}$. $f^{-1}\Ical_{Z}$ does not necessarily agree with $f^{*}\Ical_{Z}=\Ocal_{X}\otimes_{f^{-1}\Ocal_{Y}}f^{-1}\Ical_{Z}$, and $f^{*}\Ical_{Z}$ need not even be a subsheaf of $\Ocal_{X}$. There exists a morphism $f^{*}\Ical_{Z}\to\Ocal_{X}$ whose image is $f^{-1}\Ical_{Z}$ the ideal sheaf which need not be an isomorphism. 
\end{remark}
We now treat the general case by gluing.  
\begin{theorem}\label{thm: existence of blowups}
    Let $X$ be locally Noetherian and $Z\subseteq X$ a closed subscheme. The blowup of $X$ along $Z$ exists. 
\end{theorem}
\begin{proof}
    Note that for $U\subseteq X$ open, $\Bl_{(Z\cap U)}U\cong \beta^{-1}(U)$ uniquely by the universal property. Covering $X$ with affine opens, and the intersections of any two such affine opens with distinguished opens, existence and uniqueness of the blowup affine-locally \Cref{lem: blowup exists affine locally} shows that the construction glues to the blowup of $X$. 
\end{proof}
\begin{remark}\label{rmk: blowup of open subschemes}
    If $U\subseteq X$ is open, then $\Bl_{U\cap Z}U=\beta^{-1}(U)$, and if $U=X\setminus Z$ then $\beta^{-1}(U)\to U$ is an isomorphism. 
\end{remark}
We want to consider how subschemes in $X$ behave in the blowup. 
\begin{definition}[Strict Transform]\label{def: strict transform}
    Let $X$ be locally Noetherian, $Z\subseteq X$ a closed subscheme, and $\beta:\Bl_{Z}X\to X$ the blowup of $X$ along $Z$. If $Y\subseteq X$ is a closed subscheme of $X$ not contained in $Z$ the total transform of $Y$ is the scheme-theoretic preimage $Y\times_{X}\Bl_{Z}X$ of $Y$ in $\Bl_{Z}X$. 
\end{definition}
\begin{definition}[Total Transform]\label{def: total transform}
    Let $X$ be locally Noetherian, $Z\subseteq X$ a closed subscheme, and $\beta:\Bl_{Z}X\to X$ the blowup of $X$ along $Z$. If $Y\subseteq X$ is a closed subscheme of $X$ not contained in $Z$ the total transform of $Y$ is $\widetilde{Y}=\overline{\beta^{-1}(Y\setminus(Y\cap Z))}\subseteq\Bl_{Z}X$. 
\end{definition}
Let us make some computations of the line bundles associated to the exceptional divisor of blowups. 
\begin{proposition}\label{prop: exceptional divisor is O minus one}
    Let $X$ be locally Noetherian, $Z\subseteq X$ a closed subscheme, and $\beta:\Bl_{Z}X\to X$ the blowup of $X$ along $Z$. Denote the structure sheaf of $\Bl_{Z}X$ by $\Ocal_{\beta}$. Then $\Ocal_{\beta}(E_{Z}X)=\Ocal_{\beta}(-1)$. 
\end{proposition}
\begin{proof}
    It suffices to observe that the ideal sheaf of $E_{Z}X$ is $\Ocal_{\beta}(1)$, so twisting by this divisor gives the dual of the ideal sheaf $\Ocal_{\beta}(E_{Z}X)\cong\Ocal_{\beta}(-1)$
\end{proof}
\begin{remark}
    The use of $\Ocal_{\beta}$ for $\Ocal_{\Bl_{Z}X}$ is justified as $\Bl_{Z}X$ is a projective bundle as the $\underline{\proj}$ of the sheaf of graded algebras locally given by the Rees algebra as shown in the construction of the blowup \Cref{lem: blowup exists affine locally}. 
\end{remark}
Moreover, smoothess is preserved under blowups. 
\begin{proposition}
    Let $X$ be locally Noetherian, $Z\subseteq X$ a closed subscheme, and $\beta:\Bl_{Z}X\to X$ the blowup of $X$ along $Z$. If $X$ and $Z$ are smooth, then $\Bl_{Z}X$ is smooth, and $\Ncal_{E_{Z}X/\Bl_{Z}X}=\Ocal_{E_{Z}X}(-1)$.\todo{Prove this.}
\end{proposition}
\begin{example}\label{ex: blowup of A2 at origin}
    Consider $\Bl_{\{(0,0)\}}\A^{2}_{k}$. Note that $\A^{2}_{k}=\spec(k[x,y])$ and the defining ideal of $\{(0,0)\}$ is $I=(x,y)$. We know that $\Bl_{\{(0,0)\}}\A^{2}_{k}\subseteq\PP^{1}_{\A^{2}_{k}}=\PP^{1}_{A}$ where $A=k[x,y]$. We can define a ring map $\varphi:A[u,v]\mapsto\bigoplus_{d\geq0}I^{d}$ by $u\mapsto x,v\mapsto y$. The kernel is generated by $uy-vx$ which defines the blowup as a closed subscheme of $\PP^{1}_{A}$. We can easily see that the fiber over any point away from the origin is a single point, and the fiber over the origin is an entire $\PP^{1}_{k}$. Indeed for any $L$ a line through the origin defined by $\{y=tx\}$ the strict transform is the union of the line itself with the $\PP^{1}_{k}$ over the origin, and the line intersects the $\PP^{1}_{k}$ over the origin at its slope $[1:t]$. 
\end{example}
We prove the property of \Cref{rmk: blowup of open subschemes} for closed subschemes. 
\begin{lemma}\label{lem: blowup of closed subschemes}
    Let $X$ be locally Noetherian, $Z\subseteq X$ a closed subscheme, and $\beta:\Bl_{Z}X\to X$ the blowup of $X$ along $Z$. Let $Y\subseteq X$ be a closed subscheme not contained in $Z$. Then $\Bl_{(Z\cap Y)}Y\cong\widetilde{Y}$ the strict transform of $Y$ in the blowup of $X$. 
\end{lemma}
\begin{proof}
    We show this satisfies the universal property. Consider the following diagram. 
    $$% https://q.uiver.app/#q=WzAsOCxbMCwxLCJEIl0sWzEsMCwiVyJdLFsyLDEsIllcXGNhcCBaIl0sWzMsMCwiWSJdLFs0LDEsIloiXSxbNSwwLCJYIl0sWzMsMiwiXFxCbF97Wn1YIl0sWzIsMywiRV97Wn1YIl0sWzAsMl0sWzIsNF0sWzMsNV0sWzQsNV0sWzIsM10sWzAsMV0sWzEsM10sWzAsNywiIiwxLHsiY3VydmUiOjJ9XSxbMSw2LCJnIiwxLHsibGFiZWxfcG9zaXRpb24iOjcwLCJjdXJ2ZSI6M31dLFs2LDUsIiIsMSx7ImN1cnZlIjoyfV0sWzcsNCwiIiwxLHsiY3VydmUiOjN9XSxbNyw2XV0=
    \begin{tikzcd}
        & W && Y && X \\
        D && {Y\cap Z} && Z \\
        &&& {\Bl_{Z}X} \\
        && {E_{Z}X}
        \arrow[from=1-2, to=1-4]
        \arrow["g"{description, pos=0.7}, curve={height=18pt}, from=1-2, to=3-4]
        \arrow[from=1-4, to=1-6]
        \arrow[from=2-1, to=1-2]
        \arrow[from=2-1, to=2-3]
        \arrow[curve={height=12pt}, from=2-1, to=4-3]
        \arrow[from=2-3, to=1-4]
        \arrow[from=2-3, to=2-5]
        \arrow[from=2-5, to=1-6]
        \arrow[curve={height=12pt}, from=3-4, to=1-6]
        \arrow[curve={height=18pt}, from=4-3, to=2-5]
        \arrow[from=4-3, to=3-4]
    \end{tikzcd}$$
    We seek to show that the image of $g$ is contained in $\widetilde{Y}$. But this is clear from the construction of $\widetilde{Y}=\overline{\beta^{-1}(Y\setminus(Y\cap Z))}$ and uniqueness from the universal property of the blowup $\Bl_{Z}X$. 
\end{proof}
We show some properties of the blowup morphism $\beta:\Bl_{Z}X\to X$. 
\begin{proposition}\label{prop: properties of blowup map}
    Let $X$ be locally Noetherian, $Z\subseteq X$ a closed subscheme, and $\beta:\Bl_{Z}X\to X$ the blowup of $X$ along $Z$. Then $\beta$ is proper, birational, and surjective. Moreover, if $X$ is projective, so too is $\Bl_{Z}X$. 
\end{proposition}
\begin{proof}
    Properness is local on target and $\beta$ is locally given by the composition of a closed immersion and projection $\PP^{n}_{A}\to\spec(A)$, hence proper. Birationality is clear as $\Bl_{Z}X\setminus E_{Z}X\to X\setminus Z$ is an isomorphism between dense open sets of the sorce and target. If $X$ is projective, then the structrure map of $\Bl_{Z}X$ is the composition of two projective morphisms. hence projective. 
\end{proof}
\begin{remark}
    The blowup is rarely flat or unramified, hence almost never \'{e}tale. 
\end{remark}
We consider some examples of blowups of curves, illustrating that they can be used to resolve singularties. 
\begin{example}\label{ex: blowup of node}
    Consider the nodal cubic $V(y^{2}-x^{3}-x^{2})\subseteq\A^{2}_{k}$. Since this is singular at the origin, we can build on our computation of the blowup of $\A^{2}_{k}$ at the origin \Cref{ex: blowup of A2 at origin} to see that $\Bl_{\{(0,0)\}}C$ is given by $\{y^{2}-x^{3}-x^{2},uy-vx\}$. On the $u\neq0$ chart, the equations $y=xv,x^{2}v^{2}=x^{3}+x^{2}$ shows that $\Bl_{\{(0,0)\}}C$ meets the exceptional divisor at two points, since the latter equation is quadratic in the $\PP^{1}_{k}$-variable $v$. 
\end{example}
\begin{example}
    The cuspidal curve $V(y^{2}-x^{3})\subseteq\A^{2}_{k}$ is similarly singular at the origin. Repeating the computation of \Cref{ex: blowup of node}, we get that $\Bl_{\{(0,0)\}}C$ is given by the equations $\{y^{2}-x^{3},uy-vx\}$ which on the $u\neq0$ chart gives $\{y=xv,x^{2}v^{2}=x^{3}\}$ showing that the blowup meets the exceptional divisor at one point with multiplicity 2. 
\end{example}
\section{Lecture 13 -- 26th May 2025 -- Interlude: On Algebraic Varieties}\label{sec: lecture 13}
We consider the more classicial theory of algebraic varieties with the theory of schemes in hand. For this lecture, we fix an algebraically closed field $k=\overline{k}$ of characteristic zero. 
\begin{definition}[Affine Space]\label{def: affine space}
    Affine space $\A^{n}_{k}$ is the set underlying the $k$-vector space of dimension $n$ endowed with the Zariski topology where sets of the form $V(\afrak)=\{(x_{1},\dots,x_{n})\in\A^{n}_{k}:f(x_{1},\dots,x_{n})=0,\forall f\in \afrak\}$ for ideals $\afrak$ are taken to be closed. 
\end{definition}
We can similarly define projective space. 
\begin{definition}[Projective Space]\label{def: projective space}
    Projective space $\PP^{n}_{k}$ is the set underlying the projectivization of $k^{n+1}\setminus\{0\}$ endowed with the Zariski topology where sets of the form $V_{+}(\afrak)=\{(x_{0},\dots,x_{n})\in\A^{n}_{k}:f(x_{0},\dots,x_{n})=0,\forall f\in \afrak\}$ for homogeneous ideal s$\afrak$ are taken to be closed. 
\end{definition}
\begin{remark}
    One easily verifies that \Cref{def: affine space,def: projective space} satisfies the axioms of the closed sets of a topological space.
\end{remark}
This naturally recovers algebraic sets as those closed sets of affine and projective space with the Zariski topology. 
\begin{definition}[Algebraic Set]\label{def: algebraic set}
    A set $X\subseteq\A^{n}_{k}$ (resp. $X\subseteq\PP^{n}_{k}$) is an affine (resp. projective) algebraic set if it is closed in the Zariski topology of $\A^{n}_{k}$ (resp. $\PP^{n}_{k}$). 
\end{definition}
Varieties arise as a specific class of algebraic sets. 
\begin{definition}[Algebraic Variety]\label{def: algebraic variety}
    An algebraic variety is an algebraic set whose underlying topological space is irreducible. 
\end{definition}
We can define quasiaffine and quasiprojective varieties as open subsets of affine and projective varieties, respectively. 
\begin{definition}[Quasiaffine Variety]\label{def: quasiaffine variety}
    A quasiaffine variety is an open subset of an affine variety. 
\end{definition}
\begin{definition}[Quasiprojective Variety]\label{def: quasiprojective variety}
    A quasiprojective variety is an open subset of a projective variety. 
\end{definition}
Affine and projective varieties are themselves quasiaffine and quasiprojective, respectively, and quasiaffine varieties are quaisprojective by the open embedding of $\A^{n}_{k}\to\PP^{n}_{k}$. 

We can consider the category of all varieties. 
\begin{definition}[Category of Varieties]\label{def: varieties}
    The category $\Var_{k}$ has objects quasiprojective varieties and morphisms given by regular functions -- those functions that are locally rational. 
\end{definition}
By the preceding discussion, $\Var_{k}$ contains affine, projective, and quasiaffine varieties. 
\begin{theorem}\label{thm: fully faithful embedding}
    Let $k$ be an algebraically closed field. There exists a fully fatihful embedding from $k$-varieties $\Var_{k}$ to $k$-schemes $\Sch_{k}$
    $$t:\Var_{k}\longrightarrow\Sch_{k}.$$
\end{theorem}
\begin{proof}[Outline of Proof]
    For a $k$-variety $X$, let $t(X)$ be its set of irreducible components endowed with the topology that sets of the form $t(Z)$ are closed for $Z\subseteq X$ closed. By reduction to the case of affine varieties, the map $t$ takes $X$ to $\spec(A(X))$ with inverse (on the level of topological spaces) given by taking $\mathrm{mSpec}(\Gamma(t(X),\Ocal_{t(X)}))$. The structure sheaf is induced by the obvious inclusion $\mathrm{mSpec}(A(X))\to\spec(A(X))$ for $A$ the coordinate ring of $X$ which can be seen to be fully faithful. 
\end{proof}
We can also define abstract varieties and consider its embedding into $k$-schemes. 
\begin{definition}[Abstract Varieties]\label{def: abstract varieties}
    The category of abstract variteies $\AbsVar_{k}$ is the full category of $\Sch_{k}$ spanned by integral separated $k$-schemes of finite type. 
\end{definition}
\Cref{thm: fully faithful embedding} implies that there is a fully faithful embedding $\Var_{k}\to\AbsVar_{k}$ but this is not essentially surjective, with a counterexample given by Hironaka. 

We now consider function fields of varieties. 
\begin{definition}[Function Field]\label{def: function field}
    Let $X$ be an algebraic variety. Its function field $K(X)$ is the field of equivalence classes of rational functions which agree on a nonempty open subset of their intersection. 
\end{definition}
The function field of a variety is equivalent to the function field of $t(X)$. 
\begin{definition}[Birational]\label{def: birational}
    Two algebraic varieties $X,Y$ are isomorphic if $K(X)\cong  K(Y)$ as $k$-algebras. 
\end{definition}
This in turn allows us to define birationality of integral separated finite type $k$-schemes. 

\section{Lecture 14 -- 2nd June 2025}\label{sec: lecture 14}
We make preparations towards the proof of the theorem of formal functions, which allows the computation of stalks of derived pushforwards as limit of cohomology along thickened neighborhoods. 

To wit, the tools of analytic geometry, in particular the idea of functions in a small open neighborhood of a point given by power series, can be captured in the setting of algebraic geometry using complete rings. 
\begin{definition}[Graded Construction]\label{def: graded module}
    Let $A$ be a Noetherian ring and $M$ an $A$-module with decreasing filtration
    \begin{equation}\label{eqn: decreasing filtration}
        \dots\subsetneq M_{n}\subsetneq\dots\subsetneq M_{1}\subsetneq M_{0}=M.
    \end{equation}
    The graded construction $\gr^{\bullet}(M)$ of $M$ is the direct sum $\bigoplus_{0\leq i\leq n-1} M_{i}/M_{i+1}$.
\end{definition}
\begin{remark}\label{rmk: associated graded}
    Evidently we have by definition $\gr^{i}(M)\cong M_{i}/M_{i+1}$. 
\end{remark}
\begin{definition}[Completion Along Filtration]\label{def: completion along filtration}
    Let $A$ be a Noetherian ring and $M$ an $A$-module with decreasing filtration (\ref{eqn: decreasing filtration}). The completion of $M$ along this filtration is the limit $\widehat{M}=\lim_{i}M/M_{i}$. 
\end{definition}
The filtration induces a topology on $M$ with basis given by the cosets $x+M_{i}$ for $x\in M,i\in\NN$ which is Hausdorff if and only if $\bigcap_{i\geq0}M_{i}=0$. 
\begin{definition}[Equivalent Filtrations]\label{def: equivalent filtrations}
    Let $A$ be a Noetherian ring and $M$ an $A$-module. 
    \begin{equation}\label{eqn: filtration 1}
        \dots\subsetneq M_{n}\subsetneq \dots\subsetneq M_{1}\subseteq M_{0}=M
    \end{equation} 
    \begin{equation}\label{eqn: filtration 2}
        \dots\subsetneq M_{n}'\subsetneq \dots\subsetneq M_{1}'\subseteq M_{0}'=M
    \end{equation}
    be two filtrations of $M$. The filtrations (\ref{eqn: filtration 1}) and (\ref{eqn: filtration 2}) are equivalent if the systems are final in each other -- for every $M_{n}$ there exists $M_{m}'$ such that $M_{m}'\subseteq M_{n}$ and for each $M_{n}'$ there exists $M_{m}$ such that $M_{m}\subseteq M_{n}'$. 
\end{definition}
\begin{remark}
    Any two equivalent filtrations in the sense of \Cref{def: equivalent filtrations} induce the same topology on the module $M$. 
\end{remark}
In what follows, we will consider $I\subseteq A$ and $M_{d}=I^{d}M$. 
\begin{example}
    If $M=A$, the induced topology on $A$ (as an $A$-module) is the $I$-adic topology. 
\end{example}
\begin{example}
    Let $M=A=k[x]$. There is a filtration of $A$ by $I=(x)$. Two elements of $A$ being close in the $I$-adic filtration imply that the coefficients of these polynomials agree in small degree, that is, that the polynomials are the same close to $0\in k$. 
\end{example}
Let $A$ be a Noetherian ring, $M$ a finite $A$-module, and $N\subseteq M$ an $A$-module. We seek to understand the filtration $\{I^{d}N\}_{d\geq0}$ in terms of the filtration $\{I^{d}M\cap N\}_{d\geq0}$. The Artin-Rees lemma gives us an equivalence beteween these two filtrations. 
\begin{theorem}[Artin-Rees Lemma]\label{thm: Artin-Rees}
    Let $A$ be a Noetherian ring, $M$ a finitely generated $A$-module, $N\subseteq M$ a submodule, and $I\subseteq A$ an ideal. The filtrations $\{I^{d}M\}_{d\geq0}$ and $\{I^{d}M\cap N\}_{d\geq0}$ are equivalent as filtrations. 
\end{theorem}
\begin{proof}
    For any $d$, we have the containment $I^{d}N\subseteq I^{d}M\cap N$. By \Cref{def: equivalent filtrations}, it suffices to find $m\geq0$ such that $I^{m}M\cap N\subseteq I^{d}N$. For any $c\geq0$ we have that $I^{c}M\cap N\subseteq N$ so it suffices to find $c\geq0$ such that $I^{d+c}M\cap N\subseteq I^{d}(I^{c}M\cap N)$ in which case equality will hold. 
    
    Consider the graded ring $S=\bigoplus_{k\geq0}I^{k}$ and the $S$-module $\widetilde{M}=\bigoplus_{k\geq0}I^{k}M$. Since $I$ is finitely generated, $S$ is a Noetherian ring being the quotient of a finitely-generated $A$-algebra, and $\widetilde{M}$ is finite over $S$ being generated by the generators of $M$ over $A$. The submodule $\widetilde{N}=\bigoplus_{k\geq0}(I^{k}M\cap N)$ of $\widetilde{M}$ is finitely generated over $S$ as well. Let $x_{1},\dots,x_{r}$ be generators of $\widetilde{N}$ with $x_{j}$ in some $I^{k_{j}}M\cap N$. By the finite generation hypothesis, we can take $c$ such that $k_{j}<c$ for all $1\leq j\leq r$. By construction, each $x\in I^{d+c}M\cap N$ can be written as $\sum_{j=1}^{r}\alpha_{j}x_{j}$ with $\alpha_{j}\in I^{d+c-k_{j}}$. This shows $x\in I^{d}(I^{c}M\cap N)$, whence the claim. 
\end{proof}
This ``commutativity'' of intersections with submodules allows us to show that rings map injectively into their completions. 
\begin{theorem}[Krull -- Intersection]\label{thm: Krull intersection}
    Let $A$ be a Noetherian local ring with maximal ideal $\mfrak$. Then $\bigcap_{d\geq0}\mfrak^{d}=0$. 
\end{theorem}
\begin{proof}
    Let $I=\bigcap_{d\geq0}\mfrak^{d}$. We have that $I=\mfrak^{d}\cap I$ for every $d\geq0$. By the Artin-Rees lemma \Cref{thm: Artin-Rees}, there exists $k\geq0$ such that $I=\mfrak^{k}\cap I\subseteq\mfrak I$. Thus $\mfrak I=I$. By Nakayama's lemma, $I=0$. 
\end{proof}
We immediately deduce the following. 
\begin{corollary}\label{corr: injective map into completion}
    Let $A$ be a Noetherian local ring with maximal ideal $\mfrak$. Then $A\to\widehat{A}$ is injective. 
\end{corollary}
\begin{proof}
    The kernel of the map $A\to\widehat{A}$ is given by $\bigcap_{d\geq0}\mfrak^{d}$ -- the intersection of the kernels of the component maps $A\mapsto A/\mfrak^{d}$ in the system 
    $$% https://q.uiver.app/#q=WzAsNSxbMCwwLCJBIl0sWzEsMSwiQS9cXG1mcmFrXnszfSJdLFsyLDEsIkEvXFxtZnJha157Mn0iXSxbMywxLCJBL1xcbWZyYWsiXSxbMCwxLCJcXGRvdHMiXSxbMCwxXSxbMCwyXSxbMSwyXSxbMiwzXSxbNCwxXSxbMCwzXV0=
    \begin{tikzcd}
        A \\
        \dots & {A/\mfrak^{3}} & {A/\mfrak^{2}} & {A/\mfrak}
        \arrow[from=1-1, to=2-2]
        \arrow[from=1-1, to=2-3]
        \arrow[from=1-1, to=2-4]
        \arrow[from=2-1, to=2-2]
        \arrow[from=2-2, to=2-3]
        \arrow[from=2-3, to=2-4]
    \end{tikzcd}$$
    By \Cref{thm: Krull intersection}, the kernel is trivial, showing the map is injective. 
\end{proof}
We collect some important properties of complete rings. 
\begin{proposition}\label{prop: completions of rings}
    Let $A$ be a Noetherian ring, $M$ an $A$-module, and $I\subseteq A$ an ideal. 
    \begin{enumerate}[label=(\roman*)]
        \item The functor $\Mod_{A}\to\Mod_{\widehat{A}}$ by $M\mapsto \widehat{M}$ is exact on finitely generated modules. 
        \item If $M$ is a finitely generated $A$-module, then $M\otimes_{A}\widehat{A}\to\widehat{M}$ is an isomorphism. 
        \item $\widehat{A}$ is flat over $A$. 
        \item $\widehat{A}$ is Noetherian. 
        \item If $A$ is a further a local ring, then $\widehat{A}$ with maximal ideal $\widehat{\mfrak}$ is a Noetherian local ring and $\gr_{\mfrak}^{\bullet}(A)\cong\gr_{\widehat{\mfrak}}^{\bullet}(\widehat{A})$. 
        \item Let $A$ be a local Noetherian $k=A/\mfrak$-algebra of dimension $d$. The following are equivalent: 
        \begin{enumerate}[label=(\alph*)]
            \item $A$ is regular. 
            \item The local ring $\widehat{A}$ with maximal ideal $\widehat{\mfrak}$ of (v) is regular. 
            \item $\gr_{\mfrak}^{\bullet}(A)\cong\gr_{\widehat{\mfrak}}^{\bullet}(\widehat{A})=k[x_{1},\dots,x_{d}]$. 
        \end{enumerate}
        \item Let $A$ be a local Noetherian $k=A/\mfrak$-algebra of dimension $d$. Then $\widehat{A}\cong k[[x_{1},\dots,x_{d}]]$. 
    \end{enumerate}
\end{proposition}
\begin{proof}[Proof of (i)]
    If $M$ is finitely generated, there is a surjection $A^{\oplus r}\to M$ yielding a surjection $\widehat{A}^{\oplus r}\to\widehat{M}$ showing that each $\widehat{M}$ is finitely generated over $\widehat{A}$. For $0\to M_{1}\to M_{2}\to M_{3}\to0$ a short exact sequence of $A$-modules, we get a short exact sequence 
    $$0\to I^{d}M_{2}\cap M_{1}\to I^{d}M_{2}\to I^{d}M_{3}\to0$$
    where applying the Artin-Rees lemma \Cref{thm: Artin-Rees} and on passage to the limit we get the exact sequence $0\to\widehat{M_{1}}\to\widehat{M_{2}}\to\widehat{M_{3}}$ since the limit need not preserve exactness on the right. It thus remains to prove exactness on the right. Let $x=(m_{1},m_{2},\dots)\in\widehat{M_{3}}$ where $m_{d}\in M_{3}/I^{d}M_{3}$. By surjectivity of the map $\varphi:M_{2}\to M_{3}$, we have $m_{d}=\varphi(m_{d}')$ for some $m_{d}'\in M_{2}/I^{d}M_{2}$. Denote $\pi:M_{1}/(I^{2}M_{2}\cap M_{1})\to M_{1}/(IM_{2}\cap M_{1})$ and $\rho_{i}:M_{i}/I^{2}M_{i}\to M_{i}/IM_{i}$ for $i\in\{2,3\}$. This gives the diagram 
    $$% https://q.uiver.app/#q=WzAsMTAsWzEsMCwiTV97MX0vKEleezJ9TV97Mn1cXGNhcCBNX3sxfSkiXSxbMSwxLCJNX3sxfS8oSU1fezJ9XFxjYXAgTV97MX0pIl0sWzMsMCwiTV97Mn0vSV57Mn1NX3syfSJdLFszLDEsIk1fezJ9L0lNX3syfSJdLFs1LDAsIk1fezN9L0leezJ9TV97M30iXSxbNSwxLCJNX3szfS9JTV97M30iXSxbMCwwLCIwIl0sWzAsMSwiMCJdLFs2LDAsIjAiXSxbNiwxLCIwIl0sWzAsMSwiXFxwaSIsMl0sWzYsMF0sWzcsMV0sWzEsM10sWzMsNV0sWzUsOV0sWzQsOF0sWzIsNF0sWzAsMl0sWzQsNSwiXFxyaG9fezN9IiwyXSxbMiwzLCJcXHJob197Mn0iLDJdXQ==
    \begin{tikzcd}
        0 & {M_{1}/(I^{2}M_{2}\cap M_{1})} && {M_{2}/I^{2}M_{2}} && {M_{3}/I^{2}M_{3}} & 0 \\
        0 & {M_{1}/(IM_{2}\cap M_{1})} && {M_{2}/IM_{2}} && {M_{3}/IM_{3}} & 0
        \arrow[from=1-1, to=1-2]
        \arrow[from=1-2, to=1-4]
        \arrow["\pi"', from=1-2, to=2-2]
        \arrow[from=1-4, to=1-6]
        \arrow["{\rho_{2}}"', from=1-4, to=2-4]
        \arrow[from=1-6, to=1-7]
        \arrow["{\rho_{3}}"', from=1-6, to=2-6]
        \arrow[from=2-1, to=2-2]
        \arrow[from=2-2, to=2-4]
        \arrow[from=2-4, to=2-6]
        \arrow[from=2-6, to=2-7]
    \end{tikzcd}$$
    where $\pi,\rho_{2},\rho_{3}$ are surjective. Then $m_{1}=\rho_{3}(m_{2})$ by definition so $\rho_{2}(m_{2}'),m_{1}'\in M_{2}/IM_{2}$ have the same image in $M_{3}/IM_{3}$. That is, 
    $$% https://q.uiver.app/#q=WzAsNCxbMCwwLCJtX3syfSciXSxbMCwxLCJtX3sxfSc9XFxyaG9fezJ9KG1fezJ9JykiXSxbMiwwLCJtX3syfSJdLFsyLDEsIm1fezF9PVxccmhvX3szfShtX3syfSkiXSxbMCwxLCIiLDIseyJzdHlsZSI6eyJ0YWlsIjp7Im5hbWUiOiJtYXBzIHRvIn19fV0sWzEsMywiIiwyLHsic3R5bGUiOnsidGFpbCI6eyJuYW1lIjoibWFwcyB0byJ9fX1dLFsyLDMsIiIsMCx7InN0eWxlIjp7InRhaWwiOnsibmFtZSI6Im1hcHMgdG8ifX19XSxbMCwyLCIiLDAseyJzdHlsZSI6eyJ0YWlsIjp7Im5hbWUiOiJtYXBzIHRvIn19fV1d
    \begin{tikzcd}
        {m_{2}'} && {m_{2}} \\
        {m_{1}'=\rho_{2}(m_{2}')} && {m_{1}=\rho_{3}(m_{2})}
        \arrow[maps to, from=1-1, to=1-3]
        \arrow[maps to, from=1-1, to=2-1]
        \arrow[maps to, from=1-3, to=2-3]
        \arrow[maps to, from=2-1, to=2-3]
    \end{tikzcd}$$
    for the rightmost square. Thus $\rho_{2}(m_{2}')-m_{1}'\in M_{1}/IM_{1}$. Since $\pi$ is surjective, we can choose $m_{2}''\in M_{1}/(I^{2}M_{2}\cap M_{1})$ such that $\pi(m_{2}'')=\rho_{2}(m_{2}')-m_{1}'$. That is, 
    $$% https://q.uiver.app/#q=WzAsNCxbMCwwLCJtX3syfScnIl0sWzIsMSwiXFxyaG9fezJ9KG1fezJ9JyktbV97MX0nPTAiXSxbMiwwLCJtX3syfScnIl0sWzAsMSwiXFxyaG9fezJ9KG1fezJ9JyktbV97MX0nIl0sWzAsM10sWzMsMV0sWzIsMV0sWzAsMl1d
    \begin{tikzcd}
        {m_{2}''} && {m_{2}''} \\
        {\rho_{2}(m_{2}')-m_{1}'} && {\rho_{2}(m_{2}')-m_{1}'=0}
        \arrow[from=1-1, to=1-3]
        \arrow[from=1-1, to=2-1]
        \arrow[from=1-3, to=2-3]
        \arrow[from=2-1, to=2-3]
    \end{tikzcd}$$
    for the leftmost square. Substituting $m_{2}'$ by $m_{2}'-m_{2}''$ we get surjectivity, and by induction we can lift each element of the sequence, giving surjectivity $\widehat{M_{2}}\to\widehat{M_{3}}$. 
\end{proof}
\begin{proof}[Proof of (ii)]
    By the finite generation hypothesis, we have an exact sequence $A^{\oplus m}\to A^{\oplus n}\to M\to0$. Tensoring with $A$ yields a diagram 
    $$% https://q.uiver.app/#q=WzAsOCxbMCwwLCJcXHdpZGVoYXR7QX1ee1xcb3BsdXMgbX0iXSxbMCwxLCJcXHdpZGVoYXR7QX1ee1xcb3BsdXMgbX0iXSxbMiwwLCJcXHdpZGVoYXR7QX1ee1xcb3BsdXMgbn0iXSxbMiwxLCJcXHdpZGVoYXR7QX1ee1xcb3BsdXMgbn0iXSxbNCwwLCJNXFxvdGltZXNfe0F9XFx3aWRlaGF0e0F9Il0sWzQsMSwiXFx3aWRlaGF0e019Il0sWzUsMCwiMCJdLFs1LDEsIjAiXSxbMCwyXSxbMiw0XSxbNCw2XSxbNSw3XSxbMyw1XSxbMSwzXSxbMCwxLCJcXHdyIiwyXSxbMiwzLCJcXHdyIiwyXSxbNCw1XSxbNiw3LCJcXHdyIiwyXV0=
    \begin{tikzcd}
        {\widehat{A}^{\oplus m}} && {\widehat{A}^{\oplus n}} && {M\otimes_{A}\widehat{A}} & 0 \\
        {\widehat{A}^{\oplus m}} && {\widehat{A}^{\oplus n}} && {\widehat{M}} & 0
        \arrow[from=1-1, to=1-3]
        \arrow["\wr"', from=1-1, to=2-1]
        \arrow[from=1-3, to=1-5]
        \arrow["\wr"', from=1-3, to=2-3]
        \arrow[from=1-5, to=1-6]
        \arrow[from=1-5, to=2-5]
        \arrow["\wr"', from=1-6, to=2-6]
        \arrow[from=2-1, to=2-3]
        \arrow[from=2-3, to=2-5]
        \arrow[from=2-5, to=2-6]
    \end{tikzcd}$$
    with exact rows, giving the isomorphism $M\otimes_{A}\widehat{A}\to\widehat{M}$ by the four-lemma. 
\end{proof}
\begin{proof}[Proof of (iii)]
    This is immediate from (i) and (ii) which show that completion given by $-\otimes_{A}\widehat{A}$ is exact so $\widehat{A}$ is flat as an $A$-module.  
\end{proof}
\begin{proof}[Proof of (iv)]
    Let $a_{1},\dots,a_{r}$ be generators of $I$. Consider the map $A[[x_{1},\dots,x_{r}]]\to\widehat{A}$ by $x_{i}\mapsto a_{i}$ induced by the map $A[x_{1},\dots,x_{r}]\to A$ by $x_{i}\mapsto a_{i}$ -- a surjective map from a Noetherian ring. This shows that $A[[x_{1},\dots,x_{r}]]\to\widehat{A}$ is a surjection by (i). And since $A[[x_{1},\dots,x_{d}]]$ is Noetherian, we get the claim. 
\end{proof}
\begin{proof}[Proof of (v)]
    Let $\widehat{\mfrak}=\lim_{d\in\NN}\mfrak/\mfrak^{d}$. By (i) we have $\widehat{A}/\widehat{\mfrak}\cong\widehat{A/\mfrak}\cong A/\mfrak$ since $\mfrak$ is trivial in $A/\mfrak$. Hence $\widehat{\mfrak}\subseteq\widehat{A}$ is maximal. To see that $\widehat{A}$ is local, it suffices to show that every element of $\widehat{A}\setminus\widehat{m}$ is a unit. Let $x=(a_{1},a_{2},\dots)$ be such an element. Necessarily $a_{1}\in (A/\mfrak)^{\times}$ is nonzero, and $a_{1}$ is the projection in $A/\mfrak$ of each $a_{i}$ so each $a_{i}$ is a unit, and $x^{-1}(a_{1}^{-1},a_{2}^{-1},\dots)$, as desired. 

    For the claim on the gradeds, it suffices to show that $A/\mfrak^{d}\cong\widehat{A}\cong\widehat{\mfrak}^{d}$ in which case the graded rings will agree in each degree. Note that the proof of the first part of the statement already gives $\gr_{\mfrak}^{0}(A)\cong\gr_{\widehat{\mfrak}}^{0}(\widehat{A})$. By (i) we have that $\widehat{A}/\widehat{\mfrak}^{d}\cong\widehat{A/\mfrak^{d}}$. Using that $\mfrak^{n}/\mfrak^{d}=0$ for all $n\geq d$, we use the diagram 
    $$% https://q.uiver.app/#q=WzAsMTAsWzEsMCwiXFxtZnJha157ZH0vXFxtZnJha157ZCsxfSJdLFszLDAsIkEvXFxtZnJha157ZCsxfSJdLFs1LDAsIkEvXFxtZnJha157ZH0iXSxbMCwwLCIwIl0sWzAsMSwiMCJdLFsxLDEsIlxcd2lkZWhhdHtcXG1mcmFrfV57ZH0vXFx3aWRlaGF0e1xcbWZyYWt9XntkKzF9Il0sWzMsMSwiXFx3aWRlaGF0e0F9L1xcd2lkZWhhdHtcXG1mcmFrfV57ZCsxfSJdLFs1LDEsIlxcd2lkZWhhdHtBfS9cXHdpZGVoYXR7XFxtZnJha31ee2R9Il0sWzYsMCwiMCJdLFs2LDEsIjAiXSxbMywwXSxbMCwxXSxbMSwyXSxbMiw4XSxbNyw5XSxbNiw3XSxbNCw1XSxbNSw2XSxbMCw1XSxbMSw2LCJcXHdyIl0sWzIsNywiXFx3ciJdXQ==
    \begin{tikzcd}
        0 & {\mfrak^{d}/\mfrak^{d+1}} && {A/\mfrak^{d+1}} && {A/\mfrak^{d}} & 0 \\
        0 & {\widehat{\mfrak}^{d}/\widehat{\mfrak}^{d+1}} && {\widehat{A}/\widehat{\mfrak}^{d+1}} && {\widehat{A}/\widehat{\mfrak}^{d}} & 0
        \arrow[from=1-1, to=1-2]
        \arrow[from=1-2, to=1-4]
        \arrow[from=1-2, to=2-2]
        \arrow[from=1-4, to=1-6]
        \arrow["\wr", from=1-4, to=2-4]
        \arrow[from=1-6, to=1-7]
        \arrow["\wr", from=1-6, to=2-6]
        \arrow[from=2-1, to=2-2]
        \arrow[from=2-2, to=2-4]
        \arrow[from=2-4, to=2-6]
        \arrow[from=2-6, to=2-7]
    \end{tikzcd}$$
    with exact rows to deduce that $\mfrak^{d}/\mfrak^{d+1}\cong\widehat{\mfrak}^{d}/\widehat{\mfrak}^{d+1}$ so each $\gr_{\mfrak}^{i}(A)\cong\gr_{\widehat{\mfrak}}^{i}(\widehat{A})$ yielding the claim. 
\end{proof}
\begin{proof}[Proof of (vi)]
    We show (a)$\Leftrightarrow$(c)$\Leftrightarrow$(b). 

    (a)$\Rightarrow$(c) Recall from \Cref{def: regular local ring} that for $A$ regular we have $\mfrak=(a_{1},\dots,a_{d})$. Denoting $k=A/\mfrak$ the residue field of $A$, we have a map $k[x_{1},\dots,x_{d}]\to\gr^{\bullet}_{\mfrak}(A)$ by $x_{i}\mapsto a_{i}$. This is an isomorphism in degree 0 and 1, and therefore an isomorphism since $\gr^{i}_{\mfrak}(A)\cong\Sym^{i}(\mfrak/\mfrak^{2})\cong\mfrak^{i}/\mfrak^{i+1}$ by regularity. 

    (c)$\Rightarrow$(a) If $\gr_{\mfrak}^{\bullet}(A)\cong k[x_{1},\dots,x_{d}]$ then $\mfrak/\mfrak^{2}$ is generated by $d$ elements, and $A$ is regular. 

    (b)$\Rightarrow$(c) This is the argument of (a)$\Rightarrow$(c) verbatim. By regularity, we take $\widehat{\mfrak}=(\widehat{a}_{1},\dots,\widehat{a}_{d})$ and we have $k=A/\mfrak\cong \widehat{A}/\widehat{\mfrak}$ by (i). The map $k[x_{1},\dots,x_{d}]\to\gr^{\bullet}_{\widehat{\mfrak}}(\widehat{A})$ is an isomorphism in degrees 0 and 1, and an isomorphism globally by $\gr^{i}_{\widehat{A}}(\widehat{A})\cong\Sym^{i}(\widehat{\mfrak}/\widehat{\mfrak}^{2})\cong\widehat{\mfrak}^{i}/\widehat{\mfrak}^{i+1}\cong\mfrak^{i}/\mfrak^{i+1}$ due to regularity. 

    (c)$\Rightarrow$(a) This is the argument of (c)$\Rightarrow$(a) verbatim. $\gr^{\bullet}_{\widehat{\mfrak}}(\widehat{A})\cong k[x_{1},\dots,x_{d}]$ is regular, so $\widehat{\mfrak}$ is generated by $d$ elements, hence regular. 
\end{proof}
\begin{proof}[Proof of (vii)]
    If $\mfrak=(a_{1},\dots,a_{d})$, we have the map $k[x_{1},\dots,x_{d}]\to\widehat{A}$ by $x_{i}\mapsto a_{i}$. We have an isomorphism $A/\mfrak\cong k$. So by (vi), the isomorphism of graded constructions $\gr_{\mfrak}^{\bullet}(A)\cong\gr_{\widehat{\mfrak}}^{\bullet}(\widehat{A})=k[x_{1},\dots,x_{d}]$ induce an isomorphism between the completions and $k[[x_{1},\dots,x_{d}]]$. 
\end{proof}
We conclude our discussion of completion by stating Cohen's structure theorem. 
\begin{theorem}[Cohen -- Structure]\label{thm: Cohen structure}
    Let $A$ be a complete Noetherian local ring with maximal ideal $\mfrak$ containing some field. Then there exixts a field $k$ in $A$ such that $k=A/\mfrak$ is the residue field of $A$, and $A\cong k[[x_{1},\dots,x_{d}]]/I$ fo some $d\geq0$ and ideal $I$. Furthermore, if $A$ is regular, then $A\cong k[[x_{1},\dots,x_{d}]]$. 
\end{theorem}
\begin{proof}
    See \cite[\href{https://stacks.math.columbia.edu/tag/032A}{Tag 032A}]{stacks-project}.
\end{proof}
We consider some examples. 
\begin{example}
    Let $C=V(y^{2}-x^{3}-x^{2})\subseteq\A^{2}_{k},C'=V(y^{2}-x^{2})\subseteq\A^{2}_{k}$. Both of these curves are nodal at the origin, but their local rings $(k[x,y]/(y^{2}-x^{3}-x^{2}))_{(x,y)},(k[x,y]/(y^{2}-x^{2}))_{(x,y)}$ are not isomorphic. In a certain sense, they are not sufficiently ``local'' to capture the geometric behavior of these two being nodes. However, their completions at $(x,y)$ are isomorphic. 
\end{example}
\section{Lecture 15 -- 5th June 2025}\label{sec: lecture 15}
We make preparations towards the proof of the theorem of formal functions. 
\begin{definition}[Faithfully Flat]\label{def: faithfully flat}
    Let $\varphi:A\to B$ be a ring map. $\varphi$ is faithfully flat if exactness of $0\to M_{1}\to M_{2}\to M_{3}\to 0$ in $\Mod_{A}$ is equivalent to exactness of $0\to M_{1}\otimes_{A}B\to M_{2}\otimes_{A}B\to M_{3}\otimes_{A}B\to0$ in $\Mod_{B}$. 
\end{definition}
We consider some properties thereof. 
\begin{proposition}\label{prop: faithful flatness properties}
    Let $\varphi:A\to B$ be a ring map. 
    \begin{enumerate}[label=(\roman*)]
        \item If $\varphi$ is faithfully flat, then $\varphi$ is injective. 
        \item If $\varphi$ is faithfully flat then for $I\subseteq A$ an ideal, $IB\cap A=I$. 
        \item If $\varphi$ is a flat local homomorphism of local rings then $\varphi$ is faithfully flat. 
        \item If $A$ is a Noetherian local ring, then $A\hookrightarrow\widehat{A}$ is faithfully flat. 
    \end{enumerate}
\end{proposition}
\begin{proof}[Proof of (i)]
    Consider the morphism 
    $$\varphi\otimes\id_{B}=\widetilde{\varphi}:B=A\otimes_{A}B\to B\to B\otimes_{A}B.$$
    Note that the composition $B\xrightarrow{\widetilde{\varphi}}B\otimes_{A}B\xrightarrow{b\otimes b'\mapsto bb'}B$ is the identity as $a\otimes b=\varphi(a)b\mapsto \varphi(a)\otimes b\mapsto \varphi(a)b$ for all $a\in A,b\in B$. Thus $\widetilde{\varphi}$ is injective. And since $\widetilde{\varphi}$ is injective (as a map of $B$-modules), $\varphi$ is injective as well (as a map of $A$-modules). 
\end{proof}
\begin{proof}[Proof of (ii)]
    By (i), $\varphi$ is a ring extension so we have $A\cap IB\subseteq I$ and thus a (not necessarily exact) sequence $0\to IB\cap A\to A\to A/I\to0$. But applying $\otimes_{A}B$ this yields $0\to IB\to B\to B/IB\to 0$ which is exact, so $0\to IB\cap A\to A\to A/I\to0$ was exact to begin with. 
\end{proof}
\begin{proof}[Proof of (iii)]
    Let $\phi:M\to N$ be a homomorphism of $A$-modules such that its base extension $\widetilde{\phi}:M\otimes_{A}B\to N\otimes_{A}B$ is injective. We have that $\ker(\phi)\otimes_{A}B=0$. Let $m\in \ker(\phi)$ and consider $I=\{a\in A:am=0\}\subsetneq A$. The morphism $A\to \ker(\phi)$ by $a\mapsto am$ has kernel $I$ so $A/I\hookrightarrow \ker(\phi)$. By flatness, $A/I\otimes_{A}B=B/IB\hookrightarrow \ker(\phi)\otimes_{A}B=0$ so $B/IB=0$. Since $I$ is an ideal of the local ring $A$, either $I=0$ or $I=\mfrak_{A}$. In the first case, $B=0$ since the quotient by the trivial ideal is an isomorphism. In the second case $B/\mfrak B=0$ implies $B=0$ by Nakayama's lemma. This yields a contradiction in both cases as the zero ring is not a local ring. 
\end{proof}
\begin{proof}[Proof of (iv)]
    This is trivial from (iii) as \Cref{prop: completions of rings} (iii) shows that $A\hookrightarrow \widehat{A}$ is flat. 
\end{proof}
With the language of faithful flatness and completions in hand, we can prove the following characteriziation of \'{e}tale morphisms in a special case. 
\begin{proposition}\label{prop: etaleness by completions}
    Let $\varphi:A\to B$ be a local homomorphism of Noetherian local rings where $A,B$ are furthermore $k=A/\mfrak_{A}\cong B/\mfrak_{B}$-algebras. Then $\varphi$ is \'{e}tale if and only if $\widehat{\varphi}:\widehat{A}\to\widehat{B}$ is an isomorphism. 
\end{proposition}
\begin{proof}
    $(\Rightarrow)$ Since $\varphi$ is \'{e}tale, it is in particular flat and unramified \Cref{def: etale at point}. Thus $\varphi$ is injective by \Cref{prop: faithful flatness properties} (i). Moreover, since $\varphi$ is uramified, we have $\mfrak_{A}B=\mfrak_{B}$. Thus since $A/\mfrak_{A}\cong B/\mfrak_{B}$ we have 
    $$B\cong A+\mfrak_{B}\cong A+\mfrak_{A}B\cong A+\mfrak_{A}(A+\mfrak_{A}B)=A+\mfrak_{A}^{2}B\cong\dots$$
    so $B\cong A+\mfrak_{B}^{d}$ for each $d\geq0$. The homomorphism $\varphi_{d}:A/\mfrak_{A}^{d}\to B/\mfrak_{B}^{d}$ fits into a commutative diagram 
    $$% https://q.uiver.app/#q=WzAsMyxbMCwwLCJBIl0sWzIsMCwiQS9cXG1mcmFrX3tBfV57ZH0iXSxbMSwxLCJCL1xcbWZyYWtfe0J9XntkfSJdLFswLDEsIiIsMCx7InN0eWxlIjp7ImhlYWQiOnsibmFtZSI6ImVwaSJ9fX1dLFswLDIsIiIsMix7InN0eWxlIjp7ImhlYWQiOnsibmFtZSI6ImVwaSJ9fX1dLFsxLDIsIlxcdmFycGhpX3tkfSJdXQ==
    \begin{tikzcd}
        A && {A/\mfrak_{A}^{d}} \\
        & {B/\mfrak_{B}^{d}}
        \arrow[two heads, from=1-1, to=1-3]
        \arrow[two heads, from=1-1, to=2-2]
        \arrow["{\varphi_{d}}", from=1-3, to=2-2]
    \end{tikzcd}$$
    where the composition $A\to B\to B/\mfrak_{B}^{d}$ is surjective being the composition of an injective and a surjective map. Thus by cancellation, $\varphi_{d}$ is surjective too. The kernel of $\varphi_{d}$ is $(A\cap \mfrak_{B}^{d})\cap\mfrak_{A}^{d}$, but $A\cap\mfrak_{B}^{d}=A\cap\mfrak_{A}^{d}B=\mfrak_{A}^{d}$ by (i) above, so $\varphi_{d}$ is an isomorphism for each $d\geq0$. This induces an isomorphism on each of the terms of the filtration, and thus an isomorphism on completion. 

    $(\Leftarrow)$ Suppose that $\widehat{\varphi}:\widehat{A}\to\widehat{B}$ is an isomorphism. Then we have by \Cref{prop: completions of rings} (v) that 
    $$\gr^{\bullet}_{\mfrak_{A}}(A)\cong\gr^{\bullet}_{\widehat{\mfrak_{A}}}(\widehat{A})\cong\gr^{\bullet}_{\widehat{\mfrak_{B}}}(\widehat{B})\cong\gr^{\bullet}_{\mfrak_{B}}(B).$$
    and in particular $\mfrak_{A}/\mfrak_{A}^{2}\cong\widehat{\mfrak_{A}}/\widehat{\mfrak_{A}}^{2}\cong\widehat{\mfrak_{B}}/\widehat{\mfrak_{B}}^{2}\cong\mfrak_{B}/\mfrak_{B}^{2}$. In particular, $\mfrak_{B}\cong\mfrak_{A}B+\mfrak_{B}^{2}$. We have finite $B$-mdoules $\mfrak_{A}B\subseteq\mfrak_{B}$ both finite $B$-modules with $\mfrak_{B}=\mfrak_{A}B+\mfrak_{B}^{2}$ so by Nakayama's lemma we have $\mfrak_{A}B=\mfrak_{B}$. This shows that $\varphi$ is unramified. It remains to show $\varphi$ is flat. We have a commutative diagram 
    $$% https://q.uiver.app/#q=WzAsNCxbMCwwLCJBIl0sWzIsMCwiQiJdLFswLDEsIlxcd2lkZWhhdHtBfSJdLFsyLDEsIlxcd2lkZWhhdHtCfSJdLFswLDEsIlxcdmFycGhpIl0sWzIsMywiXFx3aWRlaGF0e1xcdmFycGhpfSIsMl0sWzAsMl0sWzEsM11d
    \begin{tikzcd}
        A && B \\
        {\widehat{A}} && {\widehat{B}}
        \arrow["\varphi", from=1-1, to=1-3]
        \arrow[from=1-1, to=2-1]
        \arrow[from=1-3, to=2-3]
        \arrow["{\widehat{\varphi}}"', from=2-1, to=2-3]
    \end{tikzcd}$$
    with vertical maps completions at $\mfrak_{A},\mfrak_{B}$, respectively. $A\to\widehat{A}$ is faithfully flat by \Cref{prop: completions of rings} (iii) and \Cref{prop: faithful flatness properties} (iii), and $\widehat{\varphi}$ is an isomorphism hence faithfully flat, and $B\to\widehat{B}$ is faithfully flat once again by \Cref{prop: completions of rings} (iii) and \Cref{prop: faithful flatness properties} (iii). It suffices to show that for $M\hookrightarrow N$ an injection of $A$-modules that $M\otimes_{A}B\to N\otimes_{A}B$ is injective as a map of $B$-modules. Suppose to the contrary that $M\otimes_{A}B\to N\otimes_{A}B$ is not injective, yielding an exact sequence of $B$-modules
    $$0\to \ker(M\otimes_{A}B\to N\otimes_{A}B)\to M\otimes_{A}B\to N\otimes_{A}B$$
    and by faithful flatness of $B\to\widehat{B}$ an exact sequence 
    $$0\to \ker(M\otimes_{A}B\to N\otimes_{A}B)\otimes_{B}\widehat{B}\to M\otimes_{A}\widehat{B}\to N\otimes_{A}\widehat{B}$$
    of $\widehat{B}$-modules. We have that $\ker(M\otimes_{A}B\to N\otimes_{A}B)\otimes_{B}\widehat{B}=0$ as $A\to \widehat{A}\to\widehat{B}$ is faithfully flat. So by faithful flatness of $B\to\widehat{B}$, $\ker(M\otimes_{A}B\to N\otimes_{A}B)=0$ as well, showing that $-\otimes_{A}B$ preserves injectivity, and thus flatness of $A\to B$. 
\end{proof}
\begin{remark}
    The statement of \Cref{prop: etaleness by completions} should be thought of to be the situation $f:X\to Y$ a morphism of $k$-schemes, $x\in X(k)$, $y=f(x)$, with the induced local homomorphism of local rings $\Ocal_{Y,y}\mapsto\Ocal_{X,x}$, showing that \'{e}taleness can be checked on the induced morphism $\widehat{\Ocal_{Y,y}}\to\widehat{\Ocal_{X,x}}$. 
\end{remark}
We now set up the statement of the theorem of formal functions: a result that allows us to compute cohomology of stalks in terms of a limit of cohomologies of thickenings. Let $f:X\to Y$ be a proper morphism of locally Noetherian schemes. For $\Fcal\in\Coh(X)$ a coherent sheaf, is higher direct image $R^{i}f_{*}\Fcal\in\Coh(Y)$. For $y\in Y$, we have a Cartesian square 
$$% https://q.uiver.app/#q=WzAsNCxbMCwxLCJcXHNwZWMoXFxrYXBwYSh5KSkiXSxbMiwxLCJZIl0sWzIsMCwiWCJdLFswLDAsIlhfe3l9Il0sWzIsMSwiZiJdLFswLDEsIlxcaW90YSIsMl0sWzMsMiwiXFxpb3RhJyJdLFszLDAsImYnIiwyXV0=
\begin{tikzcd}
	{X_{y}} && X \\
	{\spec(\kappa(y))} && Y
	\arrow["{\iota'}", from=1-1, to=1-3]
	\arrow["{f'}"', from=1-1, to=2-1]
	\arrow["f", from=1-3, to=2-3]
	\arrow["\iota"', from=2-1, to=2-3]
\end{tikzcd}$$
which by left exactness of global sections induces a natural transformation of functors $(\iota^{*}\circ f_{*})(-)\to (f'_{*}\circ\iota'^{*})(-)$. Observe that for $\Fcal\in\Coh(X)$ applying the source to $\Fcal$ yields $(\iota^{*}\circ f_{*})(\Fcal)$ which can be identified with $(f_{*}\Fcal)_{y}$, and applying the target functor to $\Fcal$ can be similarly identified with $\Gamma(X_{y},\Fcal|_{X_{y}})$ recalling that $\iota'^{*}$ is restriction and $f'_{*}$ the direct image to a point computes cohomology. Using the universal property of $\delta$-functors, we can pass to $\delta$-functors to get morphisms $(R^{i}f_{*}\Fcal)_{y}\to H^{i}(X_{y},\Fcal|_{X_{y}})$ for all $i\geq0$. For any $n\geq0$ we can more generally consider the Cartesian diagram 
$$% https://q.uiver.app/#q=WzAsNCxbMCwxLCJcXHNwZWMoXFxPY2FsX3tZLHl9L1xcbWZyYWtfe3l9XntufSkiXSxbMiwxLCJZIl0sWzIsMCwiWCJdLFswLDAsIlhfe3l9Xnsobil9Il0sWzIsMSwiZiJdLFswLDFdLFszLDJdLFszLDAsImZfe259IiwyXV0=
\begin{tikzcd}
	{X_{y}^{(n)}} && X \\
	{\spec(\Ocal_{Y,y}/\mfrak_{y}^{n})} && Y
	\arrow[from=1-1, to=1-3]
	\arrow["{f_{n}}"', from=1-1, to=2-1]
	\arrow["f", from=1-3, to=2-3]
	\arrow[from=2-1, to=2-3]
\end{tikzcd}$$
where $X_{y}^{(n)}$ is the $n$-th order thickening of the fiber $X_{y}$. Repeating the argument for $\delta$-functors above, we get for $\Fcal\in\Coh(X)$ morphisms $(f_{*}\Fcal)\otimes_{\Ocal_{Y,y}}(\Ocal_{Y,y}/\mfrak_{y}^{n})\to \Gamma(X_{y}^{(n)},\Fcal|_{X_{y}^{(n)}})$ inducing 
\begin{equation}\label{eqn: stepwise map on cohomology}
    (R^{i}f_{*}\Fcal)\otimes_{\Ocal_{Y,y}}(\Ocal_{Y,y}/\mfrak_{y}^{n})\to H^{i}(X_{y}^{(n)},\Fcal_{X_{y}^{(n)}})
\end{equation}
for each fixed $n$. Moreover, these maps are compatible with the restriction maps on thickenings $X_{y}^{(n-1)}\hookrightarrow X_{y}^{(n)}$ in the sense that there are commutative diagrams 
$$% https://q.uiver.app/#q=WzAsNCxbMCwwLCJcXGxlZnQoUl57aX1mX3sqfVxcRmNhbFxccmlnaHQpXFxvdGltZXNfe1xcT2NhbF97WSx5fX1cXGxlZnQoXFxPY2FsX3tZLHl9L1xcbWZyYWtfe3l9XntufVxccmlnaHQpIl0sWzAsMSwiXFxsZWZ0KFJee2l9Zl97Kn1cXEZjYWxcXHJpZ2h0KVxcb3RpbWVzX3tcXE9jYWxfe1kseX19XFxsZWZ0KFxcT2NhbF97WSx5fS9cXG1mcmFrX3t5fV57bi0xfVxccmlnaHQpIl0sWzIsMCwiSF57aX1cXGxlZnQoWF97eX1eeyhuKX0sXFxGY2FsfF97WF97eX1eeyhuKX19XFxyaWdodCkiXSxbMiwxLCJIXntpfVxcbGVmdChYX3t5fV57KG4tMSl9LFxcRmNhbHxfe1hfe3l9Xnsobi0xKX19XFxyaWdodCkiXSxbMSwzXSxbMCwyXSxbMiwzXSxbMCwxXV0=
\begin{tikzcd}
	{\left(R^{i}f_{*}\Fcal\right)\otimes_{\Ocal_{Y,y}}\left(\Ocal_{Y,y}/\mfrak_{y}^{n}\right)} && {H^{i}\left(X_{y}^{(n)},\Fcal|_{X_{y}^{(n)}}\right)} \\
	{\left(R^{i}f_{*}\Fcal\right)\otimes_{\Ocal_{Y,y}}\left(\Ocal_{Y,y}/\mfrak_{y}^{n-1}\right)} && {H^{i}\left(X_{y}^{(n-1)},\Fcal|_{X_{y}^{(n-1)}}\right)}
	\arrow[from=1-1, to=1-3]
	\arrow[from=1-1, to=2-1]
	\arrow[from=1-3, to=2-3]
	\arrow[from=2-1, to=2-3]
\end{tikzcd}$$
induced by the diagram 
$$% https://q.uiver.app/#q=WzAsNixbNCwwLCJYIl0sWzQsMSwiWSJdLFsyLDAsIlhfe3l9Xnsobil9Il0sWzIsMSwiXFxzcGVjKFxcT2NhbF97WSx5fS9cXG1mcmFrX3t5fV57bn0pIl0sWzAsMCwiWF97eX1eeyhuLTEpfSJdLFswLDEsIlxcc3BlYyhcXE9jYWxfe1kseX0vXFxtZnJha197eX1ee24tMX0pIl0sWzIsMF0sWzMsMV0sWzIsM10sWzUsM10sWzQsMl0sWzQsNV0sWzAsMV1d
\begin{tikzcd}
	{X_{y}^{(n-1)}} && {X_{y}^{(n)}} && X \\
	{\spec(\Ocal_{Y,y}/\mfrak_{y}^{n-1})} && {\spec(\Ocal_{Y,y}/\mfrak_{y}^{n})} && Y
	\arrow[from=1-1, to=1-3]
	\arrow[from=1-1, to=2-1]
	\arrow[from=1-3, to=1-5]
	\arrow[from=1-3, to=2-3]
	\arrow[from=1-5, to=2-5]
	\arrow[from=2-1, to=2-3]
	\arrow[from=2-3, to=2-5]
\end{tikzcd}$$
with rightmost square and outer rectangle Cartesian, implying that the leftmost square is Cartesian. The maps of (\ref{eqn: stepwise map on cohomology}) assemble to a diagram 
$$% https://q.uiver.app/#q=WzAsMTAsWzAsMSwiXFxsZWZ0KFJee2l9Zl97Kn1cXEZjYWxcXHJpZ2h0KVxcb3RpbWVzX3tcXE9jYWxfe1kseX19XFxsZWZ0KFxcT2NhbF97WSx5fS9cXG1mcmFrX3t5fV57bn1cXHJpZ2h0KSJdLFswLDIsIlxcbGVmdChSXntpfWZfeyp9XFxGY2FsXFxyaWdodClcXG90aW1lc197XFxPY2FsX3tZLHl9fVxcbGVmdChcXE9jYWxfe1kseX0vXFxtZnJha197eX1ee24tMX1cXHJpZ2h0KSJdLFsyLDEsIkhee2l9XFxsZWZ0KFhfe3l9Xnsobil9LFxcRmNhbHxfe1hfe3l9Xnsobil9fVxccmlnaHQpIl0sWzIsMiwiSF57aX1cXGxlZnQoWF97eX1eeyhuLTEpfSxcXEZjYWx8X3tYX3t5fV57KG4tMSl9fVxccmlnaHQpIl0sWzAsMCwiXFx2ZG90cyJdLFsyLDAsIlxcdmRvdHMiXSxbMCwzLCJcXHZkb3RzIl0sWzIsMywiXFx2ZG90cyJdLFswLDQsIihSXntpfWZfeyp9XFxGY2FsKV97eX0iXSxbMiw0LCJIXntpfVxcbGVmdChYX3t5fSxcXEZjYWx8X3tYX3t5fX1cXHJpZ2h0KSJdLFsxLDNdLFswLDJdLFsyLDNdLFswLDFdLFs0LDBdLFs1LDJdLFszLDddLFsxLDZdLFs3LDldLFs2LDhdLFs4LDldXQ==
\begin{tikzcd}
	\vdots && \vdots \\
	{\left(R^{i}f_{*}\Fcal\right)\otimes_{\Ocal_{Y,y}}\left(\Ocal_{Y,y}/\mfrak_{y}^{n}\right)} && {H^{i}\left(X_{y}^{(n)},\Fcal|_{X_{y}^{(n)}}\right)} \\
	{\left(R^{i}f_{*}\Fcal\right)\otimes_{\Ocal_{Y,y}}\left(\Ocal_{Y,y}/\mfrak_{y}^{n-1}\right)} && {H^{i}\left(X_{y}^{(n-1)},\Fcal|_{X_{y}^{(n-1)}}\right)} \\
	\vdots && \vdots \\
	{(R^{i}f_{*}\Fcal)_{y}} && {H^{i}\left(X_{y},\Fcal|_{X_{y}}\right)}
	\arrow[from=1-1, to=2-1]
	\arrow[from=1-3, to=2-3]
	\arrow[from=2-1, to=2-3]
	\arrow[from=2-1, to=3-1]
	\arrow[from=2-3, to=3-3]
	\arrow[from=3-1, to=3-3]
	\arrow[from=3-1, to=4-1]
	\arrow[from=3-3, to=4-3]
	\arrow[from=4-1, to=5-1]
	\arrow[from=4-3, to=5-3]
	\arrow[from=5-1, to=5-3]
\end{tikzcd}$$
and hence induces a map on the limits 
\begin{equation}\label{eqn: map on limits}
    \widehat{(R_{i}f_{*}\Fcal)}\longrightarrow \lim_{n\in\NN} H^{i}\left(X_{y}^{(n)},\Fcal|_{X_{y}^{(n)}}\right)
\end{equation}
where $\widehat{(R^{i}f_{*}\Fcal)}$ is the completion of $(R^{i}f_{*}\Fcal)_{y}$ as an $\Ocal_{Y,y}$-module with respect to the ideal $\mfrak_{y}$. The theorem of formal functions states that (\ref{eqn: map on limits}) is an isomorphism. Before we formally state and prove the theorem, we make a few reductions necessary for the proof. 
\begin{lemma}\label{lem: reduction to affine case}
    Let $f:X\to Y$ be a proper morphism of locally Noetherian schemes, $\Fcal\in\Coh(X)$, and $y\in Y$. Consider the Cartesian square 
    \begin{equation}\label{eqn: affine reduction cartesian square}
        % https://q.uiver.app/#q=WzAsNCxbMCwwLCJXIl0sWzAsMSwiXFxzcGVjKFxcT2NhbF97WSx5fSkiXSxbMiwxLCJZIl0sWzIsMCwiWCJdLFsxLDJdLFszLDIsImYiXSxbMCwzXSxbMCwxLCJnIiwyXV0=
        \begin{tikzcd}
            W && X \\
            {\spec(\Ocal_{Y,y})} && Y.
            \arrow[from=1-1, to=1-3]
            \arrow["g"', from=1-1, to=2-1]
            \arrow["f", from=1-3, to=2-3]
            \arrow[from=2-1, to=2-3]
        \end{tikzcd}
    \end{equation}
    Then there are isomorphisms $(R^{i}f_{*}\Fcal)\otimes_{\Ocal_{Y,y}}(\Ocal_{Y,y}/\mfrak_{y}^{n})\cong(R^{i}g_{*}\Fcal|_{W})\otimes_{\Ocal_{Y,y}}(\Ocal_{Y,y}/\mfrak_{y}^{n})$ and $H^{i}(X_{y}^{(n)},\Fcal|_{X_{y}^{(n)}})\cong H^{i}(W_{y}^{(n)},\Fcal|_{W_{y}^{(n)}})$. In particular, $\widehat{(R^{i}f_{*}\Fcal)}\cong\widehat{(R^{i}g_{*}\Fcal|_{W})}$. 
\end{lemma}
\begin{proof}
    For any $V\subseteq Y$ affine containing $y$, the morphism $\spec(\Ocal_{Y,y})\to Y$ factors as $\spec(\Ocal_{Y,y})\to V\to Y$. We have that $\spec(\Ocal_{Y,y})\to V$ is flat as it is a localization, and $V\to Y$ flat as it is an open immersion. In particular the bottom horizontal map of (\ref{eqn: affine reduction cartesian square}) is flat. By flat base change \Cref{prop: flat base change} we have isomorphisms $R^{i}f_{*}\Fcal\cong R^{i}g_{*}\Fcal|_{W}$. 
    
    For the second isomorphism, we use the diagram 
    $$% https://q.uiver.app/#q=WzAsNixbNCwwLCJYIl0sWzQsMSwiWSJdLFsyLDAsIlciXSxbMiwxLCJcXHNwZWMoXFxPY2FsX3tZLHl9KSJdLFswLDEsIlxcc3BlYyhcXE9jYWxfe1kseX0vXFxtZnJha197eX1ee259KSJdLFswLDAsIlhfe3l9Xnsobil9Il0sWzAsMSwiZiJdLFs1LDJdLFsyLDBdLFszLDFdLFsyLDNdLFs1LDRdLFs0LDNdXQ==
    \begin{tikzcd}
        {X_{y}^{(n)}} && W && X \\
        {\spec(\Ocal_{Y,y}/\mfrak_{y}^{n})} && {\spec(\Ocal_{Y,y})} && Y
        \arrow[from=1-1, to=1-3]
        \arrow[from=1-1, to=2-1]
        \arrow[from=1-3, to=1-5]
        \arrow[from=1-3, to=2-3]
        \arrow["f", from=1-5, to=2-5]
        \arrow[from=2-1, to=2-3]
        \arrow[from=2-3, to=2-5]
    \end{tikzcd}$$
    where the rightmost square and outer rectangle are Cartesian. This implies that the leftmost square is Cartesian giving an isomorphism $W_{y}^{(n)}\cong X_{y}^{(n)}$ inducing the isomorphism on cohomology $H^{i}(X_{y}^{(n)},\Fcal|_{X_{y}^{(n)}})\cong H^{i}(W_{y}^{(n)},\Fcal|_{W_{y}^{(n)}})$. 

    The final statement is obtained from the first on passage to the limit. 
\end{proof}
\begin{lemma}\label{lem: module statement}
    Let $f:X\to Y$ be a proper morphism of locally Noetherian schemes with $Y=\spec(A)$ for a Noetherian local ring $A$ with maximal ideal $\mfrak$, $\Fcal\in\Coh(X)$, and $y\in Y$. There is an isomorphism $\widehat{(R^{i}f_{*}\Fcal)}\cong H^{i}(X,\Fcal)\otimes_{A}\widehat{A}$ and $H^{i}(X_{\mfrak}^{(n)},\Fcal|_{X_{\mfrak}^{(n)}})\cong H^{i}(X,\Fcal/\mfrak^{n}\Fcal)$.
\end{lemma}
\begin{proof}
    Since $Y=\spec(A)$ is affine, $R^{i}f_{*}\Fcal$ is a coherent sheaf on $\spec(A)$ corresponding to a unique finitely generated $A$-module $H^{i}(X,\Fcal)$. The isomorphism $\widehat{(R^{i}f_{*}\Fcal)}\cong H^{i}(X,\Fcal)\otimes_{A}\widehat{A}$ is immediate from \Cref{prop: completions of rings} (ii). For the second isomorphism, we use the Cartesian square 
    $$% https://q.uiver.app/#q=WzAsNCxbMiwwLCJYIl0sWzIsMSwiWSJdLFswLDAsIlhfe1xcbWZyYWt9Xnsobil9Il0sWzAsMSwiXFxzcGVjKEEvXFxtZnJha157bn0pIl0sWzAsMSwiZiJdLFsyLDAsImpfe259Il0sWzMsMV0sWzIsM11d
    \begin{tikzcd}
        {X_{\mfrak}^{(n)}} && X \\
        {\spec(A/\mfrak^{n})} && Y
        \arrow["{j_{n}}", from=1-1, to=1-3]
        \arrow[from=1-1, to=2-1]
        \arrow["f", from=1-3, to=2-3]
        \arrow[from=2-1, to=2-3]
    \end{tikzcd}$$
    where we use that $j_{n}$ is affine as it is obtained from the affine morphism $\spec(A/\mfrak^{n})\to \spec(A)=Y$ by base change along $f$. Along affine morphisms there is an isomorphism $H^{i}(X^{(n)}_{\mfrak},\Fcal|_{X_{\mfrak}^{(n)}})\cong H^{i}(X,\Fcal/\mfrak^{n}\Fcal)$ since $j_{n*}$ is the quotient by $\mfrak^{n}$. 
\end{proof}
We are now prepared to state and prove the theorem of formal functions. 
\begin{theorem}[Formal Functions]\label{thm: formal functions}
    Let $f:X\to Y$ be a proper morphism of locally Noetherian schemes, $\Fcal\in\Coh(X)$, and $y\in Y$. Denote the $\mfrak_{y}$-adic completion of the $\Ocal_{Y,y}$-module $(R^{i}f_{*}\Fcal)_{y}$ by $\widehat{(R^{i}f_{*}\Fcal)}$ and $X_{y}^{(n)}$ the $n$-th order thickening of the fiber $X\times_{Y}\spec(\Ocal_{Y,y}/\mfrak_{y}^{n})$. The morphism (\ref{eqn: map on limits}) is an isomorphism 
    $$\widehat{(R^{i}f_{*}\Fcal)}\cong\lim_{n\in\NN}H^{i}(X_{y}^{(n)},\Fcal|_{X_{y}^{(n)}}).$$
\end{theorem}
\begin{proof}
    Without loss of generality, we can take $Y$ to be affine by \Cref{lem: reduction to affine case}. We can then apply \Cref{lem: module statement} in which case we can take $Y=\spec(A)$ with $A=\Ocal_{Y,y}$ and observe it suffices to prove that there is an isomorphism 
    $$H^{i}(X,\Fcal)\otimes_{A}\widehat{A}\longrightarrow\lim_{n\in\NN}H^{i}\left(X,\Fcal/\mfrak^{n}\Fcal\right).$$

    Note that we have a short exact sequence $0\to\mfrak^{n}\Fcal\to\Fcal\to\Fcal/\mfrak^{n}\Fcal\to0$ inducing by the long exact sequence in cohomology 
    $$H^{i}(X,\mfrak_{n}\Fcal)\xrightarrow{a_{n}} H^{i}(X,\Fcal)\xrightarrow{b_{n}}H^{i}(X,\Fcal/\mfrak^{n}\Fcal)\xrightarrow{c_{n}}H^{i+1}(X,\mfrak^{n}\Fcal)\xrightarrow{d_{n}}H^{i+1}(X,\Fcal).$$
    We seek to show that the induced maps $\beta_{n}H^{i}(X,\Fcal)/\img(a_{n})\to H^{i}(X,\Fcal/\mfrak^{n}\Fcal)$ which on passage to the limit gives
    $$\beta:\lim_{n\in\NN}H^{i}(X,\Fcal)/\img(a_{n})\to\lim_{n\in\NN}H^{i}(X,\Fcal/\mfrak^{n}\Fcal)$$
    is isomorphic to $H^{i}(X,\Fcal)\otimes_{A}\widehat{A}\to\lim_{n\in\NN}H^{i}\left(X,\Fcal/\mfrak^{n}\Fcal\right)$.

    We first show that $\lim_{n\in\NN}H^{i}(X,\Fcal)/\img(a_{n})\cong H^{i}(X,\Fcal)\otimes_{A}\widehat{A}$. By the Artin-Rees lemma \Cref{thm: Artin-Rees}, it suffices to show that the filtrations $\{\img(a_{n})\}$ and $\{\mfrak^{n}H^{i}(X,\Fcal)\}$ of $H^{i}(X,\Fcal)$ are equivalent in the sense of \Cref{def: equivalent filtrations}. Observe that for every $n\geq0$ that $\mfrak^{n}H^{i}(X,\Fcal)\subseteq\img(a_{n})$ as for each $x\in\mfrak^{n}$ we have a commutative diagram 
    $$% https://q.uiver.app/#q=WzAsMyxbMCwwLCJIXntpfShYLFxcRmNhbCkiXSxbMiwwLCJIXntpfShYLFxcRmNhbCkiXSxbMSwxLCJIXntpfShYLFxcbWZyYWtee259XFxGY2FsKSJdLFswLDEsIlxcY2RvdCB4Il0sWzAsMiwiXFxjZG90IHgiLDJdLFsyLDEsImFfe259IiwyXV0=
    \begin{tikzcd}
        {H^{i}(X,\Fcal)} && {H^{i}(X,\Fcal).} \\
        & {H^{i}(X,\mfrak^{n}\Fcal)}
        \arrow["{\cdot x}", from=1-1, to=1-3]
        \arrow["{\cdot x}"', from=1-1, to=2-2]
        \arrow["{a_{n}}"', from=2-2, to=1-3]
    \end{tikzcd}$$
    Conversely for each $n\geq0$ there is $m\geq0$ such that $\img(a_{m})\subseteq\mfrak^{n}H^{i}(X,\Fcal)$. Consider the $A$-algebra $\bigoplus_{j\geq0}\mfrak^{j}$ and the $\Ocal_{X}$-algebra $\Acal=\bigoplus_{j\geq0}\mfrak^{j}\Ocal_{X}$. We have a Cartesian diagram 
    $$% https://q.uiver.app/#q=WzAsNCxbMCwwLCJYJ1xcY29uZ1xcdW5kZXJsaW5le1xcc3BlY30oXFxBY2FsKSJdLFswLDEsIlxcc3BlYyhTKSJdLFsyLDAsIlgiXSxbMiwxLCJZIl0sWzIsMywiZiJdLFswLDJdLFswLDFdLFsxLDNdXQ==
    \begin{tikzcd}
        {X'\cong\underline{\spec}(\Acal)} && X \\
        {\spec(S)} && Y.
        \arrow[from=1-1, to=1-3]
        \arrow[from=1-1, to=2-1]
        \arrow["f", from=1-3, to=2-3]
        \arrow[from=2-1, to=2-3]
    \end{tikzcd}$$
    The morphism $\spec(S)\to Y$ is affine, so $X'\to X$ is affine and since $X'\to\spec(S)$ is proper the quasicoherent sheaf $\Ecal\bigoplus_{j\geq0}\mfrak^{j}\Fcal$ on $X$ is finitely generated as an $\Acal$-module and thus $\Ecal$ is the direct image of $\Ecal'$ on $X'$. But since $X'\to X$ is affine, we have 
    $$H^{i}(X',\Ecal')=H^{i}(X,\Ecal)=H^{i}(X,\Fcal)\oplus H^{i}(X,\mfrak\Fcal)\oplus\dots.$$
    By coherence of $\Ecal'$, the cohomology $H^{i}(X,\Ecal')$ is finitely generated as an $S$-module. By the Artin-Rees lemma \Cref{thm: Artin-Rees}, the filtration $\{a_{j}\left(H^{i}(X,\mfrak^{j}\Fcal)\right)\}$ of $H^{i}(X,\Fcal)$ is $\mfrak$-stable in the sense that $\mfrak\cdot a_{j}\left(H^{i}(X,\mfrak^{j}\Fcal)\right)=a_{j+1}\left(H^{i}(X,\mfrak^{j+1}\Fcal)\right)$ for all $j\geq c$ for some fixed $c\geq0$. So 
    $$\img(a_{n+c})=a_{n+c}\left(H^{i}(X,\mfrak^{n+c}\Fcal)\right)=\mfrak^{n}\cdot a_{c}\left(H^{i}(X,\mfrak^{c}\Fcal)\right)\subseteq\mfrak^{n}H^{i}(X,\Fcal)$$
    showing that the filtrations agree -- that is, $\lim_{n\in\NN}H^{i}(X,\Fcal)/\img(a_{n})\cong H^{i}(X,\Fcal)\otimes_{A}\widehat{A}$. 

    To show that $\beta$ is an isomorphism, we have a short exact sequence 
    $$0\to H^{i}(X,\Fcal)/\img(a_{n})\xrightarrow{\beta_{n}}H^{i}(X,\Fcal/\mfrak^{n}\Fcal)\to\ker(d_{n})\to0.$$
    Observe that the map $H^{i}(X,\Fcal)/\img(a_{n+1})\to H^{i}(X,\Fcal)/\img(a_{n})$ is surjective for each $n\geq0$. By \Cref{prop: completions of rings} (i), we get an exact sequence on passage to limits 
    $$0\to\lim_{n\in\NN}H^{i}(X,\Fcal)/\img(a_{n})\xrightarrow{\beta} \lim_{n\in\NN}H^{i}(X,\Fcal/\mfrak^{n}\Fcal)\to \lim_{n\in\NN}\ker(d_{n})\to0.$$
    It remains to show that $\lim_{n\in\NN}\ker(d_{n})=0$. The multiplication maps $\mfrak\times\ker(d_{n})\to\ker(d_{n+1})$, so $\Qcal=\bigoplus_{j\geq0}\ker(d_{n})$ is an $S$-module for which we choose a set of homogeneous generators with degree at most $N$. Since $\ker(d_{n})=\img(c_{n})$, and the image of $c_{n}$ is zero after multiplication by $\mfrak^{n}$ as $\Fcal/\mfrak^{n}\Fcal$ is. Thus $\ker(d_{n})$ is zero after multiplication by $\mfrak^{n}$ as well. That is, $\Qcal$ is zero after multiplication by $\mfrak^{N}S$ of $S$. Consider the composition $\mfrak^{r}\otimes\ker(d_{n})\xrightarrow{\times}\ker(d_{n+r})\xrightarrow{\mfrak^{n+r}\Fcal\hookrightarrow\mfrak^{n}\Fcal}\ker(d_{n})$. The multiplication map is surjective for $r\geq0$ and $n\in\NN$. So the composition is zero if $r\geq n$. The restriction $\ker(d_{n+r})\to\ker(d_{n})$ is zero if $r,n\geq N$. In particular the limit vanishes, giving the desired isomorphsim. 
\end{proof}
We conclude with a quick corollary of the theorem of formal functions. 
\begin{corollary}\label{corr: vanishing of derived pushforwards}
    Let $f:X\to Y$ be a proper morphism of locally Noetherian schemes and $r$ the maximal dimension of the fibers of $f$. Then $R^{i}f_{*}\Fcal=0$ for all $i>r$ and $\Fcal\in\Coh(X)$.  
\end{corollary}
\begin{proof}
    Fix $y\in Y$ and $i>r$. For every $n\geq 1$ the topological space underlying $X_{y}^{(n)}$ is homeomorphic to $X_{y}$ of dimension at most $r$. Thus $H^{i}(X_{y}^{(n)},\Fcal|_{X_{y}^{(n)}})=0$ for all $n$. By the theorem on formal functions $\widehat{(R^{i}f_{*}\Fcal)}=0$ and the morphism to the completion is an injection, so $(R^{i}f_{*}\Fcal)_{y}=0$. But $Y$ was arbitrary, so $R^{i}f_{*}\Fcal=0$. 
\end{proof}
\section{Lecture 16 -- 16th June 2025}\label{sec: lecture 16}
We discuss some consequences of the theorem of formal functions \Cref{thm: formal functions}. 

We begin with the following definition. 
\begin{definition}[$\Ocal$-Connected]\label{def: O-connected}
    Let $f:X\to Y$ be a proper morphism of locally Noetherian schemes. $f$ is $\Ocal$-connected if $f_{*}\Ocal_{X}\cong\Ocal_{Y}$. 
\end{definition}
The first consequence of the theorem of formal functions is that $\Ocal$-connectedness implies connectedness of fibers in the ordinary sense. 
\begin{proposition}\label{prop: O-connected implies connected fibers}
    Let $f:X\to Y$ be an $\Ocal$-connected morphism of locally Noetherian schemes. Then the fibers $X_{y}$ for all $y\in Y$ are connected. 
\end{proposition}
\begin{proof}
    Suppose to the contrary that $X_{y}$ is not connected for some $y$. For such $y$, we can reduce to by induction to the case of two connected components and write $X_{y}=X_{1}\sqcup X_{2}$ with $X_{1},X_{2}$ both open and closed in the fiber. Define $e_{i}\in H^{0}(X_{y},\Ocal_{X_{y}})$ to be the function taking value 1 on $X_{i}$ and 0 otherwise. We have $\widehat{\Ocal_{Y}}\cong\widehat{(f_{*}\Ocal_{X})}\cong\lim_{n\in\NN}H^{0}(X_{y}^{(n)},\Fcal|_{X_{y}^{(n)}})$ with the first isomorphism by $\Ocal$-connectedness and the second by the theorem of formal functions \Cref{thm: formal functions} since $f$ is proper as it is $\Ocal$-connected. Note that there are elements satisfiying the conditions of $e_{1},e_{2}$ in each term of the limit $H^{0}(X_{y}^{(n)},\Fcal|_{X_{y}^{(n)}})$. So $\widehat{\Ocal_{Y}}$ contains elements $e_{1},e_{2}$ such that $e_{1}e_{2}=0$ so $e_{1},e_{2}\in\mfrak_{y}$ which implies $e_{1}+e_{2}=1\in\mfrak_{y}$, a contradiction. 
\end{proof}
\begin{remark}
    The converse of \Cref{prop: O-connected implies connected fibers} is false. $\spec(A/I)\subseteq\spec(A)$ has connected fibers since the empty set is connected, but is clearly not $\Ocal$-connected as the direct image of the structure sheaf is the algebra sheaf $\widetilde{A/I}$. 
\end{remark}
The theorem of formal functions gives ways to factorize morphisms: the Stein factorization, Zariski's main theorem, and Grothendieck's variant of Zariski's main theorem. 
\begin{theorem}[Stein -- Factorization]\label{thm: Stein factorization}
    Let $f:X\to Y$ be a proper morphism of locally Noetherian schemes. $f$ admits a factorization as $X\xrightarrow{g}Y'\xrightarrow{\pi}Y$ with $g$ $\Ocal$-connected and $\pi$ finite. 
\end{theorem}
\begin{proof}
    Define $Y'=\underline{\spec}(f_{*}\Ocal_{X})$ and let $\pi:Y'\to Y$ be the structure morphism. By properness of $f$, $f_{*}\Ocal_{X}$ is finitely generated as an $\Ocal_{Y}$ module showing $\pi$ is finite. We want to show that $g:X\to Y'$ is $\Ocal$-connected. By the adjunction $\Mor_{\Sch_{Y}}(X,\underline{\spec}(\Acal))\cong\Mor_{\Alg_{\Ocal_{Y}}}(\Acal,f_{*}\Ocal_{X})$ natural in $\Ocal_{Y}$-algebras $\Acal$, we get a unique map $g:X\to Y'$ induced by the identity along that equivalence. To see that $g$ is $\Ocal$-connected, it suffices to check after composition with $\pi$. $(\pi\circ g)_{*}\Ocal_{X}\cong\pi_{*}\Ocal_{Y'}\cong f_{*}\Ocal_{X}$
\end{proof}
For Zariski's main theorem, we will need to introduce some additional language. 
\begin{definition}[Birational Morphism]\label{def: birational morphism}
    Let $f:X\to Y$ be a morphism. $f$ is a birational morphism if $f$ admits an inverse as a rational map. 
\end{definition}
In particular, $f|_{U}:U\to V$ is an isomorphism for nonempty dense open sets $U\subseteq X,V\subseteq Y$. 

Moreover, we will require the following preparatory lemma. 
\begin{lemma}\label{lem: birational with normal target is isomorphism}
    Let $f:X\to Y$ be a finite birational morphism between integral varieties such that $Y$ is normal. Then $f$ is an isomorphism. 
\end{lemma}
\begin{proof}
    Finite morphisms are in particular affine, so without loss of generality we can take $Y=\spec(A),X=\spec(B)$ and $f$ induced by the ring map $A\to B$. Since $f$ is dominant, $A\hookrightarrow B$ is injective and by birationality, we have $A\hookrightarrow B\hookrightarrow\Frac(B)=\Frac(A)$ with the last equality by birationality. $A$ is normal, hence integrally closed, and $B$ is an $A$-algebra that is finite as an $A$-module so $A\hookrightarrow B$ is an integral extension, but $A$ was integrally closed, so $A\cong B$ yielding the claim. 
\end{proof}
We can now state and prove Zariski's main theorem in earnest. 
\begin{theorem}[Zariski -- Main]\label{thm: Zariski main theorem}
    Let $f:X\to Y$ be a proper birational morphism between integral varieties such that $Y$ is normal. Then $f$ is $\Ocal$-connected. 
\end{theorem}
\begin{proof}
    Apply the Stein factorization \Cref{thm: Stein factorization} to get $f$ as a composite $X\xrightarrow{g}Y'\xrightarrow{\pi}Y$ where $g$ is $\Ocal$-connected and $\pi$ is finite. $g$ is $\Ocal$-connected and we claim that $\pi$ is an isomorphism. $\pi$ is birational as $\pi\circ(g\circ f^{-1}\circ\pi)=\pi$ shows that $g\circ f^{-1}$ is an inverse to $\pi$ as a rational map. In particular, $\pi$ is birational and proper as it is finite, so $\pi$ is surjective, and $g\circ f^{-1}\circ\pi=\pi^{-1}$. In particular, $\pi$ is a finite birational morphism between integral varities with normal target so \Cref{lem: birational with normal target is isomorphism} completes the proof. 
\end{proof}
\begin{example}
    Let $X$ be smooth and projective over $k$ and $\beta:\widetilde{X}\to X$ be the blowing up of $X$ smooth at a $k$-rational point $x$. We have $(R^{i}\beta_{*}\Ocal_{\widetilde{X}})_{x'}=0$ for all $i>0$ and $x'\neq x$ since $\beta$ is an isomorphism there. It remains tos how that the stalk at $x$ is also zero. For $X$ of dimension $d$, we have $E_{x}X\cong\PP^{d-1}_{k}$ and by the theorem of formal functions $\widehat{(R^{i}\beta_{*}\Ocal_{\widetilde{X}})}=0$ as the structure sheaf cohomology of projective space is acyclic and thus is zero on every thickeing of the fiber. 
\end{example}
Note that in general the blowups have cohomology that differs from the base. 

We now show Grothendieck's generalization of Zariski's main theorem. For this we introduce some further language. 
\begin{definition}[Isolated in Fiber]\label{def: isolated in fiber}
    Let $f:X\to Y$ be a proper morphism and $x\in X$ with image $f(x)=y\in Y$. $x$ is isolated in its fiber if $x$ is an irreducible connected component of the fiber $X_{y}$. 
\end{definition}
Let us consider some elementary properties of this notion. 
\begin{lemma}\label{lem: properties of O-connected morphisms}
    Let $f:X\to Y$ be an $\Ocal$-connected morphism. Then:
    \begin{enumerate}[label=(\roman*)]
        \item $f$ is surjective. 
        \item $x\in X$ is isolated in its fiber if and only if $f$ is unramified at $x$. 
    \end{enumerate}
\end{lemma}
\begin{proof}[Proof of (i)]
    Properness implies that the image of $X$ in $Y$ is closed and $\Ocal$-connectedness implies $f$ is dominant, that is, has dense image. But any closed dense subset of $Y$ is $Y$ itself, hence $f$ is surjective. 
\end{proof}
\begin{proof}[Proof of (ii)]
    $(\Rightarrow)$ Suppose that $f$ is unramified at $x$. We obtain by base change $f|_{X_{y}}:X_{y}\to\spec(\kappa(y))$ which is quasifinite as $f$ is unramified at $x$. Since further $f$ is locally of finite type, $x$ is isolated in its fiber. 

    $(\Leftarrow)$ If $x$ is isolated in its fiber, we seek to show $f^{\sharp}:\Ocal_{Y}\to f_{*}\Ocal_{X}$ is an isomorphism. But this is precisely $\Ocal$-connectedness and implies unramifiedness. It suffices to show that for every $U\subseteq X$ containing $x$ there is $V\subseteq Y$ open such that $f^{-1}(V)\subseteq U$. But by properness, $X\setminus U$ is closed so $f(X\setminus U)$ is closed. So $Y\setminus f(X\setminus U)$ satisfies the required conditions. 
\end{proof}
We are now prepared to show Grothendieck's variant of Zariski's main theorem. 
\begin{theorem}[Grothendieck -- Zariski's Main Theorem]\label{thm: Grothendieck ZMT}
    Let $f:X\to Y$ be a proper morphism. Then there exists a diagram 
    $$% https://q.uiver.app/#q=WzAsNCxbMSwwLCJYIl0sWzAsMSwiWF97MH0iXSxbMiwxLCJZJyJdLFsxLDIsIlkiXSxbMSwyLCJpIiwwLHsibGFiZWxfcG9zaXRpb24iOjQwLCJjdXJ2ZSI6MSwic3R5bGUiOnsidGFpbCI6eyJuYW1lIjoiaG9vayIsInNpZGUiOiJ0b3AifX19XSxbMCwzLCJmIiwxLHsibGFiZWxfcG9zaXRpb24iOjcwLCJjdXJ2ZSI6LTF9XSxbMSwwLCIiLDEseyJzdHlsZSI6eyJ0YWlsIjp7Im5hbWUiOiJob29rIiwic2lkZSI6InRvcCJ9fX1dLFswLDIsImciXSxbMiwzLCJcXHBpIl0sWzEsMywiZnxfe1hfezB9fSIsMl1d
    \begin{tikzcd}
        & X \\
        {X_{0}} && {Y'} \\
        & Y
        \arrow["g", from=1-2, to=2-3]
        \arrow["f"{description, pos=0.7}, curve={height=-6pt}, from=1-2, to=3-2]
        \arrow[hook, from=2-1, to=1-2]
        \arrow["i"{pos=0.4}, curve={height=6pt}, hook, from=2-1, to=2-3]
        \arrow["{f|_{X_{0}}}"', from=2-1, to=3-2]
        \arrow["\pi", from=2-3, to=3-2]
    \end{tikzcd}$$
    where:
    \begin{itemize}
        \item $X_{0}\subseteq X$ is the open subset of points isolated in their fiber, 
        \item $f|_{X_{0}}$ factors as $\pi\circ i$ where $i$ is an open embedding and $i$ is finite, 
        \item and $f=\pi\circ g$. 
    \end{itemize}
\end{theorem}
\begin{proof}
    Take the Stein factorization $X\xrightarrow{g}Y'\xrightarrow{\pi}X$. Let $X_{0}$ be the set of points isolated in their fiber with respect to $f$. But $X_{0}$ is also the set of points isolated in their fiber with respect to $g$ as $\pi$ is finite. Every point of $Y'$ is isolated in its fiber and $X_{0}$ is the set of $x\in X$ such that $g$ is unramified at $x$ by \Cref{lem: properties of O-connected morphisms} (ii) and thus $X_{0}$ is open by openness of the unramified locus. 

    It remains to show $g|_{X_{0}}:X_{0}\to Y'\setminus g(X\setminus X_{0})$ is an isomorphism. By the proof of \Cref{lem: properties of O-connected morphisms} (i) $g|_{X_{0}}$ is surjective and injective an injective by \Cref{prop: O-connected implies connected fibers} and using that each point is isolated in its fiber and injective. So $g$ is a homeomorphism that is $\Ocal$-connected, and hence an isomorphism. 
\end{proof}
We conclude with a proof of the equivalence of some conditions of morphisms. 
\begin{corollary}\label{corr: equivalent conditions on morphisms}
    Let $f:X\to Y$ be a morphism of locally Noetherian schemes. The following are equivalent: 
    \begin{enumerate}[label=(\roman*)]
        \item $f$ is finite.
        \item $f$ is affine and proper.
        \item $f$ is proper and quasifinite. 
    \end{enumerate}
\end{corollary}
\begin{proof}
    (a)$\Rightarrow$(b) and (a)$\Rightarrow$(c) are immediate. 
    
    (b)$\Rightarrow$(a) If $f$ is affine and proper, and for $V=\spec(A)\subseteq Y$ with preimage $U=f^{-1}(V)=\spec(B)\subseteq X$ since $f$ is affine, we have by properness of $f|_{U}$that $\pi_{*}\Ocal_{U}$ is a coherent $\Ocal_{V}$-module so $B$ is finitely generated over $A$ showing $f|_{U}$ and thus $f$ is finite. 

    (c)$\Rightarrow$(a) Suppose $f$ is proper and quasifinite. So $X=X_{0}$ where $X_{0}$ is the set of points isolated in their fiber since if $x\in X$ with image $y=f(x)$ we have $f|_{X_{y}}:X_{y}\to \spec(\kappa(y))$ is proper and quasifinite and hence finite with $X_{y}$ the disjoint union of the Zariski spectra of finite extensions of $\kappa(y)$. By \Cref{thm: Grothendieck ZMT}, $f$ admits a factorization as $\pi\circ g$ where $g:X\to Y$ is an open immersion and $\pi$ is finite. But $g$ is proper and hence a closed immersion, thus finite as the composition of two finite morphisms. 
\end{proof}
\section{Lecture 17 -- 23rd June 2025}\label{sec: lecture 17}
We begin a discussion of base-change theorems. The setting is as follows: for $f:X\to Y$ projective with $Y$ Noetherian and $\Fcal\in\Coh(X)$we have that $R^{i}f_{*}\Fcal$ are coherent and for $\spec(A)\subseteq Y$ an affine open subset there are isomorphisms $R^{i}f_{*}\Fcal|_{\spec(A)}\cong H^{i}(X\times_{Y}\spec(A),\Fcal|_{X\times_{Y}\spec(A)})$. The theorem of formal functions \Cref{thm: formal functions} states that the completion $H^{i}(X\times_{Y}\spec(A),\Fcal|_{X\times_{Y}\spec(A)})$ at $y$ can be computed as the limit of cohomologies of the fiber $\lim_{n\in\NN}H^{i}(X_{y}^{(n)},\Fcal|_{X_{y}^{(n)}})$ though it is not in general true that $R^{i}f_{*}\Fcal\otimes\kappa(y)\cong H^{i}(X_{y},\Fcal|_{X_{y}})$. We seek to understand the connection between the stalk of $R^{i}f_{*}\Fcal$ and $H^{i}(X_{y},\Fcal_{y})$, or, more generally, for a Cartesian square 
\begin{equation}\label{eqn: proper base change}
    % https://q.uiver.app/#q=WzAsNCxbMCwwLCJYJyJdLFswLDEsIlknIl0sWzIsMCwiWCJdLFsyLDEsIlkiXSxbMCwyLCJnJyJdLFsyLDMsImYiXSxbMCwxLCJmJyIsMl0sWzEsMywiZyIsMl1d
    \begin{tikzcd}
        {X'} && X \\
        {Y'} && Y
        \arrow["{g'}", from=1-1, to=1-3]
        \arrow["{f'}"', from=1-1, to=2-1]
        \arrow["f", from=1-3, to=2-3]
        \arrow["g"', from=2-1, to=2-3]
    \end{tikzcd}
\end{equation}
the relationship between $g^{*}R^{i}f_{*}\Fcal$ and $R^{i}f'_{*}g'^{*}\Fcal$ for $\Fcal\in\Coh(X)$. Evidently, the relationship between $R^{i}f_{*}\Fcal\otimes\kappa(y)$ and $H^{i}(X_{y},\Fcal|_{X_{y}})$ is recovered from the preceding discussion by taking $g:\spec(\kappa(y))=Y'\to Y=\spec(\Ocal_{Y,y})$. 

In the case where $\Fcal\in\Coh(X)$ is $Y$-flat, this is the content of the proper base change theorem. We follow the exposition of \cite[p. 46]{Mumford}, keeping the notation of (\ref{eqn: proper base change}).
\begin{theorem}[Proper Base Change]\label{thm: proper base change}
    Let $f:X\to Y$ be a projective morphism with $Y=\spec(A)$ a Noetherian affine scheme and $\Fcal\in\Coh(X)$ that is $Y$-flat. There exists a finite complex $K^{\bullet}=0\to K^{0}\to K^{1}\to\dots\to K^{n}\to0$ of finite projective $A$-modules $K^{i}$ such that for all $A\to B$ there is an isomorphism 
    $$H^{i}(X\times_{Y}\spec(B),g'^{*}\Fcal)\cong H^{i}(K^{\bullet}\otimes_{A}B).$$
\end{theorem}
We defer the proof to the subsequent lecture, and turn now to a discussion of applications. 

For our first application, we will require the following lemma. 
\begin{lemma}\label{lem: tensor residue field is closed}
    Let $A$ be a Noetherian ring and $\varphi:M_{1}\to M_{2}$ a morphism between finite $A$-modules. Then 
    $$\{\pfrak\in\spec(A):\varphi\otimes\id_{\kappa(\pfrak)}:M_{1}\otimes_{A}\kappa(\pfrak)\to M_{2}\otimes_{A}\kappa(\pfrak)\text{ is the zero map}\}\subseteq\spec(A)$$
    is closed. 
\end{lemma}
We now state and prove the proposition of interest. 
\begin{proposition}\label{prop: upper semicontinuity of local cohomology}
    Let $f:X\to Y$ be projective with $Y$ Noetherian and $\Fcal\in\Coh(X)$ $Y$-flat. Then:
    \begin{enumerate}[label=(\roman*)]
        \item The function $Y\to\ZZ$ by $y\mapsto h^{i}(X_{y},\Fcal|_{X_{y}})$ is upper semicontinuous: for all $c$ the set of points where $h^{i}(X_{y},\Fcal|_{X_{y}})\geq c$ is closed in $Y$. 
        \item The function $Y\to \ZZ$ by $y\mapsto \chi(X_{y},\Fcal|_{X_{y}})$ is locally constant. 
    \end{enumerate}
\end{proposition}
\begin{proof}[Proof of (i)]
    By locality on target, we can, without loss of generality, take $Y=\spec(\Ocal_{Y,y}),Y'=\spec(\kappa(y))$ and apply \Cref{thm: proper base change} to observe that $H^{i}(X_{y},\Fcal|_{X_{y}})\cong H^{i}(K^{\bullet}\otimes_{A}\kappa(y))$. In particular 
    \begin{align*}
        h^{i}(X_{y},\Fcal|_{X_{y}})&=\dim(\ker(d^{i}\otimes\id_{\kappa(y)})) - \dim(\img(d^{i}\otimes\id_{\kappa(y)})) \\
        &= \dim\ker(d^{i}\otimes\id_{\kappa(y)})-\dim(\img(d^{i}\otimes\id_{\kappa(y)})) - \dim(\img(d^{i-1}\otimes\id_{\kappa(y)}))
    \end{align*}
    so to show that $h^{i}$ is upper semicontinuous it suffices to show that $\dim(\img(d^{i}\otimes\id_{\kappa(y)})) + \dim(\img(d^{i-1}\otimes\id_{\kappa(y)}))$ is lower semicontinuous. Applying \Cref{lem: tensor residue field is closed} to $M_{1}=\bigwedge^{c}K^{i},M_{2}=\bigwedge^{c}K^{i+1}$, we know that $\dim(\img(d^{i}\otimes\id_{\kappa(y)}))<c$ if and only if $\bigwedge^{c}(d^{i}\otimes\id_{\kappa(y)})=0$ if and only if $\varphi\otimes\id_{\kappa(y)}=0$ where $\varphi:\bigwedge^{c}K^{i}\to\bigwedge^{c}K^{i+1}$. In our case, we can take $\bigwedge^{c}K^{i}=A^{\oplus n},\bigwedge^{c}K^{i+1}=A^{\oplus m}$ so $\varphi\otimes\kappa(\pfrak)$ is given by an $n\times m$ matrix which vanishes if and only if each entry of the matrix lies in $\pfrak$. That is, the function $\varphi\otimes\id_{\kappa(y)}$ decreases in rank on the closed subset defined by the vanishing locus of the matrix entries, ie. is lower semicontinuous, showing the claim. 
\end{proof}
\begin{proof}[Proof of (ii)]
    Using the second line of the displayed equation in the proof of (i), we have 
    \begin{align*}
        \chi(X_{y},\Fcal|_{X_{y}})&=\sum_{i\geq0}(-1)^{i}h^{i}(X_{y},\Fcal|_{X_{y}})\\ 
        &=\sum_{i\geq0}(-1)^{i}\dim(K^{i}\otimes\id_{\kappa(y)})\\
        &=\sum_{i\geq0}(-1)^{i}\mathrm{rank}(K^{i})
    \end{align*}
    which is constant by \Cref{thm: proper base change}. 
\end{proof}
\begin{remark}
    We note that we may not drop the $Y$-flatness assumption on $\Fcal\in\Coh(X)$. Let $f:\Bl_{0}\A^{2}_{k}\to\A^{2}_{k}$ and $E_{0}\A^{2}_{k}\cong\PP^{1}_{k}$. Let $\Fcal\cong\Ocal_{\Bl_{0}\A^{2}_{k}}(E_{0}\A^{2}_{k})$. $\Fcal$ is not $\A^{2}_{k}$-flat as $f$ is not so -- the fiber dimension is not constant. For a closed point $y\in\A^{2}_{k}$, $(\Bl_{0}\A^{2}_{k})_{y}$ is either a point or $\PP^{1}_{k}$ so 
    $$\Fcal_{y}=\begin{cases}
        \spec(\kappa(y)) & y\neq0 \\
        \Ocal_{\Bl_{0}\A^{2}_{k}}(E_{0}\A^{2}_{k})|_{E_{0}\A^{2}_{k}} & y=0.
    \end{cases}$$

    \Cref{prop: upper semicontinuity of local cohomology} (ii) also requires the $Y$-flatness assumption. $\id_{X}:X\to X$ for $X$ connected is a flat morphism. For $x\in X$ a closed point, the skyscraper sheaf $\Fcal=\iota_{x}\kappa(x)$ is a coherent sheaf but $\Fcal$ is not flat as
    $$h^{0}(X_{y},\Fcal_{y})=\begin{cases}
        0 & y\neq x \\ 1 & y=x
    \end{cases}$$
    and $h^{i}(X_{y},\Fcal_{y})=0$ for all $i\geq 1$ so the Euler characteristic is
    $$\chi(X_{y},\Fcal_{y})=\begin{cases}
        1 & y\neq x \\ 0 & y=x
    \end{cases}$$
    which is not constant. 
\end{remark}
We now relate constancy of the cohomological dimension functor with local freeness of the higher direct images. 
\begin{proposition}\label{prop: constant cohomology dimension iff higher direct images are locally free}
    Let $f:X\to Y$ be a projective morphism with $Y$ Noetherian and $\Fcal\in\Coh(X)$ $Y$-flat. Then: 
    \begin{enumerate}[label=(\alph*)]
        \item  $y\mapsto h^{i}(X_{y},\Fcal|_{X_{y}})$ is constant. 
        \item $R^{i}f_{*}\Fcal$ is locally free and there is an isomorphism $R^{i}f_{*}\Fcal\cong H^{i}(X_{y},\Fcal|_{X_{y}})$. 
    \end{enumerate}
\end{proposition}
\begin{proof}
    (a)$\Rightarrow$(b) This is the statement of \Cref{ex: local flatness}. 

    (b)$\Rightarrow$(a) If $R^{i}f_{*}\Fcal$ is locally free and $R^{i}f_{*}\Fcal\cong H^{i}(X_{y},\Fcal|_{X_{y}})$ then the local cohomological dimension is constant. 
\end{proof}
\begin{corollary}\label{corr: also free in -1}
    Let $f:X\to Y$ be a projective morphism with $Y$ Noetherian and $\Fcal\in\Coh(X)$ $Y$-flat. If $y\mapsto h^{i}(X_{y},\Fcal|_{X_{y}})$ is constant then $R^{i-1}f_{*}\Fcal\otimes\kappa(y)\cong H^{i-1}(X_{y},\Fcal|_{X_{y}})$. 
\end{corollary}
\begin{remark}
    Note that in \Cref{corr: also free in -1} $R^{i-1}f_{*}\Fcal\otimes\kappa(y)$ need not be locally free -- that is, $h^{i-1}(X_{y},\Fcal|_{X_{y}})$ need not be (locally) constant. 
\end{remark}
\begin{corollary}\label{corr: vanishing in degree -1}
    Let $f:X\to Y$ be projective with $Y$ Noetherian and $\Fcal\in\Coh(X)$ $Y$-flat. If $H^{i}(X_{y},\Fcal|_{X_{y}})=0$ for all $y\in Y$ then $R^{i-1}f_{*}\Fcal\otimes\kappa(y)\cong H^{i}(X_{y},\Fcal|_{X_{y}})$. 
\end{corollary}
We can also show a vanishing result. 
\begin{proposition}\label{prop: cohomology base change vanishing}
    Let $f:X\to Y$ be a projective morphism with $Y$ Noetherian and $\Fcal\in\Coh(X)$ $Y$-flat. Fix $i_{0}\in\NN$. If $R^{i}f_{*}\Fcal=0$ for all $i\geq i_{0}$ then $H^{i}(X_{y},\Fcal_{y})=0$ for all $i\geq i_{0}$. 
\end{proposition}
\begin{proof}
    We know that $H^{i}(X_{y},\Fcal|_{X_{y}})=0$ for $i>\dim(X)$. So we can assume $H^{i}(X_{y},\Fcal|_{X_{y}})=0$ for all $i>i_{0}$. By \Cref{corr: vanishing in degree -1}, $R^{i-1}f_{*}\Fcal\otimes\kappa(y)\cong H^{i-1}(X_{y},\Fcal|_{X_{y}})\cong H^{i}(X_{y},\Fcal|_{X_{y}})=0$ giving the claim. 
\end{proof}
An especially nice situation is when the map $g:Y'\to Y$ of (\ref{eqn: proper base change}) is flat. 
\begin{theorem}[Flat Base Change]\label{thm: flat base change}
    Let $f:X\to Y$ be a projective morphism with $Y$ Noetherian, $\Fcal\in\Coh(X)$ $Y$-flat, and $g:Y'\to Y$ flat inducing a Cartesian diagram 
    $$% https://q.uiver.app/#q=WzAsNCxbMCwwLCJYJyJdLFswLDEsIlknIl0sWzIsMCwiWCJdLFsyLDEsIlkiXSxbMCwyLCJnJyJdLFsyLDMsImYiXSxbMCwxLCJmJyIsMl0sWzEsMywiZyIsMl1d
    \begin{tikzcd}
        {X'} && X \\
        {Y'} && Y.
        \arrow["{g'}", from=1-1, to=1-3]
        \arrow["{f'}"', from=1-1, to=2-1]
        \arrow["f", from=1-3, to=2-3]
        \arrow["g"', from=2-1, to=2-3]
    \end{tikzcd}$$
    Then the natural morphism $g^{*}R^{i}f_{*}\Fcal\to R^{i}f'_{*}g'^{*}\Fcal$ is an isomorphism. 
\end{theorem}
We have encountered this statement before as \Cref{prop: flat base change}.
\begin{proof}
    Isomorphisms of quasicoherent sheaves can be checked affine-locally. It thus suffices to consider the case $Y'=\spec(A'),Y=\spec(A)$ affine. Then the assertion follows by observing 
    $$H^{i}(K^{\bullet})\otimes_{A}A'\longrightarrow H^{i}(K^{\bullet}\otimes_{A}A')$$
    is an isomorphism by flatness of $A'$ over $A$. 
\end{proof}
\newpage
\printbibliography
\end{document}